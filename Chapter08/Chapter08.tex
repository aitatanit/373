\chapter{Orbital Angular Momentum}\label{sorb}
\section{Introduction}
As is well-known, angular momentum plays a vitally important role in the classical description
of three-dimensional motion. Let us now investigate the role of angular momentum in the quantum mechanical description of such motion.

\section{Angular Momentum Operators}\label{s8.2}
In classical mechanics, the vector angular momentum, {\bf L}, of a particle of 
position vector ${\bf r}$ and linear momentum
${\bf p}$ is defined as
\begin{equation}\label{e8.0}
{\bf L} = {\bf r}\times {\bf p}.
\end{equation}
It follows that
\begin{eqnarray}\label{e8.1}
L_x &=& y\,p_z - z\,p_y,\\[0.5ex]
L_y &=& z\,p_x - x\,p_z,\\[0.5ex]
L_z &=& x\,p_y - y\,p_x.\label{e8.3}
\end{eqnarray}
Let us, first of all, consider whether it is possible to use the above expressions as the
definitions of the operators corresponding to the components of angular
momentum in quantum mechanics, assuming that the $x_i$ and $p_i$ (where $x_1\equiv x$, $p_1\equiv p_x$, $x_2\equiv y$, {\em etc.})
correspond to the appropriate quantum mechanical position and momentum operators. The first point to note is that expressions (\ref{e8.1})--(\ref{e8.3}) are {\em unambiguous} with respect to the order of the terms in multiplicative factors, since the various position and momentum operators
appearing in them all {\em commute}\/
with one another [see Eqs.~(\ref{commxp})]. Moreover, given that the $x_i$ and
the $p_i$ are Hermitian operators, it is easily seen that the $L_i$ are
also {\em Hermitian}. This is important, since only Hermitian operators can represent physical variables in quantum mechanics (see Sect.~\ref{s4.6}).
We, thus, conclude that Eqs.~(\ref{e8.1})--(\ref{e8.3}) are plausible
definitions for the quantum mechanical operators which represent the components of angular momentum.

Let us now derive the commutation relations for the $L_i$. 
For instance,
\begin{eqnarray}
[L_x, L_y]& =& [y\,p_z-z\,p_y, z\,p_x-x\,p_z] = y\,p_x\,[p_z,z] + x\,p_y\,[z,p_z]\nonumber\\[0.5ex] &= &{\rm i}\,\hbar\,(x\,p_y-y\,p_x) = {\rm i}\,\hbar\,L_z,
\end{eqnarray}
where use has been made of the definitions of the $L_i$ [see Eqs.~(\ref{e8.1})--(\ref{e8.3})], and commutation relations (\ref{commxx})--(\ref{commxp})
for the $x_i$ and $p_i$. There are two similar commutation relations:
one for $L_y$ and $L_z$, and one for $L_z$ and $L_x$. Collecting all of these
commutation relations together, we obtain
\begin{eqnarray}\label{e8.6}
[L_x, L_y]&=& {\rm i}\,\hbar\,L_z,\\[0.5ex]
[L_y, L_z]&=& {\rm i}\,\hbar\,L_x,\\[0.5ex]
[L_z,L_x]&=& {\rm i}\,\hbar\,L_y.\label{e8.8}
\end{eqnarray}

By analogy with classical mechanics, the operator $L^2$, which represents
the {\em magnitude squared}\/ of the angular momentum vector,  is defined
\begin{equation}\label{e8.9}
L^2 = L_x^{\,2} + L_y^{\,2} + L_z^{\,2}.
\end{equation}
Now, it is easily demonstrated that if $A$ and $B$ are two general
operators then
\begin{equation}\label{e8.10}
[A^2,B] = A\,[A, B] +[A,B]\,A.
\end{equation}
Hence, 
\begin{eqnarray}
[L^2,L_x] &=& [L_y^{\,2},L_x] + [L_z^{\,2},L_x]\nonumber\\[0.5ex]
&=& L_y\,[L_y,L_x] + [L_y,L_x]\,L_y + L_z\,[L_z,L_x] + [L_z, L_x]\,L_z\nonumber\\[0.5ex]
&=&  {\rm i}\,\hbar\left(-L_y\,L_z - L_z\,L_y + L_z\,L_y + L_y\,L_z\right) = 0,
\end{eqnarray}
where use has been made of Eqs.~(\ref{e8.6})--(\ref{e8.8}). In other words,
$L^2$ commutes with $L_x$. Likewise, it is easily demonstrated that
$L^2$ also commutes with $L_y$, and with $L_z$. Thus,
\begin{equation}\label{e8.12}
[L^2, L_x] = [L^2, L_y] = [L^2,L_z] = 0.
\end{equation}

Recall, from Sect.~\ref{smeas}, that in order for two physical quantities
to be (exactly) measured {\em simultaneously}, the operators which represent
them in quantum mechanics must {\em commute} with one another. Hence, 
 the commutation relations (\ref{e8.6})--(\ref{e8.8}) and (\ref{e8.12})
imply that we can only simultaneously measure the magnitude squared of
the angular momentum vector, $L^2$, together with, at most,  {\em one}\/ of its
Cartesian components. By convention, we shall always choose to measure
the $z$-component, $L_z$. 

Finally, it is helpful to define the operators
\begin{equation}\label{e8.13}
L_\pm = L_x\pm {\rm i}\,L_y.
\end{equation}
Note that $L_+$ and $L_-$ are not Hermitian operators, but are
the Hermitian conjugates of one another (see Sect.~\ref{s4.6}): {\em i.e.},
\begin{equation}\label{e8.14}
(L_\pm)^\dag = L_\mp,\\[0.5ex]
\end{equation}
Moreover, it is easily seen that
\begin{eqnarray}
L_+\,L_- &=& (L_x+{\rm i}\,L_y)\,(L_x-{\rm i}\,L_y)
= L_x^{\,2} + L_y^{\,2} - {\rm i}\,[L_x, L_y]
= L_x^{\,2} + L_y^{\,2} + \hbar\,L_z\nonumber\\[0.5ex]
&=& L^2 - L_z^{\,2} + \hbar\,L_z.\label{e8.15}
\end{eqnarray}
Likewise,
\begin{equation}\label{e8.17}
L_-\,L_+ = L^2 - L_z^{\,2} -\hbar\,L_z,
\end{equation}
giving
\begin{equation}
[L_+, L_-] = 2\,\hbar\,L_z.
\end{equation}
We also have
\begin{eqnarray}\label{e8.19}
[L_+,L_z] &=& [L_x,L_z] + {\rm i}\,[L_y,L_z] = -{\rm i}\,\hbar\,L_y - \hbar\,L_x= - \hbar\,L_+,
\end{eqnarray}
and, similarly,
\begin{equation}
[L_-,L_z] = \hbar\,L_-.
\end{equation}

\section{Representation of Angular Momentum}\label{s8.3}
Now, we saw earlier, in Sect.~\ref{s7.2}, that the operators, $p_i$,  which represent
the Cartesian components of linear momentum in quantum mechanics, can be represented
as the spatial differential operators $-{\rm i}\,\hbar\,\partial/\partial x_i$. 
Let us now investigate whether angular momentum operators can similarly
be represented as spatial differential operators.

It is most convenient to perform our investigation using conventional
{\em spherical polar coordinates}: {\em i.e.}, $r$, $\theta$, and $\phi$. These are
defined with respect to our usual Cartesian coordinates as follows:
\begin{eqnarray}\label{e8.21}
x &=& r\,\sin\theta\,\cos\phi,\\[0.5ex]
y&=& r\,\sin\theta\,\sin\phi,\\[0.5ex]
z&=& r\,\cos\theta.\label{e8.23}
\end{eqnarray}
It follows, after some tedious analysis, that
\begin{eqnarray}
\frac{\partial}{\partial x} &=& \sin\theta\,\cos\phi\,\frac{\partial}{\partial r} + \frac{\cos\theta\,\cos\phi}{r}\,\frac{\partial}{\partial\theta} - \frac{\sin\phi}{r\,\sin\theta}\,\frac{\partial}{\partial\phi},\label{e8xx}\\[0.5ex]
\frac{\partial}{\partial y} &=& \sin\theta\,\sin\phi\,\frac{\partial}{\partial r} + \frac{\cos\theta\,\sin\phi}{r}\,\frac{\partial}{\partial\theta} + \frac{\cos\phi}{r\,\sin\theta}\,\frac{\partial}{\partial\phi},\label{e8yy}\\[0.5ex]
\frac{\partial}{\partial z} &=& \cos\theta\,\frac{\partial}{\partial r} -\frac{\sin\theta}{r}\,\frac{\partial}{\partial \theta}.\label{e8zz}
\end{eqnarray}
Making use of the definitions (\ref{e8.1})--(\ref{e8.3}), (\ref{e8.9}), and (\ref{e8.13}), the fundamental representation (\ref{e6.12})--(\ref{e6.14}) of the $p_i$ operators as spatial differential operators, the Eqs.~(\ref{e8.21})--(\ref{e8zz}), and a great deal of tedious algebra, we finally obtain
\begin{eqnarray}
L_x &=& - {\rm i}\,\hbar\left(-\sin\phi\,\frac{\partial}{\partial\theta}
-\cos\phi\,\cot\theta\,\frac{\partial}{\partial\phi}\right),\\[0.5ex]
L_y &=& - {\rm i}\,\hbar\left(\cos\phi\,\frac{\partial}{\partial\theta}
-\sin\phi\,\cot\theta\,\frac{\partial}{\partial\phi}\right),\\[0.5ex]
L_z &=& -{\rm i}\,\hbar\,\frac{\partial}{\partial\phi},\label{e8.26}
\end{eqnarray}
as well as
\begin{equation}
L^2 = -\hbar^2\left[\frac{1}{\sin\theta}\frac{\partial}{\partial\theta}\left(
\sin\theta\,\frac{\partial}{\partial\theta}\right) + \frac{1}{\sin^2\theta}\frac{\partial^2}{\partial\phi^2}\right],
\end{equation}
and
\begin{equation}\label{e8.28}
L_\pm = \hbar\,{\rm e}^{\pm{\rm i}\,\phi}\left(\pm\frac{\partial}{\partial\theta} +{\rm i}\,\cot\theta\,\frac{\partial}{\partial\phi}\right).
\end{equation}
We, thus, conclude that all of our angular momentum operators can be represented
as differential operators involving the {\em angular}\/ spherical
coordinates, $\theta$ and $\phi$, but not involving the {\em radial}\/ coordinate,
$r$.

\section{Eigenstates of Angular Momentum}\label{seian}
Let us find the simultaneous eigenstates of the angular momentum
operators $L_z$ and $L^2$. Since both of these operators can
be represented as purely angular differential operators, it stands to
reason that their eigenstates  only depend on the angular coordinates
$\theta$ and $\phi$. Thus,
we can write
\begin{eqnarray}\label{e8.29}
L_z\,Y_{l,m}(\theta,\phi) &=& m\,\hbar\,Y_{l,m}(\theta,\phi),\\[0.5ex]
L^2\,Y_{l,m}(\theta,\phi) &=& l\,(l+1)\,\hbar^{\,2}\,Y_{l,m}(\theta,\phi).\label{e8.30}
\end{eqnarray}
Here, the $Y_{l,m}(\theta,\phi)$ are the eigenstates in question, whereas the dimensionless
quantities $m$ and $l$ parameterize the eigenvalues of $L_z$ and $L^2$,
which are $m\,\hbar$ and $l\,(l+1)\,\hbar^2$, respectively. Of course,
we expect the $Y_{l,m}$ to be both mutually orthogonal and properly normalized
(see Sect.~\ref{seig}), so that
\begin{equation}\label{e8.31}
\oint Y^{\,\ast}_{l',m'}(\theta,\phi)\,Y_{l,m}(\theta,\phi)\,d\Omega = \delta_{ll'}\,\delta_{mm'},
\end{equation}
where $d\Omega = \sin\theta\,d\theta\,d\phi$ is an element of solid angle,
and the integral is over all solid angle.

Now, 
\begin{eqnarray}
L_z\,(L_+\,Y_{l,m}) &=& (L_+\,L_z + [L_z, L_+])\,Y_{l,m}= (L_+\,L_z + \hbar\,L_+)\,Y_{l,m}\nonumber\\[0.5ex]
&=& (m+1)\,\hbar\,(L_+\,Y_{l,m}),
\end{eqnarray}
where use has been made of Eq.~(\ref{e8.19}). We, thus, conclude that
when the operator $L_+$ operates on an eigenstate of $L_z$ corresponding to the eigenvalue $m\,\hbar$ it converts it to an
eigenstate corresponding to the eigenvalue $(m+1)\,\hbar$.
Hence, $L_+$ is known as the {\em raising operator} (for $L_z$). It is
also easily demonstrated that
\begin{equation}\label{e8.32}
L_z\,(L_-\,Y_{l,m}) = (m-1)\,\hbar\,(L_-\,Y_{l,m}).
\end{equation}
In other words, when $L_-$ operates on an eigenstate of $L_z$ corresponding to the eigenvalue $m\,\hbar$ it converts it to an
eigenstate corresponding to the eigenvalue $(m-1)\,\hbar$.
Hence, $L_-$ is known as the {\em lowering operator} (for $L_z$).

Writing
\begin{eqnarray}
L_+\,Y_{l,m} &=& c_{l,m}^+\,Y_{l,m+1},\\[0.5ex]
L_-\,Y_{l,m} &=& c_{l,m}^-\,Y_{l,m-1},
\end{eqnarray}
we obtain
\begin{equation}
L_-\,L_+\,Y_{l,m} = c^+_{l,m}\,c^-_{l,m+1}\,Y_{l,m} =
[l\,(l+1)-m\,(m+1)]\,\hbar^2\,Y_{l,m},
\end{equation}
where use has been made of Eq.~(\ref{e8.17}). Likewise,
\begin{equation}
L_+\,L_-\,Y_{l,m} = c^+_{l,m-1}\,c^-_{l,m}\,Y_{l,m} = [l\,(l+1)-m\,(m-1)]\,\hbar^2\,Y_{l,m},
\end{equation}
where use has been made of Eq.~(\ref{e8.15}). It
follows that
\begin{eqnarray}
c^+_{l,m}\,c^-_{l,m+1}&=& [l\,(l+1)-m\,(m+1)]\,\hbar^2,\\[0.5ex]
c^+_{l,m-1}\,c^-_{l,m}&=& [l\,(l+1)-m\,(m-1)]\,\hbar^2.
\end{eqnarray}
These equations are satisfied when
\begin{equation}
c^\pm_{l,m} = [l\,(l+l) - m\,(m\pm 1)]^{1/2}\,\hbar.
\end{equation}
Hence, we can write
\begin{eqnarray}\label{eraise}
L_+\,Y_{l,m} &=& [l\,(l+1)-m\,(m+1)]^{1/2}\,\hbar\,Y_{l,m+1},\\[0.5ex]
L_-\,Y_{l,m} &=& [l\,(l+1)-m\,(m-1)]^{1/2}\,\hbar\,Y_{l,m-1}.\label{elow}
\end{eqnarray}

\section{Eigenvalues of $L_z$}\label{slz}
It seems reasonable to attempt to write the eigenstate $Y_{l,m}(\theta,\phi)$
in the separable form
\begin{equation}\label{e8.34}
Y_{l,m}(\theta,\phi) = \Theta_{l,m}(\theta)\,\Phi_m(\phi).
\end{equation}
We can satisfy the orthonormality constraint (\ref{e8.31}) provided that
\begin{eqnarray}
\int_{-\pi}^\pi \Theta^{\,\ast}_{l',m'}(\theta)\,\Theta_{l,m}(\theta)\,\sin\theta\,d\theta &=
& \delta_{ll'},\\[0.5ex]
\int_0^{2\pi}\Phi^{\,\ast}_{m'}(\phi)\,\Phi_{m}(\phi)\,d\phi &=
& \delta_{mm'}.\label{e8.36}
\end{eqnarray}

Note, from Eq.~(\ref{e8.26}), that the differential operator which represents
$L_z$ only depends on the azimuthal angle $\phi$, and is independent
of the polar angle $\theta$. It therefore follows from Eqs.~(\ref{e8.26}), (\ref{e8.29}), and (\ref{e8.34})
that
\begin{equation}
-{\rm i}\,\hbar\frac{d\Phi_m}{d\phi} = m\,\hbar\,\Phi_m.
\end{equation}
The solution to this equation is
\begin{equation}\label{e8.38}
\Phi_m(\phi)\sim {\rm e}^{\,{\rm i}\,m\,\phi}.
\end{equation}
Here, the  symbol $\sim$ just means that we are neglecting multiplicative constants. 

Now, our basic interpretation of a wavefunction
as a quantity whose modulus squared represents the probability density
of finding a particle at a particular point in space suggests that a
physical wavefunction must be {\em single-valued}\/ in space. Otherwise, the probability density at a given point would not, in general, have a unique value, which does not
make physical sense. 
Hence, we demand that the wavefunction (\ref{e8.38})
be single-valued: {\em i.e.}, $\Phi_m(\phi+2\,\pi)= \Phi_m(\phi)$
for all $\phi$. This immediately implies that the quantity $m$ is {\em quantized}. 
In fact, $m$ can only take {\em integer values}. Thus, we conclude that the eigenvalues
of $L_z$ are also quantized, and take the values $m\,\hbar$, where $m$ is an integer. [A more rigorous argument is that
$\Phi_m(\phi)$ must be continuous in order to ensure that $L_z$ is an Hermitian operator, since the proof of
hermiticity involves an  integration by parts in $\phi$ that has canceling contributions from $\phi=0$ and $\phi=2\pi$.]

Finally, we can easily normalize the eigenstate (\ref{e8.38}) by making use of the
orthonormality constraint (\ref{e8.36}). We obtain
\begin{equation}
\Phi_m(\phi) = \frac{{\rm e}^{\,{\rm i}\,m\,\phi}}{\sqrt{2\pi}}.
\end{equation}
This is the properly normalized eigenstate of $L_z$ corresponding to the eigenvalue $m\,\hbar$. 

\section{Eigenvalues of $L^2$}\label{slsq}
Consider the angular wavefunction $\psi(\theta,\phi) = L_+\,Y_{l,m}(\theta,\phi)$. We know that
\begin{equation}
\oint \psi^\ast(\theta,\phi)\,\psi(\theta,\phi)\,d\Omega \geq 0,
\end{equation}
since $\psi^\ast\,\psi\equiv |\psi|^2$ is a positive-definite real quantity.
Hence, making use of Eqs.~(\ref{e5.48}) and (\ref{e8.14}), we find that
\begin{eqnarray}
\oint (L_+\,Y_{l,m})^\ast\,(L_+\,Y_{l,m})\,d\Omega
&=& \oint Y_{l,m}^{\,\ast}\,(L_+)^\dag\,(L_+\,Y_{l,m})\,d\Omega\nonumber\\[0.5ex]
&=& \oint Y_{l,m}^{\,\ast}\,L_-\,L_+\,Y_{l,m}\,d\Omega\geq 0.
\end{eqnarray}
It follows from Eqs.~(\ref{e8.17}), and (\ref{e8.29})--(\ref{e8.31}) that
\begin{eqnarray}
\oint Y_{l,m}^{\,\ast}\,(L^2 -L_z^{\,2}-\hbar\,L_z)\,Y_{l,m}\,d\Omega
&=& \oint Y_{l,m}^{\,\ast}\,\hbar^2\left[l\,(l+1) -m\,(m+1)\right]Y_{l,m}\,d\Omega\nonumber\\[0.5ex]
&=& \hbar^2\,\left[l\,(l+1) -m\,(m+1)\right]\,\oint Y_{l,m}^{\,\ast}\,Y_{l,m}\,d\Omega\nonumber\\[0.5ex]
&=&\hbar^2\,\left[l\,(l+1) -m\,(m+1)\right]\geq 0.
\end{eqnarray}
We, thus, obtain the constraint
\begin{equation}\label{e8.42}
l\,(l+1) \geq m\,(m+1).
\end{equation}
Likewise, the inequality
\begin{equation}
\oint (L_-\,Y_{l,m})^\ast\,(L_-\,Y_{l,m})\,d\Omega
=\oint Y_{l,m}^{\,\ast}\,L_+\,L_-\,Y_{l,m}\,d\Omega\geq 0
\end{equation}
leads to a second constraint: 
\begin{equation}\label{e8.44}
l\,(l+1) \geq m\,(m-1).
\end{equation}

Without loss of generality, we can assume that $l\geq 0$. This
is reasonable, from a physical standpoint, since $l\,(l+1)\,\hbar^2$
is supposed to represent the magnitude squared of something, and
should, therefore, only take non-negative values. If $l$ is non-negative 
then the constraints (\ref{e8.42})  and (\ref{e8.44}) are equivalent
to the following constraint:
\begin{equation}
-l \leq m \leq l.
\end{equation}
We, thus, conclude that the quantum number $m$ can only take a {\em restricted range}\/ of integer values.

Well, if $m$ can only take a restricted range of integer values  then there
must exist a lowest possible value it can take. Let us call this special value $m_-$,
and let $Y_{l,m_-}$ be the corresponding eigenstate. Suppose
we act on this eigenstate with the lowering operator $L_-$.  According
to Eq.~(\ref{e8.32}), this will have the effect of converting the
eigenstate into that of a state with a lower value of $m$. However,
no such state exists. A non-existent state is represented in quantum
mechanics by the null wavefunction, $\psi=0$. Thus, we must have
\begin{equation}\label{e8.46}
L_-\,Y_{l,m_-} = 0.
\end{equation}
Now, from Eq.~(\ref{e8.15}),
\begin{equation}
L^2 = L_+\,L_-+L_z^{\,2} - \hbar\,L_z
\end{equation}
Hence,
\begin{equation}
L^2\,Y_{l,m_-} = (L_+\,L_-+L_z^{\,2} - \hbar\,L_z)\,Y_{l,m_-},
\end{equation}
or
\begin{equation}
l\,(l+1)\,Y_{l,m_-} = m_-\,(m_- -1)\,Y_{l,m_-},
\end{equation}
where use has been made of (\ref{e8.29}), (\ref{e8.30}), and (\ref{e8.46}).
It follows that
\begin{equation}
l\,(l+1) = m_-\,(m_--1).
\end{equation}
Assuming that $m_-$ is negative, the solution to the above equation
is
\begin{equation}
m_- = - l.
\end{equation}
We can similarly show that the largest possible value of $m$ is
\begin{equation}
m_+ =+ l.
\end{equation}
The above two results imply that $l$ is an {\em integer}, since
$m_-$ and $m_+$ are both constrained to be integers. 

We can now formulate the rules which determine the allowed values
of the quantum numbers $l$ and $m$. The quantum number $l$
takes the non-negative integer values $0, 1, 2, 3, \cdots$. Once
$l$ is given, the quantum number $m$ can take any integer value in the
range
\begin{equation}
-l,\,-l+1,\,\cdots\, 0, \,\cdots,\, l-1,\, l.
\end{equation}
Thus, if $l=0$ then $m$ can only take the value $0$, if $l=1$ then
$m$ can take the values $-1, 0, +1$, if $l=2$ then $m$ can take
the values $-2,-1,0,+1,+2$, and so on.

\section{Spherical Harmonics}\label{sharm}
The simultaneous eigenstates, $Y_{l,m}(\theta,\phi)$, of $L^2$ and $L_z$
are known as the {\em spherical harmonics}. Let us investigate their
functional form.

Now, we know that
\begin{equation}
L_+\,Y_{l,l}(\theta,\phi) = 0,
\end{equation}
since there is no state for which $m$ has a larger value than $+l$. 
Writing
\begin{equation}
Y_{l,l}(\theta,\phi) = \Theta_{l,l}(\theta)\,{\rm e}^{\,{\rm i}\,l\,\phi}
\end{equation}
[see Eqs.~(\ref{e8.34}) and (\ref{e8.38})], and making use of
Eq.~(\ref{e8.28}), we obtain
\begin{equation}
\hbar\,{\rm e}^{\,{\rm i}\,\phi}\left(\frac{\partial}{\partial\theta} + {\rm i}\,\cot\theta\,\frac{\partial}{\partial\phi}\right)\Theta_{l,l}(\theta)\,{\rm e}^{\,i\,l\,\phi}=0.
\end{equation}
This equation yields
\begin{equation}
\frac{d\Theta_{l,l}}{d\theta} - l\,\cot\theta\,\Theta_{l,l} = 0.
\end{equation}
which can easily be solved to give
\begin{equation}
\Theta_{l,l}\sim (\sin\theta)^l.
\end{equation}
Hence, we conclude that
\begin{equation}\label{e8.59}
Y_{l,l}(\theta,\phi) \sim (\sin\theta)^l\,{\rm e}^{\,{\rm i}\,l\,\phi}.
\end{equation}
Likewise, it is easy to demonstrate that
\begin{equation}\label{e8.60}
Y_{l,-l}(\theta,\phi) \sim (\sin\theta)^l\,{\rm e}^{-{\rm i}\,l\,\phi}.
\end{equation}

Once we know $Y_{l,l}$, we can obtain $Y_{l,l-1}$ by operating
on $Y_{l,l}$ with the lowering operator $L_-$. Thus,
\begin{equation}
Y_{l,l-1} \sim L_-\,Y_{l,l} \sim {\rm e}^{-{\rm i}\,\phi}\left(-\frac{\partial}{\partial\theta} + {\rm i}\,\cot\theta\,\frac{\partial}{\partial\phi}\right) (\sin\theta)^l\,{\rm e}^{\,{\rm i}\,l\,\phi},
\end{equation}
where use has been made of Eq.~(\ref{e8.28}). 
The above equation yields
\begin{equation}
Y_{l,l-1}\sim {\rm e}^{\,{\rm i}\,(l-1)\,\phi}\left(\frac{d}{d\theta} +l\,\cot\theta\right)(\sin\theta)^l.
\end{equation}
Now,
\begin{equation}\label{e8.64}
\left(\frac{d}{d\theta}+l\,\cot\theta\right)f(\theta)\equiv
\frac{1}{(\sin\theta)^l}\frac{d}{d\theta}\left[
(\sin\theta)^l\,f(\theta)\right],
\end{equation}
where $f(\theta)$ is a general function. Hence, we can write
\begin{equation}\label{e8.64a}
Y_{l,l-1}(\theta,\phi)\sim \frac{{\rm e}^{\,{\rm i}\,(l-1)\,\phi}}{(\sin\theta)^{l-1}}\left(\frac{1}{\sin\theta}\frac{d}{d\theta}\right)
(\sin\theta)^{2\,l}.
\end{equation}
Likewise, we can show that
\begin{equation}\label{e8.65}
Y_{l,-l+1}(\theta,\phi)\sim L_+\,Y_{l,-l}\sim \frac{{\rm e}^{-{\rm i}\,(l-1)\,\phi}}{(\sin\theta)^{l-1}}\left(\frac{1}{\sin\theta}\frac{d}{d\theta}\right)
(\sin\theta)^{2\,l}.
\end{equation}

We can now obtain $Y_{l,l-2}$ by operating on $Y_{l,l-1}$ with the
lowering operator. We get
\begin{equation}
Y_{l,l-2}\sim L_-\,Y_{l,l-1}\sim {\rm e}^{-{\rm i}\,\phi}\left(-\frac{\partial}{\partial\theta} + {\rm i}\,\cot\theta\,\frac{\partial}{\partial\phi}\right) \frac{{\rm e}^{\,{\rm i}\,(l-1)\,\phi}}{(\sin\theta)^{l-1}}\left(\frac{1}{\sin\theta}\frac{d}{d\theta}\right)
(\sin\theta)^{2\,l},
\end{equation}
which reduces to
\begin{equation}
Y_{l,l-2}\sim {\rm e}^{-{\rm i}\,(l-2)\,\phi}\left[\frac{d}{d\theta} +(l-1)\,\cot\theta\right] \frac{1}{(\sin\theta)^{l-1}}\left(\frac{1}{\sin\theta}\frac{d}{d\theta}\right)
(\sin\theta)^{2\,l}.
\end{equation}
Finally, making use of Eq.~(\ref{e8.64}), we obtain
\begin{equation}\label{e8.68}
Y_{l,l-2}(\theta,\phi) \sim \frac{{\rm e}^{\,{\rm i}\,(l-2)\,\phi}}{(\sin\theta)^{l-2}}\left(\frac{1}{\sin\theta}\frac{d}{d\theta}\right)^2
(\sin\theta)^{2\,l}.
\end{equation}
Likewise, we can show that
\begin{equation}\label{e8.69}
Y_{l,-l+2}(\theta,\phi) \sim L_+\,Y_{l,-l+1}\sim \frac{{\rm e}^{-{\rm i}\,(l-2)\,\phi}}{(\sin\theta)^{l-2}}\left(\frac{1}{\sin\theta}\frac{d}{d\theta}\right)^2
(\sin\theta)^{2\,l}.
\end{equation}

A comparison of Eqs.~(\ref{e8.59}), (\ref{e8.64a}), and (\ref{e8.68})
reveals the general functional form of the spherical harmonics:
\begin{equation}
Y_{l,m}(\theta,\phi)\sim  \frac{{\rm e}^{\,{\rm i}\,m\,\phi}}{(\sin\theta)^m}\left(\frac{1}{\sin\theta}\frac{d}{d\theta}\right)^{l-m}
(\sin\theta)^{2\,l}.
\end{equation}
Here, $m$ is assumed to be non-negative. Making the substitution $u=\cos\theta$, we can also write
\begin{equation}
Y_{l,m}(u,\phi)\sim  {\rm e}^{\,{\rm i}\,m\,\phi}\,(1-u^2)^{-m/2}\left(\frac{d}{d u}\right)^{l-m}
(1-u^2)^l.
\end{equation}
Finally, it is clear from Eqs.~(\ref{e8.60}), (\ref{e8.65}), and (\ref{e8.69})
that
\begin{equation}
Y_{l,-m} \sim Y^{\,\ast}_{l,m}.
\end{equation}

\begin{figure}
\epsfysize=3in
\centerline{\epsffile{Chapter08/fig01.eps}}
\caption{\em The $|Y_{l,m}(\theta,\phi)|^{\,2}$ plotted as a functions of $\theta$. The solid, short-dashed, and long-dashed curves correspond to 
$l,m=0,0$,  and $1,0$, and $1,\pm1$, respectively.}\label{ylm1}   
\end{figure}

We now need to normalize our spherical harmonic functions so as to ensure that
\begin{equation}
\oint |Y_{l,m}(\theta,\phi)|^2\,d\Omega = 1.
\end{equation}
After a great deal of tedious analysis, the normalized spherical
harmonic functions are found to take the form
\begin{equation}
Y_{l,m}(\theta,\phi) =(-1)^m\, \left[\frac{2\,l+1}{4\pi}\,\frac{(l-m)!}{(l+m)!}\right]^{1/2} P_{l,m}(\cos\theta)\,{\rm e}^{\,{\rm i}\,m\,\phi}
\end{equation}
for $m\geq 0$, where the $P_{l,m}$ are known as {\em associated Legendre
polynomials}, and are written
\begin{equation}
P_{l,m}(u) = (-1)^{l+m}\,\frac{(l+m)!}{(l-m)!}\,\frac{(1-u^2)^{-m/2}}{2^l\,l!}\left(\frac{d}{du}\right)^{l-m} (1-u^2)^l
\end{equation}
for $m\geq 0$. Alternatively,
\begin{equation}
P_{l,m}(u) = (-1)^{l}\,\frac{(1-u^2)^{m/2}}{2^l\,l!}\left(\frac{d}{du}\right)^{l+m} (1-u^2)^l,
\end{equation}
for $m\geq 0$. 
The spherical harmonics characterized by $m<0$ 
can be calculated from those characterized by $m>0$ via the identity
\begin{equation}
Y_{l,-m} = (-1)^m\,Y^{\,\ast}_{l,m}.
\end{equation}
The spherical harmonics are {\em orthonormal}: {\em i.e.}, 
\begin{equation}\label{spho}
\oint Y_{l',m'}^{\,\ast}\,Y_{l,m}\,d\Omega = \delta_{ll'}\,\delta_{mm'},
\end{equation}
and also form a {\em complete set}. In other words,
{\em any}\/ function of $\theta$ and $\phi$ can be represented as 
a superposition of spherical harmonics. Finally, and most importantly,
the spherical harmonics are the simultaneous eigenstates of $L_z$ and $L^2$
corresponding to the eigenvalues $m\,\hbar$ and $l\,(l+1)\,\hbar^2$,
respectively.

\begin{figure}
\epsfysize=3in
\centerline{\epsffile{Chapter08/fig02.eps}}
\caption{\em The $|Y_{l,m}(\theta,\phi)|^{\,2}$ plotted as a functions of $\theta$. The solid, short-dashed, and long-dashed curves correspond to 
$l,m=2,0$,  and $2,\pm 1$, and $2,\pm 2$, respectively.}\label{ylm2}   
\end{figure}


All of the $l=0$, $l=1$, and $l=2$ spherical harmonics are listed below:
\begin{eqnarray}
Y_{0,0} &=&\frac{1}{\sqrt{4\pi}},\\[0.5ex]
Y_{1,0} &=& \sqrt{\frac{3}{4\pi}}\,\cos\theta,\\[0.5ex]
Y_{1,\pm1} &=& \mp \sqrt{\frac{3}{8\pi}}\,\sin\theta\,{\rm e}^{\pm{\rm i}\,\phi},\\[0.5ex]
Y_{2,0} &=& \sqrt{\frac{5}{16\pi}}\,(3\,\cos^2\theta - 1),\\[0.5ex]
Y_{2,\pm 1}&=&\mp\sqrt{\frac{15}{8\pi}}\,\sin\theta\,\cos\theta\,{\rm e}^{\pm{\rm i}\,\phi},\\[0.5ex]
Y_{2,\pm 2}&=& \sqrt{\frac{15}{32\pi}}\,\sin^2\theta\,{\rm e}^{\pm 2\,{\rm i}\,\phi}.
\end{eqnarray}
The $\theta$ variation of these functions is illustrated in Figs.~\ref{ylm1} and
\ref{ylm2}.


\subsubsection*{Exercises}
{\small
\begin{enumerate}
\item A system is in the state $\psi=Y_{l,m}(\theta,\phi)$. Calculate
$\langle L_x\rangle$ and $\langle L_x^{\,2}\rangle$. 

\item Find the eigenvalues and eigenfunctions (in terms of the
angles $\theta$ and $\phi$) of $L_x$. 

\item Consider a beam of particles with $l=1$. A measurement
of $L_x$ yields the result $\hbar$. What values will be
obtained by a subsequent measurement of $L_z$, and with what probabilities? Repeat the calculation for the cases in which the
measurement of $L_x$ yields the results $0$ and $-\hbar$. 

\item The Hamiltonian for an axially symmetric rotator is given by
$$
H = \frac{L_x^{\,2}+L_y^{\,2}}{2\,I_1} + \frac{L_z^{\,2}}{2\,I_2}.
$$
What are the eigenvalues of $H$?
\end{enumerate}
}