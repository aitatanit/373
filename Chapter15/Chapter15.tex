\chapter{Scattering Theory}\label{s15}
\section{Introduction}
Historically, data regarding quantum phenomena has been
obtained from two main sources. Firstly, from the study of spectroscopic
lines, and, secondly, from scattering experiments. We have already developed
theories which account for some aspects of the spectrum of
hydrogen, and hydrogen-like, atoms. Let us now examine the quantum
theory of scattering.

\section{Fundamentals}
Consider time-independent, energy conserving  scattering in which the Hamiltonian
of the system is written
\begin{equation}
H = H_0 + V({\bf r}),
\end{equation}
where
\begin{equation}
H_0 = \frac{p^2}{2\,m} \equiv - \frac{\hbar^2}{2\,m}\,\nabla^2
\end{equation}
is the Hamiltonian of a free particle of mass $m$, and $V({\bf r})$
 the scattering potential. This potential is assumed to  only be
non-zero in a fairly localized region close to the origin. Let
\begin{equation}
\psi_0({\bf r}) = \sqrt{n}\,{\rm e}^{\,{\rm i}\,{\bf k}\cdot {\bf r}}
\end{equation}
represent an incident beam of particles, of number density $n$, and
velocity ${\bf v} = \hbar\,{\bf k}/m$. Of course,
\begin{equation}
H_0\,\psi_0= E\,\psi_0,
\end{equation}
where $E = \hbar^2\,k^2/2\,m$ is the particle energy.
Schr\"{o}dinger's equation for the scattering problem is
\begin{equation}
(H_0+V)\,\psi = E\,\psi,
\end{equation}
subject to the boundary condition $\psi\rightarrow\psi_0$ as $V\rightarrow 0$. 

The above equation can be rearranged to give
\begin{equation}\label{e15.6}
(\nabla^2+k^2)\,\psi = \frac{2\,m}{\hbar^2}\,V\,\psi.
\end{equation}
Now, 
\begin{equation}
(\nabla^2+k^2)\,u({\bf r}) = \rho({\bf r})
\end{equation}
is known as the {\em Helmholtz equation}. The solution to this
equation is well-known:\,\footnote{See Griffiths, Sect.~11.4.}
\begin{equation}
u({\bf r}) = u_0({\bf r}) - \int \frac{{\rm e}^{\,{\rm i}\,k\,|{\bf r}-{\bf r}'|}}
{4\pi\,|{\bf r}-{\bf r}'|}\,\rho({\bf r}')\,d^3{\bf r}'.
\end{equation}
Here, $u_0({\bf r})$ is any solution of $(\nabla^2+k^2)\,u_0 = 0$.
Hence, Eq.~(\ref{e15.6}) can be inverted, subject to the boundary condition
$\psi\rightarrow\psi_0$ as $V\rightarrow 0$, to give
\begin{equation}\label{e15.9}
\psi({\bf r}) = \psi_0({\bf r})- \frac{2\,m}{\hbar^2}
\int\frac{{\rm e}^{\,{\rm i}\,k\,|{\bf r}-{\bf r}'|}}
{4\pi\,|{\bf r}-{\bf r}'|}\,V({\bf r}')\,\psi({\bf r}')\,d^3{\bf r}'.
\end{equation}

Let us calculate the value of the wavefunction $\psi({\bf r})$  well outside the
scattering region. Now, if $r\gg r'$ then
\begin{equation}
|{\bf r}-{\bf r}'| \simeq r - \hat{\bf r}\cdot {\bf r}'
\end{equation}
to first-order in $r'/r$, where $\hat{\bf r}/r$ is a unit vector
which points from the scattering region to the observation point.
It is helpful to define ${\bf k}'=k\,\hat{\bf r}$. This is the wavevector
for particles with the same energy as the incoming particles ({\em i.e.},
$k'=k$) which propagate from the scattering region to the observation
point. Equation~(\ref{e15.9}) reduces to
\begin{equation}\label{e15.11}
\psi({\bf r}) \simeq \sqrt{n}\left[{\rm e}^{\,{\rm i}\,{\bf k}\cdot{\bf r}}
+ \frac{e^{\,{\rm i}\,k\,r}}{r}\,f({\bf k}, {\bf k}')\right],
\end{equation}
where
\begin{equation}\label{e5.12}
f({\bf k},{\bf k}') = -\frac{m}{2\pi\,\sqrt{n}\,\hbar^2}\int
{\rm e}^{-{\rm i}\,{\bf k}'\cdot{\bf r}'}\,V({\bf r}')\,\psi({\bf r}')\,d^3{\bf r}'.
\end{equation}
The first term on the right-hand side of Eq.~(\ref{e15.11}) represents the incident particle
beam, whereas the second term represents an outgoing {\em spherical wave}\/
of scattered particles.

The {\em differential scattering cross-section}\/ $d\sigma/d\Omega$ is
defined as the number of particles per unit time scattered into
an element of solid angle $d\Omega$, divided by the incident
particle flux. From Sect.~\ref{s7.2}, the probability flux ({\em i.e.}, the
particle flux) associated with a wavefunction $\psi$ is
\begin{equation}
{\bf j} = \frac{\hbar}{m}\,{\rm Im}(\psi^\ast\,\nabla\psi).
\end{equation}
Thus, the particle flux associated with the incident wavefunction $\psi_0$ is
\begin{equation}
{\bf j} = n\,{\bf v},
\end{equation}
where ${\bf v}=\hbar\,{\bf k}/m$ is the velocity of the incident
particles. Likewise, the particle flux associated with the scattered
wavefunction $\psi-\psi_0$ is
\begin{equation}
{\bf j}' = n\,\frac{|f({\bf k},{\bf k}')|^2}{r^2}\,{\bf v}',
\end{equation}
where ${\bf v}' = \hbar\,{\bf k}'/m$ is the velocity of the scattered particles.
Now,
\begin{equation}
\frac{d\sigma}{d\Omega}\,d\Omega = \frac{r^2\,d\Omega\,|{\bf j}'|}{|{\bf j}|},
\end{equation}
which yields
\begin{equation}\label{e15.17}
\frac{d\sigma}{d\Omega} = |f({\bf k},{\bf k}')|^2.
\end{equation}
Thus, $|f({\bf k},{\bf k}')|^2$ gives the differential cross-section
for particles with incident velocity ${\bf v}=\hbar\,{\bf k}/m$ to be scattered such that their final velocities are directed into a range of
solid angles $d\Omega$ about ${\bf v}'=\hbar\,{\bf k}'/m$. Note that the scattering
conserves energy, so that $|{\bf v}'|=|{\bf v}|$ and $|{\bf k}'|=|{\bf k}|$. 

\section{Born Approximation}
Equation~(\ref{e15.17}) is not particularly useful, as it stands, because
the quantity $f({\bf k},{\bf k}')$ depends on the, as yet, unknown wavefunction
$\psi({\bf r})$ [see Eq.~(\ref{e5.12})]. Suppose, however, that the scattering is
not particularly strong. In this case, it is reasonable to suppose that the total
wavefunction, $\psi({\bf r})$, does not differ substantially from the
incident wavefunction, $\psi_0({\bf r})$. Thus, we can
obtain an expression for $f({\bf k},{\bf k}')$ by making the
substitution $\psi({\bf r})\rightarrow\psi_0({\bf r}) = \sqrt{n}\,\exp(\,{\rm i}\,
{\bf k}\cdot{\bf r})$ in Eq.~(\ref{e5.12}). This procedure is called
the {\em Born approximation}.

The Born approximation yields
\begin{equation}
f({\bf k},{\bf k}') \simeq \frac{m}{2\pi\,\hbar^2}
\int {\rm e}^{\,{\rm i}\,({\bf k}-{\bf k'})\cdot{\bf r}'}\,V({\bf r'})\,d^3{\bf r}'.
\end{equation}
Thus, $f({\bf k},{\bf k}')$ is proportional to the {\em Fourier transform}\/ of the scattering potential $V({\bf r})$ with respect to the wavevector 
${\bf q} = {\bf k}-{\bf k}'$.

For a spherically symmetric potential, 
\begin{equation}
f({\bf k}', {\bf k}) \simeq  - \frac{m}{2\pi\, \hbar^2} \int\!\int\!\int
\exp(\,{\rm i} \, q \,r'\cos\theta') \, V(r')\,r'^{\,2}\, dr'\,\sin\theta'
 \,d\theta'\,d\phi',
\end{equation}
giving
\begin{equation}\label{e17.38}
f({\bf k}', {\bf k}) \simeq  - \frac{2\,m}{\hbar^2\,q}
\int_0^\infty r' \,V(r') \sin(q \,r') \,dr'.
\end{equation}
Note that $f({\bf k}', {\bf k})$ is just a function of $q$ for a
spherically symmetric potential.
It is easily demonstrated that
\begin{equation}\label{e17.39}
q \equiv |{\bf k} - {\bf k}'| = 2\, k \,\sin (\theta/2),
\end{equation}
where $\theta$ is the angle subtended between the vectors
${\bf k}$ and ${\bf k}'$. In other words, $\theta$ is the scattering angle. Recall that the
vectors ${\bf k}$ and ${\bf k}'$ have the same length, via energy conservation.

Consider scattering by a Yukawa potential
\begin{equation}
V(r) = \frac{V_0\,\exp(-\mu \,r)}{\mu \,r},
\end{equation}
where $V_0$ is a constant, and $1/\mu$ measures the ``range'' of the
potential. It follows from Eq.~(\ref{e17.38}) that
\begin{equation}
f(\theta) = - \frac{2\,m \,V_0}{\hbar^2\,\mu} \frac{1}{q^2 + \mu^2},
\end{equation}
since
\begin{equation}
\int_0^\infty \exp(-\mu \,r') \,\sin(q\,r') \, dr' = \frac{q}{q^2+ \mu^2}.
\end{equation}
Thus, in the Born approximation, the differential cross-section
for scattering by a Yukawa potential is
\begin{equation}
\frac{d\sigma}{d \Omega} \simeq \left(\frac{2\,m \,V_0}{ \hbar^2\,\mu}\right)^2
\frac{1}{[2\,k^2\, (1-\cos\theta) + \mu^2]^{\,2}},
\end{equation}
given that
\begin{equation}
q^2 = 4\,k^2\, \sin^2(\theta/2) = 2\,k^2\, (1-\cos\theta).
\end{equation}

The Yukawa potential reduces to the familiar Coulomb potential as
$\mu \rightarrow 0$, provided that $V_0/\mu \rightarrow
Z\,Z'\, e^2 / 4\pi\,\epsilon_0$. In this limit, the Born differential cross-section becomes
\begin{equation}
\frac{d\sigma}{d\Omega} \simeq \left(\frac{2\,m \,Z\, Z'\, e^2}{4\pi\,\epsilon_0\,\hbar^2}\right)^2
\frac{1}{ 16 \,k^4\, \sin^4( \theta/2)}.
\end{equation}
Recall that $\hbar\, k$ is equivalent to $|{\bf p}|$, so the above
equation can be rewritten 
\begin{equation}\label{e17.46}
 \frac{d\sigma}{d\Omega} \simeq\left(\frac{Z \,Z'\, e^2}{16\pi\,\epsilon_0\,E}\right)^2
\frac{1}{\sin^4(\theta/2)},
\end{equation}
where $E= p^2/2\,m$ is the kinetic energy of the incident particles. 
Of course, Eq.~(\ref{e17.46}) is  the famous  {\em Rutherford scattering cross-section} formula.

The Born approximation is valid provided that $\psi({\bf r})$ is
not too different from $\psi_0({\bf r})$ in the scattering region. 
It follows, from Eq.~(\ref{e15.9}), that the condition for  $\psi({\bf r})
\simeq \psi_0({\bf r})$ in the vicinity of ${\bf r} = {\bf 0}$ is 
\begin{equation}\label{e17.47}
\left| \frac{m}{2\pi\, \hbar^2} \int \frac{ \exp(\,{\rm i}\, k \,r')}{r'} 
\,V({\bf r}')\,d^3{\bf r'} \right| \ll 1.
\end{equation}
 Consider the special case of the Yukawa potential. At low energies,
({\em i.e.}, $k\ll \mu$) we can replace $\exp(\,{\rm i}\,k\, r')$ by unity,
giving
\begin{equation}
\frac{2\,m}{\hbar^2} \frac{|V_0|}{\mu^2} \ll 1
\end{equation}
as the condition for the validity of the Born approximation.
The condition for the Yukawa potential to develop a bound state
is
\begin{equation}
\frac{2\,m}{\hbar^2} \frac{|V_0|} {\mu^2} \geq 2.7,
\end{equation}
where $V_0$ is negative. Thus, if the potential is strong enough to
form a bound state then the Born approximation is likely to break
down. In the high-$k$ limit, Eq.~(\ref{e17.47}) yields
\begin{equation}
\frac{2\,m}{\hbar^2} \frac{|V_0|}{\mu \,k} \ll 1.
\end{equation}
This inequality becomes progressively easier to satisfy as $k$ increases,
implying that the Born approximation is more accurate at {\em high}\/
incident particle energies.

\section{Partial Waves}
We can assume, without loss of generality, that the incident wavefunction
is characterized by  a wavevector ${\bf k}$ which is aligned parallel to the $z$-axis.
The scattered wavefunction is characterized by a wavevector ${\bf k}'$
which has the same magnitude as ${\bf k}$, but, in general, points
in a different direction. The direction of ${\bf k}'$ is specified
by the polar angle $\theta$ ({\em i.e.}, the angle subtended between the
two wavevectors), and an azimuthal angle $\phi$ about the $z$-axis.
Equations~(\ref{e17.38}) and (\ref{e17.39}) strongly suggest that for a spherically symmetric
scattering potential [{\em i.e.}, $V({\bf r}) = V(r)$] the scattering amplitude
is a function of $\theta$ only: {\em i.e.},
\begin{equation}
f(\theta, \phi) = f(\theta).
\end{equation}
It follows that neither the incident wavefunction,
\begin{equation}\label{e17.52}
\psi_0({\bf r}) = \sqrt{n}\,\exp(\,{\rm i}\,k\,z)= \sqrt{n}\,\exp(\,{\rm i}\,k\,r\cos\theta),
\end{equation}
nor the large-$r$ form of the total wavefunction,
\begin{equation}\label{e17.53}
\psi({\bf r})  = \sqrt{n}
\left[ \exp(\,{\rm i}\,k\,r\cos\theta) + \frac{\exp(\,{\rm i}\,k\,r)\, f(\theta)}
{r} \right],
\end{equation}
depend on the azimuthal angle $\phi$. 

Outside the range of the scattering potential, both $\psi_0({\bf r})$ and
$\psi({\bf r})$ satisfy the free space Schr\"{o}dinger equation 
\begin{equation}\label{e17.54}
(\nabla^2 + k^2)\,\psi = 0.
\end{equation}
What is the most general solution to this equation in spherical polar
coordinates which does not depend on the azimuthal angle $\phi$?
Separation of variables yields
\begin{equation}\label{e17.55}
\psi(r,\theta) = \sum_l R_l(r)\, P_l(\cos\theta),
\end{equation}
since the {\em Legendre functions}\/ $P_l(\cos\theta)$ form a complete
set in $\theta$-space. The Legendre functions are related to the
{\em spherical harmonics}, introduced in Cha.~\ref{sorb}, via
\begin{equation}
P_l(\cos\theta) = \sqrt{\frac{4\pi}{2\,l+1}}\, Y_{l,0}(\theta,\varphi).
\end{equation}
Equations~(\ref{e17.54}) and (\ref{e17.55}) can be combined to give
\begin{equation}
r^2\,\frac{d^2 R_l}{dr^2} + 2\,r \,\frac{dR_l}{dr} +  [k^2 \,r^2 -
l\,(l+1)]R_l = 0.
\end{equation}
The two independent solutions to this equation are the
{\em spherical Bessel functions}, $j_l(k\,r)$ and 
$y_l(k\,r)$, introduced in Sect.~\ref{rwell}.
Recall that
\begin{eqnarray}\label{e17.58a}
j_l(z) &=& z^l\left(-\frac{1}{z}\frac{d}{dz}\right)^l\left( \frac{\sin z}{z}\right),
\\[0.5ex]\label{e17.58b}
y_l(z) &=& -z^l\left(-\frac{1}{z}\frac{d}{dz}\right)^l \left(\frac{\cos z}{z}\right).
\end{eqnarray}
Note that  the $j_l(z)$ are well-behaved in the limit
$z\rightarrow 0$ , whereas the $y_l(z)$ become singular.
The asymptotic behaviour of these functions in the limit $z\rightarrow
\infty$ is
\begin{eqnarray}\label{e17.59a}
j_l(z) &\rightarrow &\frac{\sin(z - l\,\pi/2)}{z},\\[0.5ex]
y_l(z) &\rightarrow & - \frac{\cos(z-l\,\pi/2)}{z}.\label{e17.59b}
\end{eqnarray}

We can write
\begin{equation}
\exp(\,{\rm i}\,k\,r \cos\theta) = \sum_l a_l\, j_l(k\,r)\, P_l(\cos\theta),
\end{equation}
where the $a_l$ are constants. Note there are no  $y_l(k\,r)$ functions in
this expression, because they are not well-behaved  as $r \rightarrow 0$. 
The Legendre functions are orthonormal,
\begin{equation}\label{e17.61}
\int_{-1}^1 P_n(\mu) \,P_m(\mu)\,d\mu = \frac{\delta_{nm}}{n+1/2},
\end{equation}
so we can invert the above expansion to give
\begin{equation}
a_l \,j_l(k\,r) = (l+1/2)\int_{-1}^1 \exp(\,{\rm i}\,k\,r \,\mu) \,P_l(\mu) \,d\mu.
\end{equation}
It is well-known that
\begin{equation}
j_l(y) = \frac{(-{\rm i})^l}{2} \int_{-1}^1 \exp(\,{\rm i}\, y\,\mu)
\,P_l(\mu)\,d\mu,
\end{equation}
where $l=0, 1, 2, \cdots$ [see M.~Abramowitz and I.A.~Stegun, {\em Handbook of mathematical functions}, (Dover, New York NY, 1965),
Eq.~10.1.14]. Thus,
\begin{equation}
a_l = {\rm i}^{\,l} \,(2\,l+1),
\end{equation}
giving
\begin{equation}\label{e15.49}
\psi_0({\bf r}) = \sqrt{n}\,\exp(\,{\rm i}\,k\,r \cos\theta) =\sqrt{n}\, \sum_l {\rm i}^{\,l}\,(2\,l+1)\, j_l(k\,r)\, P_l(\cos\theta).
\end{equation}
The above expression  tells us how to decompose
the incident plane-wave  into
a series of spherical waves. These waves are usually termed  ``partial waves''.

The most general expression for the total wavefunction outside the
scattering region is
\begin{equation}
\psi({\bf r}) = \sqrt{n}\sum_l\left[
A_l\,j_l(k\,r) + B_l\,y_l(k\,r)\right] P_l(\cos\theta),
\end{equation}
where the $A_l$ and $B_l$ are constants. 
Note that the $y_l(k\,r)$ functions are allowed to appear 
in this expansion, because
its region of validity does not include the origin. In the large-$r$
limit, the total wavefunction reduces to
\begin{equation}
\psi ({\bf r} ) \simeq \sqrt{n} \sum_l\left[A_l\,
\frac{\sin(k\,r - l\,\pi/2)}{k\,r} - B_l\,\frac{\cos(k\,r -l\,\pi/2)}{k\,r}
\right] P_l(\cos\theta),
\end{equation}
where use has been made of Eqs.~(\ref{e17.59a}) and (\ref{e17.59b}). The above expression can also
be written
\begin{equation}\label{e17.68}
\psi ({\bf r} ) \simeq \sqrt{n} \sum_l C_l\,
\frac{\sin(k\,r - l\,\pi/2+ \delta_l)}{k\,r}\, P_l(\cos\theta),
\end{equation}
where the sine and cosine functions have been combined to give a
sine function which is phase-shifted by $\delta_l$. Note that $A_l=C_l\,\cos\delta_l$ and $B_l=-C_l\,\sin\delta_l$.

Equation~(\ref{e17.68}) yields
\begin{equation}
\psi({\bf r}) \simeq \sqrt{n} \sum_l C_l\left[
\frac{{\rm e}^{\,{\rm i}\,(k\,r - l\,\pi/2+ \delta_l)}
-{\rm e}^{-{\rm i}\,(k\,r - l\,\pi/2+ \delta_l)} }{2\,{\rm i}\,k\,r}  \right] P_l(\cos\theta),\label{e17.69}
\end{equation}
which contains both incoming and outgoing spherical waves. What is the
source of the incoming waves? Obviously, they must be part of
the large-$r$ asymptotic expansion of the incident wavefunction. In fact,
it is easily seen from Eqs.~(\ref{e17.59a}) and (\ref{e15.49})
that
\begin{equation}
\psi_0({\bf r}) \simeq \sqrt{n} \sum_l {\rm i}^{\,l}\,
(2l+1)\left[\frac{
{\rm e}^{\,{\rm i}\,(k\,r - l\,\pi/2)}
-{\rm e}^{-{\rm i}\,(k\,r - l\,\pi/2)}}{2\,{\rm i}\,k\,r}  \right]P_l(\cos\theta)\label{e17.70}
\end{equation}
in the large-$r$ limit. Now, Eqs.~(\ref{e17.52}) and (\ref{e17.53}) give
\begin{equation}\label{e17.71}
\frac{\psi({\bf r} )- \psi_0({\bf r}) }{ \sqrt{n}} = 
\frac{\exp(\,{\rm i}\,k\,r)}{r}\,
f(\theta).
\end{equation}
Note that the right-hand side consists  of an {\em outgoing}\/ spherical
wave only. This implies that the coefficients of the incoming spherical waves
in the large-$r$  expansions of $\psi({\bf r})$ and $\psi_0({\bf r})$
must be the same. It follows from Eqs.~(\ref{e17.69}) and (\ref{e17.70}) that
\begin{equation}
C_l = (2\,l+1)\,\exp[\,{\rm i}\,(\delta_l + l\,\pi/2)].
\end{equation} 
Thus, Eqs.~(\ref{e17.69})--(\ref{e17.71}) yield
\begin{equation}\label{e17.73}
f(\theta) = \sum_{l=0}^\infty (2\,l+1)\,\frac{\exp(\,{\rm i}\,\delta_l)}
{k} \,\sin\delta_l\,P_l(\cos\theta).
\end{equation}
Clearly, determining the scattering amplitude
$f(\theta)$  via  a decomposition into
partial waves ({\em i.e.}, spherical waves) is equivalent to determining
the phase-shifts $\delta_l$.

Now, the differential scattering cross-section $d\sigma/d\Omega$ is simply
the modulus squared of the scattering amplitude $f(\theta)$ [see Eq.~(\ref{e15.17})]. The
total cross-section is thus given by
\begin{eqnarray}
\sigma_{\rm total}& = &\int |f(\theta)|^2\,d\Omega\nonumber\\[0.5ex]
&=& \frac{1}{k^2} \oint d\phi \int_{-1}^{1} d\mu
\sum_l \sum_{l'} (2\,l+1)\,(2\,l'+1) 
\exp[\,{\rm i}\,(\delta_l-\delta_{l'})]\nonumber\\[0.5ex]
&&\mbox{\hspace{1cm}}\times  \sin\delta_l \,\sin\delta_{l'}\,
P_l(\mu)\, P_{l'}(\mu),
\end{eqnarray}
where $\mu = \cos\theta$. It follows that
\begin{equation}\label{e17.75}
\sigma_{\rm total} = \frac{4\pi}{k^2} \sum_l (2\,l+1)\,\sin^2\delta_l,
\end{equation}
where use has been made of Eq.~(\ref{e17.61}).

\section{Determination of Phase-Shifts}
Let us now consider how the phase-shifts $\delta_l$ in Eq.~(\ref{e17.73}) can be
 evaluated. Consider a spherically symmetric potential $V(r)$ which
vanishes for $r>a$, where $a$ is termed the {\em range}\/ of the potential.
In the region $r>a$, the wavefunction $\psi({\bf r})$ 
satisfies the free-space Schr\"{o}dinger equation (\ref{e17.54}). The
most general solution which is consistent with no incoming spherical-waves is
\begin{equation}
\psi({\bf r}) = \sqrt{n}\, \sum_{l=0}^\infty
{\rm i}^l\, (2\,l+1) \, {\cal R}_l(r)\, P_l(\cos\theta),
\end{equation}
where
\begin{equation}\label{e17.80}
{\cal R}_l(r) = \exp(\,{\rm i} \,\delta_l)\,
\left[\cos\delta_l \,j_l(k\,r) -\sin\delta_l\, y_l(k\,r)\right].
\end{equation}
Note that  $y_l(k\,r)$ functions are allowed to appear in the above
expression, because its region of validity does not include the origin
(where $V\neq 0$). The logarithmic derivative of the $l$th 
radial wavefunction,
${\cal R}_l(r)$, just outside the range of the potential is given by
\begin{equation}
\beta_{l+} = k\,a \left[\frac{ \cos\delta_l\,j_l'(k\,a) -
\sin\delta_l\, y_l'(k\,a)}{\cos\delta_l \,
j_l(k\,a) - \sin\delta_l\,y_l(k\,a)}\right],
\end{equation}
where $j_l'(x)$ denotes $dj_l(x)/dx$, {\em etc.} The above equation
can be inverted to give
\begin{equation}\label{e17.82}
\tan \delta_l = \frac{ k\,a\,j_l'(k\,a) - \beta_{l+}\, j_l(k\,a)}
{k\,a\,y_l'(k\,a) - \beta_{l+}\, y_l(k\,a)}.
\end{equation}
Thus, the problem of determining the phase-shift $\delta_l$ is equivalent
to that of obtaining $\beta_{l+}$. 

The most general solution to Schr\"{o}dinger's equation inside 
the range of the potential ($r<a$) which does not depend on the
azimuthal angle $\phi$ is
\begin{equation}
\psi({\bf r}) = \sqrt{n}\,\sum_{l=0}^\infty
{\rm i}^{\,l} \,(2\,l+1)\,{\cal R}_l(r)\,P_l(\cos\theta),
\end{equation}
where
\begin{equation}
{\cal R}_l (r) = \frac{u_l(r)}{r},
\end{equation}
and
\begin{equation}\label{e17.85}
\frac{d^2 u_l}{d r^2} +\left[k^2  -\frac{l\,(l+1)}{r^2} -\frac{2\,m}{\hbar^2} \,V\right] u_l = 0.
\end{equation}
The boundary condition 
\begin{equation}\label{e17.86}
u_l(0) = 0
\end{equation}
 ensures that the radial wavefunction is well-behaved at the
origin. 
We can launch a well-behaved solution of the above equation from 
$r=0$, integrate out to $r=a$, and form the logarithmic derivative
\begin{equation}
\beta_{l-} = \left.\frac{1}{(u_l/r)} \frac{d(u_l/r)}{dr}\right|_{r=a}.
\end{equation}
Since $\psi({\bf r})$ and its first derivatives are necessarily continuous for
physically acceptible wavefunctions, it follows that
\begin{equation}
\beta_{l+} = \beta_{l-}.
\end{equation}
The phase-shift $\delta_l$ is then obtainable from Eq.~(\ref{e17.82}).

\section{Hard Sphere Scattering}
Let us test out this scheme using a particularly simple example. Consider
scattering by a {\em hard sphere}, for which  the potential is infinite 
for $r<a$, and zero for $r>a$. It follows that $\psi({\bf r})$ is
zero in the region $r<a$, which implies that  $u_l =0$ for all $l$. 
Thus,
\begin{equation}
\beta_{l-} = \beta_{l+} = \infty,
\end{equation}
for all $l$. Equation~(\ref{e17.82}) thus gives
\begin{equation}\label{e17.90}
\tan \delta_l = \frac{j_l(k\,a)}{y_l(k\,a)}.
\end{equation}

Consider the $l=0$ partial wave, which is usually referred to as the $S$-wave.
Equation~(\ref{e17.90}) yields
\begin{equation}
\tan\delta_0 = \frac{\sin (k\,a)/k\,a}{-\cos (k\,a)/ka} = -\tan (k\,a),
\end{equation}
where use has been made of Eqs.~(\ref{e17.58a}) and (\ref{e17.58b}). It follows that
\begin{equation}\label{e17.92}
\delta_0 = -k\,a.
\end{equation}
The  $S$-wave radial wave function
is [see Eq.~(\ref{e17.80})]
\begin{eqnarray}
{\cal R}_0(r) &= &\exp(-{\rm i}\, k\,a) \frac{[\cos (k\,a) \,\sin (k\,r)
-\sin (k\,a) \,\cos (k\,r)]}{k\,r}\nonumber\\[0.5ex]
&=& \exp(-{\rm i}\, k\,a)\, \frac{ \sin[k\,(r-a)]}{k\,r}.
\end{eqnarray}
The corresponding radial wavefunction for the incident wave 
takes the form [see Eq.~(\ref{e15.49})]
\begin{equation}
\tilde{\cal R}_0(r) = \frac{ \sin (k\,r)}{k\,r}.
\end{equation}
Thus,  the actual $l=0$ radial wavefunction is similar to the
incident $l=0$ wavefunction, except that it is phase-shifted by $k\,a$. 

Let us examine the low and high energy asymptotic limits of $\tan\delta_l$.
Low energy implies that $k\,a\ll 1$. In this regime, the spherical Bessel functions
reduce to:
\begin{eqnarray}
j_l(k\,r) &\simeq & \frac{(k\,r)^l}{(2\,l+1)!!},\\[0.5ex]
y_l(k\,r) &\simeq & -\frac{(2\,l-1)!!}{(k\,r)^{l+1}},
\end{eqnarray}
where $n!! = n\,(n-2)\,(n-4)\cdots 1$. It follows that
\begin{equation}
\tan\delta_l = \frac{-(k\,a)^{2\,l+1}}{(2\,l+1) \,[(2\,l-1)!!]^{\,2}}.
\end{equation}
It is clear that we can neglect  $\delta_l$, with $l>0$, with respect to
$\delta_0$. In other words, at low energy only $S$-wave scattering
({\em i.e.}, spherically symmetric scattering) is important. It follows
from Eqs.~(\ref{e15.17}), (\ref{e17.73}), and (\ref{e17.92})  that 
\begin{equation}
\frac{d\sigma}{d\Omega} = \frac{\sin^2 k\,a}{k^2} \simeq a^2
\end{equation}
for $k\,a\ll 1$. Note that the total cross-section
\begin{equation}
\sigma_{\rm total} = \int\frac{d\sigma}{d\Omega}\,d\Omega = 4\pi \,a^2
\end{equation}
is {\em four times}\/ the {\em geometric cross-section}\/ $\pi \,a^2$
({\em i.e.}, the cross-section for classical particles bouncing off a
hard sphere of radius $a$). 
However, 
low energy scattering implies relatively long wavelengths, so we would not
expect to obtain the  classical result  in this limit. 

Consider the high energy limit $k\,a\gg 1$. At high energies, all partial
waves up to $l_{\rm max} = k\,a$ contribute significantly to
the scattering cross-section. It follows from Eq.~(\ref{e17.75}) that
\begin{equation}\label{e17.99}
\sigma_{\rm total} \simeq \frac{4\pi}{k^2} \sum_{l=0}^{l_{\rm max}}
(2\,l+1)\,\sin^2\delta_l.
\end{equation}
With so many $l$ values contributing, it is legitimate to replace
$\sin^2\delta_l$ by its average value $1/2$. Thus,
\begin{equation}
\sigma_{\rm total} \simeq \sum_{l=0}^{k\,a} \frac{2\pi}{k^2} \,(2\,l+1) \simeq 
2\pi \,a^2.
\end{equation}
This is {\em twice}\/ the classical result, which is  somewhat surprizing,
since we might expect to obtain the classical result in the short
wavelength limit. For hard sphere scattering, incident waves with
impact parameters less than $a$ must be deflected. However, in order to
produce a ``shadow'' behind the sphere, there must also be  some scattering
in the forward direction in order to produce
destructive interference with the incident plane-wave. In fact, the
interference is not completely destructive, and the shadow has a bright
spot (the so-called ``Poisson spot'') in the forward direction. The effective cross-section associated with
this bright spot is $\pi \,a^2$ which, when combined with the
cross-section for classical reflection, $\pi \,a^2$, gives the actual
cross-section of $2\pi \,a^2$.

\section{Low Energy Scattering}
In general, at low energies ({\em i.e.}, when $1/k$ is much larger than the range
of the potential) partial waves with $l>0$ make a
negligible contribution to the scattering cross-section. It follows
that, at these energies, with a finite range potential, only $S$-wave
scattering is important.

As a specific example, let us consider scattering by  a finite
potential well, characterized by $V=V_0$ for $r<a$, and
$V=0$ for $r\geq a$. Here, $V_0$ is a constant. The potential
is repulsive for $V_0>0$, and attractive for $V_0<0$. 
The outside wavefunction is given by [see Eq.~(\ref{e17.80})]
\begin{eqnarray}
{\cal R}_0(r) &=& \exp(\,{\rm i}\, \delta_0)\,\left[
\cos\delta_0\,j_0(k\,r)  - \sin\delta_0\,y_0(k\,r) \right]\nonumber\\[0.5ex]
&=& \frac{ \exp(\,{\rm i} \,\delta_0)\, \sin(k\,r+\delta_0)}{k\,r},
\end{eqnarray}
where use has been made of Eqs.~(\ref{e17.58a}) and (\ref{e17.58b}).
The inside wavefunction follows from Eq.~(\ref{e17.85}). We obtain
\begin{equation}\label{e17.103}
{\cal R}_0(r) = B \,\frac{\sin (k'\,r)}{r},
\end{equation}
where use has been made of the boundary condition (\ref{e17.86}).
Here, $B$ is a constant, and 
\begin{equation}
E - V_0 = \frac{\hbar^2 \,k'^{\,2}}{2\,m}.
\end{equation}
Note that Eq.~(\ref{e17.103}) only applies when $E>V_0$. For $E<V_0$, we have
\begin{equation}
{\cal R}_0(r) = B \,\frac{\sinh(\kappa\, r)}{r},
\end{equation}
where
\begin{equation}
V_0 - E = \frac{\hbar^2 \kappa^2}{2\,m}.
\end{equation}
Matching ${\cal R}_0(r)$, and its radial derivative, at $r=a$ yields
\begin{equation}\label{e17.107}
\tan(k\,a+\delta_0) = \frac{k}{k'} \,\tan( k'\,a)
\end{equation}
for $E>V_0$, and
\begin{equation}
\tan(k\,a+ \delta_0) = \frac{k}{\kappa} \,\tanh( \kappa\, a)
\end{equation}
for $E<V_0$.

Consider an attractive potential, for which $E>V_0$. Suppose that 
$|V_0|\gg E$ ({\em i.e.}, the depth of the potential well is much larger than
the energy of the incident particles), so that $k' \gg k$. We can see
from Eq.~(\ref{e17.107}) that, unless $\tan (k'\,a)$ becomes extremely large, the right-hand side is much less that unity, so replacing the tangent of a
small quantity with the quantity itself, we obtain
\begin{equation}
k\,a + \delta_0 \simeq \frac{k}{k'}\,\tan (k'\,a).
\end{equation}
This yields
\begin{equation}
\delta_0 \simeq k\,a \left[ \frac{\tan( k'\,a)}{k'\,a} -1\right].
\end{equation}
According to Eq.~(\ref{e17.99}), the scattering cross-section is given by
\begin{equation}\label{e17.111}
\sigma_{\rm total} \simeq \frac{4\pi}{k^2}\, \sin^2\delta_0
=4\pi \,a^2\left[\frac{\tan (k\,'a)}{k'\,a} -1\right]^2.
\end{equation}
Now
\begin{equation}\label{e17.112}
k'\,a = \sqrt{ k^2 \,a^2 + \frac{2 \,m \,|V_0|\, a^2}{\hbar^2}},
\end{equation}
so for sufficiently small values of $k\,a$,
\begin{equation}
k' \,a \simeq \sqrt{\frac{2\, m \,|V_0|\, a^2}{\hbar^2}}.
\end{equation}
It follows that the total ($S$-wave) scattering cross-section is {\em independent}\/
of the energy of the incident particles (provided that this energy is
sufficiently small). 

Note that there are values of $k'\,a$ ({\em e.g.}, $k'\,a\simeq 4.49$) at which
$\delta_0\rightarrow \pi$, and 
the scattering cross-section (\ref{e17.111}) vanishes, despite the very strong
attraction of the potential. In reality, the cross-section is not
exactly zero, because of contributions from $l>0$ partial waves. But,
at low incident energies, these contributions are small. It follows that
there are certain values of $V_0$ and $k$ which give rise to almost perfect 
transmission of the incident wave. This is called the {\em Ramsauer-Townsend
effect}, and has been observed experimentally. 

\section{Resonances}
There is a significant exception to the independence of the cross-section on energy mentioned above. Suppose that the quantity $\sqrt{2\,m \,|V_0|\,a^2/\hbar^2}$
is slightly less than $\pi/2$. As the incident energy increases, $k'\,a$,
which is given by Eq.~(\ref{e17.112}), can reach the value $\pi/2$. In this case,
$\tan (k'\,a)$ becomes infinite, so we can no longer assume that the right-hand side of Eq.~(\ref{e17.107}) is small. In fact, it follows from Eq.~(\ref{e17.107}) that
at the value of the incident energy
when $k'\,a = \pi/2$ then we also have $k\,a+\delta_0 = \pi/2$,
or $\delta_0 \simeq \pi/2$ (since we are assuming that $k\,a\ll 1$). 
This implies that
\begin{equation}
\sigma_{\rm total} = \frac{4\pi}{k^2} \,\sin^2\delta_0 = 4\pi \,a^2
\left(\frac{1}{k^2 \,a^2}\right).
\end{equation}
Note that the cross-section now depends on the energy. Furthermore, the
magnitude of the cross-section is much larger than that given in Eq.~(\ref{e17.111})
for $k'\,a\neq \pi/2$ (since $k\,a\ll 1$). 

The origin of this rather strange behaviour is quite simple. The condition
\begin{equation}
\sqrt{\frac{2\,m\,|V_0 |\,a^2}{\hbar^2} } = \frac{\pi}{2}
\end{equation}
is equivalent to the condition that a spherical well of depth
$V_0$ possesses a {\em bound state}\/ at zero energy. Thus, for a potential
well which satisfies the above equation, the energy of the scattering system
is essentially the same as the energy of the bound state. In this situation,
an incident particle would like to form a bound state in the potential
well. However, the bound state is not stable, since the system has a small
positive energy. Nevertheless, this sort of {\em resonance scattering}\/
is best understood as the capture of an incident particle to form
a metastable bound state, and the subsequent decay of the bound state
and release of the particle. The cross-section for resonance scattering
is generally {\em much larger}\/ than that for non-resonance scattering.

We have seen that there is a resonant effect when the phase-shift of
the $S$-wave takes the value $\pi/2$.  There is nothing special about
the $l=0$ partial wave, so it is reasonable to assume that there
is a similar resonance when the phase-shift of the $l$th partial
wave is $\pi/2$. Suppose that $\delta_l$ attains the value
$\pi/2$ at the incident energy $E_0$, so that
\begin{equation}
\delta_l(E_0) = \frac{\pi}{2}.
\end{equation}
Let us expand $\cot \delta_l$ in the vicinity of the resonant energy:
\begin{eqnarray}
\cot \delta_l(E)& =& \cot \delta_l(E_0) +\left(
\frac{ d \cot\delta_l}{d E}\right)_{E=E_0}(E-E_0) + \cdots\nonumber\\[0.5ex]
&=& - \left(\frac{1}{\sin^2\delta_l}\frac{d\delta_l}{d E}\right)_{E=E_0}
(E-E_0)+\cdots.
\end{eqnarray}
Defining
\begin{equation}
\left(\frac{d \delta_l(E)}{d E} \right)_{E=E_0} = \frac{2}{\Gamma},
\end{equation}
we obtain
\begin{equation}
\cot\delta_l(E) = - \frac{2}{\Gamma} \,(E-E_0) + \cdots.
\end{equation}
Recall, from Eq.~(\ref{e17.75}), that the contribution of the $l$th partial wave
to the scattering cross-section is 
\begin{equation}
\sigma_l = \frac{4\pi}{k^2} \,(2\,l+1)\,\sin^2\delta_l 
= \frac{4\pi}{k^2} \,(2\,l+1)\,\frac{1}{1+\cot^2\delta_l}.
\end{equation}
Thus,
\begin{equation}
\sigma_l \simeq \frac{4\pi}{k^2} \,(2\,l+1)\,
\frac{\Gamma^2/4}{(E-E_0)^2 + \Gamma^2/4}.
\end{equation}
This is the famous {\em Breit-Wigner formula}. The variation of
the partial cross-section $\sigma_l$ with the incident energy has
the form of a classical {\em resonance curve}. The quantity $\Gamma$ is
the width of the resonance (in energy). We can interpret the
Breit-Wigner formula as describing the absorption of an incident particle
to form a metastable state, of energy $E_0$, and lifetime $\tau = \hbar/
\Gamma$.

