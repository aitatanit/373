\chapter{One-Dimensional Potentials}\label{s5}
\section{Introduction}
In this chapter, we shall investigate the interaction of
a non-relativistic particle of mass $m$ and energy $E$ with various
one-dimensional potentials, $V(x)$. Since we are searching for stationary
 solutions with unique energies, we can write the wavefunction in the form (see Sect.~\ref{sstat})
\begin{equation}
\psi(x,t) = \psi(x)\,{\rm e}^{-{\rm i}\,E\,t/\hbar},
\end{equation} 
where $\psi(x)$ satisfies the time-independent Schr\"{o}dinger equation:
\begin{equation}\label{e5.2}
\frac{d^2 \psi}{d x^2} = \frac{2\,m}{\hbar^2}
\left[V(x)-E\right]\psi.
\end{equation}
In general, the solution, $\psi(x)$,  to the above equation must be 
{\em finite}, otherwise the probability density $|\psi|^{\,2}$ would
become infinite (which is unphysical). Likewise, the solution must  be
{\em continuous}, otherwise the
probability current (\ref{eprobc}) would become infinite (which is also
unphysical).

\section{Infinite Potential Well}\label{s5.2}
Consider a particle of mass $m$ and energy $E$ moving in the following simple potential:
\begin{equation}
V(x) = \left\{\begin{array}{lcl}
0&\mbox{\hspace{1cm}}&\mbox{for $0\leq x\leq a$}\\[0.5ex]
\infty&&\mbox{otherwise}
\end{array}\right..
\end{equation}
It follows from Eq.~(\ref{e5.2}) that if $d^2\psi/d x^2$ (and, hence, $\psi$) is
to remain finite then $\psi$ must go to zero in regions where the potential
is infinite. Hence, $\psi=0$ in the regions $x\leq 0$ and $x\geq a$.
Evidently, the problem is equivalent to that of a particle trapped in a
one-dimensional box of length $a$.
The boundary conditions on $\psi$ in
the region $0<x<a$ are 
\begin{equation}\label{e5.4}
\psi(0) = \psi(a) = 0.
\end{equation}
Furthermore, it follows from Eq.~(\ref{e5.2}) that $\psi$ satisfies
\begin{equation}\label{e5.5}
\frac{d^2 \psi}{d x^2} = - k^2\,\psi
\end{equation}
in this  region, where
\begin{equation}\label{e5.6}
k^2 = \frac{2\,m\,E}{\hbar^2}.
\end{equation}
Here, we are assuming that $E>0$. It is easily demonstrated that there are
no solutions with $E<0$ which are capable of satisfying the boundary conditions (\ref{e5.4}).

The solution to Eq.~(\ref{e5.5}), subject to the boundary conditions
(\ref{e5.4}), is
\begin{equation}
\psi_n(x) = A_n\,\sin(k_n\,x),
\end{equation}
where the $A_n$ are arbitrary (real) constants, and
\begin{equation}\label{e5.8}
k_n = \frac{n\,\pi}{a},
\end{equation}
for $n=1,2,3,\cdots$. Now, it can be seen from Eqs.~(\ref{e5.6}) and (\ref{e5.8}) 
that the energy $E$ is only allowed to take certain discrete values:
{\em i.e.},
\begin{equation}\label{eenergy}
E_n = \frac{n^2\,\pi^2\,\hbar^2}{2\,m\,a^2}.
\end{equation}
In other words, the eigenvalues of the energy operator are {\em discrete}. This
is a general feature of {\em bounded}\/ solutions: {\em i.e.}, solutions in which $|\psi|\rightarrow 0$ as $|x|\rightarrow\infty$. According to the discussion in Sect.~\ref{sstat},
we expect the stationary eigenfunctions $\psi_n(x)$ to satisfy
the orthonormality constraint
\begin{equation}
\int_0^a \psi_n(x)\,\psi_m(x)\,dx = \delta_{nm}.
\end{equation}
It is easily demonstrated that this is the case, provided $A_n = \sqrt{2/a}$. 
Hence,
\begin{equation}\label{e5.11}
\psi_n(x) = \sqrt{\frac{2}{a}}\,\sin\left(n\,\pi\,\frac{x}{a}\right)
\end{equation}
for $n=1,2,3,\cdots$. 

Finally, again from Sect.~\ref{sstat}, the general {\em time-dependent}\/ solution can be written as a linear superposition of stationary solutions:
\begin{equation}
\psi(x,t) = \sum_{n=0,\infty} c_n\,\psi_n(x)\,{\rm e}^{-{\rm i}\,E_n\,t/\hbar},
\end{equation}
where
\begin{equation}\label{e5.13}
c_n = \int_0^a\psi_n(x)\,\psi(x,0)\,dx.
\end{equation}

\section{Square Potential Barrier}\label{s5.3}
Consider a particle of mass $m$ and energy $E>0$ interacting with the
simple square potential barrier
\begin{equation}
V(x) = \left\{\begin{array}{lcl}
V_0&\mbox{\hspace{1cm}}&\mbox{for $0\leq x\leq a$}\\[0.5ex]
0&&\mbox{otherwise}
\end{array}
\right.,
\end{equation}
where $V_0>0$. In the regions to the left and to the right of the
barrier, $\psi(x)$ satisfies
\begin{equation}\label{e5.15}
\frac{d^2 \psi}{d x^2} = - k^2\,\psi,
\end{equation}
where $k$ is given by Eq.~(\ref{e5.6}).

 Let us adopt the following solution
of the above equation to the left of the barrier ({\em i.e.}, $x<0$):
\begin{equation}
\psi(x) = {\rm e}^{\,{\rm i}\,k\,x} + R\,{\rm e}^{-{\rm i}\,k\,x}.
\end{equation}
This solution consists of a plane wave of unit amplitude traveling to
the right [since the time-dependent wavefunction is multiplied by 
$\exp(-{\rm i}\,\omega\,t)$, where $\omega=E/\hbar>0$], and a plane wave of complex amplitude $R$ traveling to
the left. We interpret the first plane wave as an {\em incoming particle}\/ (or, rather, a stream of incoming particles), and
the second as a particle (or stream of particles) {\em reflected}\/ by the potential barrier. Hence, $|R|^{\,2}$ is
the probability of reflection. This can be seen by calculating the
probability current (\ref{eprobc}) in the region $x<0$, which takes the form
\begin{equation}
j_l=v\,(1-|R|^{\,2}),
\end{equation}
where $v = p/m=\hbar\,k/m$ is the classical particle velocity.

Let us adopt the following solution to Eq.~(\ref{e5.15}) to the right
of the barrier ({\em i.e.} $x>a$):
\begin{equation}
\psi(x) = T\,{\rm e}^{\,{\rm i}\,k\,x}.
\end{equation}
This solution consists of a plane wave of complex amplitude $T$
traveling to the right. We interpret this as a particle (or stream of particles) {\em transmitted}\/ through
the barrier. Hence, $|T|^{\,2}$ is the probability of transmission. 
The probability current in the region $x>a$ takes the form
\begin{equation}
j_r = v\, |T|^{\,2}.
\end{equation}
Now, according to Eq.~(\ref{ediffp}), in a stationary state ({\em i.e.}, $\partial |\psi|^{\,2}/\partial t = 0$), the probability current is a {\em spatial constant}
({\em i.e.}, $\partial j/\partial x = 0$). Hence, we must have $j_l=j_r$, or
\begin{equation}\label{e5.20}
|R|^{\,2} + |T|^{\,2} = 1.
\end{equation}
In other words, the probabilities of reflection and transmission sum to
unity, as must be the case, since reflection and transmission are  the only possible outcomes for a
particle incident on the barrier.

Inside the barrier ({\em i.e.}, $0\leq x \leq a$), $\psi(x)$ satisfies
\begin{equation}\label{e5.21}
\frac{d^2 \psi}{d x^2} = - q^2\,\psi,
\end{equation}
where
\begin{equation}
q^2 = \frac{2\,m\,(E-V_0)}{\hbar^2}.
\end{equation}

Let us, first of all, consider the case where $E> V_0$. In this case, the general
solution to Eq.~(\ref{e5.21}) inside the barrier takes the
form
\begin{equation}
\psi(x) = A\,{\rm e}^{\,{\rm i}\,q\,x} +B\,{\rm e}^{-{\rm i}\,q\,x},
\end{equation}
where $q=\sqrt{2\,m\,(E-V_0)/\hbar^2}$. 

Now, the boundary conditions at the edges of the barrier ({\em i.e.}, at
$x=0$ and $x=a$) are that $\psi$ and $d\psi/d x$ are both
continuous. These boundary conditions ensure that the probability current
(\ref{eprobc}) remains finite and continuous across the edges of the boundary, as must be the
case if it is to be a spatial constant.

Continuity of $\psi$ and $d\psi/d x$ at the left edge of
the barrier ({\em i.e.}, $x=0$) yields
\begin{eqnarray}
1 + R &=& A+B,\\[0.5ex]
k\,(1-R) &=& q\,(A-B).
\end{eqnarray}
Likewise, continuity of $\psi$ and $d\psi/d x$ at the right edge of
the barrier ({\em i.e.}, $x=a$) gives
\begin{eqnarray}
A\, {\rm e}^{\,{\rm i}\,q\,a}+ B \,{\rm e}^{-{\rm i}\,q\,a} &=& T\,{\rm e}^{\,{\rm i}\,k\,a},\\[0.5ex]
q\left(A\,{\rm e}^{\,{\rm i}\,q\,a} -B \,{\rm e}^{-{\rm i}\,q\,a}\right) &=& k\,T\,{\rm e}^{\,{\rm i}\,k\,a}.
\end{eqnarray}
After considerable algebra, the above four equations yield
\begin{equation}\label{e5.28}
|R|^{\,2} = \frac{(k^2-q^2)^{\,2}\,\sin^2(q\,a)}{4\,k^2\,q^2 + (k^2-q^2)^{\,2}\,\sin^2(q\,a)},
\end{equation}
and
\begin{equation}\label{e5.29}
|T|^{\,2} = \frac{4\,k^2\,q^2}{4\,k^2\,q^2 + (k^2-q^2)^{\,2}\,\sin^2(q\,a)}.
\end{equation}
Note that the above two expression satisfy the constraint (\ref{e5.20}).

It is instructive to compare the quantum mechanical probabilities
of reflection and transmission---(\ref{e5.28}) and (\ref{e5.29}), respectively---with those derived from classical physics. Now, according
to classical physics, if a particle of energy $E$ is incident on
a potential barrier of height $V_0<E$ then the particle slows down
as it passes through the barrier, but is otherwise unaffected.
In other words, the classical probability of reflection is
{\em zero}, and the classical probability of transmission is {\em unity}.

\begin{figure}
\epsfysize=2.8in
\centerline{\epsffile{Chapter05/fig01.eps}}
\caption{\em Transmission (solid-curve) and reflection (dashed-curve) probabilities for a square potential barrier of width $a=1.25\,\lambda$, where $\lambda$ is the free-space de Broglie wavelength, as a function
of the ratio of the height of the barrier, $V_0$, to the
energy, $E$, of the incident particle.}\label{fb1}   
\end{figure}

\begin{figure}
\epsfysize=2.8in
\centerline{\epsffile{Chapter05/fig02.eps}}
\caption{\em Transmission (solid-curve) and reflection (dashed-curve) probabilities for a particle of energy $E$ incident on  a square potential barrier of height $V_0 = 0.75\,E$, as a function
of the ratio of the width of the barrier, $a$, to the free-space de Broglie
wavelength, $\lambda$.}\label{fb2}   
\end{figure}

The reflection  and transmission probabilities obtained from Eqs.~(\ref{e5.28}) and (\ref{e5.29}), respectively,  are plotted in Figs.~\ref{fb1} and
\ref{fb2}. It can be seen, from Fig.~\ref{fb1}, that the classical
result, $|R|^{\,2}=0$ and $|T|^{\,2}=1$, is obtained in the limit where the height of the barrier
is relatively small ({\em i.e.}, $V_0\ll E$). However, when $V_0$ is
of order $E$,  there is a substantial probability that the incident particle
will be {\em reflected}\/ by the barrier. According to classical physics, reflection is impossible when $V_0 < E$. 

It can also be seen, from Fig.~\ref{fb2},
that at certain barrier widths the probability of reflection goes to {\em zero}. It turns out that this is true irrespective of the energy of the incident particle.
It is evident, from Eq.~(\ref{e5.28}), that these special barrier widths
correspond to
\begin{equation}
q\,a = n\,\pi,
\end{equation}
where $n=1,2,3,\cdots$. In other words, the special barriers widths are
integer multiples of half the de Broglie wavelength of the particle {\em inside}\/ the
barrier. There is no reflection at the special barrier widths because, at these
widths,
the backward traveling wave reflected from the left edge of the barrier
interferes destructively with the similar wave reflected from the right
edge of the barrier to give zero net reflected wave.

Let us, now, consider the case $E< V_0$. In this case, the general
solution to Eq.~(\ref{e5.21}) inside the barrier takes the
form
\begin{equation}
\psi(x) = A\,{\rm e}^{\,q\,x} + B\,{\rm e}^{-q\,x},
\end{equation}
where $q=\sqrt{2\,m\,(V_0-E)/\hbar^2}$. 
Continuity of $\psi$ and $d \psi/d x$ at the left edge of
the barrier ({\em i.e.}, $x=0$) yields
\begin{eqnarray}
1 + R &=& A+B,\\[0.5ex]
{\rm i}\,k\,(1-R) &=& q\,(A-B).
\end{eqnarray}
Likewise, continuity of $\psi$ and $d\psi/d x$ at the right edge of
the barrier ({\em i.e.}, $x=a$) gives
\begin{eqnarray}
A\, {\rm e}^{\,q\,a}+ B \,{\rm e}^{-q\,a} &=& T\,{\rm e}^{\,{\rm i}\,k\,a},\\[0.5ex]
q\left(A\, {\rm e}^{\,q\,a}-B \,{\rm e}^{-q\,a}\right) &=& {\rm i}\,k\,T\,{\rm e}^{\,{\rm i}\,k\,a}.
\end{eqnarray}
After considerable algebra, the above four equations yield
\begin{equation}\label{e5.36}
|R|^{\,2} = \frac{(k^2+q^2)^{\,2}\,\sinh^2(q\,a)}{4\,k^2\,q^2 + (k^2+q^2)^{\,2}\,\sinh^2(q\,a)},
\end{equation}
and
\begin{equation}\label{e5.37}
|T|^{\,2} = \frac{4\,k^2\,q^2}{4\,k^2\,q^2 + (k^2+q^2)^{\,2}\,\sinh^2(q\,a)}.
\end{equation}
These expressions can also be obtained from Eqs.~(\ref{e5.28}) and
(\ref{e5.29}) by making the substitution $q\rightarrow -{\rm i}\,q$.
Note that Eqs.~(\ref{e5.36}) and (\ref{e5.37}) satisfy the constraint (\ref{e5.20}).

It is again instructive to compare the quantum mechanical probabilities
of reflection and transmission---(\ref{e5.36}) and (\ref{e5.37}), respectively---with those derived from classical physics. Now, according
to classical physics, if a particle of energy $E$ is incident on
a potential barrier of height $V_0>E$ then the particle is reflected.
In other words, the classical probability of reflection is
{\em unity}, and the classical probability of transmission is {\em zero}.

\begin{figure}
\epsfysize=2.8in
\centerline{\epsffile{Chapter05/fig03.eps}}
\caption{\em Transmission (solid-curve) and reflection (dashed-curve) probabilities for a square potential barrier of width $a=0.5\,\lambda$, where $\lambda$ is the free-space de Broglie wavelength, as a function
of the ratio of the energy,  $E$, of the incoming particle to the
height, $V_0$, of the barrier.}\label{fb3}   
\end{figure}

\begin{figure}
\epsfysize=2.8in
\centerline{\epsffile{Chapter05/fig04.eps}}
\caption{\em Transmission (solid-curve) and reflection (dashed-curve) probabilities for a particle of energy $E$ incident on  a square potential barrier of height $V_0 = (4/3)\,E$, as a function
of the ratio of the width of the barrier, $a$, to the free-space de Broglie
wavelength, $\lambda$.}\label{fb4}   
\end{figure}

The reflection  and transmission probabilities obtained from Eqs.~(\ref{e5.36}) and (\ref{e5.37}), respectively,  are plotted in Figs.~\ref{fb3} and
\ref{fb4}. It can be seen, from Fig.~\ref{fb3}, that the classical
result, $|R|^{\,2}=1$ and $|T|^{\,2}=0$, is obtained for relatively
thin barriers ({\em i.e.}, $q\,a\sim 1$) in the limit where the height of the barrier
is relatively large ({\em i.e.}, $V_0\gg E$). However, when $V_0$ is
of order $E$,  there is a substantial probability that the incident particle
will be {\em transmitted}\/ by the barrier. According to classical physics, transmission is impossible when $V_0 > E$. 

It can also be seen, from
Fig.~\ref{fb4}, that the transmission probability decays {\em exponentially}\/
as the width of the barrier increases. Nevertheless, even for very
wide barriers ({\em i.e.}, $q\,a\gg 1$), there is a small but {\em finite}\/
probability that a particle incident on the barrier will be
{\em transmitted}. This phenomenon, which is inexplicable within
the context of classical physics, is called {\em tunneling}. 

\section{WKB Approximation}
Consider a particle of mass $m$ and energy $E>0$ moving through some {\em slowly varying}\/
potential $V(x)$. The particle's wavefunction satisfies
\begin{equation}\label{e5.38}
\frac{d^2\psi(x)}{dx^2} = - k^2(x)\,\psi(x),
\end{equation}
where
\begin{equation}
k^2(x) = \frac{2\,m\,[E-V(x)]}{\hbar^2}.
\end{equation}
Let us try a solution to Eq.~(\ref{e5.38}) of the form
\begin{equation}\label{e5.40}
\psi(x) = \psi_0\,\exp\left(\int_0^x{\rm i}\,k(x')\,dx'\right),
\end{equation}
where $\psi_0$ is a complex constant. Note that this solution represents
a particle propagating in the positive $x$-direction [since the full
wavefunction is multiplied by $\exp(-{\rm i}\,\omega\,t)$, where $\omega=E/\hbar>0$] with the continuously varying wavenumber
$k(x)$.
It follows that
\begin{equation}
\frac{d\psi(x)}{dx} = {\rm i}\,k(x)\,\psi(x),
\end{equation}
and
\begin{equation}\label{e5.42}
\frac{d^2\psi(x)}{dx^2} = {\rm i}\,k'(x)\,\psi(x) - k^2(x)\,\psi(x),
\end{equation}
where $k'\equiv dk/dx$. A comparison of Eqs.~(\ref{e5.38}) and (\ref{e5.42})
reveals that Eq.~(\ref{e5.40}) represents an approximate solution to
Eq.~(\ref{e5.38}) provided that the first term on its right-hand side
is negligible compared to the second. This yields
the validity criterion $|k'|\ll k^2$, or
\begin{equation}\label{e5.43}
\frac{k}{|k'|}\gg k^{-1}.
\end{equation}
In other words, the variation length-scale of $k(x)$, which is
approximately the same as the variation length-scale of $V(x)$, must
be {\em much greater}\/ than the particle's de Broglie wavelength (which
is of order $k^{-1}$). Let us suppose that this is the case. Incidentally,
the approximation involved in dropping the first term on the right-hand side
of Eq.~(\ref{e5.42}) is generally known as the {\em WKB approximation}.\,\footnote{After G.~Wentzel, H.A.~Kramers, and L.~Brillouin.}
Similarly, Eq.~(\ref{e5.40}) is termed a WKB solution.

According to the WKB solution (\ref{e5.40}),  the probability
density remains constant: {\em i.e.}, 
\begin{equation}
|\psi(x)|^{\,2} = |\psi_0|^{\,2},
\end{equation}
as long as the particle
moves through a region in which $E>V(x)$, and $k(x)$ is consequently real ({\em i.e.}, an allowed region
 according to classical physics).
Suppose, however, that the particle encounters a potential barrier ({\em i.e.}, a region from which the particle is excluded according to classical
physics). By definition, $E<V(x)$ inside such a barrier, and
$k(x)$ is consequently imaginary. Let the barrier extend from $x=x_1$ to $x_2$, where
$0<x_1<x_2$. The WKB solution inside the barrier is written
\begin{equation}\label{e5.45}
\psi(x) = \psi_1\,\exp\left(-\int_{x_1}^x |k(x')|\,dx'\right),
\end{equation}
where
\begin{equation}
\psi_1=\psi_0\,\exp\left(\int_0^{x_1}{\rm i}\,k(x')\,dx'\right).
\end{equation}
Here, we have neglected the unphysical exponentially growing solution.

According to the WKB solution (\ref{e5.45}), the probability
density {\em decays exponentially}\/ inside the barrier: {\em i.e.}, 
\begin{equation}
|\psi(x)|^{\,2} = |\psi_1|^{\,2}\,\exp\left(-2\,\int_{x_1}^x |k(x')|\,dx'\right),
\end{equation}
where $|\psi_1|^{\,2}$ is the probability density at the left-hand
side of the barrier ({\em i.e.}, $x=x_1$). It follows that the
probability density at the right-hand side of the barrier ({\em i.e.}, $x=x_2$) is
\begin{equation}
|\psi_2|^{\,2} = |\psi_1|^{\,2}\,\exp\left(-2\,\int_{x_1}^{x_2} |k(x')|\,dx'\right).
\end{equation}
Note that $|\psi_2|^{\,2} < |\psi_1|^{\,2}$. Of course, in the region to the right of the
barrier ({\em i.e.}, $x>x_2$), the probability density takes the
constant value $|\psi_2|^{\,2}$. 

We can interpret the ratio of the probability densities to the right and to the left  of the potential barrier as the probability, $|T|^{\,2}$, that a particle
incident from the left will tunnel through the barrier and
emerge on the other side: {\em i.e.},
\begin{equation}\label{e5.49}
|T|^{\,2} = \frac{|\psi_2|^{\,2}}{|\psi_1|^{\,2}} = \exp\left(-2\,\int_{x_1}^{x_2} |k(x')|\,dx'\right)
\end{equation}
(see Sect.~\ref{s5.3}).
It is easily demonstrated that the probability of a particle incident from the
right  tunneling through the barrier is the same.

Note that the criterion (\ref{e5.43}) for the validity of the WKB approximation
implies that the above transmission probability is {\em very small}. Hence,
the WKB approximation only applies to situations in which there is
very little chance of a particle tunneling through the potential barrier in question.
Unfortunately, the validity criterion (\ref{e5.43}) breaks down completely
at the edges of the barrier ({\em i.e.}, at $x=x_1$ and $x_2$), since
$k(x)=0$ at these points. However, it can be demonstrated that the
contribution of those regions, around $x=x_1$ and $x_2$, in which the WKB
approximation breaks down to the integral in Eq.~(\ref{e5.49})
is fairly negligible. Hence, the above expression for the tunneling
probability is a reasonable approximation provided that the incident particle's
de Broglie wavelength is  much smaller than the spatial extent of the potential
barrier.

\section{Cold Emission}
Suppose that an unheated  metal surface is subject to a large uniform external electric field
of strength ${\cal E}$,
which is directed such that it accelerates electrons {\em away}\/ from the surface. We have
already seen (in Sect.~\ref{s3.3}) that electrons just below the surface
of a metal can be regarded as being in a potential well of depth $W$,
where $W$ is called the {\em work function}\/ of the surface. Adopting a simple
one-dimensional treatment of the problem, let the metal lie at $x<0$, and
the surface at $x=0$. Now, the applied electric field is shielded from the
interior of the metal.  Hence, the energy, $E$, say, of an electron just below the
surface is unaffected by the field.
In the absence of the electric field, the potential
barrier just above is the surface is simply $V(x)-E=W$. The electric field
modifies this to $V(x)-E=W-e\,{\cal E}\,x$. The potential barrier is
sketched in Fig.~\ref{fcold}.

\begin{figure}
\epsfysize=3in
\centerline{\epsffile{Chapter05/fig05.eps}}
\caption{\em The potential barrier for an electron  in a metal
surface subject to an external electric field.}\label{fcold}   
\end{figure}

It can be seen, from Fig.~\ref{fcold}, that an electron just below the  surface of the
metal is confined by a triangular potential barrier which extends from $x=x_1$ to $x_2$,
where $x_1=0$ and $x_2 = W / e\,{\cal E}$. Making use of the
WKB approximation (see the previous subsection), the probability of such an
electron tunneling through the barrier, and consequently being emitted from the surface,
is
\begin{equation}
|T|^{\,2} = \exp\left(-\frac{2\,\sqrt{2\,m}}{\hbar}\int_{x_1}^{x_2}
\sqrt{V(x)- E}\,dx\right),
\end{equation}
or
\begin{equation}
|T|^{\,2} = \exp\left(-\frac{2\,\sqrt{2\,m}}{\hbar}\int_{0}^{W/e\,{\cal E}}
\sqrt{W-e\,{\cal E}\,x }\,dx\right).
\end{equation}
This reduces to
\begin{equation}
|T|^{\,2} = \exp\left(-2\,\sqrt{2}\,\frac{m^{1/2}\,W^{\,3/2}}{\hbar\,e\,{\cal E}}\int_{0}^{1}
\sqrt{1-y}\,dy\right),
\end{equation}
or
\begin{equation}\label{e5.53}
|T|^{\,2} = \exp\left(-\frac{4\,\sqrt{2}}{3}\,\frac{m^{1/2}\,W^{\,3/2}}{\hbar\,e\,{\cal E}}\right).
\end{equation}
The above result is known as the {\em Fowler-Nordheim}\/ formula. Note that
the probability of emission increases {\em exponentially}\/ as the electric
field-strength above the surface of the metal increases.

The cold emission of electrons from a metal surface is the basis of
an important device known as a {\em scanning tunneling microscope}, or an STM. An STM consists of a very sharp conducting probe which is
scanned over the surface of a metal (or any other solid conducting medium). 
A large voltage difference is applied between the probe and the surface. Now, the surface electric
field-strength  immediately below the probe tip is proportional
to the applied potential difference, and inversely proportional to the spacing between
the tip and the surface. Electrons tunneling between the surface
and the probe tip give rise to a weak electric current. The magnitude of
this current is proportional to the tunneling probability (\ref{e5.53}). It
follows that the current is an {\em extremely sensitive}\/ function of the surface
electric field-strength, and, hence, of the spacing between the tip and the surface (assuming that the
potential difference is held constant). An STM can thus be used to construct
a very accurate contour map of the surface under investigation. In fact, STMs are
capable of achieving sufficient resolution to image individual atoms

\section{Alpha Decay}
Many types of heavy atomic nucleus spontaneously decay to produce  daughter nucleii
via the emission of $\alpha$-particles ({\em i.e.},  helium nucleii) of some characteristic energy.
This process is know as
$\alpha$-decay. Let us investigate the $\alpha$-decay of a particular type of atomic nucleus of radius $R$, charge-number $Z$,
and mass-number $A$. Such a nucleus thus decays to produce a daughter
nucleus of charge-number $Z_1=Z-2$ and mass-number $A_1=A-4$,
and an $\alpha$-particle of charge-number $Z_2=2$ and mass-number
$A_2=4$. Let the characteristic energy of the $\alpha$-particle
be $E$. Incidentally, nuclear radii
are found to satisfy the empirical formula
\begin{equation}
R = 1.5\times 10^{-15}\,A^{1/3}\,{\rm m}=2.0\times 10^{-15}\,Z_1^{1/3}\,{\rm m}
\end{equation}
for $Z\gg 1$.

In 1928, George Gamov proposed a very successful theory of $\alpha$-decay,
according to which the $\alpha$-particle moves freely inside the nucleus, and is emitted after {\em tunneling}\/ through the
potential barrier between itself and the daughter nucleus.  In other words,
the $\alpha$-particle, whose energy is $E$, is trapped in a potential well of radius $R$ by the
potential barrier
\begin{equation}
V(r) = \frac{Z_1\,Z_2\,e^2}{4\pi\,\epsilon_0\,r}
\end{equation}
for $r>R$. 

Making use of the WKB approximation (and neglecting the fact
that $r$ is a radial, rather than a Cartesian, coordinate), the probability
of the $\alpha$-particle tunneling through the barrier is
\begin{equation}
|T|^{\,2} = \exp\left(-\frac{2\sqrt{2\,m}}{\hbar}\int_{r_1}^{r_2}
\sqrt{V(r)-E}\,dr\right),
\end{equation}
where $r_1=R$ and $r_2 = Z_1\,Z_2\,e^2/(4\pi\,\epsilon_0\,E)$. Here,
$m=4\,m_p$ is the $\alpha$-particle mass. The above expression 
reduces to
\begin{equation}
|T|^{\,2} = \exp\left(-2\,\sqrt{2}\,\beta \int_{1}^{E_c/E}\left[\frac{1}{y}-\frac{E}{E_c}\right]^{1/2} dy\right),
\end{equation}
where 
\begin{equation}
\beta = \left(\frac{Z_1\,Z_2\,e^2\,m\,R}{4\pi\,\epsilon_0\,\hbar^2}\right)^{1/2} = 0.74\,Z_1^{2/3}
\end{equation}
is a dimensionless constant, and
\begin{equation}
E_c = \frac{Z_1\,Z_2\,e^2}{4\pi\,\epsilon_0\,R} = 1.44\,Z_1^{2/3}\,\,{\rm MeV}
\end{equation}
is the characteristic energy the $\alpha$-particle would need in order to escape
from the nucleus without tunneling. Of course, $E\ll E_c$. 
It is easily demonstrated that
\begin{equation}
\int_1^{1/\epsilon}\left[\frac{1}{y} - \epsilon\right]^{1/2} dy \simeq
\frac{\pi}{2\,\sqrt{\epsilon}}-2
\end{equation}
when $\epsilon\ll 1$. 
Hence.
\begin{equation}
|T|^{\,2} \simeq \exp\left(-2\,\sqrt{2}\,\beta\left[\frac{\pi}{2}\sqrt{\frac{E_c}{E}}-2\right]\right).
\end{equation}

Now, the $\alpha$-particle moves inside the nucleus with the characteristic
velocity $v= \sqrt{2\,E/m}$. It follows that the particle bounces backward
and forward within the nucleus at the frequency 
$\nu\simeq v/R$, giving
\begin{equation}
\nu\simeq 2\times 10^{28}\,\,{\rm yr}^{-1}
\end{equation}
for a 1 MeV $\alpha$-particle trapped inside a typical  heavy nucleus of radius $10^{-14}$\,m.
Thus, the $\alpha$-particle effectively attempts to tunnel through the potential
barrier $\nu$ times a second. If each of these attempts has a probability
$|T|^{\,2}$ of succeeding, then the probability of decay per unit time
is $\nu\,|T|^2$. Hence, if there are $N(t)\gg 1$ undecayed nuclii at time $t$ then
there are only $N+dN$ at time $t+dt$, where
\begin{equation}
dN = - N\,\nu\,|T|^2\,dt.
\end{equation}
This expression can be integrated to give
\begin{equation}
N(t) = N(0)\,\exp(-\nu\,|T|^2\,t).
\end{equation}
Now, the {\em half-life}, $\tau$,  is defined as the time  which must elapse
in order for half of the nuclii originally present to decay. It follows from
the above formula that
\begin{equation}
\tau = \frac{\ln 2}{\nu\,|T|^2}.
\end{equation}
Note that the half-life is {\em independent}\/ of $N(0)$.

Finally, making use of the above results, we obtain
\begin{equation}\label{e5.64}
\log_{10}[\tau ({\rm yr})] = -C_1 - C_2\,Z_1^{\,2/3} + C_3\,\frac{Z_1}{\sqrt{E({\rm MeV})}},
\end{equation}
where
\begin{eqnarray}
C_1 &= &28.5,\\[0.5ex]
C_2 &=& 1.83,\\[0.5ex]
C_3 &=& 1.73.
\end{eqnarray}
 
\begin{figure}
\epsfysize=4in
\centerline{\epsffile{Chapter05/fig06.eps}}
\caption{\em The experimentally determined half-life, $\tau_{ex}$, of various atomic nucleii which decay via $\alpha$ emission versus the best-fit theoretical half-life $\log_{10}(\tau_{th}) = -28.9 - 1.60\,Z_1^{\,2/3} + 1.61\,Z_1/\sqrt{E}$. Both half-lives are measured in years. Here, $Z_1=Z-2$, where $Z$ is the charge number of the nucleus, and $E$ the characteristic energy of the emitted $\alpha$-particle in MeV. In
order of increasing half-life, the points correspond to the
following nucleii: Rn 215, Po 214, Po 216, Po 197, Fm 250, Ac 225, U 230, U 232, U 234, Gd 150, U 236, U 238, Pt 190, Gd 152, Nd 144. Data obtained from IAEA Nuclear Data Centre.}\label{fal}   
\end{figure}

The half-life, $\tau$,  the daughter charge-number, $Z_1=Z-2$, and
the $\alpha$-particle energy, $E$, for atomic nucleii which undergo $\alpha$-decay
are indeed found to satisfy a relationship of the form (\ref{e5.64}). The
best fit to the data (see Fig.~\ref{fal}) is obtained using
\begin{eqnarray}
C_1 &= &28.9,\\[0.5ex]
C_2 &=& 1.60,\\[0.5ex]
C_3 &=& 1.61.
\end{eqnarray}
Note that these values are remarkably similar to those calculated above.

\section{Square Potential Well}
Consider a particle of mass $m$ and energy $E$ interacting with the
simple square potential well
\begin{equation}\label{e5.71}
V(x) = \left\{\begin{array}{lcl}
-V_0&\mbox{\hspace{1cm}}&\mbox{for $-a/2\leq x\leq a/2$}\\[0.5ex]
0&&\mbox{otherwise}
\end{array}
\right.,
\end{equation}
where $V_0>0$. 

Now, if $E>0$ then the particle is unbounded. Thus, when the particle encounters the well
it is either reflected or transmitted. As is easily demonstrated, the reflection and transmission
probabilities are given by Eqs.~(\ref{e5.28}) and (\ref{e5.29}), respectively,
where
\begin{eqnarray}
k^2&=& \frac{2\,m\,E}{\hbar^2},\\[0.5ex]
q^2 &=& \frac{2\,m\,(E+V_0)}{\hbar^2}.
\end{eqnarray}

Suppose, however, that $E<0$. In this case, the particle
is bounded ({\em i.e.}, $|\psi|^2\rightarrow 0$ as $|x|\rightarrow\infty$). 
Is is possible to find bounded solutions of Schr\"{o}dinger's equation
in  the  finite square potential well (\ref{e5.71})? 

Now, it is easily seen that  independent solutions of Schr\"{o}dinger's equation (\ref{e5.2})
in the symmetric [{\em i.e.}, $V(-x)=V(x)$] potential (\ref{e5.71})
must be either totally symmetric [{\em i.e.}, $\psi(-x)=\psi(x)$], or
totally anti-symmetric [{\em i.e.}, $\psi(-x) =-\psi(x)$]. Moreover,
the solutions must satisfy the boundary condition
\begin{equation}
\psi\rightarrow 0\mbox{\hspace{2cm}as $|x|\rightarrow\infty$}.
\end{equation}

Let us, first of all, search for a totally symmetric solution.
In the region to the left of the well ({\em i.e.} $x<-a/2$), the
solution of Schr\"{o}dinger's equation which satisfies the
boundary condition $\psi\rightarrow 0$ and $x\rightarrow-\infty$ is
\begin{equation}
\psi(x) = A\,{\rm e}^{\,k\,x},
\end{equation}
where
\begin{equation}
k^2 = \frac{2\,m\,|E|}{\hbar^2}.
\end{equation}
By symmetry, the solution in the region to the right of the well ({\em i.e.},
$x>a/2$) is
\begin{equation}
\psi(x) = A\,{\rm e}^{-k\,x}.
\end{equation}
The solution inside the well ({\em i.e.}, $|x|\leq a/2$) which
satisfies the symmetry constraint $\psi(-x)=\psi(x)$ is
\begin{equation}
\psi(x) = B\,\cos(q\,x),
\end{equation}
where
\begin{equation}
q^2 = \frac{2\,m\,(V_0+E)}{\hbar^2}.
\end{equation}
Here, we have assumed that $E> -V_0$. 
The constraint that $\psi(x)$ and its first derivative be continuous at the
edges of the well ({\em i.e.}, at $x=\pm a/2$) yields
\begin{equation}\label{e5.81}
k = q\,\tan(q\,a/2).
\end{equation}

Let $y= q\,a/2$. It follows that
\begin{equation}
E = E_0\,y^2 - V_0,
\end{equation}
where
\begin{equation}
E_0 = \frac{2\,\hbar^2}{m\,a^2}.
\end{equation}
Moreover, Eq.~(\ref{e5.81}) becomes
\begin{equation}\label{e5.84}
\frac{\sqrt{\lambda-y^2}}{y} = \tan y,
\end{equation}
with
\begin{equation}
\lambda = \frac{V_0}{E_0}.
\end{equation}
Here, $y$ must lie in the range $0< y< \sqrt{\lambda}$: {\em i.e.},
$E$ must lie in the range $-V_0< E < 0$.

\begin{figure}
\epsfysize=3in
\centerline{\epsffile{Chapter05/fig07.eps}}
\caption{\em The curves $\tan y$ (solid) and $\sqrt{\lambda - y^2}/y$ (dashed), calculated for $\lambda = 1.5\,\pi^2$. The latter curve takes the
value $0$ when $y>\sqrt{\lambda}$. }\label{well}   
\end{figure}

Now, the solutions to Eq.~(\ref{e5.84}) correspond to the
intersection of the curve $\sqrt{\lambda - y^2}/y$ with the curve
$\tan y$. Figure~\ref{well} shows these two curves plotted for
a particular value of $\lambda$. In this case, the curves intersect
twice, indicating the existence of two totally symmetric bound states in the well.
Moreover, it is evident, from the figure, that as $\lambda$ increases ({\em i.e.}, as the well becomes
deeper)  there are more and more bound states. However, it is also evident that there is
always at least one totally symmetric bound state, no matter how small $\lambda$
becomes ({\em i.e.}, no matter how shallow the well becomes). In the limit $\lambda\gg 1$
({\em i.e.}, the limit in which the well becomes very deep), the
solutions to Eq.~(\ref{e5.84}) asymptote to the roots of $\tan y =\infty$.
This gives $y = (2\,j-1)\,\pi/2$, where $j$ is a positive integer, or
\begin{equation}
q = \frac{(2\,j-1)\,\pi}{a}.
\end{equation}
These solutions are equivalent to the odd-$n$ infinite square well solutions 
specified by Eq.~(\ref{e5.8}).

\begin{figure}
\epsfysize=3in
\centerline{\epsffile{Chapter05/fig08.eps}}
\caption{\em The curves $\tan y$ (solid) and $-y/\sqrt{\lambda - y^2}$ (dashed), calculated for $\lambda = 1.5\,\pi^2$. }\label{well1}   
\end{figure}

For the case of a totally anti-symmetric bound state, similar analysis to the
above yields 
\begin{equation}\label{e5.85}
-\frac{y}{\sqrt{\lambda-y^2}} = \tan y.
\end{equation}
The solutions of this equation correspond to the intersection of the
curve $\tan y$ with the curve  $-y/\sqrt{\lambda-y^2}$. Figure~\ref{well1} shows these two curves plotted for
the same value of $\lambda$ as that used in Fig.~\ref{well}. In this
case, the curves intersect once, indicating the existence of
a single totally anti-symmetric bound state in the well. It is, again, evident, from the figure, that as $\lambda$ increases ({\em i.e.}, as the well becomes
deeper) there are more and more bound states. However, it is also evident that
when $\lambda$ becomes sufficiently small [{\em i.e.}, $\lambda < (\pi/2)^2$] then there is no totally
anti-symmetric bound state. In other words, a very shallow potential well
always possesses a totally symmetric bound state, but does not generally
possess a totally anti-symmetric bound state. In the limit $\lambda\gg 1$
({\em i.e.}, the limit in which the well becomes very deep), the
solutions to Eq.~(\ref{e5.85}) asymptote to the roots of $\tan y =0$.
This gives $y = j\,\pi$, where $j$ is a positive integer, or
\begin{equation}
q = \frac{2\,j\,\pi}{a}.
\end{equation}
These solutions are equivalent to the even-$n$ infinite square well solutions 
specified by Eq.~(\ref{e5.8}).

\section{Simple Harmonic Oscillator}\label{sosc}
The classical Hamiltonian of a simple harmonic oscillator is
\begin{equation}
H = \frac{p^2}{2\,m} + \frac{1}{2}\,K\,x^2,
\end{equation}
where $K>0$ is the so-called force constant of the oscillator. Assuming that the quantum
mechanical Hamiltonian has the same form as the classical Hamiltonian, the time-independent Schr\"{o}dinger equation for a particle of mass $m$ and energy $E$ moving in a
simple harmonic potential becomes
\begin{equation}\label{e5.90}
\frac{d^2\psi}{dx^2}  =  \frac{2\,m}{\hbar^2}\left(\frac{1}{2}\,K\,x^2-E\right)\psi.
\end{equation}
Let $\omega = \sqrt{K/m}$, where $\omega$ is the oscillator's classical angular frequency of oscillation. Furthermore, let
\begin{equation}
y = \sqrt{\frac{m\,\omega}{\hbar}}\,x,
\end{equation}
and
\begin{equation}\label{e5.92}
\epsilon = \frac{2\,E}{\hbar\,\omega}.
\end{equation}
Equation (\ref{e5.90}) reduces to
\begin{equation}\label{e5.93}
\frac{d^2\psi}{dy^2} - (y^2-\epsilon)\,\psi = 0.
\end{equation}
We need to find solutions to the above equation which are bounded
at infinity: {\em i.e.},  solutions which satisfy the boundary
condition $\psi\rightarrow 0$ as $|y|\rightarrow\infty$.

Consider the behavior of the solution to Eq.~(\ref{e5.93}) in the limit $|y|\gg 1$. As is easily seen, in this limit the equation simplifies somewhat to give
\begin{equation}
\frac{d^2\psi}{dy^2} - y^2\,\psi \simeq  0.
\end{equation}
The approximate solutions to the above equation are
\begin{equation}
\psi(y) \simeq A(y)\,{\rm e}^{\pm y^2/2},
\end{equation}
where $A(y)$ is a relatively slowly varying function of $y$.
Clearly, if $\psi(y)$ is to remain bounded as $|y|\rightarrow\infty$ then we
must chose the exponentially decaying solution. This suggests that
we should write
\begin{equation}\label{e5.96}
\psi(y) = h(y)\,{\rm e}^{-y^2/2},
\end{equation} 
where we would expect $h(y)$ to be an algebraic, rather than an exponential, function of $y$.

Substituting Eq.~(\ref{e5.96}) into Eq.~(\ref{e5.93}), we obtain
\begin{equation}\label{e5.97}
\frac{d^2h}{dy^2} - 2\,y\,\frac{dh}{dy} + (\epsilon-1)\,h = 0.
\end{equation}
Let us attempt a power-law solution of the form
\begin{equation}\label{e5.98}
h(y) = \sum_{i=0}^\infty c_i\,y^i.
\end{equation}
Inserting this test solution into Eq.~(\ref{e5.97}), and equating the
coefficients of $y^i$, we obtain the recursion relation
\begin{equation}\label{e5.99}
c_{i+2} = \frac{(2\,i-\epsilon+1)}{(i+1)\,(i+2)}\,c_i.
\end{equation}
Consider the behavior of $h(y)$ in the limit $|y|\rightarrow\infty$.
The above recursion relation simplifies to
\begin{equation}
c_{i+2} \simeq \frac{2}{i}\,c_i.
\end{equation}
Hence, at large $|y|$, when the higher powers of $y$ dominate, we
have
\begin{equation}
h(y) \sim C \sum_{j}\frac{y^{2\,j}}{j!}\sim C\,{\rm e}^{\,y^2}.
\end{equation}
It follows that $\psi(y) = h(y)\,\exp(-y^2/2)$ varies as
$\exp(\,y^2/2)$ as $|y|\rightarrow\infty$.  This behavior is unacceptable,
since it does not satisfy the boundary condition $\psi\rightarrow 0$
as $|y|\rightarrow\infty$. The only way in which we can prevent $\psi$
from blowing up as $|y|\rightarrow\infty$
is to demand that the power series (\ref{e5.98}) {\em terminate}\/ at
some finite value of $i$. This implies, from the recursion relation
(\ref{e5.99}), that
\begin{equation}
\epsilon = 2\,n+1,
\end{equation}
where $n$ is a non-negative integer. Note that the number of terms in the power
series (\ref{e5.98}) is $n+1$. Finally, using Eq.~(\ref{e5.92}), we obtain
\begin{equation}
E = (n+1/2)\,\hbar\,\omega,
\end{equation}
for $n=0,1,2,\cdots$. 

Hence, we conclude that a particle moving in a
harmonic potential has quantized energy levels which
are {\em equally spaced}. The
spacing between successive energy levels is $\hbar\,\omega$, where
$\omega$ is the classical oscillation frequency. Furthermore, the
lowest energy state ($n=0$) possesses the {\em finite}\/ energy
$(1/2)\,\hbar\,\omega$. This is sometimes called {\em zero-point energy}.
It is easily demonstrated that the (normalized) wavefunction of the lowest
energy state takes the form
\begin{equation}
\psi_0(x) = \frac{{\rm e}^{-x^2/2\,d^2}}{\pi^{1/4}\,\sqrt{d}},
\end{equation}
where $d=\sqrt{\hbar/m\,\omega}$. 

Let $\psi_n(x)$ be an energy eigenstate of the harmonic oscillator
corresponding to the eigenvalue
\begin{equation}
E_n = (n+1/2)\,\hbar\,\omega.
\end{equation}
Assuming that the $\psi_n$ are properly normalized (and real), we have
\begin{equation}\label{e5.107}
\int_{-\infty}^\infty \psi_n\,\psi_m\,dx = \delta_{nm}.
\end{equation}
Now, Eq.~(\ref{e5.93}) can be written
\begin{equation}\label{e5.108}
\left(-\frac{d^2}{d y^2}+y^2\right)\psi_n = (2n+1)\,\psi_n,
\end{equation}
where $x = d\,y$, and $d=\sqrt{\hbar/m\,\omega}$. It is helpful to
define the operators
\begin{equation}\label{e5.109}
a_\pm = \frac{1}{\sqrt{2}}\left(\mp \frac{d}{dy}+y\right).
\end{equation}
As is easily demonstrated, these operators satisfy the commutation relation
\begin{equation}
[a_+,a_-] = -1.
\end{equation}
Using these operators, Eq.~(\ref{e5.108}) can also be written
in the forms
\begin{equation}
a_+\,a_-\,\psi_n = n\,\psi_n,
\end{equation}
or
\begin{equation}
a_-\,a_+\,\psi_n = (n+1)\,\psi_n.
\end{equation}
The above two equations imply that
\begin{eqnarray}\label{e5.113}
a_+\,\psi_n &=& \sqrt{n+1}\,\psi_{n+1},\\[0.5ex]
a_-\,\psi_n &=&\sqrt{n}\,\psi_{n-1}.\label{e5.114}
\end{eqnarray}
We conclude that $a_+$ and $a_-$ are {\em raising and lowering operators},
respectively, for the harmonic oscillator: {\em i.e.}, operating on the wavefunction with $a_+$ causes the
quantum number $n$ to increase by unity, and {\em vice versa}. 
The Hamiltonian for the harmonic oscillator can be written in the form
\begin{equation}
H = \hbar\,\omega\,\left(a_+\,a_- + \frac{1}{2}\right),
\end{equation}
from which the result
\begin{equation}
H\,\psi_n = (n+1/2)\,\hbar\,\omega\,\psi_n = E_n\,\psi_n
\end{equation}
is readily deduced.
Finally, Eqs.~(\ref{e5.107}), (\ref{e5.113}), and (\ref{e5.114})
yield the useful expression
\begin{eqnarray}\label{e5.xxx}
\int_{-\infty}^\infty \psi_m\,x\,\psi_n\,dx &=& \frac{d}{\sqrt{2}}\int_{-\infty}^{\infty}\psi_m\,(a_+ + a_-)\,\psi_n\,dx\\[0.5ex]
&=& \sqrt{\frac{\hbar}{2\,m\,\omega}}\left(\sqrt{m}\,\delta_{m,n+1} + \sqrt{n}\,\delta_{m,n-1}\right).\nonumber
\end{eqnarray}

\subsubsection*{Exercises}
{\small
\begin{enumerate}
\item Show that the wavefunction of a particle of mass $m$ in an infinite one-dimensional square-well of width $a$
returns to its original form after a quantum revival time $T=4\,m\,a^2/\pi\,\hbar$. 

\item A particle of mass $m$ moves freely in one dimension between
impenetrable walls located at 
$x=0$ and $a$. Its initial wavefunction is
$$
\psi(x,0) = \sqrt{2/a}\,\sin(3\pi\,x/a).
$$
What is the subsequent time evolution of the wavefunction?
Suppose that the initial wavefunction is 
$$
\psi(x,0) = \sqrt{1/a}\,\sin(\pi\,x/a)\,[1+2\,\cos(\pi\,x/a)].
$$
What now is the subsequent time evolution? Calculate the probability
of finding the particle between 0 and $a/2$ as a function of time in
each case.

\item A particle of mass $m$ is in the ground-state of an infinite one-dimensional square-well of width $a$. Suddenly the well expands to
twice its original size, as the right wall moves from $a$ to $2a$, leaving
the wavefunction momentarily undisturbed. The energy of the particle
is now measured. What is the most probable result? What is the probability
of obtaining this result? What is the next most probable result, and
what is its probability of occurrence? What is the expectation value
of the energy? 

\item A stream of particles of mass $m$ and energy $E>0$ encounter a
potential step of height $W (<E)$: {\em i.e.}, $V(x)=0$ for $x<0$ and
$V(x)=W$ for $x>0$ with the particles incident from $-\infty$. Show that the fraction
reflected is
$$
R = \left(\frac{k-q}{k+q}\right)^2,
$$
where $k^2= (2m/\hbar^2)\,E$ and $q^{\,2}= (2m/\hbar^2)\,(E-W)$. 

\item A stream of particles of mass $m$ and energy $E>0$ encounter the
delta-function potential $V(x) = -\alpha\,\delta (x)$, where
$\alpha>0$.  Show that the fraction
reflected is
$$
R = \beta^2/(1+\beta^2),
$$
where $\beta= m\,\alpha/\hbar^2\,k$, and $k^2= (2m/\hbar^2)\,E$. 
Does such a potential have a bound state? If so, what is its
energy?

\item Two potential wells of width $a$ are separated by a distance $L\gg a$.
A particle of mass $m$ and energy $E$ is in one of the wells. Estimate
the time required for the particle to tunnel to the other well.

\item Consider the half-infinite potential well
$$
V(x) = \left\{\begin{array}{lll}
\infty&\mbox{\hspace{1cm}}&x\leq 0\\
-V_0&&0<x<L\\
0 &&x\geq L
\end{array}\right.,
$$
where $V_0> 0$. Demonstrate that the bound-states of a particle of
mass $m$ and energy $-V_0<E<0$ satisfy
$$
\tan\left(\sqrt{2\,m\,(V_0+E)}\,\,L/\hbar\right) = - \sqrt{(V_0+E)/(-E)}.
$$

\item Find the properly normalized first two excited energy eigenstates of
the harmonic oscillator, as well as  the expectation value of the potential energy in the $n$th energy eigenstate. Hint: Consider the raising and lowering operators $a_\pm$ defined
in Eq.~(\ref{e5.109}).

\end{enumerate}
}