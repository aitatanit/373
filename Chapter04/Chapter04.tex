\chapter{Fundamentals of Quantum Mechanics}\label{s4}
\section{Introduction}
The previous chapter serves as a useful introduction to many of the basic
concepts of quantum mechanics. In this chapter, we shall examine
these concepts in a more systematic 
fashion. For the sake of simplicity, we shall  concentrate on one-dimensional systems.

\section{Schr\"{o}dinger's Equation}
Consider a dynamical system consisting of a single non-relativistic particle of mass $m$ moving along the $x$-axis in some real potential $V(x)$. In quantum mechanics, the instantaneous state of the system is represented by a complex wavefunction $\psi(x,t)$. This wavefunction evolves in time
according to Schr\"{o}dinger's equation:
\begin{equation}\label{e3.1}
{\rm i}\,\hbar\,\frac{\partial\psi}{\partial t} = -\frac{\hbar^2}{2\,m}\frac{\partial^2 \psi}{\partial x^2} + V(x)\,\psi.
\end{equation}
The wavefunction is interpreted as follows: $|\psi(x,t)|^{\,2}$ is
the probability density of a measurement of the particle's
displacement yielding the value $x$. Thus, the probability of
a measurement of the displacement giving a result
between $a$ and $b$ (where $a<b$) is
\begin{equation}\label{e3.2}
P_{x\,\in\, a:b}(t) = \int_{a}^{b}|\psi(x,t)|^{\,2}\,dx.
\end{equation}
Note that this quantity is real and positive definite.

\section{Normalization of the Wavefunction}
Now, a probability is a real number between 0 and 1. An outcome
of a measurement which has a probability 0 is an impossible outcome, whereas an
outcome which has a probability 1 is a certain outcome. According to
Eq.~(\ref{e3.2}), the probability of a measurement of $x$ yielding
a result between $-\infty$ and $+\infty$ is
\begin{equation}
P_{x\,\in\, -\infty:\infty}(t) = \int_{-\infty}^{\infty}|\psi(x,t)|^{\,2}\,dx.
\end{equation}
However, a measurement of $x$ {\em must}\/ yield a value between $-\infty$ and
$+\infty$, since the particle has to be located somewhere. It follows that
$P_{x\,\in\, -\infty:\infty}=1$, or
\begin{equation}\label{e3.4}
\int_{-\infty}^{\infty}|\psi(x,t)|^{\,2}\,dx = 1,
\end{equation}
which is generally known as the {\em normalization condition}\/ for the
wavefunction.

For example, suppose that we wish to normalize the wavefunction of
a Gaussian wave packet, centered on $x=x_0$, and of characteristic
width $\sigma$ (see Sect.~\ref{s2.9}): {\em i.e.}, 
\begin{equation}\label{e3.5}
\psi(x) = \psi_0\,{\rm e}^{-(x-x_0)^{\,2}/(4\,\sigma^2)}.
\end{equation}
In order to determine the normalization constant $\psi_0$, we simply substitute
Eq.~(\ref{e3.5}) into Eq.~(\ref{e3.4}), to obtain
\begin{equation}
|\psi_0|^{\,2}\int_{-\infty}^{\infty}{\rm e}^{-(x-x_0)^{\,2}/(2\,\sigma^2)}\,dx = 1.
\end{equation}
Changing the variable of integration to $y=(x-x_0)/(\sqrt{2}\,\sigma)$, we get
\begin{equation}
|\psi_0|^{\,2}\sqrt{2}\,\sigma\,\int_{-\infty}^{\infty}{\rm e}^{-y^2}\,dy=1.
\end{equation}
However,
\begin{equation}\label{e3.8}
\int_{-\infty}^{\infty}{\rm e}^{-y^2}\,dy = \sqrt{\pi},
\end{equation}
which implies that
\begin{equation}
|\psi_0|^{\,2} = \frac{1}{(2\pi\,\sigma^2)^{1/2}}.
\end{equation}
Hence, a general normalized Gaussian wavefunction takes the form
\begin{equation}\label{eng}
\psi(x) = \frac{{\rm e}^{\,{\rm i}\,\varphi}}{(2\pi\,\sigma^2)^{1/4}}\,{\rm e}^{-(x-x_0)^{\,2}/(4\,\sigma^2)},
\end{equation}
where $\varphi$ is an arbitrary real phase-angle.

Now, it is  important to demonstrate that if a wavefunction is initially
normalized then it stays normalized as it evolves in time according
to Schr\"{o}din\-ger's equation. If this is not the case then
the probability interpretation of the wavefunction is untenable, since it
does not make sense for the probability that a measurement  of $x$ yields {\em any}\/ possible outcome (which is, manifestly, unity) to change in time.
Hence, we require that
\begin{equation}
\frac{d}{dt}\int_{-\infty}^{\infty}|\psi(x,t)|^{\,2} \,dx = 0,
\end{equation}
for  wavefunctions satisfying Schr\"{o}dinger's equation.
The above equation gives
\begin{equation}\label{e3.12}
\frac{d}{dt}\int_{-\infty}^{\infty}\psi^{\ast}\,\psi\,dx=
\int_{-\infty}^{\infty}\left(\frac{\partial\psi^{\ast}}{\partial t}\,\psi
+\psi^\ast\,\frac{\partial\psi}{\partial t}\right)\,dx=0.
\end{equation}
Now, multiplying Schr\"{o}dinger's equation by $\psi^{\ast}/({\rm i}\,\hbar)$,
we obtain
\begin{equation}
\psi^{\ast}\,\frac{\partial\psi}{\partial t}= \frac{{\rm i}\,\hbar}{2\,m}\,\psi^\ast\,\frac{\partial^2\psi}{\partial x^2} - \frac{{\rm i}}{\hbar}\,V\,|\psi|^{\,2}.
\end{equation}
The complex conjugate of this expression yields
\begin{equation}
\psi\,\frac{\partial\psi^\ast}{\partial t}= -\frac{{\rm i}\,\hbar}{2\,m}\,\psi\,\frac{\partial^2\psi^\ast}{\partial x^2} + \frac{{\rm i}}{\hbar}\,V\,|\psi|^{\,2}
\end{equation}
[since $(A\,B)^\ast = A^\ast\,B^\ast$, $A^{\ast\,\ast}=A$, and $i^\ast= -i$].
Summing the previous two equations, we get
\begin{equation}\label{e3.15}
\frac{\partial\psi^\ast}{\partial t} \,\psi+ \psi^{\ast}\,\frac{\partial\psi}{\partial t} = \frac{{\rm i}\,\hbar}{2\,m}\left(
\psi^\ast\,\frac{\partial^2\psi}{\partial x^2}-\psi\,\frac{\partial^2\psi^\ast}{\partial x^2}\right)=\frac{{\rm i}\,\hbar}{2\,m}\,\frac{\partial}{\partial x}\!\left(\psi^\ast\,\frac{\partial\psi}{\partial x} - \psi\,\frac{\partial\psi^\ast}{\partial x}\right).
\end{equation}
Equations~(\ref{e3.12}) and (\ref{e3.15}) can be combined to
produce
\begin{equation}
\frac{d}{dt}\int_{-\infty}^{\infty}|\psi|^{\,2}\,dx=
\frac{{\rm i}\,\hbar}{2\,m}\left[\psi^\ast\,\frac{\partial\psi}{\partial x} - \psi\,\frac{\partial\psi^\ast}{\partial x}\right]_{-\infty}^{\infty} = 0.
\end{equation}
The above equation is satisfied provided
\begin{equation}
|\psi|\rightarrow 0\mbox{\hspace{0.5cm}as\hspace{0.5cm}}|x|\rightarrow\infty.
\end{equation}
However, this is a necessary condition for the integral on the left-hand
side of Eq.~(\ref{e3.4}) to converge. Hence, we conclude that
all wavefunctions which are {\em square-integrable}\/ [{\em i.e.},  are such that the integral in Eq.~(\ref{e3.4}) converges] have the property
that if the normalization condition (\ref{e3.4}) is satisfied at one instant
in time then it is satisfied at all subsequent times.

It is also possible to demonstrate, via very similar analysis to the above, that
\begin{equation}\label{epc}
\frac{d P_{x\,\in\,a:b}}{dt} + j(b,t) - j(a,t) = 0,
\end{equation}
where $P_{x\,\in\,a:b}$ is defined in Eq.~(\ref{e3.2}), and
\begin{equation}\label{eprobc}
j(x,t) = \frac{{\rm i}\,\hbar}{2\,m}\left(\psi\,\frac{\partial\psi^\ast}{\partial x} - \psi^\ast\,\frac{\partial\psi}{\partial x}\right)
\end{equation}
is known as the {\em probability current}. Note that $j$ is real.
Equation~(\ref{epc}) is a {\em probability conservation equation}.
According to this equation, the probability of a measurement
of $x$ lying in the interval $a$ to $b$ evolves in time due to the
difference between the flux of probability into the interval [{\em i.e.}, $j(a,t)$],
and that out of the interval [{\em i.e.}, $j(b,t)$].
Here, we are interpreting $j(x,t)$ as the {\em flux}\/ of probability in the $+x$-direction at position $x$ and time $t$.

Note, finally, that not all wavefunctions can be normalized according to the scheme set out in Eq.~(\ref{e3.4}). For instance, a plane wave wavefunction
\begin{equation}
\psi(x,t) = \psi_0\,{\rm e}^{\,{\rm i}\,(k\,x-\omega\,t)}
\end{equation}
is  not square-integrable, and, thus, cannot be normalized. 
For such wavefunctions, the best we can say is that
\begin{equation}
P_{x\,\in\, a:b}(t) \propto \int_{a}^{b}|\psi(x,t)|^{\,2}\,dx.
\end{equation}
In the following, all wavefunctions are assumed to be square-integrable and
normalized, unless otherwise stated.

\section{Expectation Values and Variances}
We have seen that $|\psi(x,t)|^{\,2}$ is the probability density of a
measurement of a particle's displacement yielding the value $x$ at time $t$. 
Suppose that we made a large number of independent measurements of the
displacement on
an equally  large number of identical quantum systems. In general, measurements
made on different systems will yield different results. However, from the
definition of probability, the
mean  of all these results is simply
\begin{equation}\label{e3.22}
\langle x\rangle = \int_{-\infty}^{\infty} x\,|\psi|^{\,2}\,dx.
\end{equation}
Here,
$\langle x\rangle$ is called the {\em expectation value}\/ of $x$.
Similarly the expectation value of any function of $x$ is
\begin{equation}
\langle f(x)\rangle = \int_{-\infty}^{\infty} f(x)\,|\psi|^{\,2}\,dx.
\end{equation}

In general, the results of the various different measurements of $x$ will be scattered
around the expectation value $\langle x\rangle$. The
degree of scatter is parameterized by  the quantity
\begin{equation}\label{e3.24a}
\sigma^2_x = \int_{-\infty}^{\infty} \left(x-\langle x\rangle\right)^{\,2}\,|\psi|^{\,2}\,dx \equiv \langle x^2\rangle -\langle x\rangle^{2},
\end{equation}
which is known as the {\em variance}\/ of $x$. The square-root of this
quantity, $\sigma_x$, is called the {\em standard deviation}\/ of $x$.
We generally expect  the results of measurements of $x$ to lie
within a few standard deviations of the expectation value.

For instance, consider the normalized Gaussian wave packet [see Eq. (\ref{eng})]
 \begin{equation}\label{e3.24}
\psi(x) = \frac{{\rm e}^{\,{\rm i}\,\varphi}}{(2\pi\,\sigma^2)^{1/4}}\,{\rm e}^{-(x-x_0)^{\,2}/(4\,\sigma^2)}.
\end{equation}
The expectation value of $x$ associated with this wavefunction is 
\begin{equation}
\langle x\rangle = \frac{1}{\sqrt{2\pi\,\sigma^2}}\int_{-\infty}^{\infty}
x\,{\rm e}^{-(x-x_0)^{\,2}/(2\,\sigma^2)}\,dx.
\end{equation}
Let $y=(x-x_0)/(\sqrt{2}\,\sigma)$. It follows that
\begin{equation}
\langle x\rangle = \frac{x_0}{\sqrt{\pi}}\int_{-\infty}^{\infty}{\rm e}^{-y^2}\,dy+\frac{\sqrt{2}\,\sigma}{\sqrt{\pi}}\,\int_{-\infty}^{\infty}y\,{\rm e}^{-y^2}\,dy.
\end{equation}
However, the second integral on the right-hand side is zero, by symmetry. 
Hence, making use of Eq.~(\ref{e3.8}), we obtain
\begin{equation}
\langle x\rangle =x_0.
\end{equation}
Evidently,  the expectation value of $x$ for a Gaussian wave packet is
equal to the most likely value of $x$ ({\em i.e.}, the value of $x$ which
maximizes $|\psi|^{\,2}$).

The variance of $x$ associated with the Gaussian wave packet (\ref{e3.24})
is
\begin{equation}
\sigma^2_x = \frac{1}{\sqrt{2\pi\,\sigma^2}}\int_{-\infty}^{\infty}
(x-x_0)^{\,2}\,{\rm e}^{-(x-x_0)^{\,2}/(2\,\sigma^2)}\,dx.
\end{equation}
Let $y=(x-x_0)/(\sqrt{2}\,\sigma)$. It follows that
\begin{equation}
\sigma^2_x =\frac{2\,\sigma^2}{\sqrt{\pi}}\,\int_{-\infty}^{\infty} y^2\,{\rm e}^{-y^2}\,dy.
\end{equation}
However,
\begin{equation}
\int_{-\infty}^{\infty} y^2\,{\rm e}^{-y^2}\,dy = \frac{\sqrt{\pi}}{2},
\end{equation}
giving
\begin{equation}
\sigma_x^{\,2} = \sigma^2.
\end{equation}
This result is consistent with our earlier interpretation of $\sigma$ as a measure of the
spatial extent of the wave packet (see Sect.~\ref{s2.9}).
It follows that we can rewrite the Gaussian wave packet (\ref{e3.24}) in the convenient form
\begin{equation}
\psi(x) = \frac{{\rm e}^{\,{\rm i}\,\varphi}}{(2\pi\,\sigma_x^{\,2})^{1/4}}\,{\rm e}^{-(x-\langle x\rangle)^{\,2}/(4\,\sigma_x^{\,2})}.
\end{equation}

\section{Ehrenfest's Theorem}\label{s4.5}
A simple way  to calculate the expectation value of
momentum
is to evaluate the time derivative of $\langle x\rangle$, and then
multiply by the mass $m$: {\em i.e.},
\begin{equation}\label{e4.34x}
 \langle p \rangle = m\,\frac{d\langle x\rangle}{dt} = m\,\frac{d}{dt}\int_{-\infty}^{\infty}x\,|\psi|^{\,2}\,dx =m \int_{-\infty}^{\infty}x\,\frac{\partial|\psi|^{\,2}}{\partial t}\,dx.
\end{equation}
However, it is easily demonstrated that
\begin{equation}\label{ediffp}
\frac{\partial|\psi|^{\,2}}{\partial t} + \frac{\partial j}{\partial x} = 0
\end{equation}
[this is just the differential form of Eq.~(\ref{epc})],
where $j$ is the probability current defined in Eq.~(\ref{eprobc}).
Thus,
\begin{equation}
\langle p\rangle = -m\int_{-\infty}^{\infty} x\,\frac{\partial j}{\partial x}\,dx
= m\int_{-\infty}^{\infty}j\,dx,
\end{equation}
where we have integrated by parts. It follows from Eq.~(\ref{eprobc})
that
\begin{equation}
\langle p\rangle = - \frac{{\rm i}\,\hbar}{2}\int_{-\infty}^{\infty}
\left(\psi^\ast\,\frac{\partial\psi}{\partial x} - \frac{\partial\psi^\ast}{\partial x}\,\psi\right)dx = - {\rm i}\,\hbar\int_{-\infty}^{\infty}
\psi^\ast\,\frac{\partial\psi}{\partial x}\,dx,
\end{equation}
where we have again integrated by parts. Hence, the expectation value of the momentum can be written
\begin{equation}\label{e3.38}
\langle p\rangle = m\,\frac{d\langle x\rangle}{dt}= - {\rm i}\,\hbar\int_{-\infty}^{\infty}
\psi^\ast\,\frac{\partial\psi}{\partial x}\,dx.
\end{equation}

It follows from the above that
\begin{eqnarray}
\frac{d\langle p\rangle}{dt} &=& -{\rm i}\,\hbar\int_{-\infty}^{\infty}
\left(\frac{\partial\psi^\ast}{\partial t}\,\frac{\partial\psi}{\partial x} + \psi^\ast\,\frac{\partial^2\psi}{\partial t\partial x}\right)dx\nonumber\\[0.5ex] &=& 
\int_{-\infty}^{\infty}\left[ \left({\rm i}\,\hbar\,\frac{\partial\psi}{\partial t}\right)^\ast\frac{\partial \psi}{\partial x} + \frac{\partial\psi^\ast}{\partial x}\left({\rm i}\,\hbar\,\frac{\partial\psi}{\partial t}\right)\right]dx,
\end{eqnarray}
where we have integrated by parts. 
Substituting from Schr\"{o}dinger's equation (\ref{e3.1}), and simplifying, we obtain
\begin{equation}
\frac{d\langle p\rangle}{dt} = \int_{-\infty}^{\infty}
\left[-\frac{\hbar^2}{2\,m}\frac{\partial}{\partial x}\!\left(\frac{\partial\psi^{\ast}}{\partial x}\frac{\partial \psi}{\partial x}\right) + V(x)\,\frac{\partial |\psi|^{\,2}}{\partial x}\right]dx = \int_{-\infty}^{\infty} V(x)\,\frac{\partial |\psi|^{\,2}}{\partial x}\,dx.
\end{equation}
Integration by parts yields
\begin{equation}\label{e3.41}
\frac{d\langle p\rangle}{dt} =-\int_{-\infty}^{\infty} \frac{dV}{dx}\,|\psi|^{\,2}\,dx =- \left\langle \frac{dV}{dx}\right\rangle.
\end{equation}

Hence, according to Eqs.~(\ref{e4.34x}) and (\ref{e3.41}),
\begin{eqnarray}\label{e3.42}
m\,\frac{d\langle x\rangle}{dt}&=& \langle p\rangle,\\[0.5ex]
\frac{d\langle p\rangle}{dt} &=& -\left\langle \frac{dV}{dx}\right\rangle.\label{e3.43}
\end{eqnarray}
Evidently, the expectation values of displacement and momentum obey
time evolution equations which are analogous to those of classical mechanics. 
This result is known as {\em Ehrenfest's theorem}.

Suppose that the potential $V(x)$ is {\em slowly varying}. In this case, we can
expand $dV/dx$ as a Taylor series about $\langle x\rangle$. Keeping
terms up to second order, we obtain
\begin{equation}
\frac{dV(x)}{dx} = \frac{dV(\langle x\rangle)}{d\langle x\rangle}
+ \frac{dV^2(\langle x\rangle)}{d\langle x\rangle^2}\,(x-\langle x\rangle)
+ \frac{1}{2}\,\frac{dV^3(\langle x\rangle)}{d\langle x\rangle^3}\,(x-\langle x\rangle)^{\,2}.
\end{equation}
Substitution of the above expansion into Eq.~(\ref{e3.43}) yields
\begin{equation}
\frac{d\langle p\rangle}{dt} = - \frac{dV(\langle x\rangle)}{d\langle x\rangle}
- \frac{\sigma_x^{\,2}}{2}\,\frac{dV^3(\langle x\rangle)}{d\langle x\rangle^3},
\end{equation}
since $\langle 1\rangle =1$, and $\langle x-\langle x\rangle\rangle = 0$,
and $\langle (x-\langle x\rangle)^{\,2}\rangle = \sigma_x^{\,2}$.
The final term on the right-hand side of the above equation can be neglected
when the spatial extent of the particle wavefunction, $\sigma_x$, is much
smaller than the variation length-scale of the potential. In this case,
Eqs.~(\ref{e3.42}) and (\ref{e3.43}) reduce to
\begin{eqnarray}
m\,\frac{d\langle x\rangle}{dt}&=& \langle p\rangle,\\[0.5ex]
\frac{d\langle p\rangle}{dt} &=& -\frac{dV(\langle x\rangle)}{d\langle x\rangle}.
\end{eqnarray}
These equations are {\em exactly equivalent}\/ to the equations of classical
mechanics, with $\langle x\rangle$ playing the role of the particle displacement. Of course, if the spatial extent of the wavefunction is negligible
then a measurement of $x$ is almost certain to yield a result which lies
very close to $\langle x\rangle$. Hence, we conclude that quantum mechanics
corresponds to classical mechanics in the limit that the spatial
extent of the wavefunction (which is typically of order the de Boglie wavelength) is negligible. This is an important result, since we know that
classical mechanics gives the correct answer in this limit.

\section{Operators}\label{s4.6}
An operator, $O$ (say), is a mathematical entity which transforms
one function into another: {\em i.e.},
\begin{equation}
O(f(x))\rightarrow g(x).
\end{equation}
For instance, $x$ is an operator, since $x\,f(x)$ is a different function
to $f(x)$, and is fully specified once $f(x)$ is given. Furthermore,
$d/dx$ is also an operator, since $df(x)/dx$ is a different function
to $f(x)$, and is fully specified once $f(x)$ is given.
Now,
\begin{equation}
x\,\frac{df}{dx} \neq \frac{d}{dx}\left(x\,f\right).
\end{equation}
This can also be written
\begin{equation}
x\,\frac{d}{dx} \neq \frac{d}{dx}\,x,
\end{equation}
where the operators are assumed to act on everything to
their right, and a final $f(x)$ is understood [where $f(x)$ is a general function]. The above expression illustrates
an important point: {\em i.e.}, in general, {\em operators do not
commute}. Of course, some operators do commute: {\em e.g.},
\begin{equation}
x\,x^2 = x^2\,x.
\end{equation}
Finally, an operator, $O$,  is termed {\em linear}\/ if
\begin{equation}
O(c\,f(x)) =c\,O(f(x)),
\end{equation}
where $f$ is a general function, and $c$ a general complex number.
All of the operators employed in quantum mechanics are linear.

Now, from Eqs.~(\ref{e3.22}) and (\ref{e3.38}),
\begin{eqnarray}
\langle x\rangle &=& \int_{-\infty}^{\infty}\psi^\ast\,x\,\psi\,dx,\\[0.5ex]
\langle p\rangle &=& \int_{-\infty}^{\infty}\psi^{\ast}\left(-{\rm i}\,\hbar\,
\frac{\partial}{\partial x}\right)\psi\,dx.
\end{eqnarray}
These expressions suggest a number of things. First, classical dynamical
variables, such as $x$ and $p$, are represented in quantum mechanics
by {\em linear operators}\/ which act on the wavefunction. Second,
displacement is represented by the algebraic operator $x$,
and momentum by the differential operator $-{\rm i}\,\hbar\,\partial/\partial x$: {\em i.e.},
\begin{equation}\label{e3.54}
p  \equiv -{\rm i}\,\hbar\,\frac{\partial}{\partial x}.
\end{equation}
Finally, the expectation value of some dynamical variable represented by
the operator $O(x)$ is simply
\begin{equation}\label{e3.55}
\langle O \rangle = \int_{-\infty}^{\infty}\psi^\ast(x,t)\,O(x)\,\psi(x,t)\,dx.
\end{equation}

Clearly, if an operator is to represent a dynamical variable which has
physical significance then its expectation value must be {\em real}. 
In other words, if the operator $O$ represents a physical variable
then we require that $\langle O\rangle = \langle O \rangle^\ast$, or
\begin{equation}\label{e3.55a}
\int_{-\infty}^{\infty} \psi^\ast\,(O\,\psi)\,dx = \int_{-\infty}^{\infty}(O\,\psi)^\ast\,\psi\,dx,
\end{equation}
where $O^\ast$ is the complex conjugate of $O$. An operator which
satisfies the above constraint is called an {\em Hermitian}\/ operator. 
It is easily demonstrated that $x$ and $p$ are both Hermitian.
The {\em Hermitian conjugate}, $O^\dag$, of
a general operator, $O$, is defined as follows:
\begin{equation}\label{e5.48}
\int_{-\infty}^{\infty} \psi^{\ast} \,(O\,\psi)\,dx=\int_{-\infty}^\infty
(O^\dag\,\psi)^\ast\,\psi\,dx.
\end{equation}
The Hermitian conjugate of an Hermitian operator is the same as the operator
itself: {\em i.e.}, $p^\dag = p$. For a non-Hermitian operator, $O$ (say),
it is easily demonstrated that $(O^\dag)^\dag=O$, and that the operator $O+O^\dag$ is Hermitian. 
Finally, if $A$ and $B$ are two operators, then $(A\,B)^\dag = B^\dag\,A^\dag$. 

Suppose that we wish to find the operator which corresponds to the
classical dynamical variable $x\,p$. In classical mechanics, there
is no difference between $x\,p$ and $p\,x$. However, in quantum
mechanics, we have already seen that $x\,p\neq p\,x$. So,
should be choose $x\,p$ or $p\,x$? Actually, neither of these combinations
is Hermitian. However, $(1/2)\,[x\,p + (x\,p)^\dag]$ is Hermitian. 
Moreover, $(1/2)\,[x\,p + (x\,p)^\dag]=(1/2)\,(x\,p+p^\dag\,x^\dag)=(1/2)\,(x\,p+p\,x)$, which neatly resolves
our problem of which order to put $x$ and $p$. 

It is a reasonable guess that the operator corresponding to energy (which is
called the Hamiltonian, and conventionally denoted $H$) takes the form
\begin{equation}
H \equiv \frac{p^2}{2\,m} + V(x).
\end{equation}
Note that $H$ is Hermitian. Now, it follows from Eq.~(\ref{e3.54}) that
\begin{equation}
H \equiv -\frac{\hbar^2}{2\,m}\,\frac{\partial^2}{\partial x^2} + V(x).
\end{equation}
However, according to Schr\"{o}dinger's equation, (\ref{e3.1}), we have
\begin{equation}
-\frac{\hbar^2}{2\,m}\,\frac{\partial^2}{\partial x^2} + V(x) = {\rm i}\,\hbar\,\frac{\partial}{\partial t},
\end{equation}
so
\begin{equation}
H \equiv {\rm i}\,\hbar\,\frac{\partial}{\partial t}.
\end{equation}
Thus, the time-dependent Schr\"{o}dinger equation can be written
\begin{equation}\label{etimed}
{\rm i}\,\hbar\,\frac{\partial\psi}{\partial t} = H\,\psi.
\end{equation}

Finally, if $O(x,p,E)$ is a classical dynamical variable which is
a function of displacement, momentum, and energy, then a reasonable
guess for the corresponding operator in quantum mechanics is
$(1/2)\,[O(x,p,H)+ O^\dag(x,p,H)]$, where $p=-{\rm i}\,\hbar\,\partial/\partial x$, and $H={\rm i}\,\hbar\,\partial/\partial t$.

\section{Momentum Representation}
Fourier's theorerm (see Sect.~\ref{s2.9}), applied to one-dimensional wavefunctions, yields
\begin{eqnarray}
\psi(x,t) &=& \frac{1}{\sqrt{2 \pi}}\int_{-\infty}^{\infty} \bar{\psi}(k,t)\,{\rm e}^{+{\rm i}\,k\,x}\,dk,\\[0.5ex]
\bar{\psi}(k,t) &=& \frac{1}{\sqrt{2 \pi}}\int_{-\infty}^\infty \psi(x,t)\,{\rm e}^{-{\rm i}\,k\,x}\,dx,
\end{eqnarray}
where $k$ represents wavenumber. However, $p=\hbar\,k$. Hence,
we can also write
\begin{eqnarray}\label{e3.64}
\psi(x,t) &=& \frac{1}{\sqrt{2\pi\,\hbar}}\int_{-\infty}^{\infty} \phi(p,t)\,{\rm e}^{+{\rm i}\,p\,x/\hbar}\,dp,\\[0.5ex]
\phi(p,t) &=& \frac{1}{\sqrt{2\pi\,\hbar}}\int_{-\infty}^\infty \psi(x,t)\,{\rm e}^{-{\rm i}\,p\,x/\hbar}\,dx,\label{e3.65}
\end{eqnarray}
where $\phi(p,t)= \bar{\psi}(k,t)/\sqrt{\hbar}$ is the momentum-space
equivalent to the real-space wavefunction $\psi(x,t)$.

At this stage, it is convenient to introduce a  useful function called the
{\em Dirac delta-function}. This function, denoted $\delta(x)$, was
first devised by Paul Dirac, and has the following rather
unusual properties: $\delta(x)$ is zero for $x\neq 0$, and is infinite
at $x=0$. However, the singularity at $x=0$ is such that
\begin{equation}\label{e3.64a}
\int_{-\infty}^{\infty}\delta(x)\,dx = 1.
\end{equation}
The delta-function is an example of what is known as a {\em generalized function}: {\em i.e.}, 
its value is not well-defined at all $x$, but its integral is well-defined.
Consider the integral
\begin{equation}
\int_{-\infty}^{\infty}f(x)\,\delta(x)\,dx.
\end{equation}
Since $\delta(x)$ is only non-zero infinitesimally close to $x=0$, we
can safely replace $f(x)$ by $f(0)$ in the above integral (assuming $f(x)$
is well behaved at $x=0$), to give
\begin{equation}
\int_{-\infty}^{\infty}f(x)\,\delta(x)\,dx = f(0)\,\int_{-\infty}^{\infty}\delta(x)\,dx=f(0),
\end{equation}
where use has been made of Eq.~(\ref{e3.64a}). A simple generalization of this result yields
\begin{equation}\label{e3.69}
\int_{-\infty}^\infty f(x)\,\delta(x-x_0)\,dx = f(x_0),
\end{equation}
which can also be thought of as an alternative definition of a delta-function.

Suppose that $\psi(x) = \delta(x-x_0)$. It follows from Eqs.~(\ref{e3.65})
and (\ref{e3.69}) that 
\begin{equation}
\phi(p) = \frac{{\rm e}^{-{\rm i}\,p\,x_0/\hbar}}{\sqrt{2\pi\,\hbar}}.
\end{equation}
Hence, Eq.~(\ref{e3.64}) yields the important result
\begin{equation}
\delta(x-x_0)= \frac{1}{2\pi\,\hbar}\int_{-\infty}^{\infty}{\rm e}^{+{\rm i}\,p\,(x-x_0)/\hbar}\,dp.
\end{equation}
Similarly,
\begin{equation}\label{e3.72}
\delta(p-p_0)= \frac{1}{2\pi\,\hbar}\int_{-\infty}^{\infty}{\rm e}^{+{\rm i}\,(p-p_0)\,x/\hbar}\,dx.
\end{equation}

It turns out that we can just as well formulate quantum mechanics using 
momentum-space wavefunctions, $\phi(p,t)$, as real-space wavefunctions, $\psi(x,t)$. The former scheme is known as the {\em momentum representation} of quantum mechanics. In the momentum representation,
wavefunctions are the Fourier transforms of the equivalent real-space
wavefunctions, and dynamical variables are represented by {\em different}\/ operators. Furthermore, by analogy with Eq.~(\ref{e3.55}), the
expectation value of some operator $O(p)$ takes the form
\begin{equation}\label{e4.55a}
\langle O\rangle = \int_{-\infty}^{\infty}\phi^\ast(p,t)\,O(p)\,\phi(p,t)\,dp.
\end{equation}

Consider momentum. We can write
\begin{eqnarray}
\langle p\rangle& =& \int_{-\infty}^{\infty} \psi^\ast(x,t)\left(-{\rm i}\,\hbar\,
\frac{\partial}{\partial x}\right)\psi(x,t)\,dx \nonumber\\[0.5ex]&=& \frac{1}{2\pi\,\hbar}\int_{-\infty}^{\infty}\int_{-\infty}^{\infty}\int_{-\infty}^{\infty}
\phi^\ast(p',t)\,\phi(p,t)\,p\,{\rm e}^{+{\rm i}\,(p-p')\,x/\hbar}\,dx\,dp\,dp',
\end{eqnarray}
where use has been made of Eq.~(\ref{e3.64}).
However, it follows from Eq.~(\ref{e3.72})
that
\begin{equation}
\langle p\rangle = \int_{-\infty}^{\infty}\int_{-\infty}^{\infty}\phi^\ast(p',t)\,\phi(p,t)\,p\,\delta(p-p')\,dp\,dp'.
\end{equation}
Hence, using  Eq.~(\ref{e3.69}), we obtain
\begin{equation}
\langle p\rangle = \int_{-\infty}^{\infty}\phi^\ast(p,t)\,p\,\phi(p,t)\,dp =  \int_{-\infty}^{\infty}p\,|\phi|^{\,2}\,dp.
\end{equation}
Evidently,  momentum is represented by the operator $p$ in the momentum
representation. The above expression also strongly suggests [by comparison with Eq.~(\ref{e3.22})] that $|\phi(p,t)|^{\,2}$ can be interpreted as
the probability density of a measurement of momentum yielding the
value $p$ at time $t$. It follows that $\phi(p,t)$ must satisfy an analogous normalization
condition to Eq.~(\ref{e3.4}): {\em i.e.},
\begin{equation}\label{enormp}
\int_{-\infty}^{\infty} |\phi(p,t)|^{\,2}\,dp = 1.
\end{equation}

Consider displacement. We can write
\begin{eqnarray}
\langle x\rangle& =& \int_{-\infty}^{\infty} \psi^\ast(x,t)\,x\,\psi(x,t)\,dx \\[0.5ex]&=& \frac{1}{2\pi\,\hbar}\int_{-\infty}^{\infty}\int_{-\infty}^{\infty}\int_{-\infty}^{\infty}
\phi^\ast(p',t)\,\phi(p,t)\left(-{\rm i}\,\hbar\,\frac{\partial}{\partial p}\right){\rm e}^{+{\rm i}\,(p-p')\,x/\hbar}\,dx\,dp\,dp'.\nonumber
\end{eqnarray}
Integration by parts yields
\begin{equation}
\langle x\rangle=  \frac{1}{2\pi\,\hbar}\int_{-\infty}^{\infty}\int_{-\infty}^{\infty}\int_{-\infty}^{\infty}
\phi^\ast(p',t)\,{\rm e}^{+{\rm i}\,(p-p')\,x/\hbar}\left({\rm i}\,\hbar\,\frac{\partial}{\partial p}\right)\phi(p,t)\,dx\,dp\,dp'.
\end{equation}
Hence, making use of Eqs.~(\ref{e3.72}) and (\ref{e3.69}), we obtain
\begin{equation}
\langle x\rangle=  \frac{1}{2\pi\,\hbar}\int_{-\infty}^{\infty}
\phi^\ast(p) \left({\rm i}\,\hbar\,\frac{\partial}{\partial p}\right)\phi(p)\,dp.
\end{equation}
Evidently, displacement is represented by the operator
\begin{equation}
x\equiv{\rm i}\,\hbar\,\frac{\partial}{\partial p}
\end{equation}
in the momentum representation.

Finally, let us consider the normalization of the momentum-space wavefunction $\phi(p,t)$. We have
\begin{equation}
\int_{-\infty}^{\infty} \psi^\ast(x,t)\,\psi(x,t)\,dx = \frac{1}{2\pi\,\hbar}\int_{-\infty}^{\infty}\int_{-\infty}^{\infty}\int_{-\infty}^{\infty}
\phi^\ast(p',t)\,\phi(p,t)\,{\rm e}^{+{\rm i}\,(p-p')\,x/\hbar}\,dx\,dp\,dp'.
\end{equation}
Thus, it follows from Eqs.~(\ref{e3.69}) and (\ref{e3.72})
that
\begin{equation}\label{e3.83}
\int_{-\infty}^{\infty} |\psi(x,t)|^{\,2}\,dx =\int_{-\infty}^{\infty}|\phi(p,t)|^{\,2}\,dp.
\end{equation}
Hence, if $\psi(x,t)$ is properly normalized [see Eq.~(\ref{e3.4})] then $\phi(p,t)$,
as defined in Eq.~(\ref{e3.65}), is also properly normalized [see Eq.~(\ref{enormp})].

The existence of the momentum representation illustrates an important point:
{\em i.e.}, that there are many different, but entirely equivalent, ways
of mathematically formulating quantum mechanics. For instance, it
is also possible to represent wavefunctions as row and column vectors, and dynamical
variables 
as matrices which act upon these vectors.

\section{Heisenberg's Uncertainty Principle}\label{suncert}
Consider a real-space Hermitian operator $O(x)$. A straightforward generalization of Eq. (\ref{e3.55a}) yields
\begin{equation}\label{e3.84}
\int_{-\infty}^\infty \psi_1^{\ast}\,(O\,\psi_2)\,dx = \int_{-\infty}^\infty
(O\,\psi_1)^\ast\,\psi_2\,dx,
\end{equation}
where $\psi_1(x)$ and $\psi_2(x)$ are general functions. 

Let $f=(A-\langle A\rangle)\,\psi$, where $A(x)$ is an Hermitian operator,
and $\psi(x)$ a general  wavefunction. We have
\begin{equation}
\int_{-\infty}^\infty |f|^{\,2}\,dx= \int_{-\infty}^\infty f^\ast\,f\,dx = \int_{-\infty}^\infty[(A-\langle A\rangle)\,\psi]^{\,\ast}\,[(A-\langle A\rangle)\,\psi]\,dx.
\end{equation}
Making use of Eq.~(\ref{e3.84}), we obtain
\begin{equation}\label{e3.86}
\int_{-\infty}^\infty |f|^{\,2}\,dx=\int_{-\infty}^\infty \psi^\ast\,(A-\langle A \rangle)^{\,2}\,\psi\,dx = \sigma_A^{\,2},
\end{equation}
where $\sigma_A^{\,2}$ is the variance of $A$ [see Eq.~(\ref{e3.24a})].
Similarly, if $g=(B-\langle B\rangle)\,\psi$, where $B$ is a second
Hermitian operator, then
 \begin{equation}
\int_{-\infty}^\infty |g|^{\,2}\,dx = \sigma_B^{\,2},
\end{equation}

Now, there is a standard result in mathematics, known as the
{\em Schwartz inequality}, which states that
\begin{equation}
\left|\int_a^b\,f^\ast(x)\,g(x)\,dx\right|^{\,2}\leq \int_a^b|f(x)|^{\,2}\,dx\,\int_a^b |g(x)|^{\,2}\,dx,
\end{equation}
where $f$ and $g$ are two general functions. Furthermore, if $z$ is
a complex number then
\begin{equation}\label{e3.89}
|z|^{\,2} = [{\rm Re}(z)]^{\,2} + [{\rm Im}(z)]^{\,2} \geq [{\rm Im}(z)]^{\,2} = \left[\frac{1}{2\,{\rm i}}\,(z-z^\ast)\right]^{\,2}.
\end{equation}
Hence, if $z=\int_{-\infty}^\infty f^\ast\,g\,dx$ then Eqs.~(\ref{e3.86})--(\ref{e3.89}) yield
\begin{equation}\label{e3.90}
\sigma_A^{\,2}\,\sigma_B^{\,2} \geq \left[\frac{1}{2\,{\rm i}}\,(z-z^\ast)\right]^{\,2}.
\end{equation}
However,
\begin{equation}
z = \int_{-\infty}^{\infty} [(A-\langle A\rangle)\,\psi]^{\,\ast}\,[(B-\langle B\rangle)\,\psi]\,dx = \int_{-\infty}^{\infty} \psi^\ast\,(A-\langle A\rangle)\,(B-\langle B\rangle)\,\psi\,dx,
\end{equation}
where use has been made of Eq.~(\ref{e3.84}). The above
equation reduces to
\begin{equation}
z =\int_{-\infty}^\infty \psi^\ast\,A\,B\,\psi\,dx -\langle A\rangle\,\langle B\rangle.
\end{equation}
Furthermore, it is easily demonstrated that
\begin{equation}
z ^\ast=\int_{-\infty}^\infty \psi^\ast\,B\,A\,\psi\,dx -\langle A\rangle\,\langle B\rangle.
\end{equation}
Hence, Eq.~(\ref{e3.90}) gives
\begin{equation}\label{e3.94}
\sigma_A^{\,2}\,\sigma_B^{\,2} \geq \left(\frac{1}{2\,{\rm i}}\langle[A,B]\rangle\right)^{\,2},
\end{equation}
where
\begin{equation}
[A,B] \equiv A\,B-B\,A.
\end{equation}

Equation~(\ref{e3.94}) is the general form of  {\em Heisenberg's uncertainty principle}\/ in quantum mechanics. It states that if two dynamical
variables are represented by the two Hermitian operators $A$ and $B$,
and these  operators {\em do not commute}\/ ({\em i.e.}, $A\,B\neq B\,A$),
then it is {\em impossible}\/ to simultaneously (exactly) measure the two variables.
Instead, the product of the variances in the measurements  is always greater than some critical value, which
depends on the extent to which the two operators do not commute.

For instance, displacement and momentum are represented (in real-space) by the
operators $x$ and $p\equiv-{\rm i}\,\hbar\,\partial/\partial x$, respectively.
Now, it is easily demonstrated that
\begin{equation}
[x,p] = {\rm i}\,\hbar.
\end{equation}
Thus,
\begin{equation}
\sigma_x\,\sigma_p\geq \frac{\hbar}{2},
\end{equation}
which can be recognized as the standard displacement-momentum uncertainty
principle (see Sect.~\ref{sun}). It turns out that the minimum uncertainty ({\em i.e.},
$\sigma_x\,\sigma_p=\hbar/2$) is only achieved by {\em Gaussian}\/ wave packets (see Sect.~\ref{s2.9}): {\em i.e.},
\begin{eqnarray}
\psi(x) &=& \frac{{\rm e}^{+{\rm i}\,p_0\,x/\hbar}}{(2\pi\,\sigma_x^{\,2})^{1/4}}\,{\rm e}^{-(x-x_0)^{\,2}/4\,\sigma_x^{\,2}},\\[0.5ex]
\phi(p) &=&\frac{{\rm e}^{-{\rm i}\,p\,x_0/\hbar}}{(2\pi\,\sigma_p^{\,2})^{1/4}}\,{\rm e}^{-(p-p_0)^{\,2}/4\,\sigma_p^{\,2}},
\end{eqnarray}
where $\phi(p)$ is the momentum-space equivalent of $\psi(x)$.

Energy and time are represented by the operators $H\equiv{\rm i}\,\hbar\,\partial/\partial t$ and $t$, respectively. These operators do not commute,
indicating that energy and time cannot be measured simultaneously.
In fact,
\begin{equation}
[H,t] = {\rm i}\,\hbar,
\end{equation}
so
\begin{equation}
\sigma_E\,\sigma_t\geq \frac{\hbar}{2}.
\end{equation}
This can be written, somewhat less exactly, as
\begin{equation}
{\mit\Delta}E\,{\mit\Delta} t\gtapp \hbar,
\end{equation}
where ${\mit\Delta}E$ and ${\mit\Delta} t$ are the uncertainties in
energy and time, respectively. The above expression is generally
known as the {\em energy-time uncertainty principle}.

For instance, suppose that a particle passes some fixed point on the $x$-axis.
Since the particle is, in reality, an extended wave packet, it takes a certain amount
of time ${\mit\Delta}t$ for the particle to pass. Thus, there is an uncertainty,
${\mit\Delta}t$, in the arrival time of the particle. Moreover, since
$E=\hbar\,\omega$, the only wavefunctions which have unique energies
are those with unique frequencies: {\em i.e.}, plane waves. Since a
wave packet of finite extent is made up of a combination of plane waves
of different wavenumbers, and, hence, different frequencies, there will
be an uncertainty ${\mit\Delta}E$ in the particle's energy which is
proportional to the range of frequencies of the plane waves making up the
wave packet. The more compact the wave packet (and, hence, the
smaller ${\mit\Delta}t$), the larger the range of frequencies of the constituent plane waves (and, hence, the large ${\mit\Delta}E$), and
{\em vice versa}. To be more exact, if $\psi(t)$ is the wavefunction
measured at the fixed point as a function of time, then we can write
\begin{equation}
\psi(t)= \frac{1}{\sqrt{2\pi\,\hbar}}\int_{-\infty}^{\infty}\chi(E)\,{\rm e}^{-{\rm i}\,E\,t/\hbar}\,dE.
\end{equation}
In other words, we can express $\psi(t)$ as a linear combination of
plane waves of definite energy $E$. Here, $\chi(E)$ is the complex
amplitude of plane waves of energy $E$ in this combination. By Fourier's
theorem, we also have
\begin{equation}
\chi(E) = \frac{1}{\sqrt{2\pi\,\hbar}}\int_{-\infty}^{\infty}\psi(t)\,{\rm e}^{+{\rm i}\,E\,t/\hbar}\,dt.
\end{equation}
For instance, if $\psi(t)$ is a Gaussian then it is easily shown that
$\chi(E)$ is also a Gaussian: {\em i.e.}, 
\begin{eqnarray}
\psi(t) &=& \frac{{\rm e}^{-{\rm i}\,E_0\,t/\hbar}}{(2\pi\,\sigma_t^{\,2})^{1/4}}\,{\rm e}^{-(t-t_0)^{\,2}/4\,\sigma_t^{\,2}},\\[0.5ex]
\chi(E) &=&\frac{{\rm e}^{+{\rm i}\,E\,t_0/\hbar}}{(2\pi\,\sigma_E^{\,2})^{1/4}}\,{\rm e}^{-(E-E_0)^{\,2}/4\,\sigma_E^{\,2}},
\end{eqnarray}
where $\sigma_E\,\sigma_t=\hbar/2$. As before, Gaussian wave packets
satisfy the minimum uncertainty principle $\sigma_E\,\sigma_t=\hbar/2$. Conversely, non-Gaussian wave packets
are characterized by $\sigma_E\,\sigma_t>\hbar/2$.

\section{Eigenstates and Eigenvalues}\label{seig}
Consider a general real-space operator $A(x)$. When this operator
acts on a general wavefunction $\psi(x)$ the result is usually a wavefunction
with a completely different shape. However, there are certain special
wavefunctions which are such that when $A$ acts on them the
result is just a multiple of the original wavefunction. These special
wavefunctions are called {\em eigenstates}, and the multiples
are called {\em eigenvalues}. Thus, if
\begin{equation}\label{e3.107}
A\,\psi_a(x) = a\,\psi_a(x),
\end{equation}
where $a$ is a complex number, then $\psi_a$ is called an eigenstate of
$A$ corresponding to the eigenvalue $a$.

Suppose that $A$ is an Hermitian operator corresponding to some physical dynamical
variable.
Consider a particle whose wavefunction is $\psi_a$. The expectation of
value $A$ in this state is simply [see Eq.~(\ref{e3.55})]
\begin{equation}
\langle A\rangle = \int_{-\infty}^\infty \psi_a^{\ast}\,A\,\psi_a\,dx
= a\,\int_{-\infty}^\infty \psi_a^{\ast}\,\psi_a\,dx =a,
\end{equation}
where use has been made of Eq.~(\ref{e3.107}) and the normalization
condition (\ref{e3.4}). Moreover, 
\begin{equation}
\langle A^2\rangle = \int_{-\infty}^\infty \psi_a^{\ast}\,A^2\,\psi_a\,dx
= a\,\int_{-\infty}^\infty \psi_a^{\ast}\,A\,\psi_a\,dx =a^2\,\int_{-\infty}^\infty \psi_a^{\ast}\,\psi_a\,dx =a^2,
\end{equation}
so
the variance of $A$ is [{\em cf.}, Eq.~(\ref{e3.24a})]
\begin{equation}
\sigma_A^{\,2} = \langle A^2\rangle - \langle A\rangle^2 = a^2-a^2 = 0.
\end{equation}
The fact that the variance is {\em zero}\/ implies that every measurement of $A$ is bound to
yield the same result: namely,  $a$.  Thus, the eigenstate $\psi_a$ is a state which is
associated with a {\em unique}\/ value of the dynamical variable corresponding to $A$. This
unique value is simply the associated eigenvalue.

It is easily demonstrated that the eigenvalues of an Hermitian operator
are all {\em real}. Recall [from Eq.~(\ref{e3.84})] that an
Hermitian operator satisfies
\begin{equation}\label{e3.111}
\int_{-\infty}^\infty \psi_1^\ast\,(A\,\psi_2)\,dx = \int_{-\infty}^\infty
(A\,\psi_1)^\ast\,\psi_2\,dx.
\end{equation}
Hence, if $\psi_1=\psi_2=\psi_a$ then
\begin{equation}
\int_{-\infty}^\infty \psi_a^\ast\,(A\,\psi_a)\,dx = \int_{-\infty}^\infty
(A\,\psi_a)^\ast\,\psi_a\,dx,
\end{equation}
which reduces to [see Eq.~(\ref{e3.107})]
\begin{equation}
a=a^\ast,
\end{equation}
assuming that $\psi_a$ is properly normalized.

Two wavefunctions, $\psi_1(x)$ and $\psi_2(x)$, are said to be {\em orthogonal}\/
if
\begin{equation}
\int_{-\infty}^{\infty}\psi_1^\ast\,\psi_2\,dx = 0.
\end{equation}
Consider two eigenstates of $A$, $\psi_a$ and $\psi_{a'}$, which
correspond to the two {\em different}\/ eigenvalues $a$ and $a'$, respectively. Thus,
\begin{eqnarray}
A\,\psi_a&=& a\,\psi_a,\\[0.5ex]
A\,\psi_{a'}&=& a'\,\psi_{a'}.
\end{eqnarray}
Multiplying the complex conjugate of the first equation by $\psi_{a'}$,
and the second equation by $\psi_a^\ast$, and then integrating over all
$x$, we obtain
\begin{eqnarray}
\int_{-\infty}^\infty (A\,\psi_a)^\ast\,\psi_{a'}\,dx&=& a\,\int_{-\infty}^\infty\psi_a^\ast\,\psi_{a'}\,dx,\\[0.5ex]
\int_{-\infty}^\infty \psi_a^\ast\,(A\,\psi_{a'})\,dx&=& a'\,\int_{-\infty}^{\infty}\psi_a^\ast\,\psi_{a'}\,dx.
\end{eqnarray}
However, from Eq.~(\ref{e3.111}), the left-hand sides of the above two
equations are equal. Hence, we can
write
\begin{equation}
(a-a')\, \int_{-\infty}^\infty\psi_a^\ast\,\psi_{a'}\,dx = 0.
\end{equation}
By assumption, $a\neq a'$, yielding
\begin{equation}
\int_{-\infty}^\infty\psi_a^\ast\,\psi_{a'}\,dx = 0.
\end{equation} 
In other words, eigenstates of an Hermitian operator corresponding to
{\em different}\/ eigenvalues are automatically {\em orthogonal}.

Consider two eigenstates of $A$, $\psi_a$ and $\psi_a'$, which
correspond to the {\em same}\/ eigenvalue, $a$. Such eigenstates
are termed {\em degenerate}. The above proof of the orthogonality
of different eigenstates fails for degenerate eigenstates. 
Note, however, that {\em any}\/ linear combination of
$\psi_a$ and $\psi_a'$ is also an eigenstate of $A$ corresponding
to the eigenvalue $a$. Thus, even if $\psi_a$ and $\psi_a'$ are not
orthogonal, we can always choose two linear combinations
of these eigenstates which are orthogonal. For instance,
if $\psi_a$ and $\psi_a'$ are properly normalized, and
\begin{equation}
\int_{-\infty}^\infty \psi_a^\ast\,\psi_a'\,dx = c,
\end{equation}
then it is easily demonstrated that
\begin{equation}
\psi_a'' = \frac{|c|}{\sqrt{1-|c|^2}}\left(\psi_a - c^{-1}\,\psi_a'\right)
\end{equation}
is a properly normalized eigenstate of $A$, corresponding to the
eigenvalue $a$, which is orthogonal to $\psi_a$. It is straightforward
to generalize the above argument to three or more degenerate eigenstates.
Hence, we conclude that the eigenstates of an Hermitian
operator are, or can be chosen to be, {\em mutually orthogonal}.

It is also possible to demonstrate that the eigenstates of an
Hermitian operator form a {\em complete set}: {\em i.e.}, that any
general wavefunction can be written as a linear combination
of these eigenstates. However, the proof is quite difficult, and
we shall not attempt it here. 

In summary, given an Hermitian
operator $A$, any general wavefunction, $\psi(x)$, can be written
\begin{equation}\label{e3.123}
\psi = \sum_{i}c_i\,\psi_i,
\end{equation}
where the $c_i$ are complex weights, and the $\psi_i$ are the properly
normalized (and mutually orthogonal) eigenstates of $A$: {\em i.e.}, 
\begin{equation}
A\,\psi_i = a_i\,\psi_i,
\end{equation}
where $a_i$ is the eigenvalue corresponding to the eigenstate $\psi_i$,
and
\begin{equation}\label{e3.125}
\int_{-\infty}^\infty \psi_i^\ast\,\psi_j \,dx = \delta_{ij}.
\end{equation}
Here, $\delta_{ij}$ is called the {\em Kronecker delta-function}, and
takes the value unity when its two indices are equal, and zero otherwise.

It follows from Eqs.~(\ref{e3.123}) and (\ref{e3.125})
that
\begin{equation}\label{e3.126}
c_i = \int_{-\infty}^\infty \psi_i^\ast\,\psi\,dx.
\end{equation}
Thus, the expansion coefficients in Eq.~(\ref{e3.123}) are easily determined,
given the wavefunction $\psi$ and the eigenstates $\psi_i$. 
Moreover, if $\psi$ is a properly normalized wavefunction then Eqs.~(\ref{e3.123}) and (\ref{e3.125})
yield
\begin{equation}\label{e3.127}
\sum_i |c_i|^2 =1.
\end{equation}

\section{Measurement}\label{smeas}
Suppose that $A$ is an Hermitian operator corresponding to some dynamical
variable. By analogy with the discussion  in Sect.~\ref{scoll},  we expect that if a measurement of $A$
yields the result $a$  then the act of measurement will cause the wavefunction to collapse to a state in which a measurement
of $A$ is {\em bound}\/ to give the result $a$. 
What sort of wavefunction, $\psi$, is such that a measurement of $A$ is
bound to yield a certain result, $a$? Well, expressing $\psi$ as
a linear combination of the eigenstates of $A$, we have
\begin{equation}\label{e4.128}
\psi = \sum_i c_i\,\psi_i,
\end{equation}
where $\psi_i$ is an eigenstate of $A$ corresponding to the eigenvalue $a_i$. If a measurement of $A$ is bound to yield the result $a$ then
\begin{equation}
\langle A\rangle= a,
\end{equation}
and
\begin{equation}\label{e4.130}
\sigma_A^{\,2} = \langle A^2\rangle - \langle A\rangle = 0.
\end{equation}
Now it is easily seen that
\begin{eqnarray}\label{e4.131}
\langle A\rangle &=& \sum_i |c_i|^2\,a_i,\\[0.5ex]
\langle A^2\rangle &=& \sum_i |c_i|^2\,a_i^{\,2}.
\end{eqnarray}
Thus, Eq.~(\ref{e4.130}) gives
\begin{equation}
 \sum_i a_i^{\,2}\,|c_i|^2 - \left(\sum_i a_i\,|c_i|^2\right)^2=0.
\end{equation}
Furthermore, the normalization condition yields
\begin{equation}\label{e4.134}
\sum_i |c_i|^2 = 1.
\end{equation}

For instance, suppose that there are only two eigenstates. The above two
equations then reduce to $|c_1|^2=x$, and $|c_2|^2=1-x$, where $0\leq x\leq 1$,
and
\begin{equation}\label{e4.126}
(a_1-a_2)^2\,x\,(1-x) = 0.
\end{equation}
The only solutions are $x=0$ and $x=1$. This result can easily
be generalized to the case where there are more than two eigenstates.
It follows that a state associated with a definite value of $A$ is
one in which one of the $|c_i|^2$ is unity, and all of the others are zero.
In other words, the only states associated with definite values of $A$
are the {\em eigenstates} of $A$. It immediately
follows that {\em the result of a measurement of $A$ must be one of the eigenvalues of $A$}. Moreover, if a general wavefunction is expanded
as a linear combination of the eigenstates of $A$, like in Eq.~(\ref{e4.128}),
then it is clear from Eq.~(\ref{e4.131}), and the general definition of a mean,
that the probability of a measurement of $A$ yielding the eigenvalue $a_i$
is simply $|c_i|^2$, where $c_i$ is the coefficient in front of
the $i$th eigenstate in the expansion. Note, from Eq.~(\ref{e4.134}),
that these probabilities are properly normalized: {\em i.e.}, the probability
of a measurement of $A$ resulting in  any possible answer is unity.
Finally, if a measurement of $A$ results in the eigenvalue $a_i$ then
immediately after the measurement the system will be left in the
eigenstate corresponding to $a_i$.

Consider two physical dynamical variables represented by the two
Hermitian operators $A$ and $B$. Under what circumstances is
it possible to simultaneously measure these two variables (exactly)? 
Well, the possible results of measurements of $A$ and $B$ are the eigenvalues
of $A$ and $B$, respectively. Thus, to simultaneously measure $A$ and $B$ (exactly) there
must exist states which are {\em simultaneous eigenstates}\/ of $A$ and $B$. 
In fact,  in order for $A$ and $B$ to be simultaneously measurable under all
circumstances, we need {\em all}\/ of the eigenstates of $A$ to also be eigenstates of $B$, and {\em vice versa}, so that all states associated with unique values of $A$ are
also associated with unique values of $B$, and {\em vice versa}.

Now, we have already seen, in Sect.~\ref{suncert}, that if $A$ and $B$
do not commute ({\em i.e.}, if $A\,B\neq B\,A$) then they cannot
be simultaneously measured. This suggests that the condition
for simultaneous measurement is that $A$ and $B$ should commute.
 Suppose that this is the case, and  that the $\psi_i$ and $a_i$ are the normalized eigenstates and
eigenvalues of $A$, respectively. It follows that
\begin{equation}
(A\,B-B\,A)\,\psi_i = (A\,B-B\,a_i)\,\psi_i = (A-a_i)\,B\,\psi_i = 0,
\end{equation}
or
\begin{equation}
A\,(B\,\psi_i) = a_i\,(B\,\psi_i).
\end{equation}
Thus, $B\,\psi_i$ is an eigenstate of $A$ corresponding to
the eigenvalue $a_i$ (though not necessarily a normalized one). In other words, $B\,\psi_i\propto \psi_i$, or
\begin{equation}
B\,\psi_i = b_i\,\psi_i,
\end{equation}
where $b_i$ is a constant of proportionality.
Hence, $\psi_i$  is an eigenstate of $B$, and, thus,  a simultaneous
eigenstate of $A$ and $B$. We conclude that if $A$ and $B$ commute then
they possess simultaneous eigenstates, and are thus simultaneously measurable (exactly). 

\section{Continuous Eigenvalues}
In the previous two sections, it was tacitly assumed that we were dealing
with operators possessing {\em discrete}\/ eigenvalues and square-integrable
eigenstates. Unfortunately, some operators---most notably,  $x$ and 
$p$---possess
eigenvalues which lie in a {\em continuous range}  and non-square-integrable
eigenstates (in fact, these two properties go hand in hand). Let us, therefore, investigate
the eigenstates and eigenvalues of the displacement and momentum
operators.

Let $\psi_x(x,x')$ be the eigenstate of $x$ corresponding to the eigenvalue $x'$. It follows that
\begin{equation}
x\,\psi_x(x,x') = x'\,\psi_x(x,x')
\end{equation}
for all $x$. Consider the Dirac delta-function $\delta(x-x')$. We can write
\begin{equation}
x\,\delta(x-x') = x'\,\delta(x-x'),
\end{equation}
since $\delta(x-x')$ is only non-zero infinitesimally close to $x=x'$. 
Evidently, $\psi_x(x,x')$ is proportional to $\delta(x-x')$. Let us make the
constant of proportionality unity, so that
\begin{equation}
\psi_x(x,x') = \delta(x-x').
\end{equation}
Now, it is easily demonstrated that
\begin{equation}
\int_{-\infty}^{\infty} \delta(x-x')\,\delta(x-x'')\,dx = \delta(x'-x'').
\end{equation}
Hence, $\psi_x(x,x')$ satisfies the orthonormality condition
\begin{equation}\label{e4.143}
\int_{-\infty}^\infty \psi_x^\ast(x,x')\,\psi_x(x,x'')\,dx = \delta(x'-x'').
\end{equation}
This condition is analogous to the orthonormality condition (\ref{e3.125})
satisfied by  square-integrable  eigenstates.
Now, by definition, $\delta(x-x')$ satisfies
\begin{equation}
\int_{-\infty}^\infty f(x)\,\delta(x-x')\,dx = f(x'),
\end{equation}
where $f(x)$ is a general function. We can thus write
\begin{equation}\label{e4.144}
\psi(x) = \int_{-\infty}^\infty c(x')\,\psi_x(x,x')\,dx',
\end{equation}
where $c(x')=\psi(x')$, or
\begin{equation}\label{e4.145}
c(x') = \int_{-\infty}^\infty \psi_x^\ast(x,x')\,\psi(x)\,dx.
\end{equation}
In other words, we can expand a general wavefunction $\psi(x)$
as a linear combination of the eigenstates, $\psi_x(x,x')$, of the
displacement operator. Equations (\ref{e4.144}) and (\ref{e4.145})
are analogous to Eqs.~(\ref{e3.123}) and (\ref{e3.126}), respectively, 
for square-integrable eigenstates. Finally, by analogy with the
results in Sect.~\ref{seig}, the probability density of a measurement of $x$
yielding the value $x'$ is $|c(x')|^{\,2}$, which is equivalent to the
standard result
$|\psi(x')|^{\,2}$. Moreover, these probabilities are properly normalized
provided $\psi(x)$ is properly normalized [{\em cf.}, Eq.~(\ref{e3.127})]: {\em i.e.},
\begin{equation}
\int_{-\infty}^\infty |c(x')|^{\,2}\,dx'= \int_{-\infty}^\infty |\psi(x')|^{\,2}\,dx' =1.
\end{equation}
Finally, if a measurement of $x$ yields the value $x'$ then the system
is left in the corresponding displacement eigenstate, $\psi_x(x,x')$, immediately after the measurement: {\em i.e.}, the wavefunction
collapses to a ``spike-function'', $\delta(x-x')$, as discussed in Sect.~\ref{scoll}.

Now, an eigenstate of the momentum operator $p\equiv -{\rm i}\,\hbar\,\partial/\partial x$ corresponding to the eigenvalue $p'$ satisfies
\begin{equation}
-{\rm i}\,\hbar\,\frac{\partial \psi_p(x,p')}{\partial x} = p'\,\psi_p(x,p').
\end{equation}
It is evident that
\begin{equation}\label{e4.148}
\psi_p(x,p') \propto {\rm e}^{+{\rm i}\,p'\,x/\hbar}.
\end{equation}
Now, we require $\psi_p(x,p')$ to satisfy an analogous orthonormality condition
to Eq.~(\ref{e4.143}): {\em i.e.}, 
\begin{equation}
\int_{-\infty}^\infty \psi_p^\ast(x,p')\,\psi_p(x,p'')\,dx = \delta(p'-p'').
\end{equation}
Thus, it follows from Eq.~(\ref{e3.72}) that the constant of proportionality
in Eq.~(\ref{e4.148}) should be $(2\pi\,\hbar)^{-1/2}$: {\em i.e.}, 
\begin{equation}\label{e4.148a}
\psi_p(x,p') =\frac{ {\rm e}^{+{\rm i}\,p'\,x/\hbar}}{(2\pi\,\hbar)^{1/2}}.
\end{equation}
Furthermore, according to Eqs.~(\ref{e3.64}) and (\ref{e3.65}),
\begin{equation}\label{e4.152}
\psi(x) = \int_{-\infty}^\infty c(p')\,\psi_p(x,p')\,dp',
\end{equation}
where $c(p') = \phi(p')$ [see Eq.~(\ref{e3.65})], or
\begin{equation}\label{e4.153}
c(p') = \int_{-\infty}^\infty \psi_p^\ast(x,p')\,\psi(x)\,dx.
\end{equation}
In other words, we can expand a general wavefunction $\psi(x)$
as a linear combination of the eigenstates, $\psi_p(x,p')$, of the
momentum operator. Equations (\ref{e4.152}) and (\ref{e4.153})
are again analogous to Eqs.~(\ref{e3.123}) and (\ref{e3.126}), respectively, 
for square-integrable eigenstates. Likewise, the probability density
of a measurement of $p$ yielding the result $p'$ is $|c(p')|^{\,2}$, which is equivalent to the standard result $|\phi(p')|^{\,2}$. The probabilities are also properly
normalized provided $\psi(x)$ is properly normalized [{\em cf.}, Eq.~(\ref{e3.83})]:
{\em i.e.}, 
\begin{equation}
\int_{-\infty}^\infty |c(p')|^{\,2}\,dp'= \int_{-\infty}^{\infty}
|\phi(p')|^{\,2}\,dp' = \int_{-\infty}^\infty |\psi(x')|^{\,2}\,dx' =1.
\end{equation}
Finally, if a mesurement of $p$ yields the value $p'$ then the system
is left in the corresponding momentum eigenstate, $\psi_p(x,p')$, immediately after the measurement.

\section{Stationary States}\label{sstat}
An eigenstate of the energy operator $H\equiv {\rm i}\,\hbar\,\partial/\partial t$
corresponding to the eigenvalue $E_i$ satisfies
\begin{equation}
{\rm i}\,\hbar\,\frac{\partial \psi_E(x,t,E_i)}{\partial t} = E_i\,\psi_E(x,t,E_i).
\end{equation}
It is evident that this equation can be solved by writing
\begin{equation}
\psi_E(x,t,E_i) = \psi_i(x)\,{\rm e}^{-{\rm i}\,E_i\,t/\hbar},
\end{equation}
where $\psi_i(x)$ is a properly normalized stationary ({\em i.e.}, non-time-varying) wavefunction. The wavefunction $\psi_E(x,t,E_i)$  corresponds to a so-called  {\em stationary state}, since
the probability density $|\psi_E|^{\,2}$ is non-time-varying. Note that
a stationary state is associated with a {\em unique}\/ value for the energy.
Substitution of the above expression into Schr\"{o}dinger's equation (\ref{e3.1}) yields the equation satisfied by the
stationary wavefunction:
\begin{equation}\label{etimeii}
\frac{\hbar^2}{2\,m}\,\frac{d^2 \psi_i}{d x^2} = 
\left[V(x)-E_i\right]\psi_i.
\end{equation}
This is known as the {\em time-independent Schr\"{o}dinger equation}. 
More generally, this equation takes the form
\begin{equation}\label{etimei}
H\,\psi_i = E_i\,\psi_i,
\end{equation}
where $H$ is assumed not to be an explicit function of $t$.
Of course, the $\psi_i$ satisfy the usual orthonormality condition:
\begin{equation}\label{e4.157}
\int_{-\infty}^\infty \psi_i^\ast\,\psi_j\,dx = \delta_{ij}.
\end{equation}
Moreover, we can express a general wavefunction as a linear combination
of energy eigenstates:
\begin{equation}\label{e4.158}
\psi(x,t) = \sum_i c_i\,\psi_i(x)\,{\rm e}^{-{\rm i}\,E_i\,t/\hbar},
\end{equation}
where
\begin{equation}
c_i = \int_{-\infty}^{\infty} \psi_i^\ast(x)\,\psi(x,0)\,dx.
\end{equation}
Here, $|c_i|^{\,2}$ is the probability that a measurement of the energy will
yield the eigenvalue $E_i$. Furthermore, immediately after such a measurement, the system is left in the corresponding energy eigenstate.
The generalization of the above results to the case where $H$ has continuous
eigenvalues is straightforward.

If a dynamical variable is represented by some Hermitian operator $A$ which
commutes with $H$ (so that it has simultaneous eigenstates with $H$), and
contains no specific time dependence, then it is evident from Eqs. (\ref{e4.157}) and (\ref{e4.158}) that the expectation value and
variance of $A$ are {\em time independent}. In this sense, the dynamical
variable in question is a constant of the motion.

\subsubsection*{Exercises}
{\small
\begin{enumerate}
\item Monochromatic light with a wavelength of $6000\,\AA$ passes
through a fast shutter that opens for $10^{-9}$ sec. What is the
subsequent spread in wavelengths of the no longer monochromatic light?


\item Calculate $\langle x\rangle$, $\langle x^2\rangle$, and $\sigma_x$, as
well as $\langle p\rangle$, $\langle p^2\rangle$, and $\sigma_p$,
for the normalized wavefunction
$$
\psi(x) = \sqrt{\frac{2\,a^3}{\pi}}\,\frac{1}{x^2+a^2}.
$$
Use these to find $\sigma_x\,\sigma_p$. Note that $\int_{-\infty}^{\infty} dx/(x^2+a^2) = \pi/a$.


\item Classically, if a particle is not observed then the probability of finding it
in a one-dimensional box of length $L$, which extends from $x=0$ to $x=L$, is a constant $1/L$ per unit length.
Show that the classical expectation value of $x$ is $L/2$, the expectation value of 
$x^2$ is $L^2/3$, and the standard deviation of $x$ is $L/\sqrt{12}$. 


\item Demonstrate that if a particle in a one-dimensional stationary state is bound then the
expectation value of its momentum must be zero. 

\item Suppose that $V(x)$ is complex. Obtain an expression for $\partial P(x,t)/\partial t$ and $d/dt \int P(x,t)\,dx$ from Schr\"{o}dinger's equation.  What does this tell us about a complex $V(x)$? 

\item $\psi_1(x)$ and $\psi_2(x)$ are normalized eigenfunctions corresponding to
the same eigenvalue. If
$$
\int_{-\infty}^\infty \psi_1^\ast\,\psi_2\,dx = c,
$$
where $c$ is real, find normalized linear combinations of $\psi_1$ and
$\psi_2$ which are orthogonal to (a) $\psi_1$, (b) $\psi_1+\psi_2$.


\item Demonstrate that $p=-{\rm i}\,\hbar\,\partial/\partial x$ is an Hermitian
operator. Find the Hermitian conjugate of $a = x + {\rm i}\,p$.
 
\item An  operator $A$, corresponding to a physical quantity $\alpha$, has
two normalized eigenfunctions $\psi_1(x)$ and $\psi_2(x)$, with eigenvalues
$a_1$ and $a_2$. An operator $B$, corresponding to another physical
quantity $\beta$, has normalized eigenfunctions $\phi_1(x)$ and $\phi_2(x)$,
with eigenvalues $b_1$ and $b_2$. The eigenfunctions are
related via
\begin{eqnarray}
\psi_1 &=& (2\,\phi_1+3\,\phi_2) \left/ \sqrt{13},\right.\nonumber\\[0.5ex]
\psi_2 &=& (3\,\phi_1-2\,\phi_2) \left/ \sqrt{13}.\right.\nonumber
\end{eqnarray}
$\alpha$ is measured and the value $a_1$ is obtained. If $\beta$ is then measured and then $\alpha$ again, show that the probability of obtaining
$a_1$ a second time is $97/169$.

\item Demonstrate that an  operator which commutes with the
Hamiltonian, and contains no explicit time dependence, has an expectation
value which is constant in time.

\item For a certain system, the operator corresponding to the physical
quantity $A$ does not commute with the Hamiltonian. It has
eigenvalues $a_1$ and $a_2$, corresponding to properly normalized eigenfunctions
\begin{eqnarray}
\phi_1 &=& (u_1+u_2)\left/\sqrt{2},\right.\nonumber\\[0.5ex]
\phi_2 &=& (u_1-u_2)\left/\sqrt{2},\right.\nonumber
\end{eqnarray}
where $u_1$ and $u_2$ are properly normalized eigenfunctions of the
Hamiltonian with eigenvalues $E_1$ and $E_2$. If the system is in the
state $\psi=\phi_1$ at time $t=0$, show that the expectation value of $A$
at time $t$ is
$$
\langle A\rangle = \left(\frac{a_1+a_2}{2}\right) + \left(\frac{a_1-a_2}{2}\right)\cos\left(\frac{[E_1-E_2]\,t}{\hbar}\right).
$$

\end{enumerate}
}