\chapter{Variational Methods}\label{s14} 
\section{Introduction}
We have seen, in Sect.~\ref{s10.4}, that we can solve Schr\"{o}dinger's equation {\em exactly}\/ to find the stationary eigenstates
of a hydrogen atom. Unfortunately, it is not possible to find exact
solutions of Schr\"{o}dinger's equation for atoms more complicated than hydrogen, or for molecules. In such systems, the best that we can do
is to find {\em approximate}\/ solutions. Most of the methods which have been developed for finding such solutions employ the so-called {\em variational
principle}\/ discussed below.

\section{Variational Principle}
Suppose that we wish to solve the time-independent Schr\"{o}dinger equation
\begin{equation}
H\,\psi = E\,\psi,
\end{equation}
where $H$ is a known (presumably complicated) time-independent Hamiltonian. Let $\psi$ be a {\em normalized}\/ trial solution to the above equation.
The variational principle states, quite simply, that the
ground-state energy, $E_0$, is always less than or equal to the expectation
value of $H$ calculated with the trial wavefunction: {\em i.e.},
\begin{equation}
E_0 \leq \langle\psi|H|\psi\rangle.
\end{equation}
Thus, by varying $\psi$ until the expectation value of $H$ is {\em minimized}, we can
obtain  an approximation to the wavefunction and energy of the ground-state.

Let  us prove the variational principle.
Suppose that the $\psi_n$ and the $E_n$ are the true eigenstates and eigenvalues
of $H$: {\em i.e.},
\begin{equation}\label{e14.3}
H\,\psi_n = E_n\,\psi_n.
\end{equation}
Furthermore, let
\begin{equation}\label{e14.4}
E_0 < E_1 < E_2 < \cdots,
\end{equation}
so that $\psi_0$ is the ground-state, $\psi_1$ the first excited state,
{\em etc}. The $\psi_n$ are assumed to be orthonormal:
{\em i.e.},
\begin{equation}\label{e14.5}
\langle \psi_n|\psi_m\rangle = \delta_{nm}.
\end{equation}
 If our trial wavefunction $\psi$ is properly normalized then
we can write
\begin{equation}
\psi = \sum_n c_n\,\psi_n,
\end{equation}
where
\begin{equation}\label{e14.7}
\sum_n |c_n|^{\,2} = 1.
\end{equation}
Now, the expectation value of $H$, calculated with $\psi$, takes the
form
\begin{eqnarray}
\langle\psi|H|\psi\rangle & = &\left.\left\langle \sum_n c_n\,\psi_n\right|
H\left|\sum_m\,c_m\,\psi_m\right\rangle\right. = \sum_{n,m} c_n^{\,\ast}\,c_m\,\langle \psi_n|H|\psi_m\rangle\nonumber\\[0.5ex]
&=& \sum_n\,c_n^{\,\ast}\,c_m\,E_m\,\langle \psi_n|\psi_m\rangle=
\sum_n E_n\,|c_n|^{\,2},
\end{eqnarray}
where use has been made of Eqs.~(\ref{e14.3}) and (\ref{e14.5}).
So, we can write
\begin{equation}
\langle \psi|H|\psi\rangle = |c_0|^{\,2}\,E_0 + \sum_{n>0} |c_n|^{\,2}\,E_n.
\end{equation}
However, Eq.~(\ref{e14.7}) can be rearranged to give
\begin{equation}
|c_0|^{\,2} = 1-\sum_{n>0}|c_n|^{\,2}.
\end{equation}
Combining the previous two equations, we obtain
\begin{equation}
\langle \psi|H|\psi\rangle = E_0 + \sum_{n>0} |c_n|^{\,2}\,(E_n-E_0).
\end{equation}
Now, the second term on the right-hand side of the above expression
is {\em positive definite}, since $E_n-E_0>0$  for all $n>0$ [see (\ref{e14.4})].
Hence, we obtain the desired result
\begin{equation}
\langle \psi|H|\psi\rangle \geq E_0.
\end{equation}

Suppose that we have found a good approximation, $\tilde{\psi}_0$, to the ground-state
wavefunction. If $\psi$ is a normalized trial wavefunction which is 
orthogonal to $\tilde{\psi}_0$ ({\em i.e.}, $\langle \psi|\tilde{\psi}_0\rangle=0$)
then, by repeating the above analysis, we can easily demonstrate that
\begin{equation}
\langle \psi |H|\psi\rangle \geq E_1.
\end{equation}
Thus, by varying $\psi$ until the expectation value of $H$ is {\em minimized}, we can
obtain  an approximation to the wavefunction and energy of the first excited state. Obviously, we can continue this process until we have approximations
to all of the stationary eigenstates. Note, however, that  the errors are clearly cumulative in this method, 
so that any approximations to  highly excited states are unlikely to be very accurate. For this reason, the variational method is generally only
used to calculate the ground-state and first few excited states of
complicated quantum systems.

\section{Helium Atom}
A helium atom consists of a nucleus of charge $+2\,e$ surrounded
by two electrons. Let us attempt to calculate its ground-state energy.

Let the nucleus lie at the origin of our coordinate
system, and let the position vectors of the two electrons be ${\bf r}_1$
and ${\bf r}_2$, respectively. The Hamiltonian of the system thus
takes the form
\begin{equation}\label{e14.14}
H = -\frac{\hbar^2}{2\,m_e}\left(\nabla_1^{\,2} + \nabla_2^{\,2}\right)
- \frac{e^2}{4\pi\,\epsilon_0}\left(\frac{2}{r_1}+\frac{2}{r_2}-
\frac{1}{|{\bf r_2}-{\bf r_1}|}\right),
\end{equation}
where we have neglected any reduced mass effects.
The terms in the above expression represent the kinetic energy of the first
electron, the kinetic energy of the second electron, the electrostatic
attraction between the nucleus and the first electron, the electrostatic
attraction between the nucleus and the second electron, and the
electrostatic repulsion between the two electrons, respectively.
It is the final term which causes all of the difficulties. Indeed, if this
term is neglected then we can write
\begin{equation}
H = H_1 + H_2,
\end{equation}
where
\begin{equation}
H_{1,2} = -\frac{\hbar^2}{2\,m_e}\,\nabla^{\,2}_{1,2} -\frac{2\,e^2}{4\pi\,\epsilon_0\,r_{1,2}}.
\end{equation}
In other words, the Hamiltonian  just becomes the sum of separate Hamiltonians for each electron. In this case, we would expect the
wavefunction to  be separable: {\em i.e.},
\begin{equation}
\psi({\bf r}_1,{\bf r}_2) = \psi_1({\bf r}_1)\,\psi_2({\bf r}_2). 
\end{equation}
Hence, Schr\"{o}dinger's equation
\begin{equation}
H\,\psi = E\,\psi
\end{equation}
reduces to
\begin{equation}\label{e14.19}
H_{1,2}\,\psi_{1,2} = E_{1,2}\,\psi_{1,2},
\end{equation}
where
\begin{equation}
E = E_1 + E_2.
\end{equation}
Of course, Eq.~(\ref{e14.19}) is the Schr\"{o}dinger equation of a hydrogen atom whose
nuclear charge is $+2\,e$, instead of $+e$. It follows, from Sect.~\ref{s10.4} (making the substitution $e^2\rightarrow 2\,e^2$), that if both electrons are in their lowest energy
states then
\begin{eqnarray}
\psi_1({\bf r}_1) &=& \psi_0({\bf r}_1),\\[0.5ex]
\psi_2({\bf r}_2)&=& \psi_0({\bf r}_2),
\end{eqnarray}
where
\begin{equation}
\psi_0({\bf r}) = \frac{4}{\sqrt{2\,\pi}\,a_0^{\,3/2}}\,\exp\left(-\frac{2\,r}{a_0}\right).
\end{equation}
Here, $a_0$ is the Bohr radius [see Eq.~(\ref{e9.57})]. Note that $\psi_0$ is properly normalized. Furthermore,
\begin{equation}
E_1=E_2 = 4\,E_0,
\end{equation}
where $E_0=-13.6\,{\rm eV}$ is the hydrogen ground-state
energy [see Eq.~(\ref{e9.56})]. Thus, our crude estimate
for the ground-state energy of helium becomes
\begin{equation}
E = 4\,E_0 + 4\,E_0 = 8\,E_0 = -108.8\,{\rm eV}.
\end{equation}
Unfortunately, this estimate is significantly different from  the experimentally
determined value, which is $-78.98\,{\rm eV}$. This fact
demonstrates that the neglected electron-electron repulsion term makes a
large contribution to the helium ground-state energy.
Fortunately, however, we can  use the variational principle to estimate this contribution.

Let us employ the separable wavefunction discussed above as our trial
solution. Thus,
\begin{equation}\label{e14.26}
\psi({\bf r}_1, {\bf r}_2) = \psi_0({\bf r}_1)\,\psi_0({\bf r_2}) = 
\frac{8}{\pi\,a_0^{\,3}}\,\exp\left(- \frac{2\,[r_1+r_2]}{a_0}\right).
\end{equation}
The expectation value of the Hamiltonian (\ref{e14.14}) thus becomes
\begin{equation}\label{e14.27}
\langle H\rangle = 8\,E_0 + \langle V_{ee}\rangle,
\end{equation}
where
\begin{equation}\label{e14.28}
\langle V_{ee}\rangle = \left\langle \psi\left|\frac{e^2}{4\pi\,\epsilon_0\,|{\bf r}_2-{\bf r}_1|}\right|\psi\right\rangle= \frac{e^2}{4\pi\,\epsilon_0}
\int \frac{|\psi({\bf r}_1, {\bf r}_2)|^{\,2}}{|{\bf r}_2- {\bf r}_1|}\,d^3{\bf r}_1\,d^3{\bf r}_2.
\end{equation}
The variation principle only guarantees that (\ref{e14.27}) yields an
{\em upper bound}\/ on the ground-state energy. In reality, we hope
that it will give a reasonably accurate estimate of this energy.

It follows from Eqs.~(\ref{e9.56}), (\ref{e14.26}) and (\ref{e14.28}) that
\begin{equation}
\langle V_{ee}\rangle = -\frac{4\,E_0}{\pi^2}\,\int
\frac{{\rm e}^{-2\,(\hat{r}_1+ \hat{r}_2)}}{|\hat{\bf r}_1-\hat{\bf r}_2|}\,d^3\hat{\bf r}_1\,d^3\hat{\bf r}_2,
\end{equation}
where $\hat{\bf r}_{1,2} = 2\, {\bf r}_{1,2}/a_0$. Neglecting the hats, for the sake of clarity, the above
expression can also be written
\begin{equation}
\langle V_{ee}\rangle = -\frac{4\,E_0}{\pi^2}\,\int
\frac{{\rm e}^{-2\,(r_1+ r_2)}}{\sqrt{r_1^{\,2}+r_2^{\,2}-2\,r_1\,r_2\,\cos\theta}}\,d^3{\bf r}_1\,d^3{\bf r}_2,
\end{equation}
where $\theta$ is the angle subtended between vectors ${\bf r}_1$ and ${\bf r}_2$.
If we perform the integral in ${\bf r}_1$ space before that in ${\bf r}_2$
space then
\begin{equation}\label{e14.31}
\langle V_{ee}\rangle =  -\frac{4\,E_0}{\pi^2}\,\int {\rm e}^{-2\,r_2}\,I({\bf r}_2)\,d^3{\bf r}_2,
\end{equation}
where
\begin{equation}
I({\bf r}_2) = \int \frac{{\rm e}^{-2\,r_1}}{\sqrt{r_1^{\,2}+r_2^{\,2}-2\,r_1\,r_2\,\cos\theta}}\,d^3{\bf r}_1.
\end{equation}

Our first task is to evaluate the function $I({\bf r}_2)$. Let
$(r_1,\,\theta_1,\,\phi_1)$ be a set of spherical polar coordinates in ${\bf r}_1$
space whose axis of symmetry runs in the direction of ${\bf r}_2$. It follows
that $\theta=\theta_1$. Hence,
\begin{equation}
I({\bf r}_2) = \int_0^\infty\int_0^\pi\int_0^{2\pi}
\frac{{\rm e}^{-2\,r_1}}{\sqrt{r_1^{\,2}+r_2^{\,2}-2\,r_1\,r_2\,\cos\theta_1}}\,
r_1^{\,2}\,dr_1\,\sin\theta_1\,d\theta_1\,d\phi_1,
\end{equation}
which trivially reduces to 
\begin{equation}
I({\bf r}_2) = 2\pi\int_0^\infty\int_0^\pi
\frac{{\rm e}^{-2\,r_1}}{\sqrt{r_1^{\,2}+r_2^{\,2}-2\,r_1\,r_2\,\cos\theta_1}}\,
r_1^{\,2}\,dr_1\,\sin\theta_1\,d\theta_1.
\end{equation}
Making the substitution $\mu=\cos\theta_1$, we can see that
\begin{equation}
\int_0^\pi\frac{1}{\sqrt{r_1^{\,2}+r_2^{\,2}-2\,r_1\,r_2\,\cos\theta_1}}\,
\sin\theta_1\,d\theta_1 = 
\int_{-1}^1 \frac{d\mu}{\sqrt{r_1^{\,2}+r_2^{\,2}-2\,r_1\,r_2\,\mu}}.
\end{equation}
Now,
\begin{eqnarray}
\int_{-1}^1 \frac{d\mu}{\sqrt{r_1^{\,2}+r_2^{\,2}-2\,r_1\,r_2\,\mu}}
&=& \left[\frac{\sqrt{r_1^{\,2}+r_2^{\,2}-2\,r_1\,r_2\,\mu}}{r_1\,r_2}\right]_{+1}^{-1}\nonumber\\[0.5ex]
&=& \frac{(r_1+r_2) - |r_1-r_2|}{r_1\,r_2}\nonumber\\[0.5ex]
&=&\left\{\begin{array}{lcl}2/r_1&\mbox{\hspace{1cm}}&\mbox{for
$r_1>r_2$}\\
2/r_2&&\mbox{for $r_1<r_2$}\end{array}\right.,
\end{eqnarray}
giving
\begin{equation}
I({\bf r}_2) = 4\pi\left(\frac{1}{r_2}\int_0^{r_2}
{\rm e}^{-2\,r_1}\,r_1^{\,2}\,dr_1 + \int_{r_2}^\infty
{\rm e}^{-2\,r_1}\,r_1\,dr_1\right).
\end{equation}
But,
\begin{eqnarray}
\int {\rm e}^{-\beta\,x}\,x\,dx &= &-\frac{{\rm e}^{-\beta\,x}}{\beta^2}\,(1+\beta\,x),\\[0.5ex]
\int{\rm e}^{-\beta\,x}\,x^2\,dx &=& - \frac{{\rm e}^{-\beta\,x}}{\beta^3}\,(2+2\,\beta\,x+\beta^2\,x^2),
\end{eqnarray}
yielding
\begin{equation}
I({\bf r}_2) = \frac{\pi}{r_2}\left[1-{\rm e}^{-2\,r_2}\,(1+r_2)\right].
\end{equation}

Since the function $I({\bf r}_2)$ only depends on the magnitude of ${\bf r}_2$,
the integral (\ref{e14.31}) reduces to
\begin{equation}
\langle V_{ee}\rangle = -\frac{16\,E_0}{\pi}\int_0^\infty
{\rm e}^{-2\,r_2}\,I(r_2)\,r_2^{\,2}\,dr_2,
\end{equation}
which yields
\begin{equation}
\langle V_{ee}\rangle = -16\,E_0\int_{0}^\infty
{\rm e}^{-2\,r_2}\left[1-{\rm e}^{-2\,r_2}\,(1+r_2)\right]r_2\,dr_2=
-\frac{5}{2}\,E_0.
\end{equation}
Hence, from (\ref{e14.27}), our estimate for the ground-state
energy of helium is
\begin{equation}\label{e14.43}
\langle H\rangle = 8\,E_0 - \frac{5}{2}\,E_0 = \frac{11}{2}\,E_0 = -74.8\,{\rm eV}.
\end{equation}
This is remarkably close to the correct result.

We can actually refine our estimate further. The trial wavefunction (\ref{e14.26}) essentially treats the two electrons as 
{\em non-interacting}\/ particles. In
reality, we would expect one electron to partially shield the nuclear
charge from the other, and {\em vice versa}. Hence, a better
trial wavefunction might be
\begin{equation}\label{e14.44}
\psi({\bf r}_1, {\bf r}_2) = 
\frac{Z^3}{\pi\,a_0^{\,3}}\,\exp\left(- \frac{Z\,[r_1+r_2]}{a_0}\right),
\end{equation}
where $Z<2$ is effective nuclear charge number seen by each
electron. Let us recalculate the ground-state energy of helium
as a function of $Z$, using the above trial wavefunction, and then
{\em minimize}\/ the result with respect to $Z$. According to
the variational principle, this should give us an even  better estimate
for the ground-state energy.

We can rewrite the expression (\ref{e14.14}) for the Hamiltonian
of the helium atom in the form
\begin{equation}
H = H_1(Z) + H_2(Z) + V_{ee} + U(Z),
\end{equation}
where
\begin{equation}
H_{1,2}(Z) = -\frac{\hbar^2}{2\,m_e}\,\nabla^{\,2}_{1,2} -\frac{Z\,e^2}{4\pi\,\epsilon_0\,r_{1,2}}
\end{equation}
is the Hamiltonian of a hydrogen atom with nuclear charge $+Z\,e$,
\begin{equation}
V_{ee} = \frac{e^2}{4\pi\,\epsilon_0}\,\frac{1}{|{\bf r}_2-{\bf r}_1|}
\end{equation}
is the electron-electron repulsion term, and
\begin{equation}
U(Z) = \frac{e^2}{4\pi\,\epsilon_0}\left(\frac{[Z-2]}{r_1} + \frac{[Z-2]}{r_2}\right).
\end{equation}
It follows that
\begin{equation}
\langle H\rangle (Z)= 2\,E_0(Z) + \langle V_{ee}\rangle(Z) + \langle U\rangle(Z),
\end{equation}
where $E_0(Z) = Z^2\,E_0$ is the ground-state energy of a hydrogen
atom with nuclear charge $+Z\,e$, $\langle V_{ee}\rangle(Z) = -(5\,Z/4)\,E_0$ is the value of the electron-electron repulsion term when
recalculated with the wavefunction (\ref{e14.44}) [actually, all we
need to do is to make the substitution $a_0\rightarrow (2/Z)\,a_0$], and
\begin{equation}
\langle U\rangle(Z) = 2\,(Z-2)\left(\frac{e^2}{4\pi\,\epsilon_0}\right)\left\langle\frac{1}{r}\right\rangle.
\end{equation}
Here, $\langle 1/r\rangle$ is the expectation value of $1/r$ calculated
for a hydrogen atom with nuclear charge $+Z\,e$. It follows from
Eq.~(\ref{e9.74}) [with $n=1$, and making the substitution $a_0\rightarrow a_0/Z$] that
\begin{equation}
\left\langle \frac{1}{r}\right\rangle = \frac{Z}{a_0}.
\end{equation}
Hence,
\begin{equation}
\langle U\rangle(Z) = -4\,Z\,(Z-2)\,E_0,
\end{equation}
since $E_0=-e^2/(8\pi\,\epsilon_0\,a_0)$. 
Collecting the various terms, our new expression for the expectation
value of the Hamiltonian becomes
\begin{equation}
\langle H\rangle(Z) = \left[2\,Z^2 - \frac{5}{4}\,Z - 4\,Z\,(Z-2)\right] E_0
= \left[-2\,Z^2+ \frac{27}{4}\,Z\right] E_0.
\end{equation}
The value of $Z$ which minimizes this expression is the root of
\begin{equation}
\frac{d\langle H\rangle}{dZ} = \left[-4\,Z+ \frac{27}{4}\right] E_0 = 0.
\end{equation}
It follows that
\begin{equation}
Z = \frac{27}{16} = 1.69.
\end{equation}
The fact that $Z<2$ confirms our earlier conjecture that the electrons partially
shield the nuclear charge from one another. Our new estimate
for the ground-state energy of helium is
\begin{equation}
\langle H\rangle(1.69) = \frac{1}{2}\left(\frac{3}{2}\right)^6 E_0 = -77.5\,{\rm eV}.
\end{equation}
This is clearly an improvement on our previous estimate (\ref{e14.43}) [recall that the
correct result is $-78.98$ eV].

Obviously, we could get even closer to the correct value of the
helium ground-state energy by using a
more complicated trial wavefunction with more adjustable parameters.

Note, finally, that since the two electrons in a helium atom are {\em indistinguishable fermions}, the overall wavefunction must be {\em anti-symmetric}\/ with respect to exchange of particles (see Sect.~\ref{smany}).
Now, the overall wavefunction is the product of the {\em spatial wavefunction}\/
and the {\em spinor}\/ representing the spin-state. Our spatial wavefunction (\ref{e14.44}) is obviously {\em symmetric}\/ with respect to exchange of
particles. This means that the spinor must be {\em anti-symmetric}. 
It is clear, from Sect.~\ref{shalf},  that if the spin-state of
an $l=0$ system consisting of two spin one-half particles ({\em i.e.}, two electrons)
is {\em anti-symmetric}\/ with respect to interchange of particles then the system is
in the so-called {\em singlet}\/ state with overall spin zero. Hence,
the ground-state of helium has overall electron spin zero.

\section{Hydrogen Molecule Ion}
The hydrogen molecule ion consists of an electron orbiting about
two protons, and is the simplest imaginable molecule. Let us
investigate whether or not this molecule possesses a bound state: {\em i.e.}, whether or
not it possesses a ground-state whose energy is less than that of
a hydrogen atom and a free proton.
According
to the variation principle, we can deduce that the $H_2^+$ ion has a bound state if we can find {\em any}\/
trial wavefunction for which the total Hamiltonian of the system has an expectation value less than that of a hydrogen atom and a free proton.

\begin{figure}
\epsfysize=3.5in
\centerline{\epsffile{Chapter14/fig01.eps}}
\caption{\em The hydrogen molecule ion.}\label{fh2p}   
\end{figure}

Suppose that the two protons are separated by a distance $R$. In fact, let them lie on the $z$-axis, with the first at the origin, and
the second at $z=R$ (see Fig.~\ref{fh2p}). In the following, we shall treat the
protons as essentially stationary. This is reasonable, since the electron
moves far more rapidly than the protons.

Let us try
\begin{equation}\label{e14.57}
\psi({\bf r})_\pm = A\left[\psi_0({\bf r}_1) \pm \psi_0({\bf r}_2)\right]
\end{equation}
as our trial wavefunction, where
\begin{equation}
\psi_0({\bf r}) = \frac{1}{\sqrt{\pi}\,a_0^{\,3/2}}\,{\rm e}^{-r/a_0}
\end{equation}
is a normalized hydrogen ground-state wavefunction centered on the origin, and ${\bf r}_{1,2}$ are
 the position vectors of the electron with respect to each of the protons
(see Fig.~\ref{fh2p}). Obviously, this is a very simplistic wavefunction,
since it is just  a linear combination of hydrogen ground-state
wavefunctions centered on each proton. Note, however, that the wavefunction  respects
the obvious symmetries in the problem.

Our first task is to normalize our trial wavefunction. We require that
\begin{equation}
\int |\psi_\pm|^2\,d^3{\bf r} = 1.
\end{equation}
Hence, from (\ref{e14.57}),
$A = I^{-1/2}$, where
\begin{equation}
I = \int\left[|\psi_0({\bf r}_1)|^2 + |\psi_0({\bf r}_2)|^2 \pm
2\,\psi_0({\bf r}_1)\,\psi({\bf r}_2)\right] d^3{\bf r}.
\end{equation}
It follows that
\begin{equation}
I = 2\,(1\pm J),
\end{equation}
with
\begin{equation}
J = \int \psi_0({\bf r}_1)\,\psi_0({\bf r}_2)\,d^3{\bf r}.
\end{equation}

Let us employ the standard spherical polar coordinates ($r$, $\theta$, $\phi$). 
Now, it is easily seen that $r_1=r$ and $r_2=(r^2+R^2-2\,r\,R\,\cos\theta)^{1/2}$. Hence,
\begin{equation}
J = 2\int_0^\infty \int_0^\pi \exp\left[-x-(x^2+X^2-2\,x\,X\,\cos\theta)^{1/2}\right]\,x^2\,dx\,\sin\theta\,d\theta,
\end{equation}
where $X=R/a_0$. Here, we have already performed the trivial $\phi$ integral.
Let $y=(x^2+X^2-2\,x\,X\,\cos\theta)^{1/2}$. It follows that
$d(y^2)=2\,y\,dy = 2\,x\,X\,\sin\theta\,d\theta$, giving
\begin{eqnarray}
\int_0^\pi {\rm e}^{\,(x^2+X^2-2\,x\,X\,\cos\theta)^{1/2}}\,\sin\theta\,d\theta &=& \frac{1}{x\,X}\int_{|x-X|}^{x+X}
{\rm e}^{-y}\,y\,dy\\[0.5ex]
&=& - \frac{1}{x\,X}\left[{\rm e}^{-(x+X)}\,(1+x+X) - {\rm e}^{-|x-X|}\,(1+|x-X|)\right].\nonumber
\end{eqnarray}
Thus,
\begin{eqnarray}
J &=& - \frac{2}{X}\,{\rm e}^{-X}\int_0^X \left[{\rm e}^{-2\,x}\,(1+X+x)-
(1+X-x)\right]x\,dx\nonumber\\[0.5ex]
&&-\frac{2}{X}\int_X^\infty {\rm e}^{-2\,x}\left[{\rm e}^{-X}\,(1+X+x)-
{\rm e}^X\,(1-X+x)\right] x\,dx,
\end{eqnarray}
which evaluates to
\begin{equation}\label{e14.66}
J = {\rm e}^{-X}\left(1+X+\frac{X^3}{3}\right).
\end{equation}

Now, the Hamiltonian of the electron is written
\begin{equation}
H = -\frac{\hbar^2}{2\,m_e}\,\nabla^2 - \frac{e^2}{4\pi\,\epsilon_0}\left(\frac{1}{r_1}+\frac{1}{r_2}\right).
\end{equation}
Note, however, that
\begin{equation}
\left(-\frac{\hbar^2}{2\,m_e}\,\nabla^2 - \frac{e^2}{4\pi\,\epsilon_0\,r_{1,2}}\right)\psi_0({\bf r}_{1,2}) = E_0\,\psi_0({\bf r}_{1,2}),
\end{equation}
since $\psi_0({\bf r}_{1,2})$ are hydrogen ground-state wavefunctions.
It follows that
\begin{eqnarray}
H\,\psi_\pm &=& A\left[-\frac{\hbar^2}{2\,m_e}\,\nabla^2 - \frac{e^2}{4\pi\,\epsilon_0}\left(\frac{1}{r_1}+\frac{1}{r_2}\right)\right]
\left[\psi_0({\bf r}_1) \pm \psi_0({\bf r}_2)\right]\nonumber\\[0.5ex]
&=& E_0\,\psi - A\,\left(\frac{e^2}{4\pi\,\epsilon_0}\right)\left[
\frac{\psi_0({\bf r}_1)}{r_2}\pm \frac{\psi_0({\bf r}_2)}{r_1}\right].
\end{eqnarray}
Hence,
\begin{equation}
\langle H\rangle = E_0 + 4\,A^2\,(D\pm E)\,E_0,
\end{equation}
where
\begin{eqnarray}
D&=& \left\langle \psi_0({\bf r}_1)\left|\frac{a_0}{r_2}\right|\psi_0({\bf r}_1)\right\rangle,\\[0.5ex]
E&=& \left\langle \psi_0({\bf r}_1)\left|\frac{a_0}{r_1}\right|\psi_0({\bf r}_2)\right\rangle.
\end{eqnarray}

Now,
\begin{equation}
D = 2\int_0^\infty\int_0^\pi \frac{{\rm e}^{-2\,x}}{(x^2+X^2-2\,x\,X\,\cos\theta)^{1/2}}\,x^2\,dx\,\sin\theta\,d\theta,
\end{equation}
which reduces to
\begin{equation}
D =\frac{4}{X}\,\int_0^X {\rm e}^{-2\,x}\,x^2\,dx
+ 4\int_X^\infty {\rm e}^{-2\,x}\,x\,dx,
\end{equation}
giving
\begin{equation}\label{e14.75}
D = \frac{1}{X} \left( 1-[1+X]\,{\rm e}^{-2\,X}\right).
\end{equation}
Furthermore, 
\begin{equation}
E = 2\int_0^\infty \int_0^\pi \exp\left[-x-(x^2+X^2-2\,x\,X\,\cos\theta)^{1/2}\right]\,x\,dx\,\sin\theta\,d\theta,
\end{equation}
which reduces to
\begin{eqnarray}
E&=& - \frac{2}{X}\,{\rm e}^{-X}\int_0^X \left[{\rm e}^{-2\,x}\,(1+X+x)-
(1+X-x)\right]dx\nonumber\\[0.5ex]
&&-\frac{2}{X}\int_X^\infty {\rm e}^{-2\,x}\left[{\rm e}^{-X}\,(1+X+x)-
{\rm e}^X\,(1-X+x)\right] dx,
\end{eqnarray}
yielding
\begin{equation}\label{e14.78}
E = (1+X)\,{\rm e}^{-X}.
\end{equation}

Our expression for the expectation value of the electron Hamiltonian is
\begin{equation}
\langle H\rangle = \left[1+ 2\,\frac{(D\pm E)}{(1\pm J)}\right] E_0,
\end{equation}
where $J$, $D$, and $E$ are specified as functions of $X=R/a_0$ in
Eqs.~(\ref{e14.66}), (\ref{e14.75}), and (\ref{e14.78}), respectively.
In order
to obtain the total energy of the molecule, we must add to this the
potential energy of the two protons. Thus,
\begin{equation}
E_{total} = \langle H\rangle + \frac{e^2}{4\pi\,\epsilon_0\,R} = \langle H\rangle - \frac{2}{X}\,E_0,
\end{equation}
since $E_0= -e^2/(8\pi\,\epsilon_0\,a_0)$.
Hence, we can write
\begin{equation}\label{e14.81}
E_{total} = - F_\pm(R/a_0)\,E_0,
\end{equation}
where $E_0$ is the hydrogen ground-state energy, and
\begin{equation}
F_\pm(X) = -1 + \frac{2}{X}\left[\frac{(1+X)\,{\rm e}^{-2\,X}\pm(1-2\,X^2/3)
\,{\rm e}^{-X}}{1\pm (1+X+X^2/3)\,{\rm e}^{-X}}\right].
\end{equation}
The functions $F_+(X)$ and $F_-(X)$ are both plotted in Fig.~\ref{fh2pa}.
Recall that in order for  the $H_2^+$ ion to be in a bound  state it must have a lower
energy than a hydrogen atom and a free proton: {\em i.e.}, $E_{total}< E_0$. It follows from Eq.~(\ref{e14.81}) that a bound state
corresponds to $F_\pm < -1$. Clearly, the {\em even}\/ trial wavefunction $\psi_+$
possesses a bound state, whereas the {\em odd}\/ trial wavefunction $\psi_-$
does not [see Eq.~(\ref{e14.57})]. This is hardly surprising, since the
even wavefunction {\em maximizes}\/ the electron probability density between 
the two protons, thereby reducing their mutual electrostatic repulsion. On the other hand, the odd
wavefunction does exactly the opposite. The {\em binding energy}\/ of the
$H_2^+$ ion is defined as the difference between its energy and that of a
hydrogen atom and a free proton: {\em i.e.},
\begin{equation}
E_{bind} = E_{total} - E_0 = - (F_+ +1)\,E_0. 
\end{equation}
According to the variational principle, the binding energy is less than or
equal to the {\em minimum}\/ binding energy which can be inferred from Fig.~\ref{fh2pa}. 
This minimum occurs when $X\simeq 2.5$ and $F_+\simeq -1.13$. 
Thus, our estimates for the separation between the two
protons, and the binding energy, for the $H_2^+$ ion
are $R = 2.5\,a_0 = 1.33\times 10^{-10}\,{\rm m}$ and
$E_{bind} = 0.13 \,E_0 = -1.77$ eV, respectively. The experimentally
determined values   are $R=1.06\times 10^{-10}$ m, and $E_{bind}=-2.8$
eV, respectively. Clearly, our estimates are not particularly accurate. However,
our calculation does establish, beyond any doubt, the existence of
a bound state of the $H_2^+$ ion, which is all that we set out to achieve.

\begin{figure}
\epsfysize=3.5in
\centerline{\epsffile{Chapter14/fig02.eps}}
\caption{\em The functions $F_+(X)$ (solid curve) and $F_-(X)$ (dashed curve).}\label{fh2pa}   
\end{figure}
