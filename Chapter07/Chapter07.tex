 \chapter{Three-Dimensional Quantum Mechanics}\label{sthree}
 \section{Introduction}
In this chapter, we shall extend our previous one-dimensional formulation
of non-relativistic quantum mechanics to produce a fully three-dimensional theory.

\section{Fundamental Concepts}\label{s7.2}
We have  seen that in one dimension the instantaneous state
of a single non-relativistic particle is fully specified by a complex wavefunction,
$\psi(x,t)$. The probability
of finding the particle at time $t$ between $x$ and $x+dx$ is
$P(x,t)\,dx$, where
\begin{equation}
P(x,t) = |\psi(x,t)|^2.
\end{equation}
Moreover, the wavefunction is
normalized such that
\begin{equation}
\int_{-\infty}^\infty |\psi(x,t)|^2\,dx = 1
\end{equation}
at all times.

In three dimensions, the instantaneous state of a single particle is also
fully specified by a complex wavefunction, $\psi(x,y,z,t)$.
By analogy with the one-dimensional case, the probability of finding
the particle at time $t$ between $x$ and $x+dx$, between $y$ and $y+dx$, and between
$z$ and $z+dz$, is $P(x,y,z,t)\,dx\,dy\,dz$, where
\begin{equation}
P(x,y,z,t) = |\psi(x,y,z,t)|^2.
\end{equation}
As usual, this interpretation of the wavefunction only makes sense if the
wavefunction is normalized such that
\begin{equation}
\int_{-\infty}^\infty\int_{-\infty}^\infty\int_{-\infty}^\infty |\psi(x,y,z,t)|^2\,dx\,dy\,dz = 1.
\end{equation}
This normalization constraint ensures that the probability of finding the particle anywhere is space is always unity.

In one dimension, we can write the probability conservation equation (see
Sect.~\ref{s4.5})
\begin{equation}\label{e6.5}
\frac{\partial|\psi|^2}{\partial t} + \frac{\partial j}{\partial x} = 0,
\end{equation}
where
\begin{equation}
j  = \frac{{\rm i}\,\hbar}{2\,m}\left(\psi\,\frac{\partial\psi^\ast}{\partial x} - \psi^\ast\,\frac{\partial\psi}{\partial x}\right)
\end{equation}
is the flux of probability along the $x$-axis. Integrating 
Eq.~(\ref{e6.5}) over all space, and making use of the fact that $\psi\rightarrow 0$
as $|x|\rightarrow\infty$ if $\psi$ is to be square-integrable, we obtain
\begin{equation}
\frac{d}{dt}\int_{-\infty}^{\infty} |\psi(x,t)|^2\,dx = 0.
\end{equation}
In other words, if the wavefunction  is initially normalized then it stays
normalized as time progresses. This is a necessary criterion for the viability of our basic
interpretation of $|\psi|^2$ as a probability density.

In three dimensions, by analogy with the one dimensional case, the probability conservation equation becomes
\begin{equation}\label{e6.8}
\frac{\partial|\psi|^2}{\partial t} + \frac{\partial j_x}{\partial x} + \frac{\partial j_y}{\partial y} + \frac{\partial j_z}{\partial z}= 0.
\end{equation}
Here,
\begin{equation}
j_{x} =\frac{{\rm i}\,\hbar}{2\,m}\left(\psi\,\frac{\partial\psi^\ast}{\partial x} - \psi^\ast\,\frac{\partial\psi}{\partial x}\right)
\end{equation}
is the flux of probability along the $x$-axis, and
\begin{equation}
j_{y} =\frac{{\rm i}\,\hbar}{2\,m}\left(\psi\,\frac{\partial\psi^\ast}{\partial y} - \psi^\ast\,\frac{\partial\psi}{\partial y}\right)
\end{equation}
 the flux of probability along the $y$-axis, {\em etc}.
Integrating 
Eq.~(\ref{e6.8}) over all space, and making use of the fact that $\psi\rightarrow 0$
as $|{\bf r}|\rightarrow\infty$ if $\psi$ is to be square-integrable, we obtain
\begin{equation}
\frac{d}{dt}\int_{-\infty}^{\infty}\int_{-\infty}^{\infty}\int_{-\infty}^{\infty} |\psi(x,y,z,t)|^2\,dx\,dy\,dz = 0.
\end{equation}
Thus, the normalization of the wavefunction is again preserved as time
progresses, as must be the case if $|\psi|^2$ is to be interpreted as a
probability density.

In one dimension, position is represented by the algebraic operator $x$,
whereas momentum is represented by the differential operator
$-{\rm i}\,\hbar\,\partial/\partial x$ (see Sect.~\ref{s4.6}). By analogy, in
three dimensions, the Cartesian coordinates $x$, $y$, and $z$
are represented by the algebraic operators $x$, $y$, and $z$,
respectively, whereas the three Cartesian components of momentum,
$p_x$, $p_y$, and $p_z$, have the following representations:
\begin{eqnarray}\label{e6.12}
p_x &\equiv & -{\rm i}\,\hbar\,\frac{\partial}{\partial x},\\[0.5ex]
p_y &\equiv & -{\rm i}\,\hbar\,\frac{\partial}{\partial y},\\[0.5ex]
p_z &\equiv & -{\rm i}\,\hbar\,\frac{\partial}{\partial z}.\label{e6.14}
\end{eqnarray}

Let $x_1 =x$, $x_2=y$,  $x_3=z$, and $p_1=p_x$, {\em etc.}
Since the $x_i$ are {\em independent}\/ variables ({\em i.e.},
$\partial x_i/\partial x_j=\delta_{ij}$), we conclude that the
various position and momentum operators satisfy the following commutation
relations:
\begin{eqnarray}\label{commxx}
[x_i,x_j] &=& 0,\\[0.5ex]
[p_i,p_j] &=& 0,\label{commpp}\\[0.5ex]
[x_i,p_j] &=& {\rm i}\,\hbar\,\delta_{ij}.\label{commxp}
\end{eqnarray}
Now, we know, from Sect.~\ref{smeas}, that two dynamical variables
can only be (exactly) measured {\em simultaneously}\/ if the operators which represent
them in quantum mechanics {\em commute}\/ with one another. Thus,
it is clear, from the above commutation relations, that the only restriction
on measurement in a system consisting of a single particle moving in
three dimensions is that it is impossible to
simultaneously measure a given position coordinate and the corresponding
component of momentum. Note, however, that it is perfectly possible to
simultaneously measure two different positions coordinates, or two
different components of the momentum. The commutation
relations (\ref{commxx})--(\ref{commxp}) again illustrate the
point that  quantum mechanical operators corresponding to different degrees of freedom of a
dynamical system (in this case, motion in different directions) tend to commute
with one another (see Sect.~\ref{sfuncon}).

In one dimension, the time evolution of the wavefunction is given
by [see Eq.~(\ref{etimed})]
\begin{equation}\label{e6.15}
{\rm i}\,\hbar\,\frac{\partial\psi}{\partial t} = H\,\psi,
\end{equation}
where 
$H$  is the  Hamiltonian. The same equation
governs the time evolution of the wavefunction in three dimensions.

Now, in one dimension, the Hamiltonian of a non-relativistic particle
of mass $m$ takes the form
\begin{equation}
H = \frac{p_x^{\,2}}{2\,m} + V(x,t),
\end{equation}
where $V(x)$ is the potential energy. In three dimensions, this expression
generalizes to
\begin{equation}
H = \frac{p_x^{\,2}+ p_y^{\,2}+p_z^{\,2}}{2\,m} + V(x,y,z,t).
\end{equation}
Hence, making use of Eqs.~(\ref{e6.12})--(\ref{e6.14}) and (\ref{e6.15}), the three-dimensional version of the time-dependent Schr\"{o}ndiger equation
becomes [see Eq.~(\ref{e3.1})]
\begin{equation}\label{esh3d}
{\rm i}\,\hbar\,\frac{\partial\psi}{\partial t} = - \frac{\hbar^2}{2\,m}\,
\nabla^2\psi + V\,\psi.
\end{equation}
Here, the differential operator
\begin{equation}
\nabla^2 \equiv \frac{\partial^2}{\partial x^2} + \frac{\partial^2}{\partial
y^2} + \frac{\partial^2}{\partial z^2}
\end{equation}
is known as the {\em Laplacian}. Incidentally, the probability conservation equation 
(\ref{e6.8}) is easily derivable from Eq.~(\ref{esh3d}).
An eigenstate of the Hamiltonian corresponding
to the eigenvalue $E$ satisfies
\begin{equation}
H\,\psi = E\,\psi.
\end{equation}
It follows from Eq.~(\ref{e6.15}) that (see Sect.~\ref{sstat})
\begin{equation}
\psi(x,y,z,t) = \psi(x,y,z)\,{\rm e}^{-{\rm i}\,E\,t/\hbar},
\end{equation}
where the stationary wavefunction $\psi(x,y,z)$ satisfies the
three-dimensional version of the time-independent Schr\"{o}ndiger equation
[see Eq.~(\ref{etimeii})]:
\begin{equation}
\nabla^2\psi = \frac{2\,m}{\hbar^2}\,(V-E)\,\psi,
\end{equation}
where $V$ is assumed not to depend explicitly on $t$.

\section{Particle in a Box}
Consider a particle of mass $m$ trapped inside a cubic box of dimension $a$ (see Sect.~\ref{s5.2}). The particle's stationary wavefunction, $\psi(x,y,z)$, satisfies
\begin{equation}\label{e6.21}
\left(\frac{\partial^2}{\partial x^2} + \frac{\partial^2}{\partial y^2} + \frac{\partial^2}{\partial z^2}\right)\psi = -\frac{2\,m}{\hbar^2}\,E\,\psi,
\end{equation}
where $E$ is the particle energy. The wavefunction satisfies the boundary
condition that it must be zero at the edges of the box.

Let us search for a separable solution to the above equation of the
form
\begin{equation}\label{e6.22}
\psi(x,y,z) = X(x)\,Y(y)\,Z(z).
\end{equation}
The factors of the wavefunction satisfy the boundary conditions
$X(0)=X(a)=0$, $Y(0)=Y(a)=0$, and $Z(0)=Z(a)= 0$.
Substituting (\ref{e6.22}) into Eq.~(\ref{e6.21}), and rearranging, we
obtain
\begin{equation}\label{e7.28}
\frac{X''}{X} + \frac{Y''}{Y} + \frac{Z''}{Z} = -\frac{2\,m}{\hbar^2}\,E,
\end{equation}
where $'$ denotes a derivative with respect to argument. It is evident that
the only way in which the above equation can be satisfied at all points
within the box is if
\begin{eqnarray}
\frac{X''}{X} &=& - k_x^{\,2},\\[0.5ex]
\frac{Y''}{Y} &=& - k_y^{\,2},\\[0.5ex]
\frac{Z''}{Z} &=& - k_z^{\,2},\label{e7.31}
\end{eqnarray}
where $k_x^{\,2}$, $k_y^{\,2}$, and $k_z^{\,2}$ are {\em spatial constants}. Note that the right-hand
sides of the above equations must contain  negative, rather than positive,
spatial constants, because it would not otherwise be possible to satisfy the
boundary conditions. The solutions to the above equations which are properly
normalized, and satisfy the boundary conditions, are [see Eq.~(\ref{e5.11})]
\begin{eqnarray}
X(x) &=& \sqrt{\frac{2}{a}}\sin (k_x\,x),\\[0.5ex]
Y(y) &=& \sqrt{\frac{2}{a}}\sin (k_y\,y),\\[0.5ex]
Z(z) &=& \sqrt{\frac{2}{a}}\sin (k_z\,z),
\end{eqnarray}
where
\begin{eqnarray}
k_x &=& \frac{l_x\,\pi}{a},\\[0.5ex]
k_y &=& \frac{l_y\,\pi}{a},\\[0.5ex]
k_z &=& \frac{l_z\,\pi}{a}.
\end{eqnarray}
Here, $l_x$, $l_y$, and $l_z$ are {\em positive integers}.
Thus, from Eqs.~(\ref{e7.28})--(\ref{e7.31}), the energy is written [see Eq.~(\ref{eenergy})]
\begin{equation}\label{e7.38}
E = \frac{l^2\,\pi^2\,\hbar^2}{2\,m\,a^2}.
\end{equation}
where
\begin{equation}\label{e7.39}
l^2 = l_x^{\,2} + l_y^{\,2}+ l_z^{\,2}.
\end{equation}

\section{Degenerate Electron Gases}
Consider $N$ electrons trapped in a cubic box of dimension $a$. Let us
treat the electrons as essentially {\em non-interacting}\/ particles. According to Sect.~\ref{snon}, 
the total energy of a system consisting of many non-interacting particles is simply the sum of the single-particle energies of the individual
particles. Furthermore, electrons are subject to
the {\em Pauli exclusion principle}\/ (see Sect.~\ref{siden}), since
they are indistinguishable fermions. The exclusion principle
states that no two electrons in our system can occupy the same single-particle
energy level. Now, from the previous section, the single-particle
energy levels for a particle in a box are characterized by the three quantum numbers
$l_x$, $l_y$, and $l_z$. Thus, we conclude that no two electrons in
our system can have the same set of values of $l_x$, $l_y$, and $l_z$. It
turns out that this is not quite true, because electrons possess an intrinsic
angular momentum called {\em spin} (see Cha.~\ref{sspin}). The spin states of an electron are
governed by an additional quantum number, which can take one of two different
values. Hence, when spin is taken into account, we conclude that
a maximum of {\em two}\/ electrons (with different spin quantum numbers) can occupy
a single-particle energy level corresponding to a particular set of values of $l_x$, $l_y$,
and $l_z$. Note, from Eqs.~(\ref{e7.38}) and (\ref{e7.39}), that the associated particle energy is
proportional to $l^2 = l_x^{\,2} + l_y^{\,2}+ l_z^{\,2}$.

Suppose that our electrons are {\em cold}: {\em i.e.}, they have comparatively
little thermal energy. In this case, we would expect them
to fill the lowest single-particle energy levels available to them. We can imagine  the single-particle energy levels as existing in a sort of three-dimensional quantum number space whose Cartesian coordinates are
$l_x$, $l_y$, and $l_z$. Thus, the energy levels are uniformly
distributed in this space on a cubic lattice. Moreover, the distance between
nearest neighbour energy levels is unity. This implies that the
number of energy levels per unit volume is also unity. Finally, the energy of a given energy
level
is proportional to its distance, $l^2 = l_x^{\,2} + l_y^{\,2}+ l_z^{\,2}$,
from the origin.

Since we expect cold electrons to occupy the lowest  energy levels
available to them, but only two electrons can occupy a given energy
level, it follows that if the number of electrons, $N$, is very large then
the filled energy levels will be approximately distributed in a {\em sphere}\/ centered
on the origin of quantum number space. The number of energy levels contained in a sphere
of radius $l$ is approximately equal to the volume of the sphere---since
the number of energy levels per unit volume is unity. It turns out that this
is not quite correct, because we have forgotten that the quantum numbers
$l_x$, $l_y$, and $l_z$ can only take {\em positive}\/ values.
Hence, the filled energy levels actually only occupy one {\em octant}\/ of a sphere.
The radius $l_F$ of the octant of filled energy levels in quantum number space can be calculated
by equating the number of energy levels it contains to the number of electrons,
$N$.
Thus, we can write
\begin{equation}
N = 2\times\frac{1}{8}\times \frac{4\,\pi}{3}\,l_F^{\,3}.
\end{equation}
Here, the factor 2 is to take into account the two spin states of an electron,
and the factor $1/8$ is to take account of the fact that $l_x$, $l_y$, and $l_z$
can only take positive values.
Thus,
\begin{equation}
l_F = \left(\frac{3\,N}{\pi}\right)^{1/3}.
\end{equation}
According to Eq.~(\ref{e7.38}), the energy of the most energetic
electrons---which is known as the {\em Fermi energy}---is given by
\begin{equation}\label{e7.42}
E_F = \frac{l_F^{\,2}\,\pi^2\,\hbar^2}{2\,m_e\,a^2}=\frac{\pi^2\,\hbar^2}{2\,m\,a^2}\left(\frac{3\,N}{\pi}\right)^{2/3},
\end{equation}
where $m_e$ is the electron mass.
This can also be written as
\begin{equation}\label{e7.43}
E_F = \frac{\pi^2\,\hbar^2}{2\,m_e}\left(\frac{3\,n}{\pi}\right)^{2/3},
\end{equation}
where $n=N/a^3$ is the number of electrons per unit volume (in real space). Note
that the Fermi energy only depends on the {\em number density}\/
of the confined electrons. 

The mean energy of the electrons is given by
\begin{equation}
\bar{E} = E_F\left.\int_0^{l_F}l^2\,4\pi\,l^2\,dl\right/\frac{4}{3}\,\pi\,l_F^{\,5}=
\frac{3}{5}\,E_F,
\end{equation}
since $E\propto l^2$, and the energy levels are uniformly
distributed in quantum number space inside an octant of radius $l_F$. Now, according to classical physics, the mean
thermal energy of the electrons is $(3/2)\,k_B\,T$, where $T$ is the
electron temperature, and $k_B$  the
Boltzmann constant. Thus, if $k_B\,T\ll E_F$ then our original assumption
that the electrons are cold is valid. Note that, in this case, the
electron energy is {\em much larger}\/ than that predicted by classical
physics---electrons in this state are termed {\em degenerate}. On the
other hand, if $k_B\,T\gg E_F$ then the electrons are hot, and are essentially
governed by classical physics---electrons in this state are termed {\em non-degenerate}.

The total energy of a degenerate electron gas is
\begin{equation}
E_{total} = N\,\bar{E} = \frac{3}{5}\,N\,E_F.
\end{equation}
Hence, the gas pressure takes the form
\begin{equation}\label{e7.46}
P = -\frac{\partial E_{total}}{\partial V} = \frac{2}{5}\,n\,E_F,
\end{equation}
since $E_F\propto a^{-2}=V^{-2/3}$ [see Eq.~(\ref{e7.42})].
Now, the pressure predicted by classical physics is $P= n\,k_B\,T$. 
Thus, a degenerate electron gas has a {\em much higher}\/ pressure than
that which would be predicted by classical physics. This is an entirely
quantum mechanical effect, and is due to the fact that identical fermions
cannot get significantly closer together than a de Broglie wavelength
without violating the Pauli exclusion principle. Note that, according to
Eq.~(\ref{e7.43}), the mean spacing between degenerate electrons is
\begin{equation}
d\sim n^{-1/3}\sim \frac{h}{\sqrt{m_e\,E}}\sim \frac{h}{p}\sim \lambda,
\end{equation}
where $\lambda$ is the de~Broglie wavelength. Thus, an electron gas
is non-degener\-ate when the mean spacing between the electrons is much
greater than the de~Broglie wavelength, and becomes degenerate as the
mean spacing approaches the de Broglie wavelength.

In turns out that the conduction ({\em i.e.}, free) electrons inside metals are
highly degenerate (since the number of electrons per unit volume
is very large, and $E_F\propto n^{2/3}$). Indeed, most metals are hard to compress
as a direct consequence of the high degeneracy pressure of their conduction
electrons. To be more exact, resistance to compression is usually measured
in terms of a quantity known as the {\em bulk modulus}, which is defined
\begin{equation}
B = - V\,\frac{\partial P}{\partial V}
\end{equation}
Now, for a fixed number of electrons, $P\propto V^{-5/3}$ [see Eqs.~(\ref{e7.42}) and (\ref{e7.46})]. Hence,
\begin{equation}
B = \frac{5}{3}\,P = \frac{\pi^3\,\hbar^2}{9\,m}\left(\frac{3\,n}{\pi}\right)^{5/3}.
\end{equation}
For example, the number density of free electrons in magnesium is
$n\sim 8.6\times 10^{28}\,{\rm m}^{-3}$. This leads to the following estimate
for the bulk modulus: $B\sim 6.4\times 10^{10}\,{\rm N}\,{\rm m}^{-2}$.
The actual bulk modulus is $B= 4.5\times 10^{10}\,{\rm N}\,{\rm m}^{-2}$.

\section{White-Dwarf Stars}
A main-sequence hydrogen-burning star, such as the Sun, is maintained
in equilibrium via the balance of the gravitational attraction tending
to  make it collapse, and the thermal pressure tending to make it expand. 
Of course, the
thermal energy of the star is generated by nuclear reactions occurring deep inside
its core. Eventually, however, the star will run out of burnable fuel, and, therefore,
start to collapse, as it radiates away its remaining thermal energy.
What is the ultimate fate of such a star?

A burnt-out star is basically a gas of electrons and ions. As the
star collapses, its density increases, and so the mean separation between its
constituent particles decreases. Eventually, the mean separation becomes
of order the de~Broglie wavelength of the electrons, and the electron
gas becomes {\em degenerate}. Note, that the de~Broglie wavelength of the
ions is  much smaller than that of the electrons, so the ion gas remains
non-degenerate. Now, even at
zero temperature, a degenerate electron gas exerts a substantial pressure,
because the Pauli exclusion principle prevents the mean electron separation
from becoming significantly  smaller than the typical 
de~Broglie wavelength (see 
previous section). Thus, it is possible for a burnt-out star to maintain
itself against complete collapse under gravity via the {\em degeneracy pressure}
of its constituent electrons. Such stars are termed {\em white-dwarfs}. 
Let us investigate the physics of white-dwarfs in more detail.

The total energy of a white-dwarf star can be written
\begin{equation}
{\cal E} = K + U,\label{e8wd1}
\end{equation}
where $K$ is the kinetic energy of the degenerate electrons (the kinetic
energy of the ion is negligible), and $U$ is the gravitational potential
energy. Let us assume, for the sake of simplicity, that the density of the
star is {\em uniform}. In this case, the gravitational potential
energy takes the form
\begin{equation}
U = -\frac{3}{5}\frac{G\,M^2}{R},\label{e8wd2}
\end{equation}
where $G$ is the gravitational constant, $M$ is the stellar mass, and $R$ is
the stellar radius.

From the previous subsection, the kinetic energy of a degenerate electron gas is simply
\begin{equation}\label{e8wd4}
K = N\,\bar{E} = \frac{3}{5}\,N\,E_F = \frac{3}{5}\,N\,\frac{\pi^2\,\hbar^2}{2\,m_e}\left(\frac{3\,N}{\pi\,V}\right)^{2/3},
\end{equation}
where $N$ is the number of electrons, $V$ the volume of the star, and $m_e$
the electron mass.

The interior of a white-dwarf star is composed of atoms like
$C^{12}$ and $O^{16}$ which contain equal numbers of protons, neutrons, and
electrons. Thus,
\begin{equation}
M = 2\,N\,m_p,\label{e8wd6}
\end{equation}
where $m_p$ is the proton mass. 

Equations (\ref{e8wd1})--(\ref{e8wd6}) can be combined to give
\begin{equation}\label{e7.52}
{\cal E} = \frac{A}{R^2} - \frac{B}{R},
\end{equation}
where
\begin{eqnarray}
A &=& \frac{3}{20}\left(\frac{9\pi}{8}\right)^{2/3}\frac{\hbar^2}{m_e}
\left(\frac{M}{m_p}\right)^{5/3},\\[0.5ex]
B&=& \frac{3}{5}\,G\,M^2.
\end{eqnarray}
The equilibrium radius of the star, $R_\ast$, is that which
{\em minimizes} the total energy ${\cal E}$. In fact,
it is easily demonstrated that
\begin{equation}
R_\ast = \frac{2\,A}{B},
\end{equation}
which yields
\begin{equation}
R_\ast = \frac{(9\pi)^{2/3}}{8}\,\frac{\hbar^2}{G\,m_e\,m_p^{\,5/3}\,
M^{1/3}}.
\end{equation}
The above formula can also be written
\begin{equation}\label{e7.57}
\frac{R_\ast}{R_\odot}= 0.010\left(\frac{M_\odot}{M}\right)^{1/3},\label{e8wd9}
\end{equation}
where $R_\odot= 7\times 10^5\,{\rm km}$ is the solar radius, and
$M_\odot = 2\times 10^{30}\,{\rm kg}$  the solar mass. It follows that
the radius of a typical solar mass white-dwarf is about 7000\,km: 
{\em i.e.}, about the same as the radius of the Earth. The first
white-dwarf to be discovered (in 1862) was the companion of Sirius. Nowadays,
thousands of white-dwarfs have been observed, all with properties similar
to those described above.


Note from Eqs.~(\ref{e8wd4}), (\ref{e8wd6}), and (\ref{e7.57})
that $\bar{E}\propto M^{4/3}$. In other words, the mean energy of the
electrons inside a white dwarf {\em increases}\/ as the  stellar mass increases.
Hence, for a sufficiently massive white dwarf, the electrons can become
{\em relativistic}. It turns out that the degeneracy pressure for
relativistic electrons only scales as $R^{-1}$, rather that $R^{-2}$, 
and thus is unable to balance the gravitational pressure [which also
scales as $R^{-1}$---see Eq.~(\ref{e7.52})]. It follows that electron
 degeneracy pressure is only able to halt the collapse of a burnt-out star
 provided that the stellar mass does not exceed some critical value, known
 as the {\em Chandrasekhar limit},
 which turns out to be about $1.4$ times the mass of the Sun. Stars
 whose mass exceeds the Chandrasekhar limit inevitably collapse to
 produce extremely compact objects, such as neutron stars (which are
 held up by the degeneracy pressure of their constituent neutrons), 
 or black holes.

\subsubsection*{Exercises}
{\small
\begin{enumerate}
\item Consider a particle of mass $m$ moving in a three-dimensional
isotropic harmonic oscillator potential of force constant $k$. Solve the
problem via the separation of variables, and obtain an expression for
the allowed values of the total energy of the system (in a stationary state).
\item Repeat the calculation of the Fermi energy of a gas of fermions by
assuming that the fermions are massless, so that the energy-momentum relation
is $E=p\,c$. 
\item Calculate the density of states of an electron gas in a cubic
box of volume $L^3$, bearing in  mind that there are two electrons
per energy state. In other words, calculate the number of electron
states in the interval $E$ to $E+dE$. This number can be written $dN = \rho(E)\,dE$,
where $\rho$ is the density of states. 
\item Repeat the above calculation for a two-dimensional electron
gas in a square box of area $L^2$.
\item Given that the number density of free electrons in copper is
$8.5\times 10^{28}\,{\rm m}^{-3}$, calculate the Fermi
energy in electron volts, and the velocity of an electron whose kinetic
energy is equal to the Fermi energy. 
\item Obtain an expression for the Fermi energy (in eV) of an electron in a white
dwarf star as a function of the stellar mass (in solar masses). At what
mass does the Fermi energy equal the rest mass energy?
\end{enumerate}
}