\chapter{Addition of Angular Momentum}\label{sadd} 
\section{Introduction}
Consider an electron in a hydrogen atom. As we have already seen,  the electron's motion through space is parameterized by the three quantum numbers $n$, $l$,
and $m$ (see Sect.~\ref{s10.4}). To these we must now add the
two quantum numbers $s$ and $m_s$ which parameterize the electron's internal
motion (see the previous chapter). Now, the quantum numbers $l$ and $m$
specify the electron's orbital angular momentum vector, ${\bf L}$, (as much as it can be specified) whereas
the quantum numbers $s$ and $m_s$ specify its spin angular momentum vector,
${\bf S}$. But, if the electron possesses both orbital 
and spin angular momentum  then what is its total angular momentum?

\section{General Principles}\label{s11.2}
The three basic orbital angular momentum operators, $L_x$, $L_y$,
and $L_z$, obey the commutation relations (\ref{e8.6})--(\ref{e8.8}),
which can be written in the convenient vector form:
\begin{equation}
{\bf L}\times{\bf L} = {\rm i}\,\hbar\,{\bf L}.
\end{equation}
Likewise, the three basic spin angular momentum operators, $S_x$, $S_y$,
and $S_z$, obey the commutation relations (\ref{e10.1x})--(\ref{e10.2x}),
which can also be written in  vector form: {\em i.e.},
\begin{equation}
{\bf S}\times{\bf S} = {\rm i}\,\hbar\,{\bf S}.
\end{equation}
Now, since the orbital angular momentum operators are associated
with the electron's motion through space, whilst the spin angular
momentum operators are associated with its internal
motion, and these two types of motion are completely unrelated ({\em i.e.}, they correspond to different degrees of freedom---see Sect.~\ref{sfuncon}), it is reasonable to suppose that the two sets of operators commute with
one another: {\em i.e.},
\begin{equation}
[L_i, S_j] = 0,
\end{equation}
where $i,j=1,2,3$ corresponds to $x,y,z$. 

Let us now consider the electron's total angular momentum vector
\begin{equation}
{\bf J} = {\bf L} + {\bf S}.
\end{equation}
We have
\begin{eqnarray}
{\bf J}\times{\bf J} &= &({\bf L}+{\bf S})\times({\bf L}+{\bf S})\nonumber\\[0.5ex]
&=& {\bf L}\times{\bf L} + {\bf S}\times{\bf S}+{\bf L}\times{\bf S}+{\bf S}\times{\bf L} ={\bf L}\times{\bf L} + {\bf S}\times{\bf S} \nonumber\\[0.5ex]
&=& {\rm i}\,\hbar\,{\bf L} + {\rm i}\,\hbar\,{\bf S}\nonumber\\[0.5ex]
&=& {\rm i}\,\hbar\,{\bf J}.
\end{eqnarray}
In other words,
\begin{equation}
{\bf J}\times{\bf J} = {\rm i}\,\hbar\,{\bf J}.
\end{equation}
It is thus evident that the three basic total angular momentum operators, $J_x$,
$J_y$, and $J_z$, obey  analogous commutation relations to the
corresponding orbital and spin angular momentum operators. It therefore follows
that the total angular momentum  has similar properties to the
orbital and spin angular momenta. For instance, it is
only possible to simultaneously measure the magnitude
squared of the total angular momentum vector,
\begin{equation}
J^2 =J_x^{\,2}+J_y^{\,2}+J_z^{\,2},
\end{equation}
together with a single Cartesian component. By convention, we shall
always choose to measure $J_z$. A simultaneous eigenstate
of $J_z$ and $J^2$ satisfies
\begin{eqnarray}
J_z\,\psi_{j,m_j}&=& m_j\,\hbar\,\psi_{j,m_j},\\[0.5ex]
J^2\,\psi_{j,m_j} &=& j\,(j+1)\,\hbar^{\,2}\,\psi_{j,m_j},
\end{eqnarray}
where the quantum number $j$ can take positive integer, or half-integer,
values, and the quantum number $m_j$ is restricted to the following
range of values:
\begin{equation}
-j, -j+1,\cdots, j-1, j.
\end{equation}

Now
\begin{equation}
J^2 = ({\bf L}+{\bf S})\cdot({\bf L}+{\bf S}) = L^2+ S^2 + 2\,{\bf L}\cdot{\bf S},
\end{equation}
which can also be written as
\begin{equation}\label{e11.12}
J^2 = L^2+S^2 +2\,L_z\,S_z+ L_+\,S_-+L_-\,S_+.
\end{equation}
We know that the operator $L^2$ commutes with itself, with all of the
Cartesian components of ${\bf L}$ (and, hence, with the raising
and lowering operators $L_\pm$), and with all of the spin angular momentum
operators (see Sect.~\ref{s8.2}). It is therefore clear that
\begin{equation}
[J^2,L^2] = 0.
\end{equation}
A similar argument allows us to also conclude that
\begin{equation}
[J^2,S^2]=0.
\end{equation}
Now, the operator $L_z$ commutes with itself, with $L^2$, with all of
the spin angular momentum operators, but {\em not}\/ with the raising
and lowering operators $L_\pm$ (see Sect.~\ref{s8.2}). It follows that
\begin{equation}
[J^2,L_z]\neq 0.
\end{equation}
Likewise, we can also show that
\begin{equation}
[J^2,S_z]\neq 0.
\end{equation}
Finally, we have
\begin{equation}
J_z = L_z+S_z,
\end{equation}
where $[J_z,L_z]=[J_z,S_z]=0$. 

Recalling that only {\em commuting}\/ operators correspond to physical
quantities which can be simultaneously measured (see Sect.~\ref{smeas}),
it follows, from the above discussion,  that there  are {\em two}\/ alternative sets of
physical variables associated with angular momentum which we can measure simultaneously.
The first set correspond to the operators $L^2$, $S^2$, $L_z$, $S_z$, and $J_z$. The second set correspond to the operators
$L^2$, $S^2$, $J^2$, and $J_z$. In other words, we can always measure
the magnitude squared of the orbital and spin angular momentum vectors, together with
the $z$-component of the total angular momentum vector.
In addition, we can either choose to measure the $z$-components of
the orbital and spin angular momentum vectors, or the magnitude squared
of the total angular momentum vector.

Let $\psi^{(1)}_{l,s;m,m_s}$ represent a simultaneous eigenstate
of $L^2$, $S^2$, $L_z$, and $S_z$ corresponding to the following
eigenvalues:
\begin{eqnarray}
L^2\,\psi^{(1)}_{l,s;m,m_s}&=& l\,(l+1)\,\hbar^2\,\psi^{(1)}_{l,s;m,m_s},\\[0.5ex]
S^2\,\psi^{(1)}_{l,s;m,m_s}&=& s\,(s+1)\,\hbar^2\,\psi^{(1)}_{l,s;m,m_s},\\[0.5ex]
L_z\,\psi^{(1)}_{l,s;m,m_s}&=& m\,\hbar\,\psi^{(1)}_{l,s;m,m_s},\\[0.5ex]
S_z\,\psi^{(1)}_{l,s;m,m_s}&=& m_s\,\hbar\,\psi^{(1)}_{l,s;m,m_s}.
\end{eqnarray}
It is easily seen that
\begin{eqnarray}
J_z\,\psi^{(1)}_{l,s;m,m_s}& =& (L_z+S_z)\,\psi^{(1)}_{l,s;m,m_s}=
(m+m_s)\,\hbar\,\psi^{(1)}_{l,s;m,m_s}\nonumber\\[0.5ex] &= &m_j\,\hbar\,\psi^{(1)}_{l,s;m,m_s}.
\end{eqnarray}
Hence,
\begin{equation}
m_j = m+m_s.\label{e11.23}
\end{equation}
In other words, the quantum numbers controlling the $z$-components
of the various angular momentum vectors can simply be added algebraically.

Finally, let $\psi^{(2)}_{l,s;j,m_j}$ represent a simultaneous eigenstate
of $L^2$, $S^2$, $J^2$, and $J_z$ corresponding to the following
eigenvalues:
\begin{eqnarray}
L^2\,\psi^{(2)}_{l,s;j,m_j}&=& l\,(l+1)\,\hbar^2\,\psi^{(2)}_{l,s;j,m_j},\\[0.5ex]
S^2\,\psi^{(2)}_{l,s;j,m_j}&=& s\,(s+1)\,\hbar^2\,\psi^{(2)}_{l,s;j,m_j},\\[0.5ex]
J^2\,\psi^{(2)}_{l,s;j,m_j}&=& j\,(j+1)\,\hbar^2\,\psi^{(2)}_{l,s;j,m_j},\label{e11.26}\\[0.5ex]
J_z\,\psi^{(2)}_{l,s;j,m_j}&=& m_j\,\hbar\,\psi^{(2)}_{l,s;j,m_j}.
\end{eqnarray}

\section{Angular Momentum in the Hydrogen Atom}\label{s11.3}
In a hydrogen atom, the wavefunction of an electron   in a simultaneous
eigenstate of $L^2$ and $L_z$ has an angular dependence specified
by the spherical harmonic $Y_{l,m}(\theta,\phi)$ (see Sect.~\ref{sharm}).
If the electron is also in an eigenstate of $S^2$ and $S_z$ then the
quantum numbers $s$ and $m_s$ take the values $1/2$ and $\pm 1/2$,
respectively, and the internal state of the electron is  specified
by the spinors $\chi_\pm$ (see Sect.~\ref{spauli}). Hence, the
simultaneous eigenstates of $L^2$, $S^2$, $L_z$, and $S_z$ can be written
in the separable form
\begin{equation}\label{e11.28}
\psi^{(1)}_{l,1/2;m,\pm 1/2} = Y_{l,m}\,\chi_\pm.
\end{equation}
Here, it is understood that orbital angular momentum operators act on
the spherical harmonic functions, $Y_{l,m}$, whereas spin angular momentum operators act
on the spinors, $\chi_\pm$. 

Since the eigenstates $\psi^{(1)}_{l,1/2;m,\pm 1/2}$ are (presumably)
orthonormal, and form a complete set, we can express the eigenstates $\psi^{(2)}_{l,1/2;j,m_j}$ as  linear combinations of them. For instance,
\begin{equation}\label{e11.29}
\psi^{(2)}_{l,1/2;j,m+1/2} = \alpha\,\psi^{(1)}_{l,1/2;m,1/2} + \beta\,
\psi^{(1)}_{l,1/2;m+1,-1/2},
\end{equation}
where $\alpha$ and $\beta$ are, as yet, unknown coefficients. Note that
the number of $\psi^{(1)}$ states which can appear on the right-hand side
of the above expression is limited to two by the constraint that
$m_j=m+m_s$ [see Eq.~(\ref{e11.23})], and the fact that $m_s$ can only take the values $\pm 1/2$. 
Assuming that the $\psi^{(2)}$ eigenstates are properly normalized, we have
\begin{equation}\label{e11.30}
\alpha^2 + \beta^2 = 1.
\end{equation}

Now, it follows from Eq.~(\ref{e11.26}) that
\begin{equation}\label{e11.31}
J^2\,\psi^{(2)}_{l,1/2;j,m+1/2}= j\,(j+1)\,\hbar^2\,\psi^{(2)}_{l,1/2;j,m+1/2},
\end{equation}
where [see Eq.~(\ref{e11.12})]
\begin{equation}\label{e11.32}
J^2 = L^2+S^2 +2\,L_z\,S_z+ L_+\,S_-+L_-\,S_+.
\end{equation}
Moreover, according to Eqs.~(\ref{e11.28}) and (\ref{e11.29}), we can write
\begin{equation}\label{e11.33}
\psi^{(2)}_{l,1/2;j,m+1/2} = \alpha\,Y_{l,m}\,\chi_+ + \beta\,
Y_{l,m+1}\,\chi_-.
\end{equation}
Recall, from Eqs.~(\ref{eraise}) and (\ref{elow}), that
\begin{eqnarray}\label{e11.34}
L_+\,Y_{l,m} &=& [l\,(l+1)-m\,(m+1)]^{1/2}\,\hbar\,Y_{l,m+1},\\[0.5ex]
L_-\,Y_{l,m} &=& [l\,(l+1)-m\,(m-1)]^{1/2}\,\hbar\,Y_{l,m-1}.
\end{eqnarray}
By analogy, when the  spin raising and lowering operators, $S_\pm$, act on a general spinor, $\chi_{s,m_s}$, we obtain
\begin{eqnarray}
S_+\,\chi_{s,m_s} &=& [s\,(s+1)-m_s\,(m_s+1)]^{1/2}\,\hbar\,\chi_{s,m_s+1},\\[0.5ex]
S_-\,\chi_{s,m_s} &=& [s\,(s+1)-m_s\,(m_s-1)]^{1/2}\,\hbar\,\chi_{s,m_s-1}.
\end{eqnarray}
For the special case of spin one-half spinors ({\em i.e.}, $s=1/2, m_s=\pm 1/2$), the above expressions reduce to
\begin{equation}
S_+\,\chi_+=S_-\,\chi_- = 0,
\end{equation}
and
\begin{equation}\label{e11.39}
S_\pm\,\chi_\mp = \hbar\,\chi_\pm.
\end{equation}

It follows from Eqs.~(\ref{e11.32}) and (\ref{e11.34})--(\ref{e11.39}) that
\begin{eqnarray}
J^2\,Y_{l,m}\,\chi_+&=& [l\,(l+1)+3/4+m]\,\hbar^2\,Y_{l,m}\,\chi_+\nonumber\\[0.5ex]
&&+ [l\,(l+1)-m\,(m+1)]^{1/2}\,\hbar^2\,Y_{l,m+1}\,\chi_-,
\end{eqnarray}
and
\begin{eqnarray}
J^2\,Y_{l,m+1}\,\chi_-&=& [l\,(l+1)+3/4-m-1]\,\hbar^2\,Y_{l,m+1}\,\chi_-\nonumber\\[0.5ex]
&&+ [l\,(l+1)-m\,(m+1)]^{1/2}\,\hbar^2\,Y_{l,m}\,\chi_+.
\end{eqnarray}
Hence, Eqs.~(\ref{e11.31}) and (\ref{e11.33}) yield
\begin{eqnarray}\label{e11.42}
(x - m)\,\alpha - [l\,(l+1)-m\,(m+1)]^{1/2}\,\beta &=& 0,\\[0.5ex]
-[l\,(l+1)-m\,(m+1)]^{1/2}\,\alpha +(x+m+1)\,\beta&=& 0,\label{e11.43}
\end{eqnarray}
where 
\begin{equation}
x = j\,(j+1) - l\,(l+1) - 3/4.
\end{equation}
Equations (\ref{e11.42}) and (\ref{e11.43}) can be solved to give
\begin{equation}
x\,(x+1) = l\,(l+1),
\end{equation}
and
\begin{equation}\label{e11.45}
\frac{\alpha}{\beta} = \frac{[(l-m)\,(l+m+1)]^{1/2}}{x-m}.
\end{equation}
It follows that $x=l$ or $x=-l-1$, which corresponds to $j=l+1/2$ or
$j=l-1/2$, respectively. Once $x$ is specified, Eqs.~(\ref{e11.30}) and (\ref{e11.45}) can be solved for $\alpha$ and $\beta$. We obtain
\begin{equation}\label{e11.47}
\psi^{(2)}_{l+1/2,m+1/2} = \left(\frac{l+m+1}{2\,l+1}\right)^{1/2}
\psi^{(1)}_{m,1/2} + \left(\frac{l-m}{2\,l+1}\right)^{1/2}\psi^{(1)}_{m+1,-1/2},
\end{equation}
and
\begin{equation}\label{e11.48}
\psi^{(2)}_{l-1/2,m+1/2} = \left(\frac{l-m}{2\,l+1}\right)^{1/2}
\psi^{(1)}_{m,1/2} -\left(\frac{l+m+1}{2\,l+1}\right)^{1/2}\psi^{(1)}_{m+1,-1/2}.
\end{equation}
Here, we have neglected the common subscripts $l,1/2$ for the sake of
clarity: {\em i.e.}, $\psi^{(2)}_{l+1/2,m+1/2}\equiv \psi^{(2)}_{l,1/2;l+1/2,m+1/2}$, {\em etc.} The above equations can easily be inverted
to give the $\psi^{(1)}$ eigenstates in terms of the $\psi^{(2)}$ eigenstates:
\begin{eqnarray}
\psi^{(1)}_{m,1/2} &=& \left(\frac{l+m+1}{2\,l+1}\right)^{1/2}\!\psi^{(2)}_{l+1/2,m+1/2} +  \left(\frac{l-m}{2\,l+1}\right)^{1/2}\!\psi^{(2)}_{l-1/2,m+1/2},\\[0.5ex]
\!\!\!\!\!\!\psi^{(1)}_{m+1,-1/2}&=&  \left(\frac{l-m}{2\,l+1}\right)^{1/2}\!\psi^{(2)}_{l+1/2,m+1/2} - \left(\frac{l+m+1}{2\,l+1}\right)^{1/2}\!
\psi^{(2)}_{l-1/2,m+1/2}.\label{e11.50}
\end{eqnarray}
The information contained in Eqs.~(\ref{e11.47})--(\ref{e11.50})
is neatly summarized in Table~\ref{t2}. For instance, Eq.~(\ref{e11.47})
is obtained by reading the first row of this table, whereas Eq.~(\ref{e11.50})
is obtained by reading the second column. The coefficients in this type
of table are generally known as {\em Clebsch-Gordon coefficients}.

\begin{table}\centering
\begin{tabular}{c|cc|c}
&$m, 1/2$& $m+1, -1/2$&$m, m_s$\\[0.5ex]\hline
$l+1/2, m+1/2$ & ${\scriptstyle\sqrt{(l+m+1)/(2\,l+1)}}$&  ${\scriptstyle\sqrt{(l-m)/(2\,l+1)}}$&\\[0.5ex]
$l-1/2, m+1/2$& ${\scriptstyle\sqrt{(l-m)/(2\,l+1)}}$&${\scriptstyle-\sqrt{(l+m+1)/(2\,l+1)}}$&\\[0.5ex]\hline
$j, m_j$&&&
\end{tabular}
\caption{\em Clebsch-Gordon coefficients for adding spin one-half to
spin $l$.}\label{t2}
\end{table}

As an example, let us consider the $l=1$ states of a hydrogen atom. 
The eigenstates of $L^2$, $S^2$, $L_z$, and $S_z$,
 are denoted $\psi^{(1)}_{m,m_s}$. Since $m$ can take the values $-1,0,1$,
 whereas $m_s$ can take the values $\pm 1/2$, there are
 clearly six such states: {\em i.e.}, $\psi^{(1)}_{1,\pm 1/2}$, $\psi^{(1)}_{0,\pm 1/2}$,
 and $\psi^{(1)}_{-1,\pm 1/2}$. The eigenstates of $L^2$, $S^2$, $J^2$, and $J_z$,
 are denoted $\psi^{(2)}_{j,m_j}$. Since $l=1$ and $s=1/2$ can be combined
 together to form either $j=3/2$ or $j=1/2$ (see earlier), there are
 also six such states: {\em i.e.}, $\psi^{(2)}_{3/2,\pm 3/2}$, $\psi^{(2)}_{3/2,\pm 1/2}$, and $\psi^{(2)}_{1/2,\pm 1/2}$. According to
 Table~\ref{t2}, the various different eigenstates are interrelated as follows:
 \begin{eqnarray}\label{ecgs}
 \psi^{(2)}_{3/2,\pm 3/2} &=& \psi^{(1)}_{\pm 1, \pm 1/2},\\[0.5ex]
 \psi^{(2)}_{3/2,1/2} &=& \sqrt{\frac{2}{3}}\,\psi^{(1)}_{0,1/2} + 
\sqrt{ \frac{1}{3}}\,\psi^{(1)}_{1,-1/2},\label{e11.52}\\[0.5ex]
\psi^{(2)}_{1/2,1/2} &=& \sqrt{\frac{1}{3}}\,\psi^{(1)}_{0,1/2} -
\sqrt{ \frac{2}{3}}\,\psi^{(1)}_{1,-1/2},\\[0.5ex]
\psi^{(2)}_{1/2,-1/2} &=& \sqrt{\frac{2}{3}}\,\psi^{(1)}_{-1,1/2} 
-\sqrt{ \frac{1}{3}}\,\psi^{(1)}_{0,-1/2},\\[0.5ex]
\psi^{(2)}_{3/2,-1/2} &=& \sqrt{\frac{1}{3}}\,\psi^{(1)}_{-1,1/2} + 
\sqrt{ \frac{2}{3}}\,\psi^{(1)}_{0,-1/2},
 \end{eqnarray}
 and
 \begin{eqnarray}
 \psi^{(1)}_{\pm 1,\pm 1/2} &=& \psi^{(2)}_{3/2, \pm 3/2},\\[0.5ex]
\psi^{(1)}_{1,-1/2} &=& \sqrt{\frac{1}{3}}\,\psi^{(2)}_{3/2,1/2} -
\sqrt{ \frac{2}{3}}\,\psi^{(2)}_{1/2,1/2},\\[0.5ex]
 \psi^{(1)}_{0,1/2} &=& \sqrt{\frac{2}{3}}\,\psi^{(2)}_{3/2,1/2} + 
\sqrt{ \frac{1}{3}}\,\psi^{(2)}_{1/2,1/2},\label{e11.57}\\[0.5ex]
\psi^{(1)}_{0,-1/2} &=& \sqrt{\frac{2}{3}}\,\psi^{(2)}_{3/2,-1/2} 
-\sqrt{ \frac{1}{3}}\,\psi^{(2)}_{1/2,-1/2},\label{ecge}\\[0.5ex]
\psi^{(1)}_{-1,1/2} &=& \sqrt{\frac{1}{3}}\,\psi^{(2)}_{3/2,-1/2} + 
\sqrt{ \frac{2}{3}}\,\psi^{(2)}_{1/2,-1/2},
 \end{eqnarray}
Thus, if we know that an electron in a hydrogen atom is in an $l=1$ 
state characterized by $m=0$ and $m_s=1/2$ [{\em i.e.}, the state
represented by $\psi^{(1)}_{0,1/2}$] then, according to Eq.~(\ref{e11.57}),
a measurement of the total angular momentum will yield $j=3/2$, $m_j=1/2$
with probability $2/3$, and $j=1/2$, $m_j=1/2$ with probability $1/3$. 
Suppose that we make such a measurement, and obtain the result $j=3/2$, $m_j=1/2$. As a result of the measurement, the electron is thrown into
the corresponding eigenstate, $\psi^{(2)}_{3/2,1/2}$.  It thus follows
from Eq.~(\ref{e11.52}) that a subsequent measurement of $L_z$ and $S_z$
will yield $m=0$, $m_s=1/2$ with probability $2/3$, and $m=1$, $m_s=-1/2$
with probability $1/3$.

\begin{table}
\begin{tabular}{c|cccccc|c}
&$-1, -1/2$& $-1, 1/2$&$0,-1/2$&$0,1/2$&$1,-1/2$&$1,1/2$&$m,m_s$\\[0.5ex]\hline
$3/2, -3/2$ & $\scriptstyle{1}$&&&&&\\[0.5ex]
$3/2, -1/2$& &${\scriptstyle\sqrt{1/3}}$&${\scriptstyle\sqrt{2/3}}$&&&&\\[0.5ex]
$1/2, -1/2$& &${\scriptstyle\sqrt{2/3}}$&${\scriptstyle -\sqrt{1/3}}$&&&&\\[0.5ex]
$3/2, 1/2$& &&&${\scriptstyle\sqrt{2/3}}$&${\scriptstyle\sqrt{1/3}}$&&\\[0.5ex]
$1/2, 1/2$& &&&${\scriptstyle\sqrt{1/3}}$&${\scriptstyle - \sqrt{2/3}}$&&\\[0.5ex]
$3/2, 3/2$& &&&&&${\scriptstyle 1}$&\\
\hline
$j, m_j$&&&&&&&
\end{tabular}
\caption{\em Clebsch-Gordon coefficients for adding spin one-half to
spin one. Only non-zero coefficients are shown.}\label{t3}
\end{table}

The information contained in Eqs.~(\ref{ecgs})--(\ref{ecge}) is neatly summed
up in Table~\ref{t3}. Note that each row and column of this
table has unit norm, and also that the different rows and different columns
are mutually orthogonal. Of course, this is because the $\psi^{(1)}$ and
$\psi^{(2)}$ eigenstates are orthonormal.

\section{Two Spin One-Half Particles}\label{shalf}
Consider a system consisting of two spin one-half particles. Suppose
that the system does not possess any orbital angular momentum. 
Let ${\bf S}_1$ and ${\bf S}_2$ be the spin angular momentum operators
of the first and second particles, respectively, and let
\begin{equation}
{\bf S} ={\bf S}_1 + {\bf S}_2
\end{equation}
be the total spin angular momentum operator. By analogy
with the previous analysis, we conclude that it is possible to simultaneously measure either
$S_1^{\,2}$, $S_2^{\,2}$, $S^2$, and $S_z$, or
$S_1^{\,2}$, $S_2^{\,2}$, $S_{1z}$, $S_{2z}$, and $S_z$. 
Let the quantum numbers associated with measurements of
$S_1^{\,2}$, $S_{1z}$, $S_2^{\,2}$, $S_{2z}$, $S^2$, and 
$S_z$ be $s_1$, $m_{s_1}$, $s_2$, $m_{s_2}$, $s$, and $m_s$, respectively.
In other words, if the spinor $\chi_{s_1,s_2;m_{s_1},m_{s_2}}^{(1)}$ is
a simultaneous eigenstate of $S_1^{\,2}$, $S_2^{\,2}$, $S_{1z}$,
and $S_{2z}$, then
\begin{eqnarray}
S_1^{\,2}\, \chi_{s_1,s_2;m_{s_1},m_{s_2}}^{(1)}&=& s_1\,(s_1+1)\,\hbar^2
\,\chi_{s_1,s_2;m_{s_1},m_{s_2}}^{(1)},\\[0.5ex]
S_2^{\,2}\, \chi_{s_1,s_2;m_{s_1},m_{s_2}}^{(1)}&=& s_2\,(s_2+1)\,\hbar^2
\,\chi_{s_1,s_2;m_{s_1},m_{s_2}}^{(1)},\\[0.5ex]
S_{1z}\, \chi_{s_1,s_2;m_{s_1},m_{s_2}}^{(1)}&=& m_{s_1}\,\hbar
\,\chi_{s_1,s_2;m_{s_1},m_{s_2}}^{(1)},\\[0.5ex]
S_{2z}\, \chi_{s_1,s_2;m_{s_1},m_{s_2}}^{(1)}&=& m_{s_2}\,\hbar
\,\chi_{s_1,s_2;m_{s_1},m_{s_2}}^{(1)},\\[0.5ex]
S_z\, \chi_{s_1,s_2;m_{s_1},m_{s_2}}^{(1)}&=& m_s\,\hbar
\,\chi_{s_1,s_2;m_{s_1},m_{s_2}}^{(1)}.
\end{eqnarray}
Likewise, if the spinor $\chi_{s_1,s_2;s,m_s}^{(2)}$ is
a simultaneous eigenstate of $S_1^{\,2}$, $S_2^{\,2}$, $S^2$,
and $S_z$, then
\begin{eqnarray}
S_1^{\,2}\, \chi_{s_1,s_2;s,m_s}^{(2)}&=& s_1\,(s_1+1)\,\hbar^2
\,\chi_{s_1,s_2;s,m_s}^{(2)},\\[0.5ex]
S_2^{\,2}\, \chi_{s_1,s_2;s,m_s}^{(2)}&=& s_2\,(s_2+1)\,\hbar^2
\,\chi_{s_1,s_2;s,m_s}^{(2)},\\[0.5ex]
S^{2}\, \chi_{s_1,s_2;s,m_s}^{(2)}&=& s\,(s+1)\,\hbar^2
\,\chi_{s_1,s_2;s,m_s}^{(2)},\\[0.5ex]
S_z\, \chi_{s_1,s_2;s,m_s}^{(2)}&=& m_s\,\hbar
\,\chi_{s_1,s_2;s,m_s}^{(2)}.
\end{eqnarray}
Of course, since both particles have spin one-half, $s_1=s_2=1/2$, and
$s_{1z}, s_{2z}=\pm 1/2$. Furthermore, by analogy with previous
analysis,
\begin{equation}
m_s = m_{s_1}+ m_{s_2}.
\end{equation}

Now, we saw, in the previous section, that when spin $l$ is added
to spin one-half then the possible values of the total angular momentum
quantum number are $j=l\pm 1/2$. By analogy, when spin one-half
is added to spin one-half then the possible values of the
total spin quantum number are $s=1/2\pm 1/2$. In other words,
when two spin one-half particles are combined, we either obtain 
a state with overall spin $s=1$, or a state with overall spin $s=0$. To be more exact, there are
 three  possible $s=1$ states (corresponding to $m_s=-1$, 0, 1), and
one possible $s=0$ state (corresponding to $m_s=0$). The three $s=1$ states
are generally known as the {\em triplet}\/ states, whereas the
$s=0$ state is known as the {\em singlet}\/ state.

\begin{table}\centering
\begin{tabular}{c|cccc|c}
&$-1/2, -1/2$& $-1/2, 1/2$&$1/2,-1/2$&$1/2,1/2$&$m_{s_1},m_{s_2}$\\[0.5ex]\hline
$1, -1$&${\scriptstyle 1}$&&&&\\[0.5ex]
$1, 0$&&${\scriptstyle 1/\sqrt{2}}$&${\scriptstyle 1/\sqrt{2}}$&&\\[0.5ex]
$0, 0$&&${\scriptstyle 1/\sqrt{2}}$&${\scriptstyle -1/\sqrt{2}}$&&\\[0.5ex]
$1, 1$&&&&${\scriptstyle 1}$&\\
\hline
$s, m_s$&&&&&
\end{tabular}
\caption{\em Clebsch-Gordon coefficients for adding spin one-half to
spin one-half. Only non-zero coefficients are shown.}\label{t4}
\end{table}

The Clebsch-Gordon coefficients for adding spin one-half to
spin one-half can easily be inferred from Table~\ref{t2} (with $l=1/2$),
and are listed in Table~\ref{t4}. It follows from this table that the
three triplet  states are:
\begin{eqnarray}
\chi^{(2)}_{1,-1} &=& \chi^{(1)}_{-1/2,-1.2},\\[0.5ex]
\chi^{(2)}_{1,0} &=& \frac{1}{\sqrt{2}}\left(\chi^{(1)}_{-1/2,1/2}+ \chi^{(1)}_{1/2,-1/2}\right),\\[0.5ex]
\chi^{(2)}_{1,1} &=& \chi^{(1)}_{1/2,1/2},
\end{eqnarray}
where $\chi^{(2)}_{s,m_s}$ is shorthand for $\chi^{(2)}_{s_1,s_2;s,m_s}$,
{\em etc.} Likewise, the singlet
state is written:
\begin{equation}
\chi^{(2)}_{0,0} =  \frac{1}{\sqrt{2}}\left(\chi^{(1)}_{-1/2,1/2}-\chi^{(1)}_{1/2,-1/2}\right).
\end{equation}

\subsubsection*{Exercises}
{\small
\begin{enumerate}
\item An electron in a hydrogen atom occupies the combined spin
and position state
$$
R_{2,1}\,\left(\sqrt{1/3}\,Y_{1,0}\,\chi_+ + \sqrt{2/3}\,Y_{1,1}\,\chi_-\right).
$$
\begin{enumerate}
\item What values would a measurement of $L^2$ yield, and with
what probabilities?
\item Same for $L_z$.
\item Same for $S^2$.
\item Same for $S_z$.
\item Same for $J^2$.
\item Same for $J_z$.
\item What is the probability density for finding the electron at
$r$, $\theta$, $\phi$?
\item What is the probability density for finding the electron in the
spin up state (with respect to the $z$-axis) at radius $r$?
\end{enumerate}


\item In a low energy neutron-proton system (with zero orbital angular
momentum) the potential energy is given by
$$
V(r) = V_1(r) + V_2(r)\left(3\,\frac{(\bsigma_1\cdot{\bf r})\,(\bsigma_2\cdot
{\bf r})}{r^2} -\bsigma_1\cdot\bsigma_2\right) + V_3(r)\,\bsigma_1\cdot\bsigma_2,
$$
where $\bsigma_1$ denotes the vector of the Pauli matrices of the neutron,
and $\bsigma_2$ denotes the vector of the Pauli matrices of the proton. Calculate
the potential energy for the neutron-proton system:
\begin{enumerate}
\item In the spin singlet state.
\item In the spin triplet state.
\end{enumerate}


\item Consider two electrons in a spin singlet state.
\begin{enumerate}
\item If a measurement of the spin of one of the electrons shows that it
is in the state with $S_z=\hbar/2$, what is the probability that a
measurement of the $z$-component of the spin of the
other electron yields $S_z=\hbar/2$?
\item If a measurement of the spin of one of the electrons shows
that it is in the state with $S_y=\hbar/2$, what is the probability that a
measurement of the $x$-component of the spin of the
other electron yields $S_x=-\hbar/2$?
\end{enumerate}
Finally, if electron 1 is in a spin state described by $\cos\alpha_1\,\chi_+
+ \sin\alpha_1\,{\rm e}^{\,{\rm i}\,\beta_1}\,\chi_-$, and
electron 2 is in a spin state described by  $\cos\alpha_2\,\chi_+
+ \sin\alpha_2\,{\rm e}^{\,{\rm i}\,\beta_2}\,\chi_-$, what is
the probability that the two-electron spin state is a triplet state?
\end{enumerate}

}