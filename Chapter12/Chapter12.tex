\chapter{Time-Independent Perturbation Theory}\label{stimin}
\section{Introduction}
Consider the following very commonly occurring problem. The Hamiltonian of a
quantum mechanical system is written
\begin{equation}
H = H_0 + H_1.
\end{equation} 
Here, $H_0$ is a simple Hamiltonian whose eigenvalues and eigenstates
are known {\em exactly}. $H_1$ introduces some interesting additional 
physics into the problem, but is sufficiently complicated that when
we add it to $H_0$ we can no longer find the exact energy eigenvalues
and eigenstates. However, $H_1$ can, in some sense (which we shall
specify more precisely later on), be regarded as being {\em small}\/ compared
to $H_0$. Can we find approximate eigenvalues and eigenstates
of the modified Hamiltonian, $H_0+H_1$, by performing some
sort of perturbation expansion about the eigenvalues and eigenstates of the original
Hamiltonian, $H_0$? Let us investigate.

Incidentally, in this chapter, we shall only discuss so-called {\em time-independent
perturbation theory}, in which the modification to the Hamiltonian,
$H_1$, has no explicit dependence on time. It is also assumed
that the unperturbed Hamiltonian, $H_0$, is time-independent.

\section{Improved Notation}
Before commencing our investigation, it is helpful to introduce some
improved notation. Let the $\psi_i$ be a complete set of eigenstates
of the Hamiltonian, $H$, corresponding to the eigenvalues $E_i$:
{\em i.e.},
\begin{equation}
H\,\psi_i = E_i\,\psi_i.
\end{equation}
Now, we expect the $\psi_i$ to be orthonormal (see Sect.~\ref{seig}). 
In one dimension, this implies that
\begin{equation}\label{e12.1}
\int_{-\infty}^\infty \psi_i^\ast\,\psi_j\,dx = \delta_{ij}.
\end{equation}
In three dimensions (see Cha.~\ref{sthree}), the above expression generalizes to
\begin{equation}\label{e12.2}
\int_{-\infty}^\infty \int_{-\infty}^\infty \int_{-\infty}^\infty \psi_i^\ast\,\psi_j\,dx\,dy\,dz = \delta_{ij}.
\end{equation}
Finally, if the $\psi_i$ are spinors (see Cha.~\ref{sspin}) then
we have
\begin{equation}\label{e12.3}
\psi_i^\dag\,\psi_j = \delta_{ij}.
\end{equation}
The generalization to the case where $\psi$ is a product of a regular
wavefunction and a spinor is fairly obvious. We can represent all
of the above possibilities by writing
\begin{equation}
\langle \psi_i|\psi_j\rangle \equiv \langle i|j\rangle = \delta_{ij}.
\end{equation}
Here, the term in angle brackets represents the integrals in Eqs.~(\ref{e12.1})
and (\ref{e12.2}) in one- and three-dimensional regular space, respectively,
and the spinor product (\ref{e12.3}) in spin-space. The advantage of
our new notation is its great generality: {\em i.e.}, it
can deal with one-dimensional wavefunctions, three-dimensional wavefunctions,
spinors, {\em etc.}

Expanding a general wavefunction, $\psi_a$, in terms of the energy
eigenstates, $\psi_i$, we obtain
\begin{equation}\label{e12.7}
\psi_a = \sum_i c_i\,\psi_i.
\end{equation}
In one dimension, the expansion coefficients take the form (see Sect.~\ref{seig})
\begin{equation}
c_i = \int_{-\infty}^\infty\psi_i^\ast\,\psi_a\,dx,
\end{equation}
whereas in three dimensions we get
\begin{equation}
c_i = \int_{-\infty}^\infty\int_{-\infty}^\infty\int_{-\infty}^\infty\psi_i^\ast\,\psi_a\,dx\,dy\,dz.
\end{equation}
Finally, if $\psi$ is a spinor then we have
\begin{equation}
c_i = \psi_i^\dag\,\psi_a.
\end{equation}
We can represent all of the above possibilities by
writing
\begin{equation}
c_i =\langle\psi_i|\psi_a\rangle\equiv \langle i|a\rangle.
\end{equation}
The expansion (\ref{e12.7}) thus becomes
\begin{equation}\label{e12.13a}
\psi_a = \sum_i\langle\psi_i|\psi_a\rangle\,\psi_i\equiv \sum_i \langle i|a\rangle\,\psi_i.
\end{equation}
Incidentally, it follows that
\begin{equation}
\langle i|a\rangle^\ast=\langle a| i\rangle.
\end{equation}

Finally, if $A$ is a general operator, and the wavefunction
$\psi_a$ is expanded in the manner shown in Eq.~(\ref{e12.7}), then the expectation value of
$A$ is written (see Sect.~\ref{seig})
\begin{equation}\label{e12.14}
\langle A\rangle = \sum_{i,j} c_i^\ast\,c_j\,A_{ij}.
\end{equation}
Here, the $A_{ij}$ are unsurprisingly known as the {\em matrix
elements}\/ of $A$.
In one dimension, the matrix elements take the form
\begin{equation}
A_{ij} = \int_{-\infty}^\infty\psi_i^\ast\,A\,\psi_j\,dx,
\end{equation}
whereas in three dimensions we get
\begin{equation}
A_{ij} = \int_{-\infty}^\infty\int_{-\infty}^\infty\int_{-\infty}^\infty\psi_i^\ast\,A\,\psi_j\,dx\,dy\,dz.
\end{equation}
Finally, if $\psi$ is a spinor then we have
\begin{equation}
A_{ij}=\psi_i^\dag\,A\,\psi_j.
\end{equation}
We can represent all of the above possibilities by
writing
\begin{equation}
A_{ij}=\langle \psi_i|A|\psi_j\rangle \equiv \langle i|A|j\rangle.
\end{equation}
The expansion (\ref{e12.14}) thus becomes
\begin{equation}\label{e12.20a}
\langle A\rangle \equiv\langle a|A|a\rangle= \sum_{i,j} \langle a|i\rangle
\langle i|A|j\rangle \langle j|a\rangle.
\end{equation}
Incidentally, it follows that [see Eq.~(\ref{e5.48})]
\begin{equation}
 \langle i|A|j\rangle^\ast=\langle j| A^\dag|i\rangle.
\end{equation}
Finally, it is clear from Eq.~(\ref{e12.20a}) that
\begin{equation}\label{e12.20}
\sum_{i} |i\rangle \langle i| \equiv 1,
\end{equation}
where the $\psi_i$ are a {\em complete}\/ set of eigenstates, and 1 is the
identity operator.

\section{Two-State System}
Consider the simplest possible non-trivial quantum mechanical system.
In such a system,
there are only {\em two}\/ independent eigenstates
of the unperturbed Hamiltonian: {\em i.e.},
\begin{eqnarray}\label{e12.21}
H_0\,\psi_1 &=& E_1\,\psi_1,\\[0.5ex]
H_0\,\psi_2&=&E_2\,\psi_2.\label{e12.22}
\end{eqnarray}
It is assumed that these states, and their associated eigenvalues, are known.
We also expect the states to be orthonormal, and to form a complete set.

Let us now try to solve the modified energy eigenvalue problem
\begin{equation}\label{e12.23}
(H_0+H_1)\,\psi_E = E\,\psi_E.
\end{equation}
We can, in fact, solve this problem exactly. Since the eigenstates
of $H_0$ form a complete set, we can write [see Eq.~(\ref{e12.13a})]
\begin{equation}\label{e12.24}
\psi_E = \langle 1|E\rangle\,\psi_1 + \langle 2|E\rangle\,\psi_2.
\end{equation}
It follows from (\ref{e12.23}) that
\begin{equation}\label{e12.25}
\langle i|H_0 + H_1|E\rangle = E\,\langle i|E\rangle,
\end{equation}
where $i=1$ or $2$. Equations (\ref{e12.21}), (\ref{e12.22}), (\ref{e12.24}),
(\ref{e12.25}), and the orthonormality condition
\begin{equation}
\langle i|j\rangle = \delta_{ij},
\end{equation}
yield two coupled equations which can be written
in matrix form:
\begin{eqnarray}\label{e12.27}
\left(\begin{array}{cc}E_1-E+e_{11}& e_{12}\\[0.5ex]
e_{12}^\ast& E_2-E+e_{22}\end{array}\right)\left(
\begin{array}{c}\langle 1|E\rangle\\[0.5ex]
\langle 2|E\rangle\end{array}\right)=\left(
\begin{array}{c} 0\\[0.5ex]
0\end{array}\right),
\end{eqnarray}
where
\begin{eqnarray}
e_{11} &=& \langle 1|H_1|1\rangle,\\[0.5ex]
e_{22}&=&\langle 2|H_1|2\rangle,\\[0.5ex]
e_{12}&=&\langle 1|H_1|2\rangle = \langle 2|H_1|1\rangle^\ast.
\end{eqnarray}
Here, use has been made of the fact that $H_1$ is an Hermitian operator.

Consider the special (but not uncommon) case of a perturbing Hamiltonian
whose diagonal matrix elements are zero, so that
\begin{equation}
e_{11}= e_{22} = 0.
\end{equation}
The solution of  Eq.~(\ref{e12.27}) (obtained by setting
the determinant of the matrix to zero) is
\begin{equation}
E = \frac{(E_1+E_2)\pm\sqrt{(E_1-E_2)^2 + 4\,|e_{12}|^2}}
{2}.
\end{equation}
Let us expand in the supposedly small parameter
\begin{equation}
\epsilon = \frac{|e_{12}|}{|E_1-E_2|}.
\end{equation}
We obtain
\begin{equation}
E \simeq \frac{1}{2}\,(E_1+E_2) \pm \frac{1}{2}\,(E_1-E_2)(1+2\,\epsilon^2+\cdots).
\end{equation}
The above expression yields the modification of the
energy eigenvalues due to the perturbing Hamiltonian:
\begin{eqnarray}
E_1' &=& E_1 + \frac{|e_{12}|^2}{E_1-E_2}+ \cdots,\\[0.5ex]
E_2' &=& E_2 - \frac{|e_{12}|^2}{E_1-E_2}+\cdots.
\end{eqnarray}
Note that $H_1$ causes the upper eigenvalue to rise, and the lower to fall.
It is easily demonstrated that the modified eigenstates take the form
\begin{eqnarray}
\psi_1' &=&\psi_1+ \frac{e_{12}^\ast}{E_1-E_2}\,\psi_2+ \cdots,\\[0.5ex]
\psi_2'&=& \psi_2 - \frac{e_{12}}{E_1-E_2}\,\psi_1+ \cdots.
\end{eqnarray}
Thus, the modified energy eigenstates consist of one of the
unperturbed eigenstates, plus a slight admixture of the other.
Now our expansion procedure is only valid when $\epsilon\ll 1$. 
This suggests that the condition for the validity of the perturbation
method as a whole is
\begin{equation}
|e_{12}|\ll |E_1-E_2|.
\end{equation}
In other words, when we say that $H_1$ needs to be small compared to
$H_0$, what we are really saying is that the above inequality must be
satisfied.

\section{Non-Degenerate Perturbation Theory}\label{e12.4}
Let us now generalize our perturbation analysis to deal
with systems possessing more than two energy eigenstates. 
Consider a system in which the energy
eigenstates of the unperturbed Hamiltonian, $H_0$, are denoted
\begin{equation}
H_0\,\psi_n = E_n\,\psi_n,
\end{equation}
where $n$ runs from 1 to $N$. The eigenstates are assumed to
be orthonormal, so that
\begin{equation}
\langle m|n\rangle = \delta_{nm},
\end{equation}
and to form a complete set. Let us now try to
solve the energy eigenvalue problem for the perturbed Hamiltonian:
\begin{equation}
(H_0+H_1)\,\psi_E = E\,\psi_E.
\end{equation}
If follows that
\begin{equation}
\langle m|H_0+H_1|E\rangle = E\,\langle m |E\rangle,
\end{equation}
where $m$ can take any value from  1 to $N$. Now, we can express
$\psi_E$ as a linear superposition of the unperturbed energy eigenstates:
\begin{equation}
\psi_E = \sum_k \langle k|E\rangle\,\psi_k,
\end{equation}
where $k$ runs from 1 to $N$. We can combine the above
equations to give
\begin{equation}\label{e12.45}
(E_m-E+e_{mm})\,\langle m|E\rangle + \sum_{k\neq m}
e_{mk}\,\langle k|E\rangle = 0,
\end{equation}
where
\begin{equation}
e_{mk} =\langle m|H_1| k\rangle.
\end{equation}

Let us now develop our perturbation expansion. We assume that 
\begin{equation}
\frac{e_{mk}}{E_m-E_k} \sim {\cal O}(\epsilon)
\end{equation}
for all $m\neq k$, where $\epsilon\ll 1$ is our expansion parameter.
We also assume that
\begin{equation}
\frac{e_{mm}}{E_m}\sim {\cal O}(\epsilon)
\end{equation}
for all $m$. Let us search for a modified version of the $n$th unperturbed
energy eigenstate for which
\begin{equation}
E = E_n + {\cal O}(\epsilon),
\end{equation}
and
\begin{eqnarray}
\langle n|E\rangle &=& 1,\\[0.5ex]
\langle m|E\rangle&=&{\cal O}(\epsilon)
\end{eqnarray}
for $m\neq n$. Suppose that we write out Eq.~(\ref{e12.45}) for $m\neq n$,
neglecting terms which are ${\cal O}(\epsilon^2)$ according to our expansion
scheme. We find that
\begin{equation}
(E_m-E_n)\,\langle m|E\rangle + e_{mn} \simeq 0,
\end{equation}
giving
\begin{equation}
\langle m|E\rangle \simeq - \frac{e_{mn}}{E_m-E_n}.
\end{equation}
Substituting the above expression into Eq.~(\ref{e12.45}),
evaluated for $m=n$, and neglecting ${\cal O}(\epsilon^3)$ terms, we obtain
\begin{equation}
(E_n-E+e_{nn})-\sum_{k\neq n}\frac{|e_{nk}|^2}{E_k-E_n} \simeq 0.
\end{equation}
Thus, the modified $n$th energy eigenstate possesses an eigenvalue
\begin{equation}\label{e12.56}
E_n' = E_n + e_{nn} + \sum_{k\neq n}\frac{|e_{nk}|^2}{E_n-E_k}
+ {\cal O}(\epsilon^3),
\end{equation}
and a wavefunction
\begin{equation}\label{e12.57}
\psi_n' = \psi_n + \sum_{k\neq n} \frac{e_{kn}}{E_n-E_k}\,\psi_k + {\cal O}(\epsilon^2).
\end{equation}
Incidentally, it is easily demonstrated that the modified eigenstates remain orthonormal
to ${\cal O}(\epsilon^2)$. 

\section{Quadratic Stark Effect}\label{s12.5}
Suppose that a hydrogen atom is
subject to a uniform external electric field, of magnitude $|{\bf E}|$,  directed along the $z$-axis. The Hamiltonian of the system can be split into two parts. Namely, the unperturbed
Hamiltonian,
\begin{equation}\label{e12.58}
H_0 =\frac{p^2}{2\,m_e} -\frac{e^2}{4\pi\epsilon_0\,r},
\end{equation}
and the perturbing Hamiltonian
\begin{equation}
H_1 = e\,|{\bf E}|\,z.
\end{equation}

 Note that the electron spin is irrelevant to this problem 
(since the spin operators all commute with $H_1$), so we can ignore
the spin degrees of freedom of the system. Hence, the energy eigenstates
of the unperturbed Hamiltonian are characterized by three quantum
numbers---the radial quantum number $n$, and the two angular
quantum numbers $l$ and $m$ (see Cha.~\ref{scent}). Let us
denote these states as the $\psi_{nlm}$, and let their corresponding energy eigenvalues
be the $E_{nlm}$. According to the analysis in the previous section,  the change in energy of the eigenstate characterized by the quantum
numbers $n,l,m$ in  the presence of a {\em small}\/ electric field is given by
\begin{eqnarray}
\Delta E_{nlm}&=& e\,|{\bf E}|\,\langle n,l,m|z|n,l,m\rangle\nonumber\\[0.5ex]
&&+ e^2\,|{\bf E}|^2\sum_{n',l',m'\neq n,l,m}\frac{|\langle n,l,m|z|n',l',m'\rangle|^2}{E_{nlm}-E_{n'l'm'}}.\label{e12.59}
\end{eqnarray}
This energy-shift is known as the {\em Stark effect}.

The sum on the right-hand side of the above equation seems very complicated. However, it turns out that most of the terms in this sum
are zero. This follows because the matrix elements $\langle n,l,m|z|n',l',m'\rangle$
are zero for virtually all choices of the two sets of quantum number, $n,l,m$ and
$n',l',m'$. Let us try to find a set of rules which determine when these
matrix elements are non-zero. These rules are usually referred to as the
{\em selection rules}\/ for the problem in hand.

Now, since [see Eq.~(\ref{e8.3})]
\begin{equation}
L_z = x\,p_y - y\,p_x,
\end{equation}
it follows that [see Eqs.~(\ref{commxx})--(\ref{commxp})]
\begin{equation}
[L_z,z] = 0.
\end{equation}
Thus,
\begin{eqnarray}
\langle n,l,m|[L_z,z]|n',l',m'\rangle& =& \langle n,l,m|L_z\,z-z\,L_z|n',l',m'\rangle\nonumber\\[0.5ex]
&=& \hbar\,(m-m')\,\langle n,l,m|z|n',l',m'\rangle = 0,
\end{eqnarray}
since $\psi_{nlm}$ is, by definition, an eigenstate of $L_z$ corresponding
to the eigenvalue $m\,\hbar$. Hence, it is clear, from the above equation, 
that one of the selection rules is that the matrix element $\langle n,l,m|z|n',l',m'\rangle$ is zero unless
\begin{equation}\label{e12.63}
m' = m.
\end{equation}

Let us now determine the selection rule for $l$. We have
\begin{eqnarray}
[L^2,z]&=& [L_x^{\,2},z] + [L_y^{\,2},z]\nonumber\\[0.5ex]
& =& L_x\,[L_x,z] + [L_x,z]\,L_x + L_y\,[L_y,z]+[L_y,z]\,L_y\nonumber\\[0.5ex]
&=& {\rm i}\,\hbar\,(-L_x\,y-y\,L_x+L_y\,x+x\,L_y)\nonumber\\[0.5ex]
&=& 2\,{\rm i}\,\hbar\,(L_y\,x -L_x\,y + {\rm i}\,\hbar\,z)\nonumber\\[0.5ex]
&=& 2\,{\rm i}\,\hbar\,(L_y\,x - y\,L_x) = 2\,{\rm i}\,\hbar\,(x\,L_y-L_x\,y),
\end{eqnarray}
where use has been made of Eqs.~(\ref{commxx})--(\ref{commxp}), (\ref{e8.1})--(\ref{e8.3}), and (\ref{e8.10}).
Thus,
\begin{eqnarray}
[L^2,[L^2,z]]&=& 2\,{\rm i}\,\hbar\,\left(L^2, L_y\,x-L_x\,y + {\rm i}\,\hbar\,z\right)\nonumber\\[0.5ex]
&=& 2\,{\rm i}\,\hbar\,\left(L_y\,[L^2,x] - L_x\,[L^2,y] + {\rm i}\,\hbar\,[L^2,z]\right)\nonumber\\[0.5ex]
&=&-4\,\hbar^2\,L_y\,(y\,L_z-L_y\,z) + 4\,\hbar^2\,L_x\,(L_x\,z-x\,L_z)\nonumber\\[0.5ex]
&&-2\,\hbar^2\,(L^2\,z-z\,L^2),
\end{eqnarray}
which reduces to
\begin{eqnarray}
[L^2,[L^2,z]]&=& -\hbar^2\,\left\{4\,(L_x\,x+ L_y\,y+L_z\,z)\,L_z
-4\,(L_x^{\,2}+L_y^{\,2}+L_z^{\,2})\,z\right.\nonumber\\[0.5ex]
&&\left.+2\,(L^2\,z-z\,L^2)\right\}\nonumber\\[0.5ex]
&=&  -\hbar^2\,\left\{4\,(L_x\,x+ L_y\,y+L_z\,z)\,L_z-2\,(L^2\,z+z\,L^2)\right\}.
\end{eqnarray}
However, it is clear from Eqs.~(\ref{e8.1})--(\ref{e8.3}) that
\begin{equation}
L_x\,x+L_y\,y+L_z\,z = 0.
\end{equation}
Hence, we obtain
\begin{equation}
[L^2,[L^2,z]] = 2\,\hbar^2\,(L^2\,z+z\,L^2).
\end{equation}
Finally, the above expression expands to give
\begin{equation}\label{e12.69}
L^4\,z-2\,L^2\,z\,L^2 + z\,L^4 - 2\,\hbar^2\,(L^2\,z+z\,L^2) = 0.
\end{equation}

Equation (\ref{e12.69}) implies that
\begin{equation}
\langle n,l,m|L^4\,z-2\,L^2\,z\,L^2 + z\,L^4 - 2\,\hbar^2\,(L^2\,z+z\,L^2) |n',l',m\rangle = 0.
\end{equation}
Since, by definition, $\psi_{nlm}$ is an eigenstate of $L^2$ corresponding
to the eigenvalue $l\,(l+1)\,\hbar^2$, this expression yields
\begin{eqnarray}
\left\{l^2\,(l+1)^2-2\,l\,(l+1)\,l'\,(l'+1) + l'^2\,(l'+1)^2\right.&&\nonumber\\[0.5ex]
\left.-2\,l\,(l+1) - 2\,l'\,(l'+1)\right\}\langle n,l,m|z|n',l',m\rangle& =& 0,
\end{eqnarray}
which reduces to
\begin{equation}
(l+l'+2)\,(l+l')\,(l-l'+1)\,(l-l'-1)\,\langle n,l,m|z|n',l',m\rangle = 0.
\end{equation}
According to the above formula, the matrix element $\langle n,l,m|z|n',l',m\rangle$ vanishes unless $l=l'=0$ or $l'=l\pm 1$. [Of course, the
factor $l+l'+2$, in the above equation, can never be zero, since $l$ and $l'$ can never be negative.]
Recall, however, from Cha.~\ref{scent}, that an $l=0$ wavefunction
is {\em spherically symmetric}. It, therefore, follows, from symmetry,
that the matrix element $\langle n,l,m|z|n',l',m\rangle$ is zero when
$l=l'=0$. In conclusion, the selection rule for $l$ is that the
matrix element $\langle n,l,m|z|n',l',m\rangle$ is zero unless
\begin{equation}\label{e12.73}
l' = l\pm 1.
\end{equation}

Application of the selection rules (\ref{e12.63}) and (\ref{e12.73}) to
Eq.~(\ref{e12.59}) yields
\begin{equation}\label{e12.74}
\Delta E_{nlm} = e^2\,|{\bf E}|^2\sum_{n',l'=l\pm 1}
\frac{|\langle n,l,m|z|n',l',m\rangle|^2}{E_{nlm}-E_{n'l'm}}.
\end{equation}
Note that,  according to the selection rules, all of the terms in Eq.~(\ref{e12.59}) which vary linearly with
the electric field-strength vanish.
Only those terms which vary quadratically with the field-strength
survive. Hence, this type of energy-shift of an atomic state in the
presence of a small electric field is known as the {\em quadratic}\/
Stark effect. Now, the {\em electric polarizability}\/ of an atom is
defined in terms of the energy-shift of the atomic state as follows:
\begin{equation}
\Delta E = -\frac{1}{2}\,\alpha\,|{\bf E}|^2.
\end{equation}
Hence, we can write
\begin{equation}\label{e12.76}
\alpha_{nlm} = 2\,e^2\sum_{n',l'=l\pm 1}
\frac{|\langle n,l,m|z|n',l',m\rangle|^2}{E_{n'l'm}-E_{nlm}}.
\end{equation}

Unfortunately, there is one fairly obvious problem with Eq.~(\ref{e12.74}). Namely,  it predicts  an {\em infinite}\/ energy-shift if there exists some non-zero
matrix element $\langle n,l,m|z|n',l',m\rangle$ which couples
two {\em degenerate}\/ unperturbed energy eigenstates: {\em i.e.},
if $\langle n,l,m|z|n',l',m\rangle\neq 0$ and $E_{nlm}=E_{n'l'm}$.
Clearly,  our perturbation method breaks down completely in this
situation. Hence, we conclude that Eqs.~(\ref{e12.74}) and (\ref{e12.76})
are only applicable to cases where the coupled eigenstates are
{\em non-degenerate}. For this reason, the type of perturbation
theory employed here is known as {\em non-degenerate perturbation theory}.
Now, the unperturbed eigenstates of a hydrogen atom have energies which
only depend on the radial quantum number $n$ (see Cha.~\ref{scent}). It follows that
we can only apply the above results to the $n=1$ eigenstate (since
for $n>1$ there will be coupling to degenerate eigenstates with
the same value of $n$ but different values of $l$).  

Thus, according to non-degenerate perturbation theory, the polarizability of the ground-state  ({\em i.e.}, $n=1$) of a hydrogen atom is given by
\begin{equation}
\alpha = 2\,e^2\sum_{n>1}\frac{|\langle 1,0,0|z|n,1,0\rangle|^2}{E_{n00}-E_{100}}.
\end{equation}
Here, we have made use of the fact that $E_{n10}=E_{n00}$. The sum in the above expression can be
evaluated approximately by noting that (see Sect.~\ref{s10.4})
\begin{equation}
E_{n00} = -\frac{e^2}{8\pi\,\epsilon_0\,a_0\,n^2},
\end{equation}
where
\begin{equation}
a_0 = \frac{4\pi\epsilon_0\,\hbar^2}{m_e\,e^2}
\end{equation}
is the Bohr radius. Hence, we can write
\begin{equation}
E_{n00}-E_{100} \geq E_{200}-E_{100} = \frac{3}{4}\,\frac{e^2}{8\pi\,\epsilon_0\,a_0},
\end{equation}
which implies that
\begin{equation}
\alpha < \frac{16}{3}\,4\pi\epsilon_0\,a_0\,\sum_{n>1}
|\langle 1,0,0|z|n,1,0\rangle|^2.
\end{equation}
However, [see Eq.~(\ref{e12.20})]
\begin{eqnarray}
\sum_{n>1}|\langle 1,0,0|z|n,1,0\rangle|^2
&=& \sum_{n>1}\langle 1,0,0|z|n,1,0\rangle\,\langle n,1,0|z|1,0,0\rangle\nonumber\\[0.5ex]
&=&\sum_{n',l',m'}\langle 1,0,0|z|n',l',m'\rangle\,\langle n',l',m'|z|1,0,0\rangle\nonumber\\[0.5ex]
&=&\langle 1,0,0|z^2|1,0,0\rangle = \frac{1}{3}\,\langle 1,0,0|r^2|1,0,0\rangle,
\end{eqnarray}
where we have made use of the selection rules,  the fact that the
$\psi_{n',l',m'}$ form a complete set, and the fact the the
ground-state of hydrogen is spherically symmetric. Finally, it
follows from Eq.~(\ref{e9.73}) that
\begin{equation}
\langle 1,0,0|r^2|1,0,0\rangle = 3\,a_0^{\,2}.
\end{equation}
Hence, we conclude that
\begin{equation}
\alpha < \frac{16}{3}\,4\pi\epsilon_0\,a_0^{\,3}\simeq 5.3\,\,4\pi\epsilon_0\,a_0^{\,3}.
\end{equation}
The exact result (which can be obtained by solving Schr\"{o}dinger's equation
in parabolic coordinates) is
\begin{equation}
\alpha = \frac{9}{2}\,4\pi\epsilon_0\,a_0^{\,3} = 4.5\,\,4\pi\epsilon_0\,a_0^{\,3}.
\end{equation}

\section{Degenerate Perturbation Theory}\label{s12.6}
Let us, rather naively, investigate the Stark effect in an excited ({\em i.e.}, 
$n>1$) state of the hydrogen atom using standard non-degenerate
perturbation theory. We can write
\begin{equation}
H_0\,\psi_{nlm} = E_n\,\psi_{nlm},
\end{equation}
since the energy eigenstates of the unperturbed Hamiltonian only depend
on the quantum number $n$.  Making use of the selection rules
(\ref{e12.63}) and (\ref{e12.73}), non-degenerate perturbation theory
yields the following expressions for the perturbed energy levels
and eigenstates [see Eqs.~(\ref{e12.56}) and (\ref{e12.57})]:
\begin{equation}\label{e12.88}
E_{nl}' = E_n + e_{nlnl} + \sum_{n',l'=l\pm 1}\frac{|e_{n'l'nl}|^2}{E_n-E_{n'}},
\end{equation}
and
\begin{equation}\label{e12.89}
\psi'_{nlm} = \psi_{nlm} + \sum_{n',l'=l\pm 1}\frac{e_{n'l'nl}}{E_n-E_{n'}}\,\psi_{n'l'm},
\end{equation}
where
\begin{equation}
e_{n'l'nl} = \langle n',l',m|H_1|n,l,m\rangle.
\end{equation}
Unfortunately, if $n>1$ then the summations in the above expressions
are not well-defined, because there exist non-zero matrix elements, $e_{nl'nl}$,  which couple degenerate eigenstates: {\em i.e.}, there exist non-zero matrix elements which couple states with the same value of $n$, but
different values of $l$. These particular matrix elements give rise to
singular factors $1/(E_n-E_n)$ in the summations. This does not occur if
$n=1$ because, in this case, the selection rule $l'=l\pm 1$, and the
fact that $l=0$ (since $0\leq l < n$), only allow $l'$ to take the single value 1.
Of course, there is no $n=1$ state with $l'=1$.
Hence, there is only one coupled state corresponding to the
eigenvalue $E_1$. Unfortunately, if $n>1$ then there are multiple
coupled states corresponding to the eigenvalue $E_n$. 

Note that our problem would disappear if the matrix elements of the
perturbed Hamiltonian corresponding to the same value of $n$, but
different values of $l$, were all zero: {\em i.e.}, if
\begin{equation}\label{e12.91}
\langle n,l',m|H_1|n,l,m\rangle = \lambda_{nl}\,\delta_{ll'}.
\end{equation}
In this case, all of the singular
terms in Eqs.~(\ref{e12.88}) and (\ref{e12.89}) would reduce to zero.
Unfortunately, the above equation is not satisfied. Fortunately, we can
always redefine the unperturbed eigenstates corresponding to the
eigenvalue $E_n$ in such a manner that Eq.~(\ref{e12.91}) is satisfied.
Suppose that there are $N_n$ coupled eigenstates belonging
to the eigenvalue $E_n$. Let us define $N_n$ new states
which are linear combinations of our $N_n$ original degenerate eigenstates:
\begin{equation}
\psi_{nlm}^{(1)}= \sum_{k=1,N_n}\langle n,k,m|n,l^{(1)},m\rangle\,\psi_{nkm}.
\end{equation}
Note that these new states are also degenerate energy eigenstates of the
unperturbed Hamiltonian, $H_0$,  corresponding to the eigenvalue $E_n$. The
$\psi_{nlm}^{(1)}$ are chosen in such a manner that they are also
eigenstates of the  perturbing Hamiltonian, $H_1$:
{\em i.e.}, they are {\em simultaneous eigenstates}\/ of $H_0$ and $H_1$. Thus,
\begin{equation}\label{e12.93}
H_1\,\psi_{nlm}^{(1)} = \lambda_{nl}\,\psi_{nlm}^{(1)}.
\end{equation}
The $\psi_{nlm}^{(1)}$ are also chosen so as to be orthonormal:
{\em i.e.},
\begin{equation}
\langle n,l'^{(1)},m|n,l^{(1)},m\rangle = \delta_{ll'}.
\end{equation}
It follows that
\begin{equation}
\langle n,l'^{(1)},m|H_1|n,l^{(1)},m\rangle =\lambda_{nl}\, \delta_{ll'}.
\end{equation}
Thus, if we use the new eigenstates, instead of the old ones, then we
can employ Eqs.~(\ref{e12.88}) and (\ref{e12.89}) directly, since all
of the singular terms vanish. The only remaining difficulty
is to determine the new eigenstates in terms of the original ones.

Now [see Eq.~(\ref{e12.20})]
\begin{equation}
\sum_{l=1,N_n}|n,l,m\rangle\langle n,l,m|\equiv 1,
\end{equation}
where $1$ denotes the identity operator in the sub-space of all coupled unperturbed
eigenstates corresponding to the eigenvalue $E_n$. Using this completeness
relation, the eigenvalue equation (\ref{e12.93}) can be
transformed into a straightforward matrix equation:
\begin{equation}
\sum_{l''=1,N_n}\langle n,l',m|H_1|n,l'',m\rangle\,\langle n,l'',m|n,l^{(1)},m\rangle
= \lambda_{nl}\,\langle n,l',m|n,l^{(1)},m\rangle.
\end{equation}
This can be written more transparently as
\begin{equation}\label{e12.100}
{\bf U}\,{\bf x} = \lambda \,{\bf x},
\end{equation}
where the elements of the $N_n\times N_n$ Hermitian matrix ${\bf U}$
are
\begin{equation}
U_{jk} = \langle n,j,m|H_1|n,k,m\rangle.
\end{equation}
Provided that the determinant of ${\bf U}$ is non-zero, Eq.~(\ref{e12.100})
can always be solved to give $N_n$ eigenvalues $\lambda_{nl}$ (for $l=1$ to $N_n$), with $N_n$ corresponding eigenvectors ${\bf x}_{nl}$. The
normalized eigenvectors specify the weights of the new eigenstates in terms of the
original eigenstates: {\em i.e.}, 
\begin{equation}
({\bf x}_{nl})_k = \langle n,k,m|n,l^{(1)},m\rangle,
\end{equation}
for $k=1$ to $N_n$. In our new scheme, Eqs.~(\ref{e12.88}) and (\ref{e12.89}) yield
\begin{equation}
E_{nl}' = E_n +\lambda_{nl}+\sum_{n'\neq n,l'=l\pm 1}\frac{|e_{n'l'nl}|^2}{E_n-E_{n'}},
\end{equation}
and
\begin{equation}
\psi_{nlm}^{(1)'} = \psi_{nlm}^{(1)} + \sum_{n'\neq n,l'=l\pm 1}
\frac{e_{n'l'nl}}{E_n-E_{n'}}\,\psi_{n'l'm}.
\end{equation}
There are no singular terms in  these expressions, since the summations
are over $n'\neq n$: {\em i.e.}, they specifically exclude the problematic,
degenerate, unperturbed energy eigenstates corresponding to the eigenvalue
$E_n$. Note that the first-order energy shifts are equivalent to the
eigenvalues of the matrix equation (\ref{e12.100}).

\section{Linear Stark Effect}
Returning to the Stark effect, let us examine the effect of an external electric
field on the energy levels of the $n=2$ states of a hydrogen atom. 
There are four such states: an $l=0$ state, usually referred to as $2S$,
and three $l=1$ states (with $m=-1,0,1$), usually referred to as 2P. All
of these states possess the same unperturbed energy, $E_{200}
= -e^2/(32\pi\,\epsilon_0\,a_0)$. 
As before, the perturbing Hamiltonian is
\begin{equation}
H_1 = e\,|{\bf E}|\,z.
\end{equation}
According to the previously determined selection rules ({\em i.e.}, $m'=m$,
and $l'=l\pm1$), this Hamiltonian couples $\psi_{200}$ and $\psi_{210}$.
Hence, non-degenerate perturbation theory breaks down when applied to
these two states. On the other hand, non-degenerate perturbation
theory works fine for the $\psi_{211}$ and $\psi_{21-1}$ states,
since these are not coupled to any other $n=2$ states by the perturbing
Hamiltonian.

In order to apply perturbation theory to the $\psi_{200}$ and $\psi_{210}$ states, we have to solve the
matrix eigenvalue equation
\begin{equation}
{\bf U}\,{\bf x} = \lambda\,{\bf x},
\end{equation}
where ${\bf U}$ is the matrix of the matrix elements of $H_1$ between these states. Thus,
\begin{equation}
{\bf U} = e\,|{\bf E}|\left(\begin{array}{cc}
0& \langle 2,0,0|z|2,1,0\rangle\\[0.5ex]
\langle 2,1,0|z|2,0,0\rangle&0
\end{array}\right),
\end{equation}
where the rows and columns correspond to  $\psi_{200}$ and $\psi_{210}$, respectively. Here, we have again
made use of the selection rules, which tell us
that the matrix element of $z$ between two hydrogen atom
states is zero unless the states possess $l$ quantum numbers which differ by unity. It is easily demonstrated,
from the exact forms of the 2S  and 2P wavefunctions, that
\begin{equation}
\langle 2,0,0|z|2,1,0\rangle = \langle 2,1,0|z|2,0,0\rangle = 3\,a_0.
\end{equation}

It can be seen, by inspection, that the eigenvalues of ${\bf U}$
are $\lambda_1=3\,e\,a_0\,|{\bf E}|$ and $\lambda_2=-3\,e\,a_0\,|{\bf E}|$. The corresponding normalized eigenvectors are
\begin{eqnarray}
{\bf x}_1&=&\left(\begin{array}{c}
1/\sqrt{2}\\[0.5ex]
1/\sqrt{2}\end{array}\right),\\[0.5ex]
{\bf x}_2&=&\left(\begin{array}{c}
1/\sqrt{2}\\[0.5ex]
-1/\sqrt{2}\end{array}\right).
\end{eqnarray}
It follows that the simultaneous eigenstates of $H_0$ and $H_1$ take the form
\begin{eqnarray}
\psi_1 &=& \frac{\psi_{200} + \psi_{210}}{\sqrt{2}},\\[0.5ex]
\psi_2 &=& \frac{\psi_{200} -\psi_{210}}{\sqrt{2}}.
\end{eqnarray}
In the absence of an external electric field, both of these states possess the
same energy, $E_{200}$. The first-order energy shifts induced by
an external electric field are given by
\begin{eqnarray}
\Delta E_1 &=& +3\,e\,a_0\,|{\bf E}|,\\[0.5ex]
\Delta E_2 &=& -3\,e\,a_0\,|{\bf E}|.
\end{eqnarray}
Thus, in the presence of
an electric field, the energies of states 1 and 2 are shifted upwards and downwards, respectively, by an amount $3\,e\,a_0\,|{\bf E}|$. These states are orthogonal linear combinations of the
original $\psi_{200}$ and $\psi_{210}$ states. Note that the energy shifts
are {\em linear}\/ in the electric field-strength, so this effect---which is
known as the {\em linear}\/ Stark effect---is  much larger  than the quadratic effect described in Sect.~\ref{s12.5}. 
Note, also, that the energies of the $\psi_{211}$ and $\psi_{21-1}$ states are not affected by the electric field to first-order. Of course,
to second-order the energies of these states are shifted by an amount
which depends on the square of the electric field-strength (see Sect.~\ref{s12.5}).

\section{Fine Structure of Hydrogen}\label{s12.8}
According to special relativity, the kinetic energy ({\em i.e.}, the difference
between the total energy and the rest mass energy) of a particle
of rest mass $m$ and momentum $p$ is
\begin{equation}
T = \sqrt{p^2\,c^2+m^2\,c^4} - m\,c^2.
\end{equation}
In the non-relativistic limit $p\ll m\,c$, we can expand the square-root
in the above expression to give
\begin{equation}
T = \frac{p^2}{2\,m}\left[1- \frac{1}{4}\left(\frac{p}{m\,c}\right)^2+ 
{\cal O}\left(\frac{p}{m\,c}\right)^4\right].
\end{equation}
Hence,
\begin{equation}
T \simeq \frac{p^2}{2\,m} - \frac{p^4}{8\,m^3\,c^2}.
\end{equation}
Of course, we recognize the first term on the right-hand side of this equation
as the standard non-relativistic expression for the kinetic energy.
The second term is the lowest-order relativistic correction to this
energy. Let us consider the effect of this type of correction on the energy
levels of a hydrogen atom. So, the unperturbed Hamiltonian is
given by Eq.~(\ref{e12.58}), and the perturbing Hamiltonian
takes the form
\begin{equation}
H_1 = - \frac{p^4}{8\,m_e^{\,3}\,c^2}.
\end{equation}

Now, according to standard first-order perturbation theory (see Sect.~\ref{e12.4}), the lowest-order relativistic correction to the energy of a hydrogen atom state characterized
by the standard quantum numbers $n$, $l$, and $m$ is given by
\begin{eqnarray}
\Delta E_{nlm} &=& \langle n,l,m|H_1|n,l,m\rangle = - \frac{1}{8\,m_e^{\,3}\,c^2}\,
\langle n,l,m|p^4|n,l,m\rangle\nonumber\\[0.5ex]
&=& - \frac{1}{8\,m_e^{\,3}\,c^2}\,
\langle n,l,m|p^2\,p^2|n,l,m\rangle.
\end{eqnarray}
However, Schr\"{o}dinger's equation for a unperturbed hydrogen atom
can be written
\begin{equation}
p^2\,\psi_{n,l,m} = 2\,m_e\,(E_n-V)\,\psi_{n,l,m},
\end{equation}
where $V=-e^2/(4\pi\epsilon_0\,r)$. 
Since $p^2$ is an Hermitian operator, it follows that
\begin{eqnarray}
\Delta E_{nlm} &=& -\frac{1}{2\,m_e\,c^2}\,\langle n,l,m|(E_n -V)^2|n,l,m\rangle\nonumber\\[0.5ex]
&=& -\frac{1}{2\,m_e\,c^2}\left(E_n^{\,2} - 2\,E_n\,\langle
n,l,m|V|n,l,m\rangle + \langle n,l,m|V^2|n,l,m\rangle\right)\nonumber\\[0.5ex]
&=& -\frac{1}{2\,m_e\,c^2}\left[
E_n^{\,2} + 2\,E_n\left(\frac{e^2}{4\pi\epsilon_0}\right)\left\langle
\frac{1}{r}\right\rangle + \left(\frac{e^2}{4\pi\epsilon_0}\right)^2\left\langle\frac{1}{r^{\,2}}\right\rangle\right].
\end{eqnarray}
It follows from Eqs.~(\ref{e9.74}) and (\ref{e9.75}) that
\begin{eqnarray}
\Delta E_{nlm} &=& -\frac{1}{2\,m_e\,c^2}\left[
E_n^{\,2} + 2\,E_n\left(\frac{e^2}{4\pi\epsilon_0}\right)\frac{1}{n^2\,a_0} + \left(\frac{e^2}{4\pi\epsilon_0}\right)^2\frac{1}{(l+1/2)\,n^3\,a_0^{\,2}}\right].\nonumber\\[0.5ex]&&
\end{eqnarray}
Finally, making use of Eqs.~(\ref{e9.55}), (\ref{e9.56}), and (\ref{e9.57}), the above expression reduces to
\begin{equation}\label{e12.121}
\Delta E_{nlm} = E_n\,\frac{\alpha^2}{n^2}\left(\frac{n}{l+1/2}-\frac{3}{4}\right),
\end{equation}
where
\begin{equation}
\alpha = \frac{e^2}{4\pi\epsilon_0\,\hbar\,c}\simeq \frac{1}{137}
\end{equation}
is the dimensionless  {\em fine structure constant}.

Note that the above derivation implicitly assumes that $p^4$ is an Hermitian
operator. It turns out that this is not the case for $l=0$ states. However,
somewhat fortuitously, 
our calculation still gives the correct answer when $l=0$. Note, also,
that we are able to use {\em non-degenerate}\/ perturbation theory in the
above calculation, using the $\psi_{nlm}$ eigenstates, because the perturbing Hamiltonian commutes
with both $L^2$ and $L_z$. It follows that there is no
coupling between states with different $l$ and $m$ quantum numbers.
Hence, all coupled states have different $n$ quantum numbers, and
therefore have different energies.

Now, an electron in a hydrogen atom experiences an electric field
\begin{equation}
{\bf E} = \frac{e\,{\bf r}}{4\pi\epsilon_0\,r^3}
\end{equation}
due to the charge on the nucleus. However, according to
electromagnetic theory, a non-relativistic particle moving in a
electric field ${\bf E}$ with velocity ${\bf v}$ also experiences an effective
magnetic field
\begin{equation}\label{e12.124}
{\bf B} = -\frac{{\bf v}\times {\bf E}}{c^2}.
\end{equation}
Recall, that an electron possesses a magnetic moment [see Eqs.~(\ref{e10.58})
and (\ref{e10.59})]
\begin{equation}
\bmu = - \frac{e}{m_e}\,{\bf S}
\end{equation}
due to its spin angular momentum, ${\bf S}$. We, therefore, expect
an additional contribution to the Hamiltonian of a hydrogen atom of the form [see Eq.~(\ref{e10.60a})]
\begin{eqnarray}
H_1 &=& - \bmu\cdot {\bf B}\nonumber\\[0.5ex]
&=& - \frac{e^2}{4\pi\epsilon_0\,m_e\,c^2\,r^3}\,{\bf v}\times {\bf r}\cdot{\bf S}\nonumber\\[0.5ex]
&=& \frac{e^2}{4\pi\epsilon_0\,m_e^{\,2}\,c^2\,r^3}\,{\bf L}\cdot {\bf S},
\end{eqnarray}
where ${\bf L} = m_e\,{\bf r}\times {\bf v}$  is the electron's orbital angular momentum. This effect is known as {\em spin-orbit coupling}. It turns
out that the above expression is too large, by a factor 2, due to an
obscure relativistic effect known as {\em Thomas precession}. Hence, the true
spin-orbit correction to the Hamiltonian is
\begin{equation}\label{e12.127}
H_1 = \frac{e^2}{8\pi\,\epsilon_0\,m_e^{\,2}\,c^2\,r^3}\,{\bf L}\cdot {\bf S}.
\end{equation}
Let us now apply perturbation theory to the hydrogen atom, using the
above expression as the perturbing Hamiltonian.

Now
\begin{equation}
{\bf J} = {\bf L} + {\bf S}
\end{equation}
is the total angular momentum of the system. Hence,
\begin{equation}
J^2 = L^2+S^2+ 2\,{\bf L}\cdot{\bf S},
\end{equation}
giving
\begin{equation}
{\bf L}\cdot {\bf S} = \frac{1}{2}\,(J^2-L^2-S^2).
\end{equation}
Recall, from Sect.~\ref{s11.2}, that whilst $J^2$ commutes with both $L^2$
and $S^2$, it does not commute with either $L_z$ or $S_z$. It follows
that the perturbing Hamiltonian (\ref{e12.127}) also commutes with both $L^2$ and $S^2$,  but does not commute with either $L_z$ or $S_z$.
Hence, the simultaneous eigenstates of the unperturbed Hamiltonian (\ref{e12.58})
and the perturbing Hamiltonian (\ref{e12.127}) are the same as the simultaneous
eigenstates of $L^2$, $S^2$, and $J^2$ discussed in Sect.~\ref{s11.3}.
It is important to know this since, according to Sect.~\ref{s12.6}, we
can only safely apply perturbation theory to the simultaneous
eigenstates of the unperturbed and perturbing Hamiltonians.

Adopting the notation introduced in Sect.~\ref{s11.3}, let
$\psi^{(2)}_{l,s;j,m_j}$ be a simultaneous eigenstate of $L^2$, $S^2$,
$J^2$, and $J_z$ corresponding to the eigenvalues
\begin{eqnarray}
L^2\,\psi^{(2)}_{l,s;j,m_j} &=& l\,(l+1)\,\hbar^2\,\psi^{(2)}_{l,s;j,m_j},\\[0.5ex]
S^2\,\psi^{(2)}_{l,s;j,m_j} &=& s\,(s+1)\,\hbar^2\,\psi^{(2)}_{l,s;j,m_j},\\[0.5ex]
J^2\,\psi^{(2)}_{l,s;j,m_j} &=& j\,(j+1)\,\hbar^2\,\psi^{(2)}_{l,s;j,m_j},\\[0.5ex]
J_z\,\psi^{(2)}_{l,s;j,m_j} &=& m_j\,\hbar\,\psi^{(2)}_{l,s;j,m_j}.
\end{eqnarray}
According to standard first-order perturbation theory, the energy-shift induced in such a state
by spin-orbit coupling is given by
\begin{eqnarray}
\Delta E_{l,1/2;j,m_j} &=& \langle l,1/2;j,m_j|H_1|l,1/2;j,m_j\rangle\nonumber\\[0.5ex]
&=&  \frac{e^2}{16\pi\,\epsilon_0\,m_e^{\,2}\,c^2}\left\langle
1,1/2;j,m_j\left|\frac{J^2-L^2-S^2}{r^3}\right|l,1/2;j,m_j\right\rangle\nonumber\\[0.5ex]
&=& \frac{e^2\,\hbar^2}{16\pi\,\epsilon_0\,m_e^{\,2}\,c^2}\,\left[j\,(j+1)-l\,(l+1)-3/4\right]\,\left\langle\frac{1}{r^3}\right\rangle.
\end{eqnarray}
Here, we have made use of the fact that $s=1/2$ for an electron. It follows
from Eq.~(\ref{e9.75a}) that
\begin{equation}
\Delta E_{l,1/2;j,m_j}=  \frac{e^2\,\hbar^2}{16\pi\,\epsilon_0\,m_e^{\,2}\,c^2\,a_0^{\,3}}\left[\frac{j\,(j+1)-l\,(l+1)-3/4}{l\,(l+1/2)\,(l+1)\,n^3}\right],
\end{equation}
where $n$ is the radial quantum number. Finally, making use of Eqs.~(\ref{e9.55}), (\ref{e9.56}), and (\ref{e9.57}), the above expression reduces to
\begin{equation}\label{e12.137}
\Delta E_{l,1/2;j,m_j}=  E_n\,\frac{\alpha^2}{n^2}\left[
\frac{n\,\left\{3/4+l\,(l+1)-j\,(j+1)\right\}}{2\,l\,(l+1/2)\,(l+1)}\right],
\end{equation}
where $\alpha$ is the fine structure constant. A comparison of this
expression with Eq.~(\ref{e12.121}) reveals that the energy-shift
due to spin-orbit coupling is of the same order of magnitude as that due
to the lowest-order relativistic correction to the Hamiltonian. We can
add these two corrections together (making use of the fact that
$j=l\pm 1/2$ for a hydrogen atom---see Sect.~\ref{s11.3}) to obtain
a net energy-shift of
\begin{equation}\label{e12.138}
\Delta E_{l,1/2;j,m_j}=  E_n\,\frac{\alpha^2}{n^2}\left(\frac{n}{j+1/2}-\frac{3}{4}\right).
\end{equation}
This modification of the energy levels of a hydrogen atom due to a combination
of relativity and spin-orbit coupling is
known as {\em fine structure}. 

Now, it is conventional to refer to the energy eigenstates of a hydrogen
atom which are also simultaneous eigenstates of $J^2$ as $nL_j$ states,
where $n$ is the radial quantum number, $L=(S,P,D,F,\cdots)$ as $l=(0,1,2,3,\cdots)$, and $j$ is the total angular momentum quantum number.
Let us examine the effect of the fine structure energy-shift (\ref{e12.138})
on these eigenstates for  $n=1,2$ and 3.

For $n=1$, in the absence of fine structure, there are two degenerate $1S_{1/2}$ states. 
According to Eq.~(\ref{e12.138}), the fine structure induced energy-shifts of
these two states are the same. Hence, fine structure does not
break the degeneracy of the two $1S_{1/2}$ states of hydrogen. 

For $n=2$, in the absence of fine structure, there are two $2S_{1/2}$
states, two $2P_{1/2}$ states, and four $2P_{3/2}$ states, all
of which are degenerate.
According to Eq.~(\ref{e12.138}), the fine structure induced energy-shifts of
the $2S_{1/2}$ and $2P_{1/2}$ states are the same as one another, but are different
from the  induced
energy-shift of the $2P_{3/2}$ states.
Hence, fine structure does not break the
degeneracy of the $2S_{1/2}$ and $2P_{1/2}$ states of hydrogen, but
does break the degeneracy of these states relative to the $2P_{3/2}$
states.

 For $n=3$, in the absence of fine structure, there are two $3S_{1/2}$
states, two $3P_{1/2}$ states, four $3P_{3/2}$ states,  four $3D_{3/2}$
states, and six $3D_{5/2}$ states, all of
which are degenerate. According to Eq.~(\ref{e12.138}), fine structure
breaks these states into three groups: the $3S_{1/2}$ and $3P_{1/2}$ states,
the $3P_{3/2}$ and $3D_{3/2}$ states, and the $3D_{5/2}$ states.

The effect of the fine structure energy-shift on the $n=1$, 2, and 3 energy
states of a hydrogen atom is illustrated in Fig.~\ref{ffine}.

\begin{figure}
\epsfysize=3.5in
\centerline{\epsffile{Chapter12/fig01.eps}}
\caption{\em Effect of the fine structure energy-shift on the
$n=1,2$ and 3 states of a hydrogen atom. Not to scale.}\label{ffine}   
\end{figure}

Note, finally, that although expression (\ref{e12.137}) does not
have a well-defined value for $l=0$, when added to expression (\ref{e12.121})  it, somewhat fortuitously, gives rise to an expression
(\ref{e12.138}) which is both well-defined and correct when $l=0$.

\section{Zeeman Effect}
Consider a hydrogen atom placed in a uniform $z$-directed external
magnetic field of strength $B$. The modification to the Hamiltonian
of the system is
\begin{equation}
H_1 = -\bmu\cdot{\bf B},
\end{equation}
where
\begin{equation}
\bmu = - \frac{e}{2\,m_e}\,({\bf L} + 2\,{\bf S})
\end{equation}
is the total electron magnetic moment, including both orbital and spin contributions
[see Eqs.~(\ref{e10.57})--(\ref{e10.59})].  Thus,
\begin{equation}
H_1 = \frac{e\,B}{2\,m_e}\,(L_z+ 2\,S_z).
\end{equation}

Suppose that the applied magnetic field is much weaker than the atom's internal
magnetic field (\ref{e12.124}). Since the magnitude of the internal
field is about 25 tesla, this is a fairly reasonable assumption. In this
situation, we can treat $H_1$ as a small perturbation acting
on the simultaneous eigenstates of the unperturbed Hamiltonian and
the fine structure Hamiltonian. Of course, these states
are the simultaneous eigenstates of $L^2$, $S^2$, $J^2$, and $J_z$ (see
previous section). Hence, from standard perturbation theory, the
first-order energy-shift induced by a weak external magnetic field
is
\begin{eqnarray}
\Delta E_{l,1/2;j,m_j} &=& \langle l,1/2;j,m_j|H_1|l,1/2;j,m_j\rangle\nonumber\\[0.5ex]
&=& \frac{e\,B}{2\,m_e}\,\left(m_j\,\hbar + \langle l,1/2;j,m_j|S_z|l,1/2;j,m_j\rangle\right),
\end{eqnarray}
since $J_z=L_z+S_z$. Now, according to Eqs.~(\ref{e11.47}) and
(\ref{e11.48}), 
\begin{equation}\label{e12.143}
\psi^{(2)}_{j,m_j} = \left(\frac{j+m_j}{2\,l+1}\right)^{1/2}\psi^{(1)}_{m_j-1/2,1/2} + \left(\frac{j-m_j}{2\,l+1}\right)^{1/2}\,\psi^{(1)}_{m_j+1/2,-1/2}
\end{equation}
when $j=l+1/2$, and
\begin{equation}
\psi^{(2)}_{j,m_j} = \left(\frac{j+1-m_j}{2\,l+1}\right)^{1/2}\psi^{(1)}_{m_j-1/2,1/2} - \left(\frac{j+1+m_j}{2\,l+1}\right)^{1/2}\,\psi^{(1)}_{m_j+1/2,-1/2}
\end{equation}
when $j=l-1/2$. Here, the $\psi^{(1)}_{m,m_s}$ are the
simultaneous eigenstates of $L^2$, $S^2$, $L_z$, and $S_z$, whereas
the $\psi^{(2)}_{j,m_j}$ are the simultaneous eigenstates of
$L^2$, $S^2$, $J^2$, and $J_z$. In particular,
\begin{equation}\label{e12.145}
S_z\,\psi^{(1)}_{m,\pm 1/2} = \pm \frac{\hbar}{2}\,\psi^{(1)}_{m,\pm 1/2}.
\end{equation}
It follows from Eqs.~(\ref{e12.143})--(\ref{e12.145}), and the
orthormality of the $\psi^{(1)}$, that
\begin{equation}
\langle l,1/2;j,m_j|S_z|l,1/2;j,m_j\rangle = \pm \frac{m_j\,\hbar}{2\,l+1}
\end{equation}
when $j=l\pm 1/2$.  Thus, the induced energy-shift when a hydrogen atom is placed in  an external magnetic field---which is known as the {\em Zeeman effect}---becomes
\begin{equation}\label{e12.147}
\Delta E_{l,1/2;j,m_j} = \mu_B\,B\,m_j\left[1\pm \frac{1}{2\,l+1}\right]
\end{equation}
where the $\pm$ signs correspond to $j=l\pm 1/2$. Here,
\begin{equation}
\mu_B = \frac{e\,\hbar}{2\,m_e} = 5.788\times 10^{-5}\,{\rm eV/T}
\end{equation}
is known as the {\em Bohr magnetron}. Of course, the quantum number $m_j$ takes values differing by unity in the range $-j$ to $j$. It, thus,
follows from Eq.~(\ref{e12.147}) that the Zeeman effect splits
degenerate states characterized by $j=l+1/2$ into $2\,j+1$
equally spaced states of interstate spacing
\begin{equation}\label{e12.149}
\Delta E_{j=l+1/2} = \mu_B\,B\,\frac{2\,l+2}{2\,l+1}.
\end{equation}
Likewise, the Zeeman effect splits degenerate states characterized by
$j=l-1/2$ into $2\,j+1$ equally spaced states of interstate spacing
\begin{equation}\label{e12.150}
\Delta E_{j=l-1/2} = \mu_B\,B\,\frac{2\,l}{2\,l+1}.
\end{equation}

In conclusion, in the presence of a weak external magnetic field, the two degenerate $1S_{1/2}$
states of the hydrogen atom are split by $2\,\mu_B\,B$. Likewise,
the four degenerate  $2S_{1/2}$ and $2P_{1/2}$ states are split by
$(2/3)\,\mu_B\,B$, whereas the four degenerate $2P_{3/2}$ states
are split by $(4/3)\,\mu_B\,B$. This is illustrated in Fig.~\ref{fzee}.
Note, finally, that since the $\psi^{(2)}_{l,m_j}$ are not simultaneous
eigenstates of the unperturbed and perturbing Hamiltonians, 
Eqs.~(\ref{e12.149}) and (\ref{e12.150}) can only be regarded as the expectation
values of the magnetic-field induced energy-shifts. However, as long as
the external magnetic field is much weaker than the internal magnetic
field, these expectation  values are almost identical to the actual
measured values of the energy-shifts.

\begin{figure}
\epsfysize=3.5in
\centerline{\epsffile{Chapter12/fig02.eps}}
\caption{\em The Zeeman effect for the
$n=1$ and $2$ states of a hydrogen atom. Here, $\epsilon= \mu_B\,B$. Not to scale.}\label{fzee}   
\end{figure}

\section{Hyperfine Structure}
The proton in a hydrogen atom is a spin one-half charged particle, and therefore
possesses a magnetic moment. By analogy with Eq.~(\ref{e10.58}),
we can write
\begin{equation}\label{e12.151}
\bmu_p = \frac{g_p\,e}{2\,m_p}\,{\bf S}_p,
\end{equation}
where $\bmu_p$ is the proton magnetic
moment, ${\bf S}_p$ is the proton spin, and the proton gyromagnetic ratio $g_p$ is found experimentally to take that value $5.59$. Note that the
magnetic moment of a proton is much smaller (by a factor of order $m_e/m_p$)
than that of an electron.
 According
to classical electromagnetism, the proton's magnetic moment generates a
magnetic field of the form
\begin{equation}
{\bf B} = \frac{\mu_0}{4\pi\,r^3}\,\left[3\,(\bmu_p\cdot{\bf e}_r)\,{\bf e}_r - \bmu_p\right] + \frac{2\,\mu_0}{3}\,\bmu_p\,\delta^3({\bf r}),
\end{equation}
where ${\bf e}_r = {\bf r}/r$. We can understand the origin of the delta-function term
in the above expression by thinking of the proton as a tiny current loop centred on the origin.
All magnetic field-lines generated by the loop must pass through the loop.
Hence, if the size of the  loop goes to zero then the field will be infinite at the origin, and this contribution is what is reflected by the delta-function term. Now, the Hamiltonian of the electron in the magnetic
field generated by the proton is simply
\begin{equation}
H_1 = - \bmu_e\cdot {\bf B},
\end{equation}
where
\begin{equation}
\bmu_e = - \frac{e}{m_e}\,{\bf S}_e.
\end{equation}
Here, $\bmu_e$ is the electron  magnetic moment [see Eqs.~(\ref{e10.58})
and (\ref{e10.59})], and ${\bf S}_e$ the electron spin. Thus, the
perturbing Hamiltonian is written
\begin{equation}
H_1=\frac{\mu_0\,g_p\,e^2}{8\pi\,m_p\,m_e}\,
\frac{3\,({\bf S}_p\cdot{\bf e}_r)\,({\bf S}_e\cdot{\bf e}_r) - {\bf S}_p\cdot{\bf S}_e}{r^3} + \frac{\mu_0\,g_p\,e^2}{3\,m_p\,m_e}\,{\bf S}_p\cdot{\bf S}_e\,\delta^3({\bf r}).
\end{equation}
Note that, since we have neglected coupling between the proton
spin and the magnetic field generated by the electron's orbital motion,
the above expression is only valid for $l=0$ states.

According to standard first-order perturbation theory, the energy-shift induced
by spin-spin coupling between the proton and the electron is the expectation
value of the perturbing Hamiltonian. Hence,
\begin{equation}
\Delta E = \frac{\mu_0\,g_p\,e^2}{8\pi\,m_p\,m_e}\left\langle\frac{3\,({\bf S}_p\cdot{\bf e}_r)\,({\bf S}_e\cdot{\bf e}_r) - {\bf S}_p\cdot{\bf S}_e}{r^3}\right\rangle+ \frac{\mu_0\,g_p\,e^2}{3\,m_p\,m_e}\,\langle{\bf S}_p\cdot{\bf S}_e\rangle\,|\psi(0)|^2.
\end{equation}
For the ground-state of hydrogen, which is spherically symmetric,
the first term in the above expression vanishes by symmetry. 
Moreover, it is easily demonstrated that $|\psi_{000}(0)|^2=
1/(\pi\,a_0^{\,3})$. Thus, we obtain
\begin{equation}
\Delta E = \frac{\mu_0\,g_p\,e^2}{3\pi\,m_p\,m_e\,a_0^{\,3}}\,\langle{\bf S}_p\cdot{\bf S}_e\rangle.
\end{equation}

Let
\begin{equation}
{\bf S} = {\bf S}_e + {\bf S}_p
\end{equation}
be the total spin. We can show that
\begin{equation}
{\bf S}_p\cdot{\bf S}_e = \frac{1}{2}\,(S^2-S_e^{\,2}-S_p^{\,2}).
\end{equation}
Thus, the simultaneous eigenstates of the perturbing Hamiltonian
and the main Hamiltonian are the simultaneous eigenstates of $S_e^{\,2}$,
$S_p^{\,2}$, and $S^2$. However, both the proton and
the electron are spin one-half particles. According to Sect.~\ref{shalf},
when two spin one-half particles are combined (in the absence of orbital
angular momentum) the net state  has either spin 1 or spin 0.
In fact, there are three spin 1 states, known as triplet states, and a single
spin 0 state, known as the singlet state. For all states,
the eigenvalues of $S_e^{\,2}$ and $S_p^{\,2}$ are $(3/4)\,\hbar^2$.
The eigenvalue of $S^2$ is 0 for the singlet state, and $2\,\hbar^2$
for the triplet states. Hence,
\begin{equation}
\langle {\bf S}_p\cdot{\bf S}_e\rangle = - \frac{3}{4}\,\hbar^2
\end{equation}
for the singlet state, and
\begin{equation}
\langle {\bf S}_p\cdot{\bf S}_e\rangle =  \frac{1}{4}\,\hbar^2
\end{equation}
for the triplet states. 

It follows, from the above analysis, that spin-spin coupling breaks
the degeneracy of the two $1S_{1/2}$ states in hydrogen, lifting the
energy of the triplet configuration, and lowering that of the singlet.
This splitting is known as {\em hyperfine structure}.
The net energy difference between the singlet and the triplet states
is
\begin{equation}
\Delta E = \frac{8}{3}\,g_p\,\frac{m_e}{m_p}\,\alpha^2\,E_0 = 5.88\times 10^{-6}\,{\rm eV},
\end{equation}
where $E_0=13.6\,{\rm eV}$ is the (magnitude of the) ground-state energy.
Note that the hyperfine energy-shift is much smaller, by a factor $m_e/m_p$, than
a typical fine structure energy-shift.
If we convert the above energy into a wavelength then we obtain
\begin{equation}
\lambda = 21.1\,{\rm cm}.
\end{equation}
This is the wavelength of the radiation emitted by a hydrogen atom
which is collisionally excited from the singlet to the triplet
state, and then decays back to the lower energy singlet state. 
The 21\,cm line is famous in  radio astronomy because it was used to
map out the spiral structure of our galaxy in the 1950's. 