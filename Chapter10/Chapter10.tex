\chapter{Spin Angular Momentum}\label{sspin}
\section{Introduction}
Broadly speaking, a classical extended object ({\em e.g.}, the Earth) can possess two types of
angular momentum. The first type is due to the rotation of the
object's center of mass about some  fixed external point ({\em e.g.}, the Sun)---this
is generally known as {\em orbital angular momentum}. The
second type is due to the object's internal motion---this is generally
known as {\em spin angular momentum} (since, for a rigid object, the internal
motion consists of spinning about an axis passing through the center of
mass). By analogy,  quantum  particles can possess
both orbital angular momentum due to their motion through space
 (see
Cha.~\ref{sorb}),
and spin angular momentum due to their internal motion. 
Actually, the analogy with classical extended objects is not entirely accurate, since electrons, for instance,
are structureless point particles. In fact, in quantum mechanics, it is best to think of spin angular momentum
 as  a kind of {\em intrinsic}\/ angular momentum possessed by particles. It turns out that each type of elementary particle
has a characteristic spin angular momentum, just as each type
has a characteristic charge and  mass.

\section{Spin Operators}\label{s10.2}
Since spin is a type of angular momentum, it is reasonable to suppose
that it possesses similar properties to orbital angular momentum.
Thus, by analogy with Sect.~\ref{s8.2}, we would expect to be able
to define three operators---$S_x$, $S_y$, and $S_z$---which represent
the three Cartesian components of spin angular momentum. Moreover,
it is plausible that these operators
possess
analogous commutation relations to the three corresponding orbital
angular momentum operators, $L_x$, $L_y$, and $L_z$ [see Eqs.~(\ref{e8.6})--(\ref{e8.8})]. In other words,
\begin{eqnarray}\label{e10.1x}
[S_x, S_y]&=& {\rm i}\,\hbar\,S_z,\\[0.5ex]
[S_y, S_z]&=& {\rm i}\,\hbar\,S_x,\\[0.5ex]
[S_z,S_x]&=& {\rm i}\,\hbar\,S_y.\label{e10.2x}
\end{eqnarray}
We can represent the magnitude squared of the spin angular momentum vector by the operator
\begin{equation}
S^2 = S_x^{\,2} + S_y^{\,2}+ S_z^{\,2}.
\end{equation}
By analogy with the analysis in Sect.~\ref{s8.2}, it is
easily demonstrated that
\begin{equation}
[S^2, S_x] = [S^2, S_y] = [S^2,S_z] = 0.
\end{equation}
We thus conclude (see Sect.~\ref{smeas}) that we can simultaneously measure the magnitude squared
of the spin angular momentum vector, together with, at most, one Cartesian component.
By convention, we shall always choose to measure the $z$-component, $S_z$. 

By analogy with Eq.~(\ref{e8.13}), we can define raising and lowering
operators for spin angular momentum:
\begin{equation}
S_\pm = S_x \pm {\rm i}\,S_y.
\end{equation}
If $S_x$, $S_y$, and $S_z$ are Hermitian operators, as must be
the case if they are to represent physical quantities, then $S_\pm$ are
the Hermitian conjugates of one another: {\em i.e.},
\begin{equation}\label{e10.7}
(S_\pm)^\dag = S_\mp.
\end{equation}
Finally, by analogy with Sect.~\ref{s8.2}, it is easily
demonstrated that
\begin{eqnarray}
S_+\,S_- &=& S^2-S_z^{\,2}+\hbar\,S_z,\label{e10.7a}\\[0.5ex]
S_-\,S_+&=& S^2-S_z^{\,2}-\hbar\,S_z,\label{e10.8}\\[0.5ex]
[S_+,S_z]&=& - \hbar\,S_+,\label{e10.9}\\[0.5ex]
[S_-,S_z]&=& +\hbar\,S_-.\label{e10.10}
\end{eqnarray}

\section{Spin Space}\label{s10.3}
We now have to discuss the wavefunctions upon which the previously introduced
spin operators act. Unlike regular wavefunctions, spin wavefunctions
do not exist in real space. Likewise, the spin angular momentum operators
cannot be represented as differential operators in real space.
Instead, we need to think of spin wavefunctions as
existing in an abstract (complex) vector space. The different members of
this space correspond to the different internal configurations of the particle
under investigation.
Note that only the {\em directions}\/ of our vectors have any physical significance
(just as only the shape of a regular wavefunction has any physical
significance). Thus,
if the vector $\chi$  corresponds to a particular internal state then
$c\,\chi$ corresponds to the same state, where $c$ is a complex number.
Now, we expect the internal states of our particle to be {\em superposable},
since the superposability of states is one of the fundamental assumptions of
quantum mechanics.
It follows that the vectors making up our vector space must also be superposable.
Thus, if $\chi_1$ and $\chi_2$ are two vectors corresponding to two
different internal states then $c_1\,\chi_1+c_2\,\chi_2$ is another vector
corresponding to the state obtained by superposing $c_1$ times state 1
with $c_2$ times state 2 (where $c_1$ and $c_2$ are complex numbers). Finally, the dimensionality of our vector
space is simply the number of linearly independent vectors required to span
it ({\em i.e.}, the number of linearly independent internal states of the
particle under investigation).

We now need to define the length of our vectors. We can do this by
introducing a second, or {\em dual}, vector space whose elements are in one to one
correspondence with the elements of our first space. Let the element of the second
space which corresponds to the element $\chi$ of the first space
be called $\chi^\dag$. Moreover, the element of the second space
which corresponds to $c\,\chi$ is $c^\ast\,\chi^\dag$. We shall assume
that it is possible to combine $\chi$ and $\chi^\dag$ in a multiplicative
fashion to generate a real
positive-definite number which we interpret as the length, or {\em norm},
of $\chi$. Let us denote this number $\chi^\dag\,\chi$. Thus, we
have
\begin{equation}\label{e10.11}
\chi^\dag\,\chi\geq 0
\end{equation}
for all $\chi$. We shall also assume that it is possible to combine unlike states
in an analogous multiplicative fashion to produce complex numbers. The
product of two unlike states $\chi$ and $\chi'$ is denoted $\chi^\dag\,\chi'$.
Two states $\chi$ and $\chi'$ are said to be mutually orthogonal, or independent,
if $\chi^\dag\,\chi' = 0$.

Now, when a general spin operator, $A$, operates on a general spin-state, $\chi$, it coverts it into a different spin-state which we shall denote
$A\,\chi$. The dual of this state is $(A\,\chi)^\dag\equiv \chi^\dag\,A^\dag$, where $A^\dag$ is the Hermitian conjugate of $A$ (this is the definition of an
Hermitian conjugate in spin space).  An eigenstate
of $A$ corresponding to the eigenvalue $a$ satisfies
\begin{equation}
A\,\chi_{a} = a\,\chi_{a}.
\end{equation}
As before, if $A$ corresponds to a physical variable then a measurement
of $A$ will result in one of its eigenvalues (see Sect.~\ref{smeas}). In order to ensure that
these eigenvalues are all real, $A$ must be Hermitian: {\em i.e.}, $A^\dag=A$ (see Sect.~\ref{seig}). We expect the $\chi_a$ to be mutually orthogonal. We
can also normalize them such that they all have unit length. In other words,
\begin{equation}
\chi_{a}^\dag\,\chi_{a'} = \delta_{aa'}.
\end{equation}
Finally, a general spin state can be written
as a superposition of the normalized eigenstates of $A$: {\em i.e.}, 
\begin{equation}
\chi = \sum_a c_a\, \chi_{a}.
\end{equation}
A measurement of $\chi$ will then yield the result $a$ with probability $|c_a|^2$. 

\section{Eigenstates of $S_z$ and $S^2$}
Since the operators $S_z$ and $S^2$ commute, they must possess simultaneous
eigenstates (see Sect.~\ref{smeas}). Let these eigenstates take the form [see Eqs. (\ref{e8.29}) and (\ref{e8.30})]:
\begin{eqnarray}
S_z\,\chi_{s,m_s}&=& m_s\,\hbar\,\chi_{s,m_s},\label{e10.16}\\[0.5ex]
S^2\,\chi_{s,m_s} &=& s\,(s+1)\,\hbar^{\,2}\,\chi_{s,m_s}.\label{e10.17}
\end{eqnarray}

Now, it is easily demonstrated, from the commutation relations (\ref{e10.9}) and
(\ref{e10.10}), that
\begin{equation}
S_z\,(S_+\,\chi_{s,m_s}) = (m_s+1)\,\hbar\,(S_+\,\chi_{s,m_s}),
\end{equation}
and
\begin{equation}
S_z\,(S_-\,\chi_{s,m_s}) = (m_s-1)\,\hbar\,(S_-\,\chi_{s,m_s}).
\end{equation}
Thus, $S_+$ and $S_-$ are indeed the raising and lowering operators,
respectively, for spin angular momentum (see Sect.~\ref{seian}).
The eigenstates of $S_z$ and $S^2$ are assumed to be orthonormal: {\em i.e.},
\begin{equation}\label{e10.20}
\chi^\dag_{s,m_s}\,\chi_{s',m_s'} =\delta_{ss'}\,\delta_{m_s m_s'}.
\end{equation}

Consider the wavefunction $\chi=S_+\,\chi_{s,m_s}$. Since we know,
from Eq. (\ref{e10.11}), that $\chi^\dag\,\chi\geq 0$, it follows that
\begin{equation}
(S_+\,\chi_{s,m_s})^\dag\,(S_+\,\chi_{s,m_s}) = \chi_{s,m_s}^\dag\,
S_+^\dag\,S_+\,\chi_{s,m_s} = \chi_{s,m_s}^\dag\,S_-\,S_+\,\chi_{s,m_s}\geq 0,
\end{equation}
where use has been made of Eq.~(\ref{e10.7}). Equations~(\ref{e10.8}), (\ref{e10.16}), (\ref{e10.17}), and (\ref{e10.20})
yield
\begin{equation}
s\,(s+1) \geq m_s\,(m_s+1).
\end{equation}
Likewise, if $\chi=S_-\,\chi_{s,m_s}$ then we obtain
\begin{equation}
s\,(s+1)\geq m_s\,(m_s-1).
\end{equation}
Assuming that $s\geq 0$, the above two inequalities
imply that
\begin{equation}
-s \leq m_s\leq s.
\end{equation}
Hence, at fixed $s$, there is both a maximum and a minimum possible value that $m_s$
can take.

Let $m_{s\,min}$ be the minimum possible value of $m_s$. It follows
that (see Sect.~\ref{slsq})
\begin{equation}
S_-\,\chi_{s,m_{s\,min}}= 0.
\end{equation}
Now, from Eq.~(\ref{e10.7a}), 
\begin{equation}
S^2 = S_+\,S_-+S_z^{\,2}-\hbar\,S_z.
\end{equation}
Hence,
\begin{equation}
S^2\,\chi_{s,m_{s\,min}} = (S_+\,S_- +S_z^{\,2}-\hbar\,S_z)\,\chi_{s,m_{s\,min}},
\end{equation}
giving
\begin{equation}
s\,(s+1) = m_{s\,min}\,(m_{s\,min}-1).
\end{equation}
Assuming that $m_{s\,min}<0$, this equation yields
\begin{equation}
m_{s\,min} = -s.
\end{equation}
Likewise, it is easily demonstrated that
\begin{equation}
m_{s\,max} = +s.
\end{equation}
Moreover,
\begin{equation}\label{e10.31}
S_-\,\chi_{s,-s} = S_+\,\chi_{s,s} = 0.
\end{equation}

Now, the raising operator $S_+$, acting upon $\chi_{s,-s}$, converts
it into some multiple of $\chi_{s,-s+1}$. Employing the raising operator
a second time, we obtain a multiple of $\chi_{s,-s+2}$. However, this
process cannot continue indefinitely, since there is a maximum possible
value of $m_s$. Indeed, after acting upon $\chi_{s,-s}$ a sufficient number
of times with the raising operator $S_+$, we must obtain a multiple
of $\chi_{s,s}$, so that employing the raising operator one more time
leads to the null state [see Eq.~(\ref{e10.31})]. If this is not the case then we will inevitably obtain eigenstates
of $S_z$ corresponding to $m_s>s$, which we have already demonstrated is impossible.

It follows, from the above argument, that
\begin{equation}
m_{s\,max}-m_{s\,min} = 2\,s = k,
\end{equation}
where $k$ is a positive integer. Hence, the quantum number $s$
can either take {\em positive integer}\/ or {\em positive half-integer}\/ values.
Up to now, our analysis has been very similar to that which we used earlier to investigate orbital
angular momentum (see Sect.~\ref{sorb}). Recall, that for orbital angular momentum the quantum number $m$, which is analogous to $m_s$,
is restricted to take {\em integer}\/ values (see Cha.~\ref{slz}). This implies
that the quantum number $l$, which is analogous to $s$, is also
restricted to take integer values.
However,
the origin of these restrictions is the representation of the orbital
angular momentum operators as differential operators in real space
(see Sect.~\ref{s8.3}). There is no equivalent representation of the
corresponding spin angular momentum operators. Hence, we conclude
that there is no reason why the quantum number $s$ cannot take half-integer,
as well as integer, values.

In 1940, Wolfgang Pauli proved the so-called {\em spin-statistics theorem}\/
using relativistic quantum mechanics. According to this theorem, all
{\em fermions}\/ possess {\em half-integer spin}\/ ({\em i.e.}, a half-integer value of $s$),
whereas all {\em bosons}\/ possess {\em integer spin}\/ ({\em i.e.}, an integer value of $s$). In fact, all presently known
fermions, including electrons and protons, possess {\em spin one-half}. In other words,
electrons and protons are characterized by $s=1/2$ and $m_s=\pm 1/2$.

\section{Pauli Representation}\label{spauli}
Let us denote the two independent spin eigenstates of an electron as
\begin{equation}
\chi_\pm \equiv \chi_{1/2,\pm 1/2}.
\end{equation}
It thus follows, from Eqs.~(\ref{e10.16}) and (\ref{e10.17}), that
\begin{eqnarray}
S_z\,\chi_\pm &=& \pm \frac{1}{2}\,\hbar\,\chi_\pm,\label{e10.34}\\[0.5ex]
S^2\,\chi_\pm &=& \frac{3}{4}\,\hbar^2\,\chi_\pm.
\end{eqnarray}
Note that $\chi_+$ corresponds to an electron whose spin angular momentum vector has a positive component along the $z$-axis. Loosely speaking,
we could say that the spin vector points in the $+z$-direction (or its spin is
``up''). Likewise,
$\chi_-$ corresponds to an electron whose spin points in  the $-z$-direction
(or whose spin is ``down'').
These two eigenstates satisfy the orthonormality requirements
\begin{equation}\label{e10.35}
\chi_+^\dag\,\chi_+ = \chi_-^\dag\,\chi_- = 1,
\end{equation}
and
\begin{equation}\label{e10.36}
\chi_+^\dag\,\chi_- = 0.
\end{equation}
A general spin state can be represented as a linear combination of
$\chi_+$ and $\chi_-$: {\em i.e.},
\begin{equation}
\chi = c_+\,\chi_+ + c_-\,\chi_-.
\end{equation}
It is thus evident that electron spin space is {\em two-dimensional}.

Up to now, we have discussed spin space in rather abstract terms. In the
following, we shall describe a particular representation of electron
spin space due to Pauli. This so-called {\em Pauli representation}\/ allows us
to visualize spin space, and also facilitates calculations involving spin.

Let us attempt to represent a general spin state as a complex {\em column vector}\/ in some two-dimensional space: {\em i.e.},
\begin{equation}
\chi \equiv \left(\begin{array}{c}c_+\\c_-\end{array}\right).
\end{equation}
The corresponding dual vector is represented as a {\em row vector}: {\em i.e.},
\begin{equation}
\chi^\dag\equiv  (c_+^\ast, c_-^\ast).
\end{equation}
Furthermore, the product $\chi^\dag\,\chi$ is obtained according to the
ordinary rules of matrix multiplication: {\em i.e.},
\begin{equation}
\chi^\dag\,\chi =  (c_+^\ast, c_-^\ast)\left(\begin{array}{c}c_+\\c_-\end{array}\right) = c_+^\ast\,c_+ + c_-^\ast\,c_- = |c_+|^2 + |c_-|^2\geq 0.
\end{equation}
Likewise, the product $\chi^\dag\,\chi'$ of two different spin states
is also obtained from the rules of matrix multiplication: {\em i.e.},
\begin{equation}
\chi^\dag\,\chi' =  (c_+^\ast, c_-^\ast)\left(\begin{array}{c}c_+'\\c_-'\end{array}\right) = c_+^\ast\,c_+' + c_-^\ast\,c_-'.
\end{equation}
Note that this particular representation of spin space is in complete accordance with the discussion in Sect.~\ref{s10.3}. For obvious reasons,
 a vector used to represent a spin state is generally known as
{\em spinor}.

A general spin operator $A$ is represented as a $2\times 2$ matrix
which operates on a spinor: {\em i.e.},
\begin{equation}
A\,\chi \equiv \left(\begin{array}{cc}A_{11},& A_{12}\\
A_{21},& A_{22}\end{array}\right)\left(\begin{array}{c}c_+\\c_-\end{array}\right).
\end{equation}
As is easily demonstrated, the Hermitian conjugate of $A$ is represented by
the transposed complex conjugate of the matrix used to represent $A$: {\em i.e.},
\begin{equation}
A^\dag \equiv \left(\begin{array}{cc}A_{11}^\ast,& A_{21}^\ast\\
A_{12}^\ast,& A_{22}^\ast\end{array}\right).
\end{equation}

Let us represent the spin eigenstates $\chi_+$ and $\chi_-$  as
\begin{equation}
\chi_+ \equiv \left(\begin{array}{c}1\\0\end{array}\right),
\end{equation}
and
\begin{equation}
\chi_- \equiv \left(\begin{array}{c}0\\1\end{array}\right),
\end{equation}
respectively. Note that these forms automatically
satisfy the orthonormality constraints (\ref{e10.35}) and (\ref{e10.36}). 
It is convenient to write the spin operators $S_i$ (where $i=1,2,3$ corresponds to
$x,y,z$) as
\begin{equation}\label{e10.46}
S_i = \frac{\hbar}{2}\,\sigma_i.
\end{equation}
Here, the $\sigma_i$ are dimensionless $2\times 2$ matrices. According
to Eqs.~(\ref{e10.1x})--(\ref{e10.2x}), the $\sigma_i$ satisfy the commutation
relations
\begin{eqnarray}
[\sigma_x, \sigma_y]&=& 2\,{\rm i}\,\sigma_z,\\[0.5ex]
[\sigma_y, \sigma_z]&=& 2\,{\rm i}\,\sigma_x,\\[0.5ex]
[\sigma_z,\sigma_x]&=& 2\,{\rm i}\,\sigma_y.
\end{eqnarray}
Furthermore, Eq.~(\ref{e10.34}) yields
\begin{equation}
\sigma_z\,\chi_\pm = \pm \chi_\pm.
\end{equation}
It is easily demonstrated, from the above expressions, that the $\sigma_i$ are represented by the
following matrices:
\begin{eqnarray}
\sigma_x&\equiv&  \left(\begin{array}{cc}0,&1\\
1,& 0\end{array}\right),\\[0.5ex]
\sigma_y&\equiv&  \left(\begin{array}{cc}0,&-{\rm i}\\
{\rm i},& 0\end{array}\right),\\[0.5ex]
\sigma_z&\equiv&  \left(\begin{array}{cc}1,&0\\
0,& -1\end{array}\right).\label{e10.53}
\end{eqnarray}
Incidentally, these matrices are generally known as the {\em Pauli matrices}. 

Finally, a general spinor takes the form
\begin{equation}
\chi = c_+\,\chi_++c_-\,\chi_- =  \left(\begin{array}{c}c_+\\c_-\end{array}\right).
\end{equation}
If the spinor is properly normalized then
\begin{equation}
\chi^\dag\,\chi = |c_+|^2 + |c_-|^2 =1.
\end{equation}
In this case, we can interpret $|c_+|^2$ as the probability that
an observation of $S_z$ will yield the result $+\hbar/2$, and
$|c_-|^2$ as the probability that an observation of $S_z$
will yield the result $-\hbar/2$.

\section{Spin Precession}\label{sspinp}
According to classical physics, a small current loop possesses a {\em magnetic moment}\/ of magnitude $\mu=I\,A$, where
$I$ is the current circulating around the loop, and $A$ the area of the loop.
The direction of the magnetic moment is conventionally taken to be
normal to the plane of the loop, in the sense  given by a standard
right-hand circulation rule. Consider a small current loop consisting of an electron in uniform circular motion. It is
easily demonstrated that the electron's orbital angular momentum ${\bf L}$
is related to the magnetic moment $\bmu$ of the loop via
\begin{equation}\label{e10.57}
\bmu = -\frac{e}{2\,m_e}\,{\bf L},
\end{equation}
where $e$ is the magnitude of the electron charge, and $m_e$ the electron mass.

The above expression suggests that there may be a similar relationship between
magnetic moment and spin angular momentum. We can write
\begin{equation}\label{e10.58}
\bmu = -\frac{g\,e}{2\,m_e}\,{\bf S},
\end{equation}
where $g$ is called the {\em gyromagnetic ratio}. Classically, we would
expect $g=1$. In fact,
\begin{equation}\label{e10.59}
g = 2\left(1+\frac{\alpha}{2\pi}+\cdots\right) = 2.0023192,
\end{equation}
where $\alpha= e^2/(2\,\epsilon_0\,h\,c) \simeq 1/137$ is the so-called
{\em fine-structure constant}. The fact that the gyromagnetic ratio is
(almost) twice that expected from classical physics is only explicable using relativistic
quantum mechanics. Furthermore, the small corrections to the relativistic result $g=2$ come from quantum field theory.

The energy of a classical magnetic moment $\bmu$ in a uniform magnetic field ${\bf B}$ is
\begin{equation}\label{e10.60a}
H = - \bmu\cdot {\bf B}.
\end{equation}
Assuming that the above expression also holds good in quantum mechanics,
the Hamiltonian of an electron in a $z$-directed magnetic field of magnitude
$B$ takes the form
\begin{equation}\label{e10.60}
H = \Omega\,S_z,
\end{equation}
where
\begin{equation}
\Omega = \frac{g\,e\,B}{2\,m_e}.
\end{equation}
Here, for the sake of simplicity, we are neglecting the electron's translational degrees of freedom.

Schr\"{o}dinger's equation can be written
[see Eq.~(\ref{etimed})]
\begin{equation}\label{e10.62}
{\rm i}\,\hbar\,\frac{\partial\chi}{\partial t} = H\,\chi,
\end{equation}
where the spin state of the electron is characterized by the spinor $\chi$.
Adopting the Pauli representation, we obtain
\begin{equation}\label{e10.63}
\chi = \left(\begin{array}{c}c_+(t)\\c_-(t)\end{array}\right),
\end{equation}
where $|c_+|^2+|c_-|^2=1$. Here, $|c_+|^2$ is the probability of observing the
spin-up state, and $|c_-|^2$ the probability of observing the spin-down
state. It follows from Eqs.~(\ref{e10.46}), (\ref{e10.53}), (\ref{e10.60}),
(\ref{e10.62}), and (\ref{e10.63}) that
\begin{equation}
{\rm i}\,\hbar\left(\begin{array}{c}\dot{c}_+\\\dot{c}_-\end{array}\right)
=\frac{\Omega\,\hbar}{2} \left(\begin{array}{cc}1,&0\\
0,& -1\end{array}\right)\left(\begin{array}{c}c_+\\c_-\end{array}\right),
\end{equation}
where $\dot{~}\equiv d/dt$.
Hence, 
\begin{equation}\label{e10.65}
\dot{c}_\pm = \mp {\rm i}\,\frac{\Omega}{2}\,c_\pm.
\end{equation}
Let
\begin{eqnarray}
c_+(0) &=& \cos(\alpha/2),\label{e10.66}\\[0.5ex]
c_-(0) &=& \sin(\alpha/2).\label{e10.67}
\end{eqnarray}
The significance of the angle $\alpha$ will become apparent presently.
Solving Eq.~(\ref{e10.65}),  subject to the initial conditions
(\ref{e10.66}) and (\ref{e10.67}), we obtain
\begin{eqnarray}\label{e10.68}
c_+(t) &=& \cos(\alpha/2)\,\exp(-{\rm i}\,\Omega\,t/2),\\[0.5ex]
c_-(t)&=& \sin(\alpha/2)\,\exp(+{\rm i}\,\Omega\,t/2).\label{e10.69}
\end{eqnarray}

We can most easily visualize the effect of the time dependence in the above
expressions for $c_\pm$ by calculating the expectation
values of the three Cartesian components of the electron's spin angular momentum. By analogy
with Eq.~(\ref{e3.55}), the expectation value of a general spin operator
$A$ is simply
\begin{equation}
\langle A \rangle = \chi^\dag\,A\,\chi.
\end{equation}
Hence, the expectation value of $S_z$ is
\begin{equation}
\langle S_z\rangle= \frac{\hbar}{2}\left(c_+^\ast, c_-^\ast\right) \left(\begin{array}{cc}1,&0\\
0,& -1\end{array}\right)\left(\begin{array}{c}c_+\\ c_-\end{array}\right),
\end{equation}
which reduces to
\begin{equation}\label{e10.72}
\langle S_z \rangle = \frac{\hbar}{2}\,\cos\alpha
\end{equation}
with the help of Eqs.~(\ref{e10.68}) and (\ref{e10.69}). Likewise,
the expectation value of $S_x$ is
\begin{equation}
\langle S_x\rangle= \frac{\hbar}{2}\left(c_+^\ast, c_-^\ast\right) \left(\begin{array}{cc}0,&1\\
1,& 0\end{array}\right)\left(\begin{array}{c}c_+\\ c_-\end{array}\right),
\end{equation}
which reduces to
\begin{equation}\label{e10.74}
\langle S_x\rangle = \frac{\hbar}{2}\,\sin\alpha\,\cos(\Omega\,t).
\end{equation}
Finally, the expectation value of $S_y$ is
\begin{equation}\label{e10.75}
\langle S_y\rangle = \frac{\hbar}{2}\,\sin\alpha\,\sin(\Omega\,t).
\end{equation}
According to Eqs.~(\ref{e10.72}), (\ref{e10.74}), and (\ref{e10.75}),
the {\em expectation value}\/ of the spin angular momentum vector subtends
a constant angle $\alpha$ with the $z$-axis, and {\em precesses}\/ about
this axis at the frequency 
\begin{equation}
\Omega \simeq \frac{e\,B}{m_e}.
\end{equation}
This behaviour is actually equivalent to that predicted by classical physics.
Note, however, that a {\em measurement}\/ of $S_x$, $S_y$, or $S_z$ will always
yield either $+\hbar/2$ or $-\hbar/2$. It is the {\em relative probabilities}\/
of obtaining these two results which varies as the expectation value
of a given component of the spin varies.

\subsubsection*{Exercises}
{\small
\begin{enumerate}
\item Find the Pauli representations of  $S_x$, $S_y$, and $S_z$ for a spin-1 particle.

\item  Find the Pauli representations of the normalized eigenstates of $S_x$ and $S_y$ for
a spin-$1/2$ particle. 
\item Suppose that a spin-$1/2$ particle
has a spin vector which lies in the $x$-$z$ plane, making an
angle $\theta$ with the $z$-axis. Demonstrate that a measurement of $S_z$
yields $\hbar/2$ with probability $\cos^2(\theta/2)$, and $-\hbar/2$
with probability $\sin^2(\theta/2)$. 

\item An electron is in the spin-state 
$$
\chi = A\,\left(\begin{array}{c}1-2\,{\rm i}\\2\end{array}\right)
$$
in the Pauli representation. Determine the constant $A$ by normalizing
$\chi$. If a measurement of $S_z$ is made, what values will be
obtained, and with what probabilities? What is the expectation
value of $S_z$? Repeat the above calculations for $S_x$ and $S_y$. 


\item Consider a spin-$1/2$ system represented by the normalized spinor
$$
\chi =\left(\begin{array}{c}\cos\alpha\\\sin\alpha\,\exp(\,{\rm i}\,\beta)\end{array}\right)
$$
in the Pauli representation, where $\alpha$ and $\beta$ are real. What is the probability that a measurement of
$S_y$ yields $-\hbar/2$? 

\item An electron is at rest in an oscillating magnetic field
$$
{\bf B} = B_0\,\cos(\omega\,t)\,{\bf e}_z,
$$
where $B_0$ and $\omega$ are real positive constants. 
\begin{enumerate}
\item Find the Hamiltonian of the system.
\item If the electron starts in the spin-up state with respect to the
$x$-axis, determine the spinor $\chi(t)$ which represents the state
of the system in the Pauli representation at all subsequent times.
\item Find the probability that a measurement of $S_x$ yields
the result $-\hbar/2$ as a function of time.
\item What is the minimum value of $B_0$ required to force a
complete flip in $S_x$?
\end{enumerate}
\end{enumerate}}