\chapter{Multi-Particle Systems}\label{smany}
\section{Introduction}
In this chapter, we shall extend the single particle, one-dimensional formulation of non-relativistic 
quantum mechanics introduced in the previous sections in order to investigate one-dimensional chapters containing {\em multiple}\/ particles.

\section{Fundamental Concepts}\label{sfuncon}
We have already seen that the instantaneous
state of a system consisting of a single non-relativistic particle, whose position coordinate
is $x$, is fully specified by
a complex wavefunction $\psi(x,t)$. This wavefunction is interpreted
as follows. The probability of finding the particle between $x$ and
$x+dx$ at time $t$ is given by $|\psi(x,t)|^2\,dx$. This interpretation only makes sense if the wavefunction is
normalized such that
\begin{equation}
\int_{-\infty}^\infty |\psi(x,t)|^2\,dx = 1
\end{equation}
at all times. The physical significance of this normalization requirement is
that the probability of the particle being found anywhere on the $x$-axis
must always be unity (which corresponds to certainty).

Consider a system containing $N$ non-relativistic particles, labeled $i=1, N$, moving in one dimension. Let $x_i$ and $m_i$ be the position coordinate
and mass, respectively, of the $i$th particle.  
By analogy with the single-particle case, the instantaneous state of a multi-particle system is 
specified by a complex wavefunction $\psi(x_1,x_2,\ldots, x_N,t)$.
The probability of finding the first particle between $x_1$ and $x_1+dx_1$,
the second particle between $x_2$ and $x_2+dx_2$, {\em etc}., at time
$t$ is given by $|\psi(x_1,x_2,\ldots, x_N,t)|^2\,dx_1\,dx_2\ldots dx_N$.
It follows that the wavefunction must satisfy the normalization condition
\begin{equation}\label{en1}
\int |\psi(x_1,x_2,\ldots, x_N,t)|^2\,dx_1\,dx_2\ldots dx_N = 1
\end{equation}
at all times, where the integration is taken over all $x_1\,x_2\,\ldots x_N$
space.

In a single-particle system, position is represented by the algebraic operator $x$,
whereas momentum is represented by the differential operator
$-{\rm i}\,\hbar\,\partial/\partial x$ (see Sect.~\ref{s4.6}). By analogy,
in a multi-particle system, the position of the $i$th particle
is represented by the algebraic operator $x_i$, whereas the corresponding
momentum is represented by the differential operator
\begin{equation}\label{ex3}
p_i = - {\rm i}\,\hbar\,\frac{\partial}{\partial x_i}.
\end{equation}

Since the $x_i$ are {\em independent}\/ variables ({\em i.e.},
$\partial x_i/\partial x_j=\delta_{ij}$), we conclude that the
various position and momentum operators satisfy the following commutation
relations:
\begin{eqnarray}\label{xe6.4}
[x_i,x_j] &=& 0,\\[0.5ex]
[p_i,p_j] &=& 0,\\[0.5ex]
[x_i,p_j] &=& {\rm i}\,\hbar\,\delta_{ij}.\label{xe6.6}
\end{eqnarray}
Now, we know, from Sect.~\ref{smeas}, that two dynamical variables
can only be (exactly) measured {\em simultaneously}\/ if the operators which represent
them in quantum mechanics {\em commute}\/ with one another. Thus,
it is clear, from the above commutation relations, that the only restriction
on measurement in a one-dimensional multi-particle system is that it is impossible to
simultaneously measure the position and momentum of the {\em same}\/
particle. Note, in particular, that a knowledge of the position or momentum of a given
particle does not in any way preclude a similar knowledge for a different
particle. The commutation relations (\ref{xe6.4})--(\ref{xe6.6}) illustrate
an important point in quantum mechanics: namely, that operators corresponding to {\em different degrees
of freedom}\/ of a dynamical system tend to {\em commute}\/ with one another.
In this case, the different degrees of freedom correspond to the different
motions of the various particles making up the system.

Finally, if $H(x_1,x_2,\ldots, x_N,t)$ is the Hamiltonian of the system then
the multi-particle wavefunction $\psi(x_1,x_2,\ldots,x_N,t)$ satisfies
the usual time-dependent Schr\"{o}\-dinger equation [see Eq.~(\ref{etimed})]
\begin{equation}\label{ex7}
{\rm i}\,\hbar\,\frac{\partial\psi}{\partial t} = H\,\psi.
\end{equation}
Likewise, a multi-particle state of definite energy $E$ ({\em i.e.}, an
eigenstate of the Hamiltonian with eigenvalue $E$) is written (see Sect.~\ref{sstat})
\begin{equation}
\psi(x_1,x_2,\ldots, x_N,t)  = \psi_E(x_1,x_2,\ldots, x_N)\,{\rm e}^{-{\rm i}\,E\,t/\hbar},
\end{equation}
where the stationary wavefunction $\psi_E$ satisfies the time-independent
Schr\"{o}d\-inger equation [see Eq.~(\ref{etimei})]
\begin{equation}\label{ex9}
H\,\psi_E = E\,\psi_E.
\end{equation}
Here, $H$ is assumed not to be an explicit function of $t$.

\section{Non-Interacting Particles}\label{snon}
In general, we expect the Hamiltonian of a multi-particle system to take the form
\begin{equation}\label{ex10}
H(x_1,x_2,\ldots, x_N, t) = \sum_{i=1,N}\frac{p_i^{\,2}}{2\,m_i}
+ V(x_1,x_2,\ldots, x_N, t).
\end{equation}
Here, the first term on the right-hand side represents the total kinetic
energy of the system, whereas the potential $V$ specifies the nature of the interaction between the
various particles making up the system, as well as the interaction of the
particles with any external forces.

 Suppose that the particles do not interact with one another. This
 implies that each particle moves in a common potential: {\em i.e.},
 \begin{equation}
V(x_1,x_2,\ldots, x_N,t) = \sum_{i=1,N} V(x_i,t).
 \end{equation}
  Hence, we can write
 \begin{equation}\label{ex11}
 H(x_1,x_2,\ldots, x_N,t)=\sum_{i=1,N} H_i(x_i,t),
 \end{equation}
 where 
 \begin{equation}\label{ex12}
 H_i = \frac{p_i^{\,2}}{2\,m_i} + V(x_i,t).
 \end{equation}
 In other words, for the case of non-interacting particles, the
 multi-particle Hamiltonian of the system can be written as the
 sum of $N$ independent single-particle Hamiltonians. Here, $H_i$
 represents the energy of the $i$th particle, and is completely
 unaffected by the energies of the other particles.
 Furthermore, given that the various particles which make up the
 system are non-interacting, we expect their instantaneous positions to be completely
 {\em uncorrelated}\/ with one another. This immediately implies that
 the multi-particle wavefunction $\psi(x_1,x_2,\ldots x_N,t)$ can
 be written as the product of $N$ independent single-particle 
 wavefunctions: {\em i.e.}, 
 \begin{equation}\label{ex13}
 \psi(x_1,x_2,\ldots, x_N, t) = \psi_1(x_1,t)\,\psi_2(x_2,t)\ldots\psi_N(x_N,t).
 \end{equation}
 Here, $|\psi_i(x_i,t)|^2\,dx_i$ is the probability of finding the
 $i$th particle between $x_i$ and $x_i+dx_i$ at time $t$. This probability
 is completely unaffected by the positions of the other particles. It
 is evident that $\psi_i(x_i,t)$ must satisfy the normalization constraint
 \begin{equation}
 \int_{-\infty}^\infty |\psi_i(x_i,t)|^2\,dx_i = 1.
 \end{equation}
 If this is the case  then the normalization constraint (\ref{en1}) for the
 multi-particle wavefunction is automatically satisfied.
 Equation (\ref{ex13}) illustrates an important point in quantum mechanics: namely, that we can generally write the total wavefunction of a many degree of freedom system as a product of different wavefunctions corresponding to each degree
 of freedom.
 
 According to Eqs.~(\ref{ex11}) and (\ref{ex13}), the time-dependent
 Schr\"{o}dinger equation (\ref{ex7}) for a system of $N$ non-interacting
 particles factorizes into $N$ independent equations of the form
 \begin{equation}
 {\rm i}\,\hbar\,\frac{\partial \psi_i}{\partial t} = H_i\,\psi_i.
 \end{equation}
 Assuming that $V(x,t)\equiv V(x)$,  the time-independent Schr\"{o}dinger equation (\ref{ex9}) 
 also factorizes to give
 \begin{equation}
 H_i\,\psi_{E_i} =E_i\,\psi_{E_i},
 \end{equation}
 where 
  $\psi_i(x_i,t) = \psi_{E_i}(x_i)\,\exp(-{\rm i}\, E_i\,t/\hbar)$,
  and $E_i$ is the energy of the $i$th particle.
 Hence, a multi-particle state of definite energy $E$ has a
 wavefunction of the form
 \begin{equation}
 \psi(x_1,x_2,\ldots, x_n, t) =\psi_E(x_1,x_2,\ldots, x_N)\,{\rm
 e}^{-{\rm i}\,E\,t/\hbar},
 \end{equation}
 where
 \begin{equation}
\psi_E(x_1,x_2,\ldots, x_N) =  \psi_{E_1}(x_1)\,\psi_{E_2}(x_2)\ldots\psi_{E_N}(x_N),
 \end{equation}
 and
 \begin{equation}
 E = \sum_{i=1,N} E_i.
 \end{equation}
 Clearly, for the
 case of non-interacting particles, the energy of
 the whole system is simply the sum of the energies of the component
 particles.
 
 \section{Two-Particle Systems}\label{stwo}
 Consider a system consisting of two particles, mass $m_1$ and $m_2$, 
 interacting via the potential $V(x_1-x_2)$ which only depends on the
 {\em relative positions}\/ of the particles.  According to Eqs.~(\ref{ex3})
 and (\ref{ex10}), the Hamiltonian of the system is written
 \begin{equation}
 H(x_1,x_2) = -\frac{\hbar^2}{2\,m_1}\frac{\partial^2}{\partial x_1^{\,2}}
 - \frac{\hbar^2}{2\,m_2}\frac{\partial^2}{\partial x_2^{\,2}}+ V(x_1-x_2).
 \end{equation}
 Let
 \begin{equation}
 x' = x_1-x_2
 \end{equation}
 be the particles' relative position, and
 \begin{equation}
 X = \frac{m_1\,x_1+m_2\,x_2}{m_1+m_2}
 \end{equation}
 the position of the center of mass. 
 It is easily demonstrated that
 \begin{eqnarray}
 \frac{\partial}{\partial x_1} = \frac{m_1}{m_1+m_2}\frac{\partial}{\partial X} + \frac{\partial}{\partial x'},\\[0.5ex]
  \frac{\partial}{\partial x_2} = \frac{m_2}{m_1+m_2}\frac{\partial}{\partial X} - \frac{\partial}{\partial x'}.
 \end{eqnarray}
 Hence, when expressed in terms of the new variables, $x'$ and $X$, 
 the Hamiltonian becomes
 \begin{equation}\label{ex6.24}
 H(x',X) = -\frac{\hbar^2}{2\,M} \frac{\partial^2}{\partial X^2}
 -\frac{\hbar^2}{2\,\mu}\frac{\partial^2}{\partial x'^{\,2}} + V(x'),
 \end{equation}
 where
 \begin{equation}
 M = m_1+ m_2
 \end{equation}
 is the total mass of the system, and
 \begin{equation}
 \mu = \frac{m_1\,m_2}{m_1+m_2}
 \end{equation}
 the so-called {\em reduced mass}.
 Note that the total momentum of the system can be written
 \begin{equation}\label{exa}
P= -{\rm i}\,\hbar\left(\frac{\partial}{\partial x_1} + \frac{\partial}{\partial x_2}\right) = -{\rm i}\,\hbar\,\frac{\partial}{\partial X}.
 \end{equation}
 
 The fact that the Hamiltonian (\ref{ex6.24}) is separable when expressed
 in terms of the new coordinates  [{\em i.e.},
 $H(x',X) = H_{x'}(x') + H_X(X)]$ suggests, by analogy with the analysis
 in the previous section, that the wavefunction can be factorized: {\em i.e.}, 
 \begin{equation}\label{exb}
 \psi(x_1,x_2,t) = \psi_{x'}(x',t)\,\psi_X(X,t).
 \end{equation}
 Hence, the time-dependent Schr\"{o}dinger equation (\ref{ex7})
 also factorizes to give
 \begin{equation}
 {\rm i}\,\hbar\,\frac{\partial\psi_{x'}}{\partial t} = -\frac{\hbar^2}{2\,\mu}
 \frac{\partial^2\psi_{x'}}{\partial x'^{\,2}} + V(x')\,\psi_{x'},
 \end{equation}
 and
 \begin{equation}
 {\rm i}\,\hbar\,\frac{\partial\psi_X}{\partial t} = -\frac{\hbar^2}{2\,M}\frac{\partial^2\psi_X}{\partial X^2}.
 \end{equation}
 The above equation can be solved to give
 \begin{equation}\label{ex33}
 \psi_X(X,t) = \psi_{0}\,{\rm e}^{\,{\rm i}\,(P'\,X/\hbar - E'\,t/\hbar)},
 \end{equation}
 where $\psi_0$, $P'$, and $E' = P'^{\,2}/2\,M$ are constants. It is clear, from Eqs.~(\ref{exa}),  (\ref{exb}),  and (\ref{ex33}), that
 the total momentum of the system takes the constant value $P'$:
 {\em i.e.}, momentum is conserved.
 
 Suppose that we work in the {\em centre of mass frame}\/ of the system, which is characterized by
 $P'=0$. It follows that
 $\psi_X=\psi_0$. In this case, we can write the wavefunction of the system in the form
 $\psi(x_1,x_2,t) = \psi_{x'}(x',t)\,\psi_0\equiv \psi(x_1-x_2,t)$, where 
 \begin{equation}
 {\rm i}\,\hbar\,\frac{\partial\psi}{\partial t} = -\frac{\hbar^2}{2\,\mu}
 \frac{\partial^2\psi}{\partial x^{\,2}} + V(x)\,\psi.
 \end{equation}
 In other words, in the center of mass frame, two particles of mass $m_1$
 and $m_2$, moving in the potential $V(x_1-x_2)$, are equivalent
 to a single particle of mass $\mu$, moving in the potential $V(x)$,
 where $x=x_1-x_2$. This is a familiar result from classical dynamics.

\section{Identical Particles}\label{siden}
Consider a system consisting of two {\em identical}\/ particles of mass $m$.
As before, the instantaneous state of the system is specified by the
complex wavefunction $\psi(x_1,x_2,t)$. However, the only thing
which this wavefunction tells us is that the probability of finding the
first particle between $x_1$ and $x_1+dx_1$, and the second 
between $x_2$ and $x_2+dx_2$, at time $t$ is $|\psi(x_1,x_2,t)|^2\,dx_1\,dx_2$. However, since the particles are
identical, this must be the same as the probability of finding the 
first particle between $x_2$ and $x_2+dx_2$, and the second
between $x_1$ and $x_1+dx_1$, at time $t$ (since, in both
cases, the physical outcome of the measurement is exactly the same).
Hence, we conclude that
\begin{equation}
|\psi(x_1,x_2,t)|^2 = |\psi(x_2,x_1,t)|^2,
\end{equation}
or
\begin{equation}
\psi(x_1,x_2,t) = {\rm e}^{\,{\rm i}\,\varphi}\,\psi(x_2,x_1,t),
\end{equation}
where $\varphi$ is a real constant. However, if we swap the labels on
particles 1 and 2 (which are, after all, arbitrary for identical particles), and repeat the argument, we also conclude that
\begin{equation}
\psi(x_2,x_1,t) = {\rm e}^{\,{\rm i}\,\varphi}\,\psi(x_1,x_2,t).
\end{equation}
Hence,
\begin{equation}
{\rm e}^{\,2\,{\rm i}\,\varphi} = 1.
\end{equation}
The only solutions to the above equation are $\varphi=0$ and $\varphi=\pi$.
Thus, we infer that for a system consisting of two identical particles, the wavefunction
must be either {\em symmetric}\/ or {\em anti-symmetric}\/ under interchange
of particle labels: {\em i.e.}, either
\begin{equation}
\psi(x_2,x_1,t) = \psi(x_1,x_2,t),
\end{equation}
or
\begin{equation}
\psi(x_2,x_1,t) = -\psi(x_1,x_2,t).
\end{equation}
The above argument can easily be extended to systems containing more
than two identical particles.

It turns out that whether the wavefunction of a
system containing many identical particles is symmetric or anti-symmetric
under interchange of the labels on any two particles  is determined by the nature
of the particles themselves. Particles with wavefunctions which are {\em symmetric}\/ under
label interchange
are said to obey {\em Bose-Einstein}\/ statistics, and
are called {\em bosons}---for instance, photons are bosons. Particles with 
wavefunctions which are {\em anti-symmetric}\/ 
under label interchange are said to obey {\em Fermi-Dirac}\/
statistics, and are called {\em fermions}---for instance, electrons,
protons, and neutrons are fermions.

Consider a system containing two identical and non-interacting bosons.
Let $\psi(x,E)$ be a properly normalized, single-particle, stationary wavefunction corresponding to a state of definite energy $E$.
The stationary wavefunction of the
whole system is written
\begin{equation}
\psi_{E\,boson}(x_1,x_2) = \frac{1}{\sqrt{2}}\left[\psi(x_1,E_a)\,\psi(x_2,E_b)+\psi(x_2,E_a)\,\psi(x_1,E_b)\right],
\end{equation}
when the energies of the two particles are $E_a$ and $E_b$. This
expression automatically satisfies the symmetry requirement on the
wavefunction. Incidentally, since the particles are
identical, we cannot be sure which particle
has energy $E_a$, and which has energy $E_b$---only that one particle
has energy $E_a$, and the other $E_b$.

For a system consisting of two identical and non-interacting fermions,
the stationary wavefunction of the whole system takes
the form
\begin{equation}
\psi_{E\,fermion}(x_1,x_2) = \frac{1}{\sqrt{2}}\left[\psi(x_1,E_a)\,\psi(x_2,E_b)-\psi(x_2,E_a)\,\psi(x_1,E_b)\right],
\end{equation}
Again, this expression automatically satisfies the symmetry requirement on
the wavefunction. Note that if $E_a=E_b$ then the total wavefunction
becomes zero everywhere. Now, in quantum mechanics, a null wavefunction
corresponds to the absence of a state. We thus conclude that it
is impossible for the two fermions in our system to occupy the
same single-particle stationary state. 

Finally, if the two particles are somehow distinguishable then the stationary
wavefunction of the system is simply
\begin{equation}
\psi_{E\,dist}(x_1,x_2) = \psi(x_1,E_a)\,\psi(x_2,E_b).
\end{equation}

Let us evaluate the variance of the distance, $x_1-x_2$, between the
two particles, using the above three wavefunctions. It is easily
demonstrated that if the particles are distinguishable then
\begin{equation}
\langle (x_1-x_2)^2\rangle_{dist} = \langle x^2\rangle_a + \langle x^2\rangle_b - 2\,\langle x\rangle_a\,\langle x\rangle_b,
\end{equation}
where
\begin{equation}
\langle x^n\rangle_{a,b} = \int_{-\infty}^\infty\psi^\ast(x,E_{a,b})\,x^n\,\psi(x,E_{a,b})\,dx.
\end{equation}
For the case of two identical bosons, we find
\begin{equation}\label{ebos}
\langle (x_1-x_2)^2\rangle_{boson} = \langle (x_1-x_2)^2\rangle_{dist} -
2\,|\langle x\rangle_{ab}|^2,
\end{equation}
where
\begin{equation}
\langle x \rangle_{ab} = \int_{-\infty}^\infty \psi^\ast(x,E_a)\,x\,\psi(x,E_b)\,dx.
\end{equation}
Here, we have assumed that $E_a\neq E_b$, so that
\begin{equation}
\int_{-\infty}^\infty \psi^\ast(x,E_a)\,\psi(x,E_b)\,dx = 0.
\end{equation}
Finally, for the case of two identical fermions,  we obtain
\begin{equation}\label{efer}
\langle (x_1-x_2)^2\rangle_{fermion} = \langle (x_1-x_2)^2\rangle_{dist} +
2\,|\langle x\rangle_{ab}|^2,
\end{equation}
Equation~(\ref{ebos})  shows that
the symmetry requirement on the total wavefunction of two identical {\em bosons}\/ forces the
particles to be, on average, {\em closer together}\/ than two similar distinguishable
particles. Conversely, Eq.~(\ref{efer}) shows that the symmetry requirement on the total wavefunction of two identical {\em fermions}\/ forces the
particles to be, on average, {\em further apart}\/ than two similar distinguishable
particles. However, the strength of this effect depends on square of the magnitude of $\langle x\rangle_{ab}$, which measures the {\em overlap}\/ between the
wavefunctions $\psi(x,E_a)$ and $\psi(x,E_b)$. It is evident, then, that
if these two wavefunctions do not overlap to any great extent then identical
bosons or fermions will act very much like distinguishable particles. 

For a system containing $N$ identical and non-interacting fermions,
the anti-symmetric stationary wavefunction of the system
is written
\begin{eqnarray}
\psi_{E}(x_1,x_2,\ldots x_N)& =&\frac{1}{\sqrt{N!}} \left|
\begin{array}{cccc}
\psi(x_1,E_1)&\psi(x_2,E_1)&\ldots&\psi(x_N,E_1)\\[0.5ex]
\psi(x_1,E_2)&\psi(x_2,E_2)&\ldots&\psi(x_N,E_2)\\[0.5ex]
\vdots&\vdots&\vdots&\vdots\\[0.5ex]
\psi(x_1,E_N)&\psi(x_2,E_N)&\ldots&\psi(x_N,E_N)
\end{array}\right|.
\end{eqnarray}
This expression is known as the {\em Slater determinant}, and automatically
satisfies the symmetry requirements on the wavefunction.
Here, the energies of the particles are $E_1, E_2, \ldots, E_N$.
Note, again, that if any two particles in the system have the same energy
({\em i.e.}, if $E_i=E_j$ for some $i\neq j$)
then the total wavefunction is null. We conclude that it is impossible
for any two identical fermions in a multi-particle system to
occupy the same single-particle stationary state. This important result is known as
the {\em Pauli exclusion principle}. 

\subsubsection*{Exercises {\rm (N.B.\ Neglect spin in the following questions.)}}
{\small
\begin{enumerate}
\item Consider a system consisting of two non-interacting particles, and three
one-particle states, $\psi_a(x)$, $\psi_b(x)$, and $\psi_c(x)$. How
many different two-particle states can be constructed if the particles are
(a) distinguishable, (b) indistinguishable bosons, or (c) indistinguishable
fermions? 

\item Consider two non-interacting particles, each of mass $m$, in 
a one-dimensional harmonic oscillator potential of classical oscillation
frequency $\omega$. If one particle is in the ground-state, and the
other in the first excited state, calculate $\langle (x_1-x_2)^2\rangle$
assuming that the particles are (a) distinguishable, (b) indistinguishable bosons, or (c) indistinguishable fermions.

\item Two non-interacting particles, with the same mass $m$, are
in a one-dimensional box of length $a$. What are the four lowest
energies of the system? What are the degeneracies of these
energies if the two particles are (a) distinguishable, (b) indistinguishable
bosons, or (c) indistingishable fermions? 

\item Two particles in a one-dimensional box of length $a$ occupy
the $n=4$ and $n'=3$ states. Write the properly normalized
wavefunctions if the particles are (a) distinguishable, (b) indistinguishable
bosons, or (c) indistinguishable fermions.

\end{enumerate}
}