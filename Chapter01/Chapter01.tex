\chapter{Introduction}
\section{Intended audience}
These lecture notes outline a single semester course on {\em non-relativistic quantum mechanics}\/   which  is primarily intended for {\em upper-division undergraduate physics majors}.
The course assumes some previous knowledge of physics and mathematics. In particular, prospective students should be reasonably familiar with Newtonian dynamics, elementary classical electromagnetism and special relativity, the physics and mathematics of waves (including the representation of
waves via complex functions), basic
probability theory, ordinary and partial differential equations, linear algebra, vector
algebra, and Fourier series and transforms.

\section{Major Sources}
The textbooks which I have consulted most frequently whilst developing course material are:
\begin{description}
\item {\em The Principles of Quantum Mechanics},
P.A.M.~Dirac, 4th Edition (revised), (Oxford University Press, Oxford UK, 1958).
\item {\em Quantum Mechanics}, E.~Merzbacher, 2nd Edition, (John Wiley \& Sons, New York NY, 1970).
\item {\em Introduction to the Quantum Theory}, D.~Park, 2nd Edition,
(McGraw-Hill, New York NY, 1974).
\item {\em Modern Quantum Mechanics}, J.J.~Sakurai, (Benjamin/Cummings,
Menlo Park CA, 1985).
\item {\em Quantum Theory}, D.~Bohm, (Dover, New York NY, 1989).
\item {\em Problems in Quantum Mechanics}, G.L.~Squires, (Cambridge University Press, Cambridge UK, 1995).
\item {\em Quantum Physics}, S.~Gasiorowicz, 2nd Edition,  (John Wiley \& 
Sons, New York NY, 1996).
\item {\em Nonclassical Physics}, R.~Harris, (Addison-Wesley, Menlo Park CA, 1998).
\item {\em Introduction to Quantum Mechanics},  D.J.~Griffiths, 2nd Edition, (Pearson
Prentice Hall, Upper Saddle River NJ, 2005).
\end{description}

\section{Aim of Course}
The aim of this course is to develop non-relativistic quantum mechanics
as a complete theory of microscopic dynamics, capable of making detailed
predictions, with a minimum of abstract mathematics.

\section{Outline of Course}
The first part of the course is devoted to an in-depth exploration of the basic
principles of quantum mechanics.  
After
a brief review of probability theory, in Chapter~2, we shall start, in Chapter~3,  by examining how many of the central
ideas of quantum mechanics are a direct consequence of wave-particle
duality---{\em i.e.}, the concept that waves sometimes act as particles, and particles
as waves. We shall then proceed to investigate the rules of quantum mechanics in a more systematic fashion in Chapter~4. Quantum mechanics is used to examine the motion of a single
particle in one dimension, many particles in one dimension, and a single
particle in three dimensions, in Chapters~5, 6,  and 7,
respectively. Chapter~8 is devoted to the investigation of orbital
angular momentum, and Chapter~9 to the closely related subject of
particle motion in a central potential. Finally, in Chapters~10 and 11,
we shall examine spin angular momentum, and the addition of orbital and spin
angular momentum, respectively.

The second part of this course describes
selected practical applications of quantum mechanics. In Chapter~12, time-independent perturbation theory is used to investigate the Stark effect,
the Zeeman effect, fine structure, and hyperfine structure, in the hydrogen
atom. Time-dependent perturbation theory is employed to study
radiative transitions in the hydrogen atom in Chapter~13. Chapter~14
illustrates the use of variational methods in quantum mechanics.
Finally, Chapter~15 contains an introduction to quantum scattering theory.
